\section{10-12-2022}

% \epigraph{``I ''}{Charlie}

Outer Hausdorff measure.

the $\alpha$-dimensional outer Hausdorff measure of a set $A \subseteq \mathbb{R}^d$ is $H^{\alpha} (A) = \sup H_{r}^{\alpha} (A)$ 

where
\begin{align*}
	H_{r}^{\alpha} (A) = \inf\{ \sum_{k \geq 0} diam (B_k)^{\alpha} : A \subseteq \bigcup_{k \geq 0}, \quad \sup_{k \geq 0} diam (B_{k} ) \leq r \}
\end{align*}

\begin{example}
	$H_{1}^{\alpha} \leq 1$ for each $A \subseteq B_{\frac{1}{2}}$ 

	$H_{1}^{1}$ does not measure length.

	$\sup_{r > 0} \{H_{r}^{1}\}$ does measure length
\end{example}

\begin{lemma}
	$H^{\alpha}$ is an outer measure.
\end{lemma}

\begin{proof} several steps.
	\begin{enumerate}
		\item[step 1:] $H_{r}^{\alpha}$ is an outer measure by same argument for $L$.

			\begin{align*}
			H_{r}^{\alpha} (\emptyset)= 0 \quad \text{and} \quad \implies H_{r}^{\alpha} (A) \leq H_{r}^{\alpha} (B)
			\end{align*} 
			immediate from definition.

			Subadditivity:
			\begin{align*}
			H_{r}^{\alpha} ( \cup_{k \geq 0} A_{k}) \leq \sum_{K \geq 0} H_{r}^{\alpha} (A_k)
			\end{align*} Follows from countable union of countable is countable.

		\item[step 2:] Immediate that $H^{\alpha} (\emptyset) = 0$ and 
			\begin{align*}
			A \subseteq B \implies H^{\alpha} (A) \leq H^{\alpha}(B)
			\end{align*} Assume
			\begin{align*}
				\infty > H^{\alpha}( \bigcup_{k \geq 0} A_k ) &= \sup_{r > 0} H_{r}^{\alpha} ( \bigcup_{k \geq 0} A_k) \\
															  &\leq \varepsilon + H_{r}^{\alpha} ( \bigcup_{k \geq 0} A_k) \\ 
															  &\leq \varepsilon +  \sum_{k \geq 0} \sup_{r > 0} H_{r}^{\alpha} ( A_k) \\ 
															  &= \varepsilon +  \sum_{k \geq 0} H^{\alpha} ( A_k) \\ 
			\end{align*} since $\varepsilon$ arbirary, 
			\begin{align*}
			H^{\alpha} ( \bigcup_{k \geq 0} A_k) \leq \sum_{k \geq 0} H^{\alpha} ( A_k)
			\end{align*}
			In infinite case, use similar argument to show
		\begin{align*}
					H^{\alpha} ( \bigcup_{k \geq 0} A_k) = \infty \implies \sum_{k \geq 0} H^{\alpha} ( A_k) = \infty
		\end{align*} 
	\end{enumerate}
\end{proof}

\begin{definition}[Metric outer measure]
	A metric outer measure on a metric space is an outer measure $m \mathcal{P}(X) \to [0,\infty]$ s.t. 
	\begin{align*}
	m(A \cup B) = m(A) + m(B) \quad \text{whenever} \quad d(A,B) > 0
	\end{align*} 
	where  
	\begin{align*}
		d(A,B) = \inf_{\underset{a \in A}{b \in B}} d(a,b)
	\end{align*} 
\end{definition}

\begin{example}
	$H^{\alpha}$ is a metric outer measure. indeed
	\begin{align*}
	H_{r}^{\alpha}(A \cup B) = H_{r}^{\alpha}(A) + H_{r}^{\alpha}(B)  \quad \text{where} \quad d(A,B) > 0
	\end{align*} 
	Because we can effectviely split any cover of $A \cup B$ into a cover of $A$ and a cover of $B$.
\end{example}

\begin{lemma}
	Borel sets are caratheodory for metric outer measure.
\end{lemma}

come back to
\begin{proof}:
	\begin{enumerate}
		\item[step 1]
	\end{enumerate}
\end{proof}


\begin{theorem}
	The restriction of $H^{\alpha}$ to borel sets is a measure.
\end{theorem}


\begin{definition}
	$\alpha$-dimensional Hausdorff measure is restriction of $H^{\alpha}$ to Borel sets.
\end{definition}


\begin{example}
	$H^{0}$ is counting measure.
\end{example}


\begin{example}
	$H^{\alpha} (A) = 0$ when $\alpha > 0$ 
	(insert image) we can conver  $[0,1]^{d}$ by $2^{nd}$ subcubes of size $2^{-n}$.

	this shows:
	\begin{align*}
		H^{\alpha} ([0,1]^d) \leq \lim_{n \to \infty} 2^{nd}( \sqrt{d} 2^{-n})^{\alpha} = 0
	\end{align*}
\end{example}

\begin{example}
	$H^{d} = \beta L$ for some $\beta > 0$
\end{example}

\begin{theorem}
	If $m$ borel measure on $\mathbb{R}^{d}$, $m(x + A) = m(A)$ for $A \in \mathbb{R}^d$ and $A \subseteq \mathbb{R}^d$ Borel and $m([0,1]^{d}) =1$, then $m = L$
\end{theorem}

\begin{proof}
	Homework.
\end{proof}

\begin{definition}
	The Hausdorff dimension of a set $A \subseteq \mathbb{R}^d$ is
	\begin{align*}
		\dim_{H} (A) = \inf\{\alpha \geq 0 : H^{\alpha} (A) = 0 \}
	\end{align*}
\end{definition}

\begin{lemma}
	For any $d$, $A \subseteq \mathbb{R}^d$ there is an $\alpha \in [0,d]$ s.t. 
	\begin{align*}
					H^{\beta} (A) &= 0 \quad \text{for} \quad  \beta > \alpha \\
		\text{and} \quad H^{\beta} (A) &= \infty \quad \text{for} \quad \beta < \alpha
	\end{align*} 
\end{lemma}


\begin{example}
	the $\frac{1}{3}$ cantor set $\mathcal{C}$ as Hausdorff dim: $\alpha = \frac{\log 2}{\log 3}$ 
	\begin{proof}
		At spet $K$ have cover of $\mathcal{C}$ consisting of $2^{k}$ intervals of length $3^{-k}$ this gives us 
		\begin{align*}
		\lim_{k \to \infty} (2^{k}\cdot 3^{-\alpha}) \in (0, \infty) \\
		\iff \alpha = \frac{\log 2}{\log 3}
		\end{align*} 
	\end{proof}
	
\end{example}

\begin{example}
	The Kock curve has dimension: $\alpha = \frac{\log 4}{\log 3} \in [1,2)$ 
	\begin{proof}
		Can cover with $4^{k}$ intervals of length $3^{-k}$
	\end{proof}
	
\end{example}

