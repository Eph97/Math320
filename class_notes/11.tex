\section{October 12, 2022}

\epigraph{*accidentally kicks over classroom phone* ``I don't understand why there is a phone is here at all.''}{Charlie}

\subsection{Outer Hausdorff Measure}
The $\alpha$-dimensional outer Hausdorff measure of a set $A \subseteq \mathbb{R}^d$ is
\[
	H^{\alpha} (A) = \sup_{r > 0} H_{r}^{\alpha} (A)
\]
where
\[
	H_{r}^{\alpha} (A) = \inf\left\{ \sum_{k \geq 0} \diam (B_k)^{\alpha} : A \subseteq \bigcup_{k \geq 0} B_k, \; \sup_{k \geq 0} \diam (B_{k} ) \leq r \right\}
\]

\begin{example}
	$H_{1}^{\alpha} \leq 1$ for each $A \subseteq B_{\frac{1}{2}}$
  \begin{itemize}
		\item $H_{1}^{1}$ \underline{does not} measure length.
		\item $\sup_{r > 0} \{H_{r}^{1}\}$ \underline{does} measure length
  \end{itemize}
\end{example}

\begin{lemma}
	$H^{\alpha}$ is an outer measure.
\end{lemma}

\begin{proof} The proof consists of several steps.
	\begin{enumerate}
		\item[Step 1:] $H_{r}^{\alpha}$ is an outer measure by same argument for $L$.
			\[
				H_{r}^{\alpha} (\emptyset)= 0 \quad \text{and} \quad A \subseteq B \implies H_{r}^{\alpha} (A) \leq H_{r}^{\alpha} (B)
			\]
			is immediate from definition.
			The subadditivity:
			\begin{align*}
			H_{r}^{\alpha} ( \cup_{k \geq 0} A_{k}) \leq \sum_{K \geq 0} H_{r}^{\alpha} (A_k)
			\end{align*} follows from countable union of countable sets is countable.

		\item[Step 2:] It is immediate that $H^{\alpha} (\emptyset) = 0$ and
			\begin{align*}
			A \subseteq B \implies H^{\alpha} (A) \leq H^{\alpha}(B)
			\end{align*} Assume
			\begin{align*}
				\infty > H^{\alpha} \bigg( \bigcup_{k \geq 0} A_k \bigg) &= \sup_{r > 0} H_{r}^{\alpha} \bigg( \bigcup_{k \geq 0} A_k \bigg) \\
																&\leq \varepsilon + H_{r}^{\alpha} \bigg( \bigcup_{k \geq 0} A_k \bigg) \\
																&\leq \varepsilon + \sum_{k \geq 0} \sup_{r > 0} H_{r}^{\alpha} ( A_k) \\
																&= \varepsilon + \sum_{k \geq 0} H^{\alpha} ( A_k)
			\end{align*}
			since $\varepsilon$ arbitrary,
			\begin{align*}
				H^{\alpha} \bigg( \bigcup_{k \geq 0} A_k \bigg) \leq \sum_{k \geq 0} H^{\alpha} ( A_k)
			\end{align*}
			In the infinite case, use a similar argument to show
			\begin{align*}
				H^{\alpha} \bigg( \bigcup_{k \geq 0} A_k \bigg) = \infty \implies \sum_{k \geq 0} H^{\alpha} ( A_k) = \infty
			\end{align*}
	\end{enumerate}
\end{proof}

\subsection{Metric Outer Measures}

\begin{definition}[Metric Outer Measure]
	A metric outer measure on a metric space is an outer measure $m : \mathcal{P}(X) \to [0,\infty]$ such that
	\[
		m(A \cup B) = m(A) + m(B) \quad \text{whenever} \quad d(A,B) > 0.
	\]
	Here
	\[
		d(A,B) = \inf_{\substack{a \in A \\ b \in B}} d(a,b).
	\]
\end{definition}

\begin{example}
	$H^{\alpha}$ is a metric outer measure. Indeed
	\[
		H_{r}^{\alpha}(A \cup B) = H_{r}^{\alpha}(A) + H_{r}^{\alpha}(B) \quad \text{where} \quad d(A,B) > 0
	\]
	because we can effectively split any cover of $A \cup B$ into a cover of $A$ and a cover of $B$.
\end{example}

\begin{lemma}
	Borel sets are Caratheodory for metric outer measures.
\end{lemma}

\begin{proof} \hfill
	\begin{enumerate}
		\item[Step 1] Show if $A = \bigcup_{k \geq 1} A_k$, $A_{k+1} \contains A_k$,
			and $d(A_k, A \setminus A_{k+1}) >0$, then 
			\[
				m(A) = \lim_{k \to \infty} m(A_k).
			\]
			\begin{enumerate}
				\item[Case 1:]
					If $\sum_{k \geq 1} m(A_{k+1} \setminus A_k) = \infty$ for $j=1$ or $2$,
					then
					\begin{align*}
						\infty = \sum_{k \geq 1} m(A_{j+2k+1} \setminus A_{j + 2k})
						& = \lim_{n \to \infty} \sum_{k = 1}^{n} m(A_{j+2k+1} \setminus A_{j + 2k}) \\
						& = \lim_{n \to \infty} m \left(\bigcup_{k = 1}^{n} (A_{j+2k+1} \setminus A_{j + 2k})\right) \\
						& \leq \lim_{n \to \infty} m(A_{j+2k+1})
					\end{align*}
					So both $m(A)$ and $\lim_{n \to \infty} m(A_n)$ are $\infty$.

				\item[Case 2:]
					If $\sum_{n=1}^{\infty} m(A_{k+1} \setminus A_{k}) < \infty$, then
					\begin{align*}
					& \lim_{k \to \infty}  m(A_{k+1} \setminus A_{k}) = 0 \\
					& \lim_{k \to \infty}  m(A \setminus A_{k+1}) = 0
					\end{align*}
					Now
					\begin{align*}
						m(A) &\leq m(A_{k}) + m(A_{k+1} \setminus A_{k}) + m(A \setminus A_{k+1}) \\
							 &\leq \limsup_{k \to \infty} m(A_{k}) \\
							 &= \lim_{k \to \infty} m(A_{k}) \\
							 &\leq m(A)
					\end{align*}
			\end{enumerate}
		\item[Step 2] For $A$ closed, we check
			\begin{align*}
			m(B) = m(B \cap A) + m(B \cap A^{c}) \quad \forall \, B \subseteq \mathbb{R}^d
			\end{align*}
			Write $A^{c} = \bigcup_{k \geq 1} D_k$ where $D_{k} = \{x \in \mathbb{R}^d : d(x,A) > 2^{-k} \}$.

			Since $d(A^{c} \setminus D_{k+1} , D_k) > 0$, step 1 implies
			\begin{align*}
				m(B \cap A^{c}) = \lim_{n \to \infty} m(B \cap D_k)
			\end{align*}

			And since $d(\mathbb{R}^d \setminus D_{k+1}, D_k ) > 0$, step 1 implies
			\begin{align*}
				m(B ) = \lim_{n \to \infty} m(B \cap (A \cup D_{k}))
			\end{align*}

			And finally since $d(A, D_{k}) > 0$,
			\begin{align*}
				m(B \cap (A \cup D_{k})) = m(B \cap A) + m(B \cap D_{k}).
			\end{align*}
			Taking $k \to \infty$ gives 
			\[
				m(B) = m(B \cap A) + m(B \cap A^{c}).
			\]
	\end{enumerate}

\end{proof}


\begin{theorem}
	The restriction of $H^{\alpha}$ to Borel sets is a measure.
\end{theorem}


\begin{definition}
	$\alpha$-dimensional Hausdorff measure is restriction of $H^{\alpha}$ to Borel sets.
\end{definition}


\begin{example}
	$H^{0}$ is counting measure.
\end{example}


\begin{example}
	$H^{\alpha} (A) = 0$ when $\alpha > d$
	(insert image) we can cover  $[0,1]^{d}$ by $2^{nd}$ sub-cubes of size $2^{-n}$.

	This shows:
	\begin{align*}
		H^{\alpha} ([0,1]^d) \leq \lim_{n \to \infty} 2^{nd}( \sqrt{d} 2^{-n})^{\alpha} = 0
	\end{align*}
\end{example}

\begin{example}
	$H^{d} = \beta L$ for some $\beta > 0$
\end{example}

\begin{theorem}
	If $m$ Borel measure on $\mathbb{R}^{d}$, $m(x + A) = m(A)$ for $x \in \mathbb{R}^d$ and $A \subseteq \mathbb{R}^d$ Borel, and $m([0,1]^{d}) =1$, then $m = L$
\end{theorem}

\begin{proof}
	Homework.
\end{proof}

\subsection{Hausdorff Dimension}

\begin{definition}[Hausdorff Dimension]
	The Hausdorff dimension of a set $A \subseteq \mathbb{R}^d$ is
	\begin{align*}
		\dim_{H} (A) = \inf\{\alpha \geq 0 : H^{\alpha} (A) = 0 \}
	\end{align*}
\end{definition}

\begin{lemma}
	For any $d$, $A \subseteq \mathbb{R}^d$ there is an $\alpha \in [0,d]$ s.t.
	\begin{align*}
		H^{\beta} (A) &= 0 \quad \text{for } \beta > \alpha \\
		\text{and} \quad H^{\beta} (A) &= \infty \quad \text{for } \beta < \alpha
	\end{align*}
\end{lemma}


\begin{example}
	The $\frac{1}{3}$ cantor set $\mathcal{C}$ has Hausdorff dimension: $\alpha = \frac{\log 2}{\log 3}$
	\begin{proof}
		At step $K$ have cover of $\mathcal{C}$ consisting of $2^{k}$ intervals of length $3^{-k}$ this gives us
		\begin{align*}
		\lim_{k \to \infty} (2^{k}\cdot 3^{-\alpha}) \in (0, \infty) \iff \alpha = \frac{\log 2}{\log 3}
		\end{align*}
	\end{proof}

\end{example}

\begin{example}
	The Koch curve has dimension: $\alpha = \frac{\log 4}{\log 3} \in (1,2)$
	\begin{proof}
		Can cover with $4^{k}$ intervals of length $3^{-k}$
	\end{proof}

\end{example}

\begin{example}
	The Serpinski triangle has dimension: $\alpha = \frac{\log 3}{\log 2} \in (1,2)$
\end{example}

