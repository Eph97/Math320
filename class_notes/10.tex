\section{October 5th, 2022}

\epigraph{``I wonder at the end of my career, all you have to do is install measure theory.''}{Charlie}

\subsection*{Review}

\begin{theorem}[Riesz]
	$\Lambda : C_c (X) \to \mathbb{R}$ positive linear on nice space $X$, then $\Lambda(f) = \int f dm$ for a unique Borel measure on $X$.
\end{theorem}

We have tools for constructing measure. Namely, the Caratheodory theorem for outer measure: Every outer measure is a positive measure on its Caratheodory sets.

Now we have two candidates for Lebesgue measure.

\begin{enumerate}
	\item $\Lambda (f) = \underset{Riemann}{\int f}$
		which should be given by: $\Lambda(f) = \int f \,dL$
	\item The outer Lebesgue measure \[
		L^* (A) = \inf \left\{\sum_K |Q_k| : A \subseteq \bigcup_{K} Q_k \text{ closed rectangles}\right\}.
	\]
  should be $L$ on its Caratheodory sets.
\end{enumerate}

These two are essentially the same.

\begin{lemma}
	$L^*$ is an outer measure on $\mathcal{P}(\R^d)$
\end{lemma}

\begin{proof}
	$L^* (\emptyset)= 0$ because any collection of closed rectangles covers $\emptyset$. Also,
	\[
		A \subseteq B \implies L^*(A) \leq L^*(B).
	\]
  So,
	 \begin{align*}
	L^*(A) & = \inf \left\{\sum_{k} |Q_k| : A \subseteq \bigcup_{k} Q_K \right\} \\
         & \leq \inf\left\{\sum_{k} |Q_k| : B \subseteq \bigcup_{k} Q_K \right\} \\
         & = L^*(B)
	\end{align*}
	Next we want
	\begin{align*}
	L^* \left( \bigcup_{K} A_K \right) \leq \sum_{K} L^* (A_K)
	\end{align*}
	Choose $Q_{k,j}$ so that
	\[
		A_k \subseteq \bigcup_{j} Q_{k,j} \quad \text{and} \quad \sum_{j} | Q_{k,j} | \leq L^* (A_{k}) + \frac{\varepsilon}{2^{k}}.
	\]
	Let $\{\tilde{Q}_{l}\} = \{Q_{k,j}\}$, then
	\[
		\bigcup_{k}A_{k} \leq \bigcup_{l} \tilde{Q}_{l} \quad \text{and} \quad
		\sum_{l} | \tilde{Q}_{l} | \leq \sum_{k} |Q_{k,j} | \leq \sum_{k} L^* (A_k) + \varepsilon.
	\]
\end{proof}

\begin{lemma}
	Borel sets are $L^*$-Caratheodory.
\end{lemma}

\begin{proof}
	Recall $A \subseteq \mathbb{R}^d$ is Caratheodory if
	\begin{align*}
	L^*(B) = L^*(B \cap A ) + L^*(B \cap A^{c}) \quad \text{for } B \subseteq \mathbb{R}^d
	\end{align*}
	Since both Caratheodory sets and Borel sets form $\sigma$-algebras, it is enough to prove the family generated by Borel sets are Caratheodory.
\end{proof}

Borel sets are generated by open sets.

\begin{claim}
	Open sets are unions of closed rectangles (countably many)

	So it is enough to prove that
	\begin{align*}
	L^*(B) \geq L^*(B \cap Q) + L^* (B\cap Q^{c})
	\end{align*}
\end{claim}

Suppose $B \subseteq \bigcup_{k} Q_{k}$. We want to make equally efficient covers of $B \cap Q$ and $B \cap Q^{c}$. Choose $Q_{k,1}, \ldots , Q_{k,3^d}$ so that
\begin{gather*}
	Q_{k,1} = Q_{k} \cap Q \\
	Q_{k,2} \cup \ldots \cup Q_{k,3^d} = Q_{k} \cap Q^{c} \\
	|Q_{k}| = |Q_{k,1}| + \cdots + | Q_{k,3^d} |
\end{gather*}
Let $\{Q_{l}'\} = \{Q_{k,1}\}$ and $\{Q_{l}''\} = \{Q_{k,j} : j = 2, \ldots , 3^d\}$. \\
Now:
 \begin{align*}
	 B & \cap Q \subseteq \bigcup_{l} Q_{l}' \\
	 B & \cap Q^{c} \subseteq \bigcup_{l} Q_{l}''
\end{align*}
so,
\[
	\sum_{l} |Q_{l}'| + \sum_{l} | Q_{l}''| = \sum_{k} |Q_k|. 
\]

\begin{theorem}
	If $L$ is restriction of $L^*$ to Borel sets, then $L$ is a positive measure, and for $f \in C_{c} (\mathbb{R}^d)$
	\[
		\int f \,dL = \underset{Riemann}{\int f}
	\]
\end{theorem}

\begin{proof}
	The lemma tells us $L$ is a positive measure on Borel sets. It is enough to prove
	\[
		\int \1_{\mathbb{Q}} \,dL = \underset{Riemann}{\int} \kern-1em \1_{\mathbb{Q}}
	\]
	Enough to prove
	\[
		\int \1_{\mathbb{Q}} \,dL = L(Q) = |Q|
	\]
	This is true by HW1.
\end{proof}

\subsection{\texorpdfstring{Two constructions of $L$}{Two constructions of L}}
\epigraph{``I used to be a logician in a former life. If you ask me about logic, I'll take up the rest of class talking about it.''}{Charlie}

Two constructions of $L$ (via Riesz or outer measure) differ in that $L^*$ may have Caratheodory sets that are not Borel.

\begin{theorem}
	Let $S \subseteq \mathcal{P}(\mathbb{R}^d)$ denote the $L^*$-Caratheodory sets. For $E \subseteq
	\mathbb{R}^d$, $E \in S$ if and only if there are Borel $B_1$ and $B_2$ such that 
	$B_1 \subseteq E \subseteq B_2$ and $L^*(B_2 \setminus B_1) = 0$. 
	Moreover, in this case  $L^*(E) = L^*(B_1) = L^*(B_2)$
\end{theorem}
\begin{proof}
	Exercise or see Rudin.
\end{proof}
For now, ignore Lebesgue measurable sets that are not Borel.

\begin{example}[Vitali set]
	Two Ingredients:
	\begin{enumerate}
		\item Translation invariance of $L$ on $\mathbb{R}$.
		\item Axiom of choice.
	\end{enumerate}

	For $x, y \in \R | \Z$, say $x \sim y \iff x - y \in  \mathbb{Q} | \Z$.
	Let $A \subseteq \mathbb{R}$ be maximal, subject to $x - y \notin \mathbb{Q} | \Z$ for distinct $x,y \in A$.
	By maximality, $\mathbb{R} = \bigcup_{q \in \mathbb{Q} | Z} (q + A)$
	and $(q + A) \cap (\tilde{q} + A) = \emptyset$ for $q \neq \tilde{q}$.
	By translational invariance, $L^*(q + A) + L^*(A)$. By subadditivity, 
	\[
		1 = L^*(\R | \Z) \leq \sum_{a \in \Q | \Z} L^*(q + A) = \sum_{q \in \Q | \Z} L^*(A).
	\]
	If $\geq$, then $1 = \infty \cdot L^*(A)$.\footnote{In some cases, it is necessary to worry about completion of Lebesgue measures, but will bypass this for now and only worry about Borel measures.} 
\end{example}
pic (come back to)

\begin{definition}[Hausdorff Measure]
	Choose a smooth closed curve $\gamma$ in $\mathbb{R}^2$. Define $\Lambda : C_c(\mathbb{R}^2) \to \mathbb{R}$ by $\Lambda (f) = \int_{\gamma} f \, ds$, where $\gamma :[0,1] \to \mathbb{R}^2$.
	\[
		\Lambda (f) = \int_{0}^{1} f(\gamma(t)) |\gamma'(f)| \, dt.
	\]
	Use Riesz to write $\Lambda (f) = \int f \, dm$ where $m$ is a Borel measure on $\mathbb{R}^2$.
	$m$ is 1-dimensional Lebesgue measure ``pushed onto'' $\gamma$.
\end{definition}

\begin{example}[Koch Snowflake]
	\begin{enumerate}
		\item start with O---O
		\item replace every line with bump (add images)
		\item Go to step 2.
	\end{enumerate}
	At each step, length is multiplied by $\tau > 1$
\end{example}
Is there a natural length measure?

\begin{definition}[Diameter]
	A set $X \subseteq \mathbb{R}^d$ has diameter 
	\[
		\diam(X) = \sup_{x, y < k} |x - y|
	\]
	where $x, y \in X$
\end{definition}

\begin{definition}[Outer Hausdorff Measure]
	For $\alpha \geq 0$, the outer $\alpha$-dim Hausdorff measure of $A \subseteq \mathbb{R}^d$ is
	\[
		H_{\alpha}^* = \lim_{r \to \infty} \inf\left\{ \sum_{k} \left(\diam(X_{k}) \right)^{\alpha} : A \subseteq \bigcup_{k} X_k, \,
		\sup_{k} \diam(X_k) \leq r \right\}.
	\]
	$H_{\alpha}^*$ is comparable to $L^*$. \\
	$H_{1}^*$ measures length.
\end{definition}

