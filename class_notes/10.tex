\section{ 2022-10-05 }

\epigraph{``''}{Charlie}

review

\begin{theorem}
	Riesz $ \Lambda : C_c (X) \to \mathbb{R}$ positive linear on nice space $X$ then $\Lambda(f) = \int f dm$ for a unique borel measure on $X$.
\end{theorem}

We have tools for constructing measure. Namely, the Caratheodory theorem for outer measure: Every outer measure is a positive measure on its Caratheodory sets.

Now we have two candidates for Lebesgue measure.

\begin{enumerate}
	\item $\Lambda (f) = \underset{Riemann}{\int f}$
		Which should be given by: $\Lambda(f) = \int f dL$
	\item The outer Lebesgue measure  $l^* (A) = \inf\{\sum_K |Q_k| : A \subseteq \bigcup_{K} Q_k \}$ Should be $L$ on its Caratheodory sets.
\end{enumerate}

These are essentially the same.

\begin{lemma}
	$L^*$ is an outer measure on $\mathcal{P}(R^d)$
\end{lemma}

\begin{proof}
	$L^* (\emptyset)= 0$ because any collection of closed rectangles covers  $\emptyset$

	also
	 \begin{align*}
	A \subseteq B \implies L^*(A) \leq L^*(B) \\
	L^*(A) = \inf\{\sum_{k} |Q_k| : A \subseteq \bigcup_{k} Q_K \} \leq \inf\{\sum_{k} |Q_k| : B \subseteq \bigcup_{k} Q_K \} = L^*(B)
	\end{align*} 
	Next we want
	\begin{align*}
	L^*(\bigcup_{K} A_K) \leq \sum_{K} L^* (A_K)
	\end{align*} 
	Choose $Q_{K,J}$ so 

\end{proof}

\begin{lemma}
	Borel sets are $L^*$-Caratheodory.
\end{lemma}

\begin{proof}
	Recall $A \subseteq \mathbb{R}^d$ is Caratheodory if
	\begin{align*}
	L^*(B) = L^*(B \cap A ) + L^*(B \cap A^{c}) \quad \text{for} \quad B \subseteq \mathbb{R}^d
	\end{align*} 

	Since both caratheodory sets and borel sets fom $\sigma$-algebra, it is enough to prove family generated by Borel sets are caratheodory.


\end{proof}

come back to

	Two constructions of $L$ (via Riesz or outer measure). They differ in that $L^*$ may have caratheodory sets that are not Borel.

\begin{theorem}
	Let $S \subseteq \mathbb{P}(\mathbb{R}^d)$ denote the $L^*$-Caratheodory sets. For $E \subseteq 
	\mathbb{R}^d$, $E \in S$ $\iff$ there are borel $B_1$ and $B_2$ s.t. $B_1 \subseteq E \subseteq B_2$ and
	$L^*(B_2 \setminus B_1) = 0$.

	Moreover, In this case  $L^*(E) = L^*(B_1) = L^*(B_2)$
\end{theorem}
\begin{proof}
	Exerise 1 or see Rudin.
\end{proof}

For now, ignore Lebesgue measurable sets that are not Borel.

\begin{example}[Vitali set]
	Two Ingredients
	\begin{enumerate}
		\item Translation invarience of $L$ on $\mathbb{R}$.
		\item Axiom of choice.
	\end{enumerate}

	Say $x \sim y \iff y - y \in  \mathbb{Q}$

	Let $A \subseteq \mathbb{R}$ be maximal subject to $x - y \notin \mathbb{Q}$ for distinct $x,y \in A$

	By maximality,  $\mathbb{R} = \bigcup_{q \in \mathbb{Q}} q + A$.

	and $(q + A) \cap (\tilde{q} + A) = q$.\,  For $q \neq \tilde{q}$
\end{example}
pic (come back to)

\begin{definition}[Hausdorff Measure]
	Choose a smooth closed curve $\delta$ in $\mathbb{R}^2$

	Define $\Lambda : C_c(\mathbb{R}^2) \to \mathbb{R}$ by $\Lambda (f) = \int_{\delta} f ds$

	were 
	\begin{align*}
		\delta :[0,1] \to \mathbb{R}^2 \qquad \Lambda (f) = \int_{0}^{1} f(\delta(t)) |\delta'(f) |dt
	\end{align*} 
	Use Riesz to write $\Lambda (f) = \int f dm$  where $m$ is a Borel measure on $\mathbb{R}^2$
	$m$ is 1-dimensional lebesgue measure "pushed onto $\delta$
\end{definition}

\begin{example}[Kach Snowflake]
	\begin{enumerate}
		\item sstart with O---O
		\item replace every line with bump (add images)
		\item Go to step 2.
	\end{enumerate}
	At each step, length is multiplied by $\tau > 1$
\end{example}
Is there a natural length measure?

\begin{definition}
	A set $X \subseteq \mathbb{R}^d$ has diameter $diam(X) = \sup|x - y |$ where $x, y \in X$
\end{definition}

\begin{definition}
	For $\alpha \geq 0$, the outer $\alpha$-dim Hausdorff measure of $A \subseteq \mathbb{R}^d$ is
	\begin{align*}
		H_{\alpha}^* = \lim_{r \to \infty} \inf\{ \sum_{k} \left(diam(X_{k}) \right)^{\alpha} : A \subseteq \cup_{k} X_k, \,
		\sup_{k} diam(X_k) \leq r \}
	\end{align*} 

	$H_{\alpha}^*$ is comparable to $L^*$ 

	$H_{1}^*$ measure length.
\end{definition}

