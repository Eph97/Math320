\section{2022-10-24}

\epigraph{``Today everything is named Hausdorff''}

\subsection{Hausdorff measure on Fractals}

\begin{remark}[hypothesis]
	if $r_1, \ldots, r_k \in [0,1)$ and $\phi_{1}, \ldots, \phi_{k} : \mathbb{R}^d \to \mathbb{R}^d$, then
	\begin{align*}
	|\phi_{j} (x) - \phi_{j} (y) | \leq r_j |x - y|
	\end{align*} 
\end{remark}

\begin{theorem}
	There is a unique compact $K \subseteq \mathbb{R}^d $ so $ K = \bigcup_{j=1}^{J} \phi_{j}(k)$
\end{theorem}

\begin{proof}
	Consider Hausdorff metric

	\begin{align*}
		d_H (K,G) = \max\{\max_{x\in K} \min_{y \in G} |x - y|, \max_{y\in G} \min_{x \in K} |y - k|\}
	\end{align*} 

	Define $\phi(K) =  \bigcup_{j=1}^{J} \phi_{j}(k)$

	\begin{enumerate}
		\item[claim 1] $d_H(\phi(K), \phi(G)) \leq (\max_{j} r_j ) d_{H}(K,G)$

			\begin{align*}
				d_H (\phi(K) \phi(G)) &\leq \max_{j} d_{H} (\phi_{j}(K), \phi_{j} (G)) \\
									  &\leq \max_{j} r_{j} d_{H}(K,G) \\
									  &\leq (\max_{j} r_{j}) d_{H}(K,G)
			\end{align*} 

		\item[Claim 2] the space of compact $K \subseteq \mathbb{R}^d$ with $d_H$ is complete metric space.

			Suppose $G_{K}$ compact and, $d_{H} (G_{K}, G_{K+1}) \leq 2^{-k}$

		$G = \bigcap_{n \geq 0} \overline{\bigcup_{k \geq n} G_{K}}$
		is compact since it is the limit of descending chain of compact $d_{H}(G, G_{K}) \leq 2^{-k}$

	\end{enumerate}
	Now back to the hypothesis. If we let 
	$r_1, \ldots, r_j \in (0,1)$ then
	\begin{align*}
		|\phi_j (x) - \phi_{j} (y) | = r_{j} |x - y| \qquad \text{Note equality!}
	\end{align*} 

	Note we also need an open set condition

	\begin{remark}
		There is a non-empty bounded open $U \subseteq \mathbb{R}^d$ s.t.
		\begin{align*}
			\phi_{1}(U), \ldots, \phi_{j}(U) \subseteq U \quad \text{and} \quad \phi_{1} (U), \ldots, \phi_{j} (U) \quad \text{disjoint}
		\end{align*} 

	\end{remark}
	
	\begin{example}
		$d = 2$, $j=3$.
		 \begin{align*}
		\phi_{j} (x,y) = \left(\frac{x}{2} + \cos( \frac{2\pi}{3}j), \frac{y}{2} + \sin( \frac{2\pi}{3}j)\right)
		\end{align*} 
	\end{example}
	
\end{proof}



\subsection{Moran's Theorem}

\begin{theorem}[Moran's 1946]

	For $0 < H^{\alpha} (K) < \infty$ where $\alpha \in [0,d]$, then the dimension is the unique $\alpha$ s.t.
	\begin{align*}
	r_{1}^{\alpha} + \ldots + r_{j}^{\alpha} = 1
	\end{align*} 
	
\end{theorem}


\subsubsection{scalling analysis}

$rA = \{rx : x \in A \}$ where  $r > 0$

 \begin{align*}
	 H^{\alpha} (rA) &= r^{\alpha} H^{\alpha} (A) \\
				K	&= \bigcup_{j=1}^{J} \phi_{j} (K)
\end{align*} 

If unione disjoint, then 
\begin{align*}
	H^{\alpha}(K) = \sum_{j=1}^{J} r_{j}^{\alpha} H^{\alpha} (K)
\end{align*} 
 The open set condition guarentees

 \begin{align*}
 	H^{\alpha} (\phi_{i} (k) \cap \phi_{j} (k) ) = 0 \quad \text{for $i \neq j$  (as we will see)}
 \end{align*} 

 \begin{proof}[Proof of Moran]
	 The main idea is to construct a finitely subadditive
	 \begin{align*}
		 m^{*} : \mathcal{P}(\R^{d}) \to [0, \infty]
	 \end{align*}
	 that lies on $K$ and then rescale so $\diam(V) = 1$

	 [insert pic]

	 \begin{align*}
		 \mathcal{U}_0 &= \{U \} \\
		 \mathcal{U}_{n+1} &= \{\phi_{j} (V) : V \in \mathcal{U}_{n}, \, j \in \{1, \ldots , J \}\} \\
		 \mathcal{U} &= \bigcup_{n \geq 0} \mathcal{U}_{n}
	 \end{align*} 
$\mathcal{U}$ is true of open sets ordered by inclusion, $\mathcal{U}_{n}$ ets at level $n$ are dosjiont (by induction).

\begin{enumerate}
	\item
		\begin{align*}
			\sum_{v \in \mathcal{U}_n} \diam(V)^{\alpha} = 1
		\end{align*} By induction 
		\begin{align*}
			\sum_{v \in \mathcal{U}_0} \diam(V)^{\alpha} &= \diam(V)^{\alpha} = 1^{\alpha} = 1 \\
			\sum_{v \in \mathcal{U}_{n+1}} \diam(V)^{\alpha} &=
			\sum_{j=1}^{J} \sum_{v \in \mathcal{U}_n} \diam(\phi(V))^{\alpha} = 
			\sum_{j=1}^{J} r_{j}^{\alpha} \sum_{v \in \mathcal{U}_n} \diam(V)^{\alpha} \\
			= \sum_{j=1}^{J} r_{j}^{\alpha} = 1
		\end{align*}

	\item $\max_{V \in \mathcal{U}_{n+1}} \diam (V) = (\max_{j} r_{j})^{n}$ goes to $0$ as $n \to 0$

	\item $K \subseteq \bigcup_{V \in \mathcal{U}_{n}} \bar{V}$ because $d_{H} ( \phi^{n}(\{x\}, K) \to 0$ and $\phi^{m}(\{x\}) \subseteq \bigcup_{V \in \mathcal{U}_{n}} V	$ for $m \geq n$

	\item $H^{\alpha} (K) \leq 1$ by (1), (2), and (3)
\end{enumerate}

For $A \subseteq \R^{d}$ define
\begin{align*}
	\mu_{n}(A) &= \sum_{\substack{V \in \mathcal{U}_n \\ \bar V \cap A \neq \emptyset}} \diam (V)^{\alpha}, \\
	\mu (A) &= \lim_{n \to \infty} \mu_{n}(A)
\end{align*}

\begin{remark}
	$\mu_n (A)$, is decreasing in $n$ because for $V \in \mathcal{U}_n$,
	\begin{align*}
		\sum_{\substack{w \in \mathcal{U}_{n+1} \\ w \subseteq V}} \diam (w)^{\alpha} = \diam (V)^{\alpha}
	\end{align*} 
\end{remark}

Check $\mu$ is a measure.

\begin{align*}
	\mu(\R^{\alpha}) &= 1 \text{   check??} \\
	A \subseteq B &\implies \mu(A) \leq \mu (B) \\
	\mu_{n} (A \cup B ) &\leq \mu_{n} (A) + \mu_{n} (B) \\
	\mu (A \cup B ) &\leq \mu (A) + \mu (B) \quad \text{by passing to limits} \\
\end{align*} 
finally,
$u(K) = 1$ : if $V \in \mathcal{U}_{n}$, then
\begin{align*}
	\bar{V} &= \phi_{j1}( \phi_{j2} ( \ldots \phi_{jn} (\bar{U}) \ldots )) \\
			&\contains \phi_{j1}( \phi_{j2} ( \ldots \phi_{jn} (K) \ldots )) \\
\end{align*} So $\mu_n (K) = 1$


Now $\mu (B_{\delta} (x) \leq C_{d,r_{1},\ldots, r_{d}, v} \delta^{\alpha}$ 

Let $V_{1}, \ldots, V_{2} \in \mathcal{U}$ be maximal elements of tree so $\bar{V} \cap B_{j} (x) \neq \emptyset$ \\
and $\diam(v) \leq \delta$ 


\begin{remark}
	Note $\diam(V_{l}) \geq (\min_{j} r_{j} ) \delta$ by maximality.

	and $B_{c}(x) \subseteq U \subseteq B_{c} (x)$ for some constants $0 < c < 1 < C < \infty$ (check)
\end{remark}

Because the maps $\phi_{j}$ are isometry-dialations
\begin{align*}
	B_{c \diam(v)} (X_{v}) \leq V \leq B_{C \diam (v)} (X_{v}) \quad \text{for all } v \in \mathcal{U}.
\end{align*} [insert pic]

\begin{align*}
	B_{\delta} + C \delta^{(x)} \contains \bigcup_{l =1}^{L} v_{l} \contains L^{m}\text{-any disjoint } B_{c} (\min_{j} r_{j}) \delta \text{ balls}
\end{align*} 
So $(1+c)^{d} \delta^{d} \geq L (c \min_{j} r_{j})^{d} \delta^{d}$ or $L \leq C_{U, r_{1}, \ldots, r_{J}, d}$

For $n$ large enough that $V_{1}, \ldots, V_{L} \in \mathcal{U}_{n} \cup \ldots \cup \mathcal{U}_r$ we have
\begin{align*}
	\mathcal{U}_{n} (B_{\delta}(x)) &\leq \mathcal{U}_{n} \left( \bigcup_{l=1}^{L} V_{l} \right)
									\leq \sum_{l=1}^{L} u_{n} (v_{l}) \leq \sum_{l=1}^{L} \diam (V_{l}) \\
									&\leq L \delta^{\alpha} \leq C_{V, r_{1}, \ldots, r_{J} , d	} \delta^{\alpha}
\end{align*} 

$\mu (K) = 1$, and $\mu (B_{\delta} (X)) \leq C \delta^{\alpha}$ 
Now suppose $K \subseteq \bigcup_{m \geq 1} B_{m}$. Since $K$ compact, we may assume $K = \bigcup_{m \geq 1} B_{m}$

thus
\begin{align*}
	\sum_{m=1}^{M} \diam(B_{m})^{\alpha} \geq \sum_{m=1}^{M} c \mu(B_{m}) \geq c \mu (\bigcup_{m=1}^{M} B_{m}) \geq c \mu (K) = c > 0
\end{align*} 
That is $H^{\alpha} (K) > 0$

\end{proof}


\begin{remark}
	this is basically using Calder\'{o}n-Zygmond. \\
	This was also the hardest proof so far.
\end{remark}

\begin{example}
	The restriction of $u$ to Borel sets $\mathcal{B}$ is given by
	\begin{align*}
		\mu ( \mathcal{B} ) = \frac{H^{\alpha} (\mathcal{B} \cap K)}{H^{\alpha} (K)}
	\end{align*} 
\end{example}

\begin{exercise}
	Suppose $C \subseteq  [0,1]$ is $\frac{1}{3}$ cantor set where $C = \bigcup_{n \geq 0} C_{n}$

	If $f \in C([0,1])$, then
	\begin{align*}
		\frac{\int f \1_{C_n}}{\int \1_{C_n}}  \xrightarrow[\enskip n \to \infty\enskip]{} 
		\frac{\int f \1_{c} d H^{\frac{\log(2)}{\log(3)}}}{\int \1_{c} d H^{\frac{\log(2)}{\log(3)}}}
	= \int f d\mu
	\end{align*} 
\end{exercise}


