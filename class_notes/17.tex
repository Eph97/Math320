\section{2022-11-09}

review

\begin{definition}[Absolutely continuous]
	$m_1, m_2 : S \to [0, \infty)$ measures,
	\begin{align*}
		m_1 << m_2
	\end{align*} if and only if $m_{2} (A) = 0 $ implies $m_{1} (A) = 0$ for every $A \in S$.
\end{definition}


\begin{theorem}
	For measure $m_1, m_2 : S \to [0, \infty)$ there are unique measures  $m_a, m_{s} :S \to [0, \infty]$ so
	\begin{align*}
		m_1 = m_a + m_s, \quad m_a << m_2, \quad m_s \perp m_2
	\end{align*},
\end{theorem}

\begin{proof}
	Let $E = \bigcup_{n \geq 0} E_{n}$ where $E_{n} \in S$ satisfies 
	 $m_{2} (E_{n}) = 0$ and
	 \begin{align*}
	 	\lim_{n \to \infty} (E_{n}) = 0
	 \end{align*} and
	 \begin{align*}
		 \lim_{n \to \infty} m_{1} (E_{n}) &= \sup_{m_{2} (A)} m_1 (A) \\
		 m_s (A) &= m_1 (A \cap E) \\
		 m_a (A) &= m_{1} (A \cap E^c)
	 \end{align*} 
	 $m_1 = m_s + m_a$ is immediate, and
	 $m_2 (E) = 0$ and $m_2 (E^c) = 0$ implies $m_2 \perp m_s$.

	 If $A \subseteq E^c$ and $m_2 (A) = 0$, then 
	 \begin{align*}
		 m_1 (A) + m_1 (E_n) &= m_1 (A \cup E_n) \\
							 &\leq \sup_{m_2 (B) = 0} m_1 (B) \\
							 &= \lim_{n \to \infty} m_1 (E_n)
	 \end{align*} 
	 and 
	 \begin{align*}
	 	m_1 (A) + \sup_{m_2 (B) = 0} m_1 (B) = \sup_{m_2 (B) = 0} m_1 (B)
	 \end{align*} 

	 Since $m_1 (X) < \infty$, conclude $m_1 (A) = 0$.
	 We have proved  $A \subseteq  E^c$ and $m_2 (A) = 0 \implies m_1 (A) = 0$ .
	 This gives $m_2 (A) = 0 \implies m_a (A) = 0$.
\end{proof}

\begin{theorem}[Radon-Nikodym]
	\begin{lemma}: Suppose $m_1, m_2 : S \to [0, \infty)$ and $m_1 << m_2$.
		For $\varepsilon > 0$, there are $A \in S$ and $\delta > 0$ so 
		\begin{align*}
					m_1 (A) &\geq (1 - \varepsilon) m_1 (X) \\
			\text{and} \quad m_1 (B) &\geq \delta m_2 (B) \quad \text{for} \quad B \subseteq A
		\end{align*} 
	\end{lemma}
\end{theorem}


\begin{proof}
	Assume $m_1 (X) > 0$.

	Choose $\delta > 0$ so $m_1 (X) - \delta m_2 (X) \geq \varepsilon m_1 (X)$

	Define: 
	\begin{align*}
		S (B) = m_1 (B) - \delta m_2 (B)
	\end{align*} 
	Let $A_0 = X$ and choose  $A_{n+1} \subseteq A_{n}$. so
	\begin{align*}
		S( A_{n + 1} ) &\geq S (A_{n}) \\
		\text{and} \quad S(A_{n+1} ) &\geq \frac{-1}{n+1} + \sup_{A \subseteq A_{n}	} S (A)
	\end{align*} 

	Now let $A = \bigcap_{n \geq 0} A_n$

	Compute
	\begin{align*}
		S(A) &= S(X) - \sum_{n \geq 0} S( A_n \setminus A_{n+1} ) \\
			 &= S(X) - \sum_{n \geq 0} [S(A_{n}) - S(A_{n+1} ) ] \\
			 &\geq S(X)
	\end{align*} 

	For $B \subseteq A$, compute
	\begin{align*}
		S(A_{n + 1} &= S(B) + S(A_{n+1} \cap B^c ) \\
					&\leq S(B) + \sup_{A \subseteq A_{n}} S(A) \\
					&\leq S(B) + \frac{1}{n+1} + S(A_{n+1} )
	\end{align*} Sending $n \to \infty$ gives $S(B) \geq 0$. Unroll definition of $S$.
	\begin{align*}
		S(A) &\geq S(X) \implies \\
		m_1 (A) - \delta m_2 (A) &\geq m_1 (X) - \delta m_2 (X) \\
								 &\geq \varepsilon m_1 (X) \\
	\end{align*} implies $m_1 (A) \geq ( \varepsilon + \delta) m_1 (X)$.

	Now for $B \subseteq A$. $S(B) \geq 0$ implies $m_1 (B) \geq \delta m_2 (B)$.
\end{proof}


\begin{theorem}[Radon-Nikodym]
	For $m_1, m_2 : S \to [0, \infty) $ measures with \\
	$m_1 << m_2$, there is  $f \in L^1 (m_2)$ so
	$m_1 (A) = \int \1_A f dm_2$ for $A \in S$.
\end{theorem}

\begin{proof}
	Let $G$ denote set of $g \in L^1 (m_2)$ this satisfies
	\begin{align*}
		m_1 (A) \geq \int \1_A g \, dm_2 \quad \text{for } A \in S
	\end{align*} 
	Note $0 \in G$, also
	$g_1 , g_2 \in G \implies \max \{g_1, g_2\} \in G$.

	Let $B = \{g_1 \geq g_2\}$
	\begin{align*}
		m_1 (A) &= m_1 (Aa \cap B) + m_2 (A \cap B^c) \\
				&\geq \int \1_{A \cap B} \, g_1 \, d m_2 + \int \1_{A \cap B^c} \, g_2 \, dm_2 \\
				&= \int \1_{A \cap B} \, g \, d m_2 + \int \1_{A \cap B^c} \, g \, dm_2 \\
				&= \int \1_A g dm_2
	\end{align*} 

	Choose $g_n \in G$ with 
	\begin{align*}
		g_{n+1} &\geq g_n \\
		\lim_{n \to \infty} \int g_{m} d m_2 &= \sup_{g \in G} \int g dm_2 \\
											 &\leq m_1 (X) < \infty
	\end{align*}

	Let $f = \lim_{n\to \infty} g_n$. By monotone convergence,
	\begin{align*}
		m_1 (A) \geq \int \1_A g_n dm_2 \to \int \1_{A} f dm_2
	\end{align*} In particular, $f \in G$.

	Want $m_1 (A) \leq \int \1_{A} f dm_2$.

	Consider the "residual" measure, $m_3 (A) = m_1 (A) - \int \1_{A} f dm_2$
	
	Observe: $m_3$ is a measure and $m_3 << m_1$. 
	By lemma, choose $A \in S$ and $\delta > 0$ so
	\begin{align*}
							m_3 (E) &\geq \frac{1}{2} m_33 (X) \\
		\text{and} \quad 	m_3 (B) &\geq \delta m_2 (B) \quad \text{for} B \subseteq A. 
	\end{align*} (Should be E?)
	Rewrite as 
	\begin{align*}
		m_3 (B) &\geq \delta m_2 ( B \cap E) \\
				&= \int \1_{B} \delta \1_E d m_2 \quad \text{for} B \in S. 
	\end{align*} 

	using def of $m_3$, 
	\begin{align*}
		m_1 (A) \geq \int \1_A (f + \delta \1_E ) d m_2
	\end{align*} 
	for all $A \in S$, by maximality of $f,$ 
	\begin{align*}
		\int \delta \1_E dm_2 = 0, \\
		m_2 (E) = 0 \implies m_3 (E) = 0 \implies m_3 (X) = 0
	\end{align*} 
\end{proof}



\begin{corollary}[To Radon-Nikodym]
	If $m_1, m_2 : S \to [0, \infty)$	 measures, then there are unique $m_s :S \to [0, \infty)$ measure and 
	$f \in L^{1} (m_2)$ so $m_s \perp m_2$ and $m_1 (A) = m_s (A) + \int \1_A f d m_2$
\end{corollary}

\begin{remark}[Notation] \\
	$d m_1 = f dm_2$ $\iff $ $\int \1_A d m_1 = \int \1_A d m_2$ for $A \in S$.
\end{remark}



\begin{remark}
	Both proofs have the same ingredients: approximate from below and the monotone connvergence.

	Riesz proof makes series for $\frac{d m_1}{d m_2}$
\end{remark}

