\section{November 9th, 2022}

\subsection*{Last Time}

\begin{definition}[Absolutely Continuous]
	If $m_1, m_2 : S \to [0, \infty)$ are measures, $m_1 \ll m_2$
	if and only if $m_{2} (A) = 0 $ implies $m_{1} (A) = 0$ for every $A \in S$.
\end{definition}

\begin{definition}[Mutually Singular]
	$m_1 \perp m_2$ if $m_1(A) = 0$ and $m_2(A^c) = 0$ for some $A \in S$.
\end{definition}


\begin{theorem}
	For measures $m_1, m_2 : S \to [0, \infty)$, there are unique measures  $m_a, m_{s} : S \to [0, \infty]$ so
	\begin{align*}
		m_1 = m_a + m_s, \quad m_a \ll m_2, \quad m_s \perp m_2
	\end{align*}
\end{theorem}

\begin{proof}
	Let $E = \bigcup_{n \geq 0} E_{n}$ where $E_{n} \in S$ satisfies
	 $m_{2} (E_{n}) = 0$ and
	 \begin{align*}
		 \lim_{n \to \infty} m_{1} (E_{n}) &= \sup_{m_{2} (A)} m_1 (A)
	 \end{align*}
	Let $m_s (A) = m_1 (A \cap E)$ and $m_a (A) = m_{1} (A \cap E^c)$.
	$m_1 = m_s + m_a$ is immediate, and
	$m_2 (E) = 0$ and $m_2 (E^c) = 0$ implies $m_2 \perp m_s$.

	 If $A \subseteq E^c$ and $m_2 (A) = 0$, then
	 \begin{align*}
		 m_1 (A) + m_1 (E_n) &= m_1 (A \cup E_n) \\
							 &\leq \sup_{m_2 (B) = 0} m_1 (B) \\
							 &= \lim_{n \to \infty} m_1 (E_n)
	 \end{align*}
	 and
	 \begin{align*}
	 	m_1 (A) + \sup_{m_2 (B) = 0} m_1 (B) = \sup_{m_2 (B) = 0} m_1 (B)
	 \end{align*}

	 Since $m_1 (X) < \infty$, conclude $m_1 (A) = 0$.
	 We have proved  $A \subseteq  E^c$ and $m_2 (A) = 0 \implies m_1 (A) = 0$.
	 This gives $m_2 (A) = 0 \implies m_a (A) = 0$.
\end{proof}

\subsection{Radon-Nikodym}

\begin{lemma}: Suppose $m_1, m_2 : S \to [0, \infty)$ and $m_1 \ll m_2$.
For $\varepsilon > 0$, there are $A \in S$ and $\delta > 0$ so
\begin{align*}
			m_1 (A) &\geq (1 - \varepsilon) m_1 (X) \\
	\text{and} \quad m_1 (B) &\geq \delta m_2 (B) \quad \text{for} \quad B \subseteq A
\end{align*}
\end{lemma}


\begin{proof}
	Assume $m_1 (X) > 0$. Choose $\delta > 0$ so $m_1 (X) - \delta m_2 (X) \geq \varepsilon m_1 (X)$.

	\noindent Define:
	\begin{align*}
		S (B) = m_1 (B) - \delta m_2 (B)
	\end{align*}
	Let $A_0 = X$ and choose  $A_{n+1} \subseteq A_{n}$. so
	\begin{align*}
		S( A_{n + 1} ) &\geq S (A_{n}) \\
		\text{and} \quad S(A_{n+1} ) &\geq \frac{-1}{n+1} + \sup_{A \subseteq A_{n}	} S (A)
	\end{align*}

	Now let $A = \bigcap_{n \geq 0} A_n$

	Compute
	\begin{align*}
		S(A) &= S(X) - \sum_{n \geq 0} S( A_n \setminus A_{n+1} ) \\
			 &= S(X) - \sum_{n \geq 0} [S(A_{n}) - S(A_{n+1} ) ] \\
			 &\geq S(X)
	\end{align*}

	For $B \subseteq A$, compute
	\begin{align*}
		S(A_{n + 1} &= S(B) + S(A_{n+1} \cap B^c ) \\
					&\leq S(B) + \sup_{A \subseteq A_{n}} S(A) \\
					&\leq S(B) + \frac{1}{n+1} + S(A_{n+1} )
	\end{align*} Sending $n \to \infty$ gives $S(B) \geq 0$. Unroll definition of $S$.
	\begin{align*}
		S(A) &\geq S(X) \implies \\
		m_1 (A) - \delta m_2 (A) &\geq m_1 (X) - \delta m_2 (X) \\
								 &\geq \varepsilon m_1 (X) \\
	\end{align*} implies $m_1 (A) \geq ( \varepsilon + \delta) m_1 (X)$.

	Now for $B \subseteq A$. $S(B) \geq 0$ implies $m_1 (B) \geq \delta m_2 (B)$.
\end{proof}


\begin{theorem}[Radon-Nikodym]
	For $m_1, m_2 : S \to [0, \infty) $ measures with \\
	$m_1 \ll m_2$, there is  $f \in L^1 (m_2)$ so
	$m_1 (A) = \int \1_A f dm_2$ for $A \in S$.
\end{theorem}

\begin{proof}
	Let $G$ denote set of $g \in L^1 (m_2)$ this satisfies
	\begin{align*}
		m_1 (A) \geq \int \1_A g \, dm_2 \quad \text{for } A \in S
	\end{align*}
	Note $0 \in G$, also
	$g_1 , g_2 \in G \implies \max \{g_1, g_2\} \in G$.

	Let $B = \{g_1 \geq g_2\}$
	\begin{align*}
		m_1 (A) &= m_1 (Aa \cap B) + m_2 (A \cap B^c) \\
				&\geq \int \1_{A \cap B} \, g_1 \, d m_2 + \int \1_{A \cap B^c} \, g_2 \, dm_2 \\
				&= \int \1_{A \cap B} \, g \, d m_2 + \int \1_{A \cap B^c} \, g \, dm_2 \\
				&= \int \1_A g dm_2
	\end{align*}

	Choose $g_n \in G$ with
	\begin{align*}
		g_{n+1} &\geq g_n \\
		\lim_{n \to \infty} \int g_{m} d m_2 &= \sup_{g \in G} \int g dm_2 \\
											 &\leq m_1 (X) < \infty
	\end{align*}

	Let $f = \lim_{n\to \infty} g_n$. By monotone convergence,
	\begin{align*}
		m_1 (A) \geq \int \1_A g_n dm_2 \to \int \1_{A} f dm_2
	\end{align*} In particular, $f \in G$.

	Want $m_1 (A) \leq \int \1_{A} f dm_2$.

	Consider the "residual" measure, $m_3 (A) = m_1 (A) - \int \1_{A} f dm_2$

	Observe: $m_3$ is a measure and $m_3 \ll m_1$.
	By lemma, choose $A \in S$ and $\delta > 0$ so
	\begin{align*}
							m_3 (E) &\geq \frac{1}{2} m_33 (X) \\
		\text{and} \quad 	m_3 (B) &\geq \delta m_2 (B) \quad \text{for} B \subseteq A.
	\end{align*} (Should be E?)
	Rewrite as
	\begin{align*}
		m_3 (B) &\geq \delta m_2 ( B \cap E) \\
				&= \int \1_{B} \delta \1_E d m_2 \quad \text{for} B \in S.
	\end{align*}

	using def of $m_3$,
	\begin{align*}
		m_1 (A) \geq \int \1_A (f + \delta \1_E ) d m_2
	\end{align*}
	for all $A \in S$, by maximality of $f,$
	\begin{align*}
		\int \delta \1_E dm_2 = 0, \\
		m_2 (E) = 0 \implies m_3 (E) = 0 \implies m_3 (X) = 0
	\end{align*}
\end{proof}



\begin{corollary}[To Radon-Nikodym]
	If $m_1, m_2 : S \to [0, \infty)$	 measures, then there are unique $m_s :S \to [0, \infty)$ measure and
	$f \in L^{1} (m_2)$ so $m_s \perp m_2$ and $m_1 (A) = m_s (A) + \int \1_A f d m_2$
\end{corollary}

\begin{remark*}[Notation] \\
	$d m_1 = f dm_2$ $\iff $ $\int \1_A d m_1 = \int \1_A d m_2$ for $A \in S$.
\end{remark*}

\begin{theorem}[Radon-Nikodym] we restate the theorem: \\
	$m_1 \ll m_2$ if and only if
	\begin{align*}
		d m_1 = f d m_2 \quad \text{for some } f \in L^1 (m_2) .
	\end{align*}
	Call f the "derivative of $m_1$ with respect to $m_2$."
\end{theorem}

% Pardon the many times this has iterated, but this is to fix the overfull hbox problem
\subsection[Alternative proof using Hilbert Space and Riesz-Representation]{\texorpdfstring{Alternative proof using Hilbert Space \\ and Riesz-Representation}{Alternative proof using Hilbert Space and Riesz-Representation}}

We assume basic knowledge of Hilbert spaces (complete inner-product space).

Direct proof of the corollary

\begin{proof}
	Start with measures $m_1, m_2 :S \to [0,\infty)$.
	Let  $m_3 = m_1 + m_2$

	Now compute for  $f \in L^2 (m_3)$
	\begin{align*}
		\left| \int f dm_1 \right| &\leq \int |f| dm_1 \leq \int |f| dm_3 \\
						&\leq \left( \int |f|^2 dm_3 \right)^{1/2} \left( \int 1 dm_3 \right)^{1/2}\\
						&= \sqrt{m_3 (X)} \, \|f\|_{L^{2}(m_3)}
	\end{align*}
	So $\Lambda f = \int f dm_3$ is a bounded linear functional.

	Since $L^{2} (m_3)$ is a Hilbert space, Riesz gives a unique $g \in L^{2} (m_3)$ so
	\begin{align*}
		\Lambda f = \int f g dm_3 \quad \text{for } f \in L^{2}(m_3)
	\end{align*}

	Rewrite
	\begin{align*}
		\int f dm_1 = \int f g d m_3 \quad \text{where $m_3 = m_1 + m_2$}
	\end{align*}

	For $f = \1_{A}$, $m_1 (A) = \int \1_{A} g dm_3$ or
	\begin{align*}
		0 \leq \frac{\int \1_{A} g dm_3}{m_3 (A)} = \frac{ m_1 (A)}{m_3 (A)} \leq 1
	\end{align*}
	Conclude $m_3 (\{0 \leq g \leq 1\}^c) = 0$. In other words, change $g$ on a set of $m_3$-measure zero so $0 \leq g \leq 1$ everywhere.

	\begin{align*}
		\text{let  } E &= \{ 0 \leq g < 1 \} \\
		m_a (A) &= m_1 (A \cap E), \text{ and} \\
		M_s (A) &= m_1 (A \cap E^c),
	\end{align*}
	Rewrite again
	\begin{align*}
		\int (1 - g) f \, dm_1 = \int fg \,dm_2.
	\end{align*}
	since $(1 - g) \1_{E^c} \equiv 0$ and $\1_{E^c} g - \1_{E^c}$, obtain $m_2 (E^c) = 0$
	and so $m_2 \perp m_s$. 


	Now for  $A \subseteq E$, setting
	\begin{align*}
		f = (1 + g + \cdots + g^n) \1_{A } \in L^2 (m_3)
	\end{align*} gives
	\begin{align*}
		\int ( 1 - g^{n+1}) \1_{A} dm_1 = \int (g + g^2 + \cdots + g^{n+1}) \1_{A} dm_2
	\end{align*}
	let $h = \sum_{n \geq 0} g^{n} \1_{E}$.

	Since $A \subseteq E = \{ 0 \leq g \leq 1\}$
	\begin{align*}
		&(1 - g^{n+1}) \1 \searrow \1_{A} \\
		&(g + g^2 + \cdots + g^{n+1} ) \1_{A} \nearrow h \1_{A}
	\end{align*}
	So by monotone convergene
	\begin{align*}
		\int \1_{A} dm_1 = \int h \1_{A} dm_2
	\end{align*}
\end{proof}



\begin{remark}
	Both proofs have the same ingredients: approximate from below and the monotone connvergence.

	Riesz proof makes series for $\frac{d m_1}{d m_2}$
\end{remark}

