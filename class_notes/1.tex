\section{2022-08-31}

Following the universal rule for first class, we covered the syllabus and logistics. We then dove into review on the Riemann integral to motivate measure theory. Prof. Smart seemed much more comfortable talking about math than logistics.

\subsection{Course Plan}
\begin{itemize}
	\item Review Riemann Int
	\item Abstract Measure Theory (Majority of class)
	\item Applications in Probability
	\item Apps in Dynamics
	\item Some brief apps in geometry
\end{itemize}

\begin{problem} The Riemann Integral 
\item Works for most functions of interest but notably
	\begin{enumerate}
		\item Is not closed under important limits
		\item It is hard to eneralize to new geometries.
	\end{enumerate}
\end{problem}

\begin{definition}[Riemann Integral]
	We can define the riemann integral as $\int_{{\Q}}^{{}} {f}$ of a bounded $f : \Q \to \R$ 
	defined in closed rectangles
	$\Q = [a_1,b_1] \cross [a_2,b_2] \ldots [a_d,b_d] \subseteq \R$. We use Darboux sums
\end{definition}

\vline
\begin{definition}[partition]
	A \underline{partition} of $\Q$	is a finite set of \emph{closed} rectanlges whose |underline{interiors} are disjoint and whose union is $\Q$.
\end{definition}

\vline

The upper and lower Darboux sums are $U(p,f) =\sum_{R \in P}^{} \sup f \cdot |R|$ and \newline 
$L(p,f) = \inf f \cdot |R|$

\begin{lemma}
	For any 2 partitions $p_1$ and $p_2$ of  $\Q$,  $L(p_1,f) \leq U(p_2, f)$
\end{lemma}

\begin{solution}
	We have
	\begin{align*}
		L(p_1, f) &= \sum_{R}^{}  inf f \cdot |R| = \sum_{R_1 \in P_1}^{} inf f 
		\sum_{R_2 \in P_2}^{} |R_1 \intersection R_2 | \\
				  &= \sum_{\substack{R_1 \in P_1 \\ R_2 \in P_2 \\ |R_1 \intersection R_2 | > 0} }^{} \inf_{R_1} f \cdot |R_1 \intersection R_2 | 
				  \leq \sum_{\substack{R_1 \in P_1 \\ R_2 \in P_2 \\ |R_1 \intersection R_2 | > 0} }^{} \inf_{R_1 \intersection R_2} f \cdot |R_1 \intersection R_2 | \\
				  &\leq \sum_{\substack{R_1 \in P_1 \\ R_2 \in P_2 \\ |R_1 \intersection R_2 | > 0} }^{} \sup_{R_1 \intersection R_2} f \cdot |R_1 \intersection R_2 |
				  \leq \sum_{\substack{R_1 \in P_1 \\ R_2 \in P_2 \\ |R_1 \intersection R_2 | > 0} }^{} \sup_{R_2} f \cdot |R_1 \intersection R_2 | \\
				  &= \sum_{R_2 \in P_2}^{} \sup_{R_2} f \sum_{\substack{R_1 \in P_1 \\ |R_1 \intersection R_2 | > 0}}^{} |R_1 \intersection R_2 |  = \sum_{R_2 \in P_2}^{} \sup_{R_2} f \cdot |R_2| = U(p_2, f)
	\end{align*} 
\end{solution}


Note we say a bounded $f : \Q \to \R$ is Riemann integrable when  $\sup_p L(p,f) = \inf_p U(p,f)$ in which case we let  $\int_{\Q} f$ denote the common value.


\begin{theorem}
	If $f : \Q \to \R$	 is continuous, then f is Rieman integrable.
\end{theorem}

\begin{solution}
	$\Q$ is compact, so  $f$ is uniformly cont. Thus, given a $\varepsilon >0 $, choose a $\delta >0$ so  $$
\end{solution}


