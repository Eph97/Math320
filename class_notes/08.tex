\section{September 26th, 2022}

\epigraph{``What's one thing good about these cables [all over the place] is that you are constantly reminded
	of Thermodynamics. Entropy. (said after he trips on some cables on the classroom floor)''}{Charlie}

\subsection{Outer Measures}
A general tool (devised by Caratheodory) to constrain measures. Outer measures is a relaxation of measure.
It is easier to construct outer measures. The main point of this section is a theorem about
constructing measures from outer measures.

\begin{definition}[Outer Measure]
	An outer measure on a measurable space $(X,S)$ is a function $m: S \to [0, \infty]$ such that
	\begin{enumerate}
		\item $m(\emptyset) = 0$
		\item $A \subseteq B$ implies $m(A) \leq m(B)$.
		\item $m \left(\bigcup_{k=1}^{\infty} A_k\right) \leq \sum_{k = 1}^{\infty} m(A_k)$.
	\end{enumerate}
	these three conditions can be summed up by: non-triviality, monotonicity, and countable subadditivity.
\end{definition}

\begin{example}
  Lebesgue outer measure.
	\[
		L^* : \R^d \to [0, \infty].
	\]
	given by
	\[
		L^*(A) = \inf \left\{\sum_k \left|Q_k\right| : A \subseteq \bigcup_{k = 1}^{\infty} Q_k \right\}.
	\]
\end{example}

\begin{question}
Is $J^*$ an outer measure?
\end{question}
No -- fails countability subadditivity condition. $J^*$ is only finitely subadditive.

\begin{definition}[m-Caratheodory]
	If $(X,S,m)$ is an outer measure space, then call $A \in S$ m-Caratheodory if
	\[
		m(B) = m(B \cap A) + m(B \cap A^c) \quad \text{ for } B \in S.
	\]
\end{definition}

Let $S_m \{ A \in S : A \textrm{m-Caratheodory}\}$. Caratheodory basically means ``good for splitting''.

\begin{theorem}
	$S_m$ is a $\sigma$-algebra and $(X, S_m, m|_{S_m})$ is a positive measure space.
\end{theorem}

\begin{proof}
	Step 1: $\emptyset \in S_m$. 
	\begin{align*}
		m(B \cap \emptyset) + m(B \cap \emptyset^c) & = m(\emptyset) + m(B) \\
																								& = 0 + m(B)
	\end{align*}

	Step 2: $A \in S_m \implies A^c \in S_m$.
	\begin{align*}
	  1 = 1
	\end{align*}
\end{proof}

