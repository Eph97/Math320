\section{September 26th, 2022}

\epigraph{``What's one thing good about these cables [all over the place] is that you are constantly reminded
	of Thermodynamics... Entropy.''}{Charlie (said after he tripped on some cables on the classroom floor)}

\subsection{Outer Measures}
A general tool (devised by Caratheodory) to constrain measures. Outer measures is a relaxation of measure.
It is easier to construct outer measures. The main point of this section is a theorem about
constructing measures from outer measures.

\begin{definition}[Outer Measure]
	An outer measure on a measurable space $(X,S)$ is a function $m: S \to [0, \infty]$ such that
	\begin{enumerate}
		\item $m(\emptyset) = 0$
		\item $A \subseteq B$ implies $m(A) \leq m(B)$.
		\item $m \left(\bigcup_{k=1}^{\infty} A_k\right) \leq \sum_{k = 1}^{\infty} m(A_k)$.
	\end{enumerate}
	these three conditions can be summed up by: non-triviality, monotonicity, and countable subadditivity.
\end{definition}

\begin{example}
  Lebesgue outer measure.
	\[
		L^* : \R^d \to [0, \infty].
	\]
	given by
	\[
		L^*(A) = \inf \left\{\sum_k \left|Q_k\right| : A \subseteq \bigcup_{k = 1}^{\infty} Q_k \right\}.
	\]
\end{example}

\begin{question}
Is $J^*$ an outer measure?
\end{question}
No -- fails countability subadditivity condition. $J^*$ is only finitely subadditive.

\begin{definition}[m-Caratheodory]
	If $(X,S,m)$ is an outer measure space, then call $A \in S$ m-Caratheodory if
	\[
		m(B) = m(B \cap A) + m(B \cap A^c) \quad \text{ for } B \in S.
	\]
\end{definition}

Let $S_m = \{ A \in S : A \textrm{ m-Caratheodory}\}$. Caratheodory basically means ``good for splitting''.

\begin{theorem}
	$S_m$ is a $\sigma$-algebra and $(X, S_m, m|_{S_m})$ is a positive measure space.
\end{theorem}

\begin{proof}
	Step 1: $\emptyset \in S_m$.
	\begin{align*}
		m(B \cap \emptyset) + m(B \cap \emptyset^c) & = m(\emptyset) + m(B) \\
																								& = 0 + m(B)
	\end{align*}

	Step 2: $A \in S_m \implies A^c \in S_m$.
	\begin{align*}
		m(B) & = m(B \cap A) + m(B \cap A^c) \\
				 & = m(B \cap (A^c)^c) + m(B \cap A^c)
	\end{align*}

	Step 3: $A_1, A_2 \in S_m \implies A_1 \cup A_2 \in S_m$.
	\begin{align}
		m(B) & = m(B \cap A_1) + m(B \cap A_1^c) \\
				 & = m(B \cap A_1 \cap A_2) + m(B \cap A_1 \cap A_2^c) \nonumber \\
				 & \phantom{=} + m(B \cap A_1^c \cap A_2) + m(B \cap A_1^c \cap A_2^c) \label{step3.2} \\
				 & \geq m(B \cap (A_1 \cup A_2)) + m(B \cap A_1^c \cap A_2^c) \\
				 & = m(B \cap (A_1 \cup A_2)) + m(B \cap (A_1 \cup A_2)^c) \label{step3.4} \\
				 & \geq m(B)
	\end{align}

	Step 3$'$: $A_1, A_2 \in S_m$ disjoint and $m(B) \leq \infty$ implies,
	\[
		m(B \cap (A_1 \cup A_2)) = m(B \cap A_1) + m(B \cap (A_1 \cup A_2)^c).
	\]
	Using disjointness, lines \ref{step3.2} and \ref{step3.4} give
	\begin{multline*}
		m(B \cap A_1) + m(B \cap A_2) + m(B \cap (A_1 \cup A_1)^{c})  \\
		= m(B \cap (A_1 \cup A_2) + m(B \cap (A_1 \cup A_2)^{c}).
	\end{multline*}

	Step 4: $A_k \in S_m$ disjoint $\implies \bigcup_{n = 1}^{\infty} A_k \in S_m$
	and $m(\bigcup_{k=1}^{\infty} A_k = \sum_{k=1}^{\infty} m(A_k)$.
	\[
		m(B) \leq m\left(B \cap \bigcup_{k=1}^{\infty} A_k \right) + m\left(B \cap \left(\bigcup_{k=1}^{\infty} A_k\right)^{c}\right).
	\]
	We need to prove the other direction i.e. $\geq$

	We may assume $m(B) \leq \infty$.
	\begin{align*}
	  m \left(B \cap \bigcup_{k = 1}^{\infty} A_k \right) & \geq m \left(B \cap \bigcup_{k = 1}^{n} A_k \right) \\
		& = \sum_{k = 1}^{n} m(B \cap A_k) \tag{Step 3$'$} \\
		& \xrightarrow[n \to \infty]{} \sum_{k = 1}^{\infty} m(B \cap A_k) \\
		& \geq m \left( B \cap \bigcup_{k = 1}^{\infty} A_k \right)
	\end{align*}
	\begin{align*}
	  m & \left(B \cap \bigcup_{k = 1}^{\infty} A_k \right) + m \left(B \cap \left( \bigcup_{k = 1}^{\infty} A_k\right)^{c}\right) \\
		& = \lim_{n \to \infty} \left\{ m \left( B \cap \bigcup_{k = 1}^{n} A_k \right) + m \left(B \cap \left(\bigcup_{k = 1}^{\infty} A_k\right)^{c}\right)\right\} \\
		& \leq \liminf_{n \to \infty} \left\{ m \left(B \cap \bigcup_{k = 1}^{n} A_k\right) + m \left(B \cap \left(\bigcup_{k = 1}^{n} A_k\right)^{c}\right)\right\} \\
		& = m(B)
	\end{align*}
\end{proof}

\subsection{Riesz Representation Theorem (cont.)}

Suppose that $(X,T)$ is a  nice topological space.
If $K_{compact} \subseteq U_{open}$, then there is $f \in C_c(x)$ with $\1_k \leq f \leq \1_u$.
Suppose $\Lambda : C_c(x) \to \R$ is positive linear.
\begin{align*}
  & m(\underset{open}{U}) = \sup \left\{\Lambda(f) : f \in C_c(x), 0 \leq f \leq \1_k \right\} \\
  & m(\underset{arbitrary}{A}) = \inf \left\{ m(U): \underset{open}{U} \contains A\right\} \\
\end{align*}

\begin{lemma}
  $m$ is an outer measure on $(X, \mathcal{P}(X))$.
\end{lemma}

\begin{proof}
	$\emptyset$ open, so $m(\emptyset) = 0$.
	\[
		A \subseteq B \implies m(A) \leq m(B).
	\]
	automatically from the definition.
	We need to check
	\[
		m \left(\bigcup_{k = 1}^{\infty} A_k \right) \leq \sum_{k = 1}^{\infty} m(A_k).
	\]
	Choose $U_k \supseteq A_k$ open with $m(U_k) \leq m(A_k) + \frac{\varepsilon}{2^k}$.
	\[
		m \left(\bigcup_{k = 1}^{\infty} A_k \right) \leq m \left( \bigcup_{k = 1}^{\infty} U_k \right).
	\]
	so it is enough to show $\leq \sum_{k = 1}^{\infty} m(U_k) \leq \varepsilon + \sum_{k = 1}^{\infty} m(A_k)$.

	Suppose $0 \leq f \leq \1_{\bigcup_{n = 1}^{\infty} U_k}$. Then $f \in C_{c}(x)$.
	Since $f$ has support, there is an $n \geq 1$ so $0 \leq f \leq \1_{\bigcup_{n = 1}^{\infty} U_k}$.
	Using partitions of unity, $f = f_1 + \ldots + f_n$ with $0 \leq f_k \leq \1_{U_k}$. So
	\begin{align*}
		\Lambda(f) & = \Lambda(f_1) + \ldots + \Lambda(f_n) \\
							 & = m(U_1) + \ldots m(U_n)
	\end{align*}
	So,
	\[
		m \left(\bigcup_{k = 1}^{\infty} A_k \right) \leq \sum_{k = 1}^{\infty} m(A_k).
	\]
\end{proof}

Our next goal is to show that $S_m \supseteq \textrm{Borel}$.

\begin{lemma}
  If $U$ open, $m(U) < \infty$, and $\varepsilon > 0$, then there is compact $K \subseteq U$ with $m(U \setminus K) < \varepsilon$.
\end{lemma}

\begin{proof}
	Since $m(U) < \infty$, we can choose $f \in C_c(X)$ with $0 \leq f \leq \1_U$ 
	and $\Lambda(f) \geq m(U) - \varepsilon$.

	Let $k = \left\{x \in X : f(x) \neq 0 \right\}$. Note $K \subseteq U$ and $K$ compact. 
	If $g \in C_c(x)$ and $0 \leq g \leq \1_{U \setminus K}$, then $0 \leq f + g \leq \1_k$. 
	\[
		m(U) \geq \Lambda(f + g) = \Lambda (f) + \Lambda (g) > \Lambda(U) - \varepsilon + \Lambda(g).
	\]
	since $g$ arbitrary, $m(U \setminus K) \leq \varepsilon$.
\end{proof}

\begin{lemma}
  If $K_1, K_2$ disjoint compact, then $m(K_1 \cup K_2) = m(K_1) + m(K_2)$.
\end{lemma}

\begin{proof}
	Since $(X,T)$ nice, there are open $U_1 \supseteq K_1$ and $U_2 \supseteq K_2$ with $U_1 \cup U_2 = \emptyset$.

	[insert image]

	making $U_1, U_2$ smaller, we may assume
	\[
		m(U_1) \leq m(K) + \varepsilon \quad \textrm{ and } \quad m(U_2) \leq m(K_2) + \varepsilon.
	\]
	From the definitions, $m(U_1 \cup U_2) = m(U_1) + m(U_2)$.
	\footnote{Why? If $0 \leq f \leq \1_{U_1 \cup U_2}$, then $\1_{U_k} - f \in C_c(x)$.}
	We then calculate 
	\begin{align*}
		m(K_1 \cup K_2) & \leq m(U_1 \cup U_2) \\
										& = m(U_1) + m(U_2) \\
										& \leq m(K_1) + m(K_2)
	\end{align*}
\end{proof}

\begin{lemma}
  The Borel Sets are $m$-Caratheodory, $B \subseteq S_m$.
\end{lemma}

\begin{proof}
	Since $B$ generated by the open sets and $S_m$ is a $\sigma$-algebra,
	it is enough to show every open $U$ is Caratheodory. 
	Since $(X,T)$ locally compact, we need only consider $U$ with compact closure.
	Let $B \subseteq X$ be arbitrary. We need 
	\[
		m(B) \geq m(B \cap U) + m(B \cap U^{c}).
	\]
	Choose $V \supseteq B$ open such that $m(V) \leq m(B) + \varepsilon$. We then need, 
	\[
		m(V) \geq m(V \cap U) + m(B \cap U^{c}).
	\]
	We may assume $U \subseteq V$ (an exercise!).
	Choose compact $K \subseteq U$ and $G \subseteq V$ with $m(U \setminus K) \leq \varepsilon$ 
	and $m(U \setminus G) \leq \varepsilon$.
	\begin{align*}
		m(U) + m(V \cap U^{c}) & \leq m(U \setminus K) + m(K) + m(V \cap G^{c}) + m(G \cap U^{c}) \\
													 & \leq m(K) + m(G \cap U^{c}) + 2\varepsilon \\
													 & = m(K \cup (G \cap U^{c})) + 2\varepsilon \\
													 & \leq m(V) + 2 \varepsilon
	\end{align*}
\end{proof}

