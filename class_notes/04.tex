\section{2022-09-12}

\epigraph{Shirt of the day: ``I have eated all the dinner''}{Charlie}

Last Time: Basic results for measurable spaces $(X, S)$.

\begin{theorem}
  There is a smallest $\sigma$-algebra $\sigma(c)$ containing any collection $C$ of subsets of a set $X$.
\end{theorem}

\begin{lemma}
	If $(X,S_x), (Y,S_y)$ are measurable spaces, $f: X \to Y$ is $(S_x, S_y)$ measurable, $S_x \supseteq S_y$ and $S_y\supseteq S_y$ s.t. $S_x' \subseteq S_X$ and $S_Y' \subseteq S_Y$, then $f$ is $(S_x', S_y')$ measurable.
\end{lemma}

\begin{theorem}
  If $(X, S)$ is a measurable space and $f: X \to Y$, then 
	\[
		f^*(s) = \left\{A \subseteq Y \mid f^{-1}(A) \in S \right\}.
	\]
	is the largest $\sigma$-algebra such that $f$ is $(S, f^*(S))$-measurable
\end{theorem}

\begin{theorem}
	If $(Y,S)$ is a measurable space and $f : X \to Y$, then 
	 \begin{align*}
		 f^*(S) &= \{f^{-1}(A) : A \in S\}
	\end{align*}
	is the smallest $\sigma$-algebra such that $f$ is $(f^*(S), S)$- measurable.
\end{theorem}

\begin{theorem}
	If $(X, T_X)$ $(Y,T_Y)$ are topological spaces, and  $f : X \to Y$ is $T_X, T_Y$-continuous, then $f$ is also $(\sigma(T_X), \sigma(T_Y))$-measurable.
\end{theorem}

\begin{proof}
	$f$ is $(T_X,T_Y)-continuous$
	\begin{align*}
		&\iff \{f^{-1}(A) : A \in T_Y\} \subseteq T_X \\
		&\iff \{f^{-1}(A) : A \in \sigma(T_Y)\} \subseteq \sigma(T_X) \\
		&\iff f \text{ is } (\sigma(T_X), \sigma(T_Y)) \text{ -- measurable.}
	\end{align*}  
\end{proof}

abuse notation:
If $(X, S_X)$ is a measurable space and  $(Y, T_Y)$ is a topological space, then call $f : X \to Y$
$(S_X, T_Y)$-measurabl $\iff f$ is $(S_X, \sigma(T_Y))$- measurable.

\begin{theorem}
	$f : X \to Y$ is $(S_X, \sigma(T_Y))$- measurable $\iff \forall A \in T_Y$ we have $f^{-1}(A) \in S_X$.
\end{theorem}

\begin{proof}
	Since $T_Y \subseteq \sigma(T_Y)$ 
	then $\implies$ directions trivial 

	for the $\impliedby$ direction, observe that 
	\begin{align*}
		&\{f^{-1}(A) : A \in T_Y\} \subseteq S_X \\
		&\implies \sigma(\{f^{-1}(A) : A \in T_Y\}) \subseteq \sigma(S_X) = S_X \\
		&\iff \sigma(\{f^{-1}(A) : A \in \sigma(T_Y)\}) \subseteq  S_X \text{--measurable}
	\end{align*}
\end{proof}

If $(X,T)$ is a topological spae, then call $\sigma(T)$ the Borel subsets of  $X.$ 

If $(X, T_X)$ and $(Y,T_Y)$ are topological spaces, then $f:X \to Y$ is measurabe.

If it is  $(\sigma(T_X), \sigma(T))$-measurable.

comeback too

\begin{definition}
	The extended Reals.
	$[-\infty,\infty]$ is $\mathbb{R} \bigcup \{-\infty, +\infty\}$.

	Add new open sets $[-\infty, a)$ and $(a, +\infty]$ 
	and the usual arith rules
	\begin{align*}
		a + \infty = \infty, \quad a\cdot \infty = \infty, \quad a > 0
	\end{align*} etc defined by demanding $+$ and $\cdot$ extend continuously to
	\begin{align*}
		&[-\infty, \infty]^2 \setminus \{(-\infty, \infty), (\infty, -\infty)\} \text{ and } \\
		&[-\infty, \infty]^2 \setminus \{(0, \pm \infty), (\pm \infty, 0)\}
	\end{align*}
\end{definition}

\begin{corollary}
	If $(X,S)$ measurable, then $f : X \to [-\infty, \infty]$ is measurabnle 
	$iff$ $f^{-1}((a,\infty]) \in S$ for all $a \in [-\infty,\infty]$
\end{corollary}

\begin{proof}
	The main point is
	\begin{align*}
		&\sigma(\{(a,\infty] : a \in [-\infty, a)\}) \\
		= &\sigma(\{ A \subseteq [-\infty, \infty] : A \text{ open} \})
	\end{align*} 
\end{proof}

\begin{theorem}
	If $(X,S)$ is measurable space, 
	$f_1, \ldots, f_d : X \to \mathbb{R}$ are measurable and $g:\mathbb{R}^d \to \mathbb{R}$ continuous, then $g \circ (f_1, \ldots, f_d) : X \to \mathbb{R}$ is measurable.
\end{theorem}

\begin{proof}
	It is enough to show that $(f_1, \ldots, f_d) : X \to \mathbb{R}^d$ is measurable. But given $A \subseteq \mathbb{R}^d$ is open, we need to show $f^{-1}(A) \in S$.

	Write $A = \bigcup_{k=1}^{\infty} R_k$ with $R_k$ open rectangles. It is enough to show $f^{-1}(R) \in S$ for any open rectangle $\mathbb{R} \subseteq \mathbb{R}^d$.

	Wirte $R = (a,b) \times \ldots \times (a_d, b_d)$ and compute 
	$f^{-1}(R) =  f^{-1}((a,b)) \cap \ldots \cap f^{-1}((a_d, b_d)) \in S$
	by measurability of $f_1, \ldots , f_d$
\end{proof}

\begin{corollary}
	If $(X,S)$ is a measurable space and $f,g : X \to [-\infty, \infty]$ is measurable, 
	then $f+g$, $f\circ g$ $\min\{f,g\}$ $max\{f,g\}$, and $|f|$ are all measurable (when defined).
\end{corollary}

\begin{theorem}
	If $(X,S)$ measurable space and $f_1, f_2, \ldots : X \to [-\infty, \infty]$ measurable, then $f = \sup_k f_k$ is measurable.
\end{theorem}

\begin{proof}
	The main part is
	\begin{align*}
		f^{-1}((a, \infty]) = \bigcup_{k=1}^{\infty} f_k^{-1}((a,\infty]) \in S
	\end{align*}
\end{proof}

\begin{corollary}
	\begin{align*}
		\inf_k f_k, \, \lim \sup_k f_k, \, \text{and } \, \lim \inf_k f_k
	\end{align*} are also measurable.
\end{corollary}

\begin{proof}
	\begin{align*}
		&\inf_k f_k = - \sup_{k} (-f_k) \\
		&\lim_{R} \sup f_k = \inf_{k} (\sup_{j} f_{k+j}) \\
		&\lim_{k}\inf f_k = - \lim_{k} \sup (-f_k)
	\end{align*} 
\end{proof}

\begin{corollary}
	If $(X,S)$ measurabnle space, $f_k : X \to (-\infty, \infty]$ measurable for  $k \geq 0$ 
	 \begin{align*}
	    f (x) = \lim_{k \to  \infty} f_k (x) \, \forall x \in X
	\end{align*} then $f$ is measurable.
\end{corollary}

\begin{proof}
	 \begin{align*}
	f = \lim_{k} f_k \implies f = \lim_{k} \sup f_k
	\end{align*} 
\end{proof}

\underline{Review:} From Real analysis

check
\begin{enumerate}
	\item Locally uniform limits of continuous functions are continuous.
	\item Pointwise limits of continuous funcions need not be continuous.
\end{enumerate}

\subsection{Positive Measure}

\begin{definition}
	A positive measure is a function $m : S \to [0, \infty]$ on a $\sigma$-algebra $S$ on a set $X$ such that $m(\emptyset) = 0$
	and $m( \bigcup_{k=1}^{\infty} A_k) = \sum_{k=1}^{\infty} m(A_k)$ when 
	$A_1, A_2, \ldots \in S$ disjoint.

	{\tiny note $m(\emptyset) = 0$ allows us to exclude $m$ which gives everything $+\infty$ measure.}

	Call the triple $(X,S,m)$ a positive measure space.
\end{definition}

\begin{example}
	\begin{enumerate}
		\item $X = \{x_1, \ldots, x_n\}$,
			\begin{align*}
				S = \mathcal{P}(X) \text{ and } m(A) = \#A \quad \text{\tiny the counting measure}
			\end{align*} 
		\item $X$, arbitrary
			\begin{align*}
				S = \mathcal{P}(X) \text{ and } m(A) = 
				\begin{cases}
					\#A & A\text{-finite} \\
					+\infty & A\text{-finite} \\
				\end{cases}
			\end{align*} 

		\item The Dirac Measure:
			$X$ arbitrary, $S$ arbitrary, and $X_0 \in X$ and 
			\begin{align*}
				m(A) = 
				\begin{cases}
					1 & x_0 \in A \\
					0 & x_0 \notin A
				\end{cases}
			\end{align*}
	\end{enumerate}
\end{example}


\begin{definition}[Basic Properties]
	\begin{enumerate}
		\item If $A,B \in S$  and $A \subseteq B$, then $m(A) \leq m(B)$
		\item If  $A_1, \ldots, A_n \in S$ disjoint, then 
			$m(A_1 \cup A_2 \cup \ldots \cup A_n) = m(A_1) + \ldots + m(A_n)$
		\item If  $A_k \in S$ and $A_1 \subseteq A_2 \subseteq ldots$, then
			\begin{align*}
				m(\bigcup_{k=1}^{\infty} A_k ) = \lim_{k \to \infty} m(A_k)
			\end{align*} 
		\item If $A_k \in S$, and $A_1 \supseteq A_2 \supseteq \ldots$ and $m(A_1) < \infty$,
			then  
			 \begin{align*}
				m(\bigcap_{k=1}^{\infty} A_k) = \lim_{k \to \infty}  m(A_k)
			\end{align*} 
	\end{enumerate}
\end{definition}

\begin{proof}(of properties)
	\begin{enumerate}
		\item[2] let $A_k = \emptyset$ for $k >n$. Now $A_1, A_2, \ldots$ disjoint and 
			\begin{align*}
				m(A_1 \cup \ldots \cup A_n) &= m(\bigcup_{k=1}^{\infty} A_k ) \\ 
				= \sum_{k=1}^{\infty} m(A_k) &= m(A_1) + \cdots + m(A_n)
			\end{align*}
		\item[1] Write $B = A \cup (B \setminus A)$, and $B \setminus A \in S$ since $S$ is a $\sigma$-algebra
			\begin{align*}
				m(B) = m(A) + m(B \setminus A) \geq m(A)
			\end{align*} 
		\item[3] Compute (Check)
			\begin{align*}
				m(\bigcup_{k=1}^{\infty} A_k) &= m(A_1 \cup \bigcup_{k=1}^{\infty} m(A_{n+1} \setminus A_k) \\
											  &= m(A_1) + \sum_{k=1}^{\infty} m(A_{k+1} \setminus A_k) \\
											  &= m(A_1) + \lim_{n \to \infty}  \sum_{k=1}^{n} m(A_{k+1} \setminus A_k) \\ 
											  &= \lim_{n \to \infty} \left(m(A_1) \sum_{k=1}^{n} m(A_{k+1} \setminus A_k) \right) \\ 
											  &= \lim_{n \to \infty} m \left(m(A_1 \cup \bigcup_{k=1}^{\infty} (A_{k+1} \setminus A_k) \right) \\ 
											  &= \lim_{n \to \infty} m (A_{n+1})  \\ 
			\end{align*} 
		\item[4] Let $B_k = A_1 \setminus A_k \in S$. Observe: $B_1 \subseteq B_2 \subseteq \ldots$

			\underline{Conclude}:
			\begin{align*}
				m(\bigcup_{k=1}^{\infty} B_k) = \lim_{n \to \infty} m(B_n) \quad \text{Rewrite as:} 
				\quad m(A_1 \setminus \bigcap_{k=1}^{\infty} A_k) = 
				\lim_{n \to \infty} m(A_1 \setminus A_n)
			\end{align*} rewrite as 
			\begin{align*}
				m(A_1) - m(\bigcap_{k=1}^{\infty} A_k) = m(A_1) - \lim_{n \to \infty} m(A_n)
			\end{align*} 
			Since $m(A_1) < \infty$ we can cancel.
	\end{enumerate}
\end{proof}

\begin{example}
	Let $m$ be the counting measure on  $\mathbb{N}$ and consider $A_n = \{n, n+2, n+2, \ldots \}$. 
	Then $A_1 \supseteq A_2 \supseteq \ldots$ , $m(A_k) = \infty$ and  
	\begin{align*}
		m(\bigcap_{k=1}^{\infty} A_k ) = m(\emptyset) = 0
	\end{align*} 
\end{example}

