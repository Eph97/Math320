\section{2022-10-26}

\noindent
Last time: \\
Moran's theorem for fractals \\

\noindent
Today: \\
Return to rudin Narrative following chapter $3$: $L^{p}$-spaces.

\subsection{$L^{p}$-spaces}

Recall definition of $L^{1}$.

$m$ is a measure on $\sigma$-algebra $S$ on a set $X$.

$\mathcal{L}^{1}(m)$ is the space of measurable $f : [-\infty, \infty]$ such that

\begin{align*}
	[f]_{\mathcal{L}^{1}(m)} = \int |f| dm < \infty
\end{align*}

let $\mathcal{N}(m) = \{f : X \to [-\infty, \infty] \text{ meas }: m(\{x : f(x) \neq 0\} = 0\}$ be the set of functions that vanish away from a set of measure zero,

Observe
\begin{align*}
	[f]_{\mathcal{L}^1 (m) } = 0 \text{ for } f \in \mathcal{N}(m) \\
\end{align*} let
 \begin{align*}
	 L^{1} (m) &= \mathcal{L}^1 (m) / \mathcal{N} (m) \\
	 \|\{f\}\|_{L^1 (m) } &= \|f\|_{\mathcal{L}^1 (m) } 
\end{align*} 


\subsection{ Jensen's Inequality }
	
\begin{definition}[Convex]
	$\phi : \R \to \R$ is convex if and only if
	\begin{align*}
		\phi (( 1 - t) a + tb) \leq (1-t) \phi (a) + t \phi (b)
	\end{align*} for $a \leq b$ and $0 \leq t \leq 1$

	equivalently, 
	\begin{align*}
		\frac{\phi (c) - \phi (a)}{c - a} \leq \frac{\phi (b) - \phi (c)}{b - c} 
		\quad \text{for} \quad  a < b
	\end{align*} 
\end{definition}
	
\begin{lemma}
	Convex $\implies$ continuous
\end{lemma}

\begin{proof}
	$a < x < y_{k+1} < y_{k}$ where $y_{k} \xrightarrow[k \to \infty]{} \infty$ then
	\begin{align*}
		\frac{\phi(x) - \phi(a)}{x - a} &\leq \frac{\phi(y_{k+1}) - \phi (x)}{y_{k+1} - x} \\
										&\leq \frac{\phi(y_{k} - \phi(x)}{y_{k} - x}
	\end{align*} then
	\begin{align*}
		\frac{\phi(y_{k} - \phi(x)}{y_{k} - x} \to \alpha \quad \text{as} \quad k \to \infty \quad \text{and} \\
		\phi(y) = \phi(x) + \alpha (y - x) + \mathcal{O}( | y -x|) \quad \text{for} \quad y > x
	\end{align*} 
\end{proof}

\begin{remark}
	we could extend the notion of convexity to allow $\phi : [-\infty, \infty] \to (-\infty, \infty]$ 
	\begin{lemma}
		okay except for jumps [insert pic]
		\begin{align*}
			f(x) =
			\begin{cases}
				\infty & |x| > 1 \\
				0 &|x| \leq 1
			\end{cases}
		\end{align*} 
	\end{lemma}
	
\end{remark}

\begin{theorem}[Jensen's Inequality]
	$m(x) = 1$, $f \in C^{1}(m)$, $a < f < b$ and \\ 
	$\phi : (a,b) \to \R$ convex, then
	\begin{align*}
		\phi\left(\int f dm \right) \leq \int \phi(f) dm
	\end{align*} 
\end{theorem}

\begin{proof}
	Let $c = \int f dm$. Choose $\alpha \in \R$ so
	\begin{align*}
		\phi (x) \geq \phi (c) + \alpha (x - c) \quad x \in (a,b)
	\end{align*} 
	compute
	\begin{align*}
		\int \phi (f) dm &\geq \int \phi (c) + \alpha (f -c) dm \\
						 &= \phi (c) \int 1 dm + \alpha \int f dm - \alpha c \int 1 dm \\
						 &= \phi (c) + \alpha c - \alpha \\
						 &= \phi (c) = \phi \left( \int f dm \right)
	\end{align*}
\end{proof}

\begin{example}[Arithmatic - geometric mean inequality]
	$e^{x}$ convex
	($\frac{d^{2}}{dx^{2}} e^{x} = e^{x} \geq 0$ or $e^{x} = \sum \frac{x^{k}}{k!} =$ sum of convexterms.
\end{example}

\begin{align*}
	e^{\int f dm} \leq \int e^{f} dm \\
	e^{\log f dm} \leq \int f dm \quad f > 0 \\ 
\end{align*} let $m$ be the $\frac{1}{N} \times$ counting measure on $\{1, \ldots, N \}$ 

\begin{align*}
	e^{\frac{1}{N} \log f_{1} + \cdots + \log f_{N}} \leq \frac{1}{N} (f_{1} + \cdots + f_{N} ) \\
	(f_{1}, \ldots, f_{N})^{\frac{1}{N}} \leq \frac{f_{1} + \cdots + f_{N}}{N} \quad \text{for} \quad f_{1}, \ldots, f_{N} > 0
\end{align*} 

\begin{theorem}[Holder Inequality]
	Suppose $1 < p,q < \infty$ and $\frac{1}{p} + \frac{1}{q} = 1$.
	By convexity,
	\begin{align*}
		e^{\frac{s}{p} + \frac{t}{q}} \leq \frac{e^{s}}{p} + \frac{e^{t}}{q}
	\end{align*} 
\end{theorem}

\begin{remark}
	If $0 < a,b < \infty$, then letting $s = pbqa$ and $t = qbga$ [check]?
	gives
	\begin{align*}
		ab \leq \frac{a^{p}}{p} + b^{q}
	\end{align*} also true for $0 \leq a,b \leq \infty$
\end{remark}

\begin{theorem}[Holder]
	If $f,g: X \to [0,\infty]$ measurable, then 
	 \begin{align*}
		\int fg dm \leq \left( \int f^{p} dm\right)^{\frac{1}{p}} \left( \int g^{q} dm\right)^{\frac{1}{q}}
	\end{align*} 
\end{theorem}

\begin{proof}
	Assume $f,g \notin \mathscr{N}(m)$
	Scale so $\int f^{p} dm = \int g^{q} dm = 1$ then
	\begin{align*}
		\int fg dm &\leq \int \frac{f^{p}}{p} + \frac{g^{q}}{q} dm = \frac{1}{p} + \frac{1}{q} = 1 \\
				   &= \left( \int f^{p} dm \right)^{\frac{1}{p}} \left( \int g^{q} dm\right)^{\frac{1}{q}}
	\end{align*} 
\end{proof}

\begin{corollary}[Weak Minkowski]
\begin{align*}
	\left( \int (f + g)^{p} dm \right)^{1/p} \leq \left( \int f^{p} dm \right)^{1/q} + \left( \int g^{p} dm \right)^{1/p}
\end{align*} 	
\end{corollary}

\begin{proof}
	$\frac{1}{p} + \frac{p-1}{p} = 1$, $q = \frac{p}{p-1}$.

	\begin{align*}
		(f + g)^{p} &= (f+g) ( f+g)^{p-1} \\
		\int f ( f + g)^{p-1} dm &\leq \left( \int f^{p} dm \right)^{1/p} \left( \int (f+g)^{p} dm \right)^{\frac{p-1}{p}} \\
		\int g ( f + g)^{p-1} dm &\leq \left( \int g^{p} dm \right)^{1/p} \left( \int (f+g)^{p} dm \right)^{\frac{p-1}{p}} \\
	\int ( f + g)^{p} dm &\leq \left[ \left( \int f^{p} dm \right)^{1/p} + \left( \int g^{p} dm \right)^{1/p} \right] \left( \int (f+g)^{p} dm \right)^{\frac{p-1}{p}} \\
	\end{align*} 
\end{proof}

\subsection{$L^{p}$-Spaces}

\noindent For $1 \leq p < \infty$:

\noindent $\mathcal{L}^{p} (m)$ is space of measurable $f : X \to [-\infty, \infty]$ such that
 \begin{align*}
	\|f\|_{\mathcal{L}^{p} (m)} = \left( \int |f|^{p} dm \right)^{1/p} < \infty
\end{align*} 



\noindent $\mathcal{L}^{\infty} (m)$ is space of measurable $f : X \to [-\infty, \infty]$ such that
 \begin{align*}
 \|f\|_{\mathcal{L}^{\infty} (m)} = \inf\{\alpha \geq 0 : m ( \{ x \in X : |f(x)| > \alpha \}) =0 \}
\end{align*} 

Observe

\begin{align*}
	\|\alpha f\|_{\mathcal{L}^{p}(m)} &= \alpha \|f \|_{\mathcal{L}^{p} (m)} \quad \text{and by minkowski} \\
	\| f + g\|_{\mathcal{L}^{p}(m)} &=  \|f \|_{\mathcal{L}^{p} (m)} + \|g \|_{\mathcal{L}^{p} (m)}  \forall \alpha \geq 0 \quad \text{and} \quad f,g \in \mathcal{L}^{p} (m) \\
\end{align*} 
 let 
 \begin{align*}
 	L^{p} (m) = \mathcal{L}^{p} (m) / \mathscr{N} (m)
 \end{align*} 


 \begin{theorem}
 	$L^{p}(m)$ is complete normed vector space.
 \end{theorem}
 
 \begin{proof}
 	Need to show completeness.
Assume $f_{k} \in L^{p} (m)$ and 
$\| f_{k + 1} - f_{k} \|_{L^{p} (m)} \leq 2^{-k}$

Assume $1 \leq p < \infty$. Let  $g_{n} = \sum_{k=1}^{n} | f_{k}- f_{k+1} |$, $g = \lim_{n \to \infty} g_{n}$

By Minkowski:
\begin{align*}
	\left( \int g_{n}^{p} dm \right)^{1/p} \leq \sum_{k=1}^{n} \|f_{k} - f_{k+1} \|_{L^{p}(m)} \leq 1
\end{align*} By monotone convergence theorem. $\int g^p dm \leq 1$ 

In particular,
\begin{align*}
	f(x) = f_{1} (x) + \sum_{k=1}^{\infty} (f_{k+1}(x) - f_{k}(x))
\end{align*} is absolutely convergent away from measure zero set of $X$.

compute
\begin{align*}
	\int | f - f_{k} |^{p} dm &= \int \liminf_{j \to \infty} |f_{j} - j_{k} |^{p} dm \\
							  &\leq \liminf_{j \to \infty}\int  |f_{j} - j_{k} |^{p} dm \\
							  &\leq (2^{1-k})^{p} \xarrow{}[k \to \infty] 0
\end{align*} 
When $p = \infty$, let
\begin{align*}
	B = \bigcup_{k \geq 1} \{x \in X : |f_{k} (x) - f_{k +1} (x) | > 2^{-k}\}
\end{align*} by definition of $L^{\infty}$, $m(B) = 0$ for  $x \in XX \setminus B$, we have
\begin{align*}
	|f_{k +1} (x) - f_{k} (x)| &\leq 2^{-k} \\
	f(x) &=\lim_{k \to \infty} f_{k}(x) \quad \text{exists} \\
	|f (x) - f_{k} (x) | &\leq 2^{1-k}
\end{align*} So we conclude that
\begin{align*}
	\| f - f_{k} \|_{L^{\infty} (m)} \leq 2^{1 - k}
\end{align*} 
 \end{proof}
 
\subsubsection{$L^{p}$ on nice spaces}

$X$ locally compact Hausdorff. $S$ a Borel $\sigma$-algebra.

 
 \begin{theorem}[Lusin]
	 If $f: X \to [-\infty, \infty]$  measurable and $m(\{x \in X : f(x) \neq 0 \} ) < \infty$ and  $\varepsilon > 0$

	 Then there exists a $g \in C_{c} (X) $ such that $m( \{ x \in X : f(x) \neq g(x) \} ) < \varepsilon$.
	 Moreover, if $|f| \leq 1$, then $ \|g\| \leq 1$
 \end{theorem}
 
 \begin{remark}
 	This says that measurable functions are almost continuous.
 \end{remark}
 
\begin{proof}
	Reduce to case $X$ compact and $0 \leq f \leq 1$ (exercise).

	Write $f = \sum_{k=1}^{\infty} 2^{-k} \1_{A_{k}}$ where 
	\begin{align*}
		A_{1} &= \{ x \in X : f(x) \geq \frac{1}{2} \} \\
		A_{k} &= \{ x \in X : f(x) - \sum_{j=1}^{k} 2^{-k} \1_{A_{j}} \geq 2^{-k} \}
	\end{align*}
\end{proof}
 
