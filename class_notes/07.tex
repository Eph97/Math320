\section{September 21st, 2022}

\epigraph{``Fill In Quote''}{Charlie}

\begin{definition}
	$(X,T)$ \underline{nice} topological space if: Hausdorff, locally compact, and every open set is the union of countably many compact sets.
\end{definition}

\begin{example}
	$(K,d)$ a compact metric space, $C_c (x)$ space of $f \in C(x)$ such that $\{x\in X: f(x) \neq 0 \}$ is compact.

	That is, $C_c (x)$ is the space of compactly supported continuous function:
		$\Lambda : C_c(X) \to \mathbb{R}$ is posiive linear if:
	\begin{align*}
		&\Lambda(\alpha f + \beta g) = \alpha \Lambda(f) + \beta \Lambda(g) \quad \text{linearity} \\
		&\Lambda(f) \geq 0 \quad \text{ when } f \geq 0
	\end{align*} 
\end{example}


\begin{theorem}
	There is a unique measure $m$ on (X,B) s.t.
	\begin{align*}
		\Lambda(f) = \int f dm \quad \forall f \in C_c (X)
	\end{align*} 
\end{theorem}

\begin{definition}[defining m]
	for $A \subseteq X$ borel and $m$ a borel measure, $m(A) = \int \1_{A} dm$
\end{definition}

We want to approximate $\1_A$ with elements of $C_c (X)$

For $U \subseteq X $ open, let $m(U) = \sup\{\Lambda (f) : f \in C_c (X), \, 0 \leq f \leq \1_{U} \}$

insert image.

For $A \subseteq X$ arbitrary, let $m(A) = \inf\{m(U) : A \subseteq U_{open} \subseteq X\}$

We argue the restriction of $m$ to Borel $\sigma$-algebra is a measure.

\subsection{Building Test functions}

\begin{lemma}
	If $K_{1}, K_{2}$ compact and $K_{1} \cap K_2 = \emptyset$, then there are open $U_{1} , U_{2}$ with $K_{1} \subseteq U_{1}$, $K_{1} \subseteq U_{2}$, and $U_1 \cap U_2 = \emptyset$
\end{lemma}

\begin{proof}
	Since $X$ hausdorff and $K_1 \cap K_2 = \emptyset$, for  $x \in K_1$ and $y \in K_2$, there are open $U_{x,y}$, and $V_{x,y}$ with $x \in U_{x,y}$ and $y \in V_{x,y}$ and $U_{x,y} \cap V_{x,y} = \emptyset$, for any $x \in K_1$,
	\begin{align*}
		K_2 \subseteq \bigcup_{y \in K_2} V_{x,y}
	\end{align*} 

	So there are $y_1, ,\ldots , y_n$ with $K_2 \subseteq \bigcup_{k=1}^n V_{x, y_{k}} = V_{x}$

	Now have $x \in U_{x}$, $K_2 \subseteq V_x$, and $U_{x} \cap V_{x} = \emptyset$. Since $K_{1} \subseteq bigcup_{x \in K_1} V_x$, have 
	\begin{align*}
		K_1 \subseteq U_{x} \cup \ldots \cup U_{x_{n}} = U
	\end{align*} 
	Let $V = V_{x_{1}} \cap \ldots \cap V_{x_{n}}$. Have $K_{1} \subseteq U$, $K_{2} \subseteq V$ and $U \cap V = \emptyset$
\end{proof}

\begin{lemma}
	If $K$ compact, then there is $K \subseteq U_{open} \subseteq \bar{U}_{compact} \subseteq X$.
\end{lemma}

\begin{proof}
	Since $(X,T)$ locally compact, every $x \in K$ has $x \in U_{x} \subseteq \bar{U}_x \subseteq X$
	Since $K$ compact, $K \subseteq U_{x_1} \cup \ldots \cup U_{x_n} \subseteq \bar{U}_x \cup \ldots \cup \bar{U}_{x_{n}}$
\end{proof}

\begin{lemma}
	If $K_1 \subseteq U_1$, then there are  $K_1 \subseteq U_{2} \subseteq K_1 \subseteq U$
\end{lemma}

\begin{proof}
	By the previous lemma, there is $K_1 \subseteq V_{open} \subseteq \bar{V}_{compact}$.

	Replace $U_{1}$ by $U_{1} \cap V$ to obtain $\bar{V}_{1}$ compact. By 2nd previous lemma, we can separate $K$ and $\bar{U}_1 \setminus U_{1}$

	Choose open $V_1$ and $V_2$ with $K_{1} \subseteq V_{1}$,  $\bar{U}_1 \setminus U_1 \subseteq V_{2}$
and $V_{1} \cap V_{2} = \emptyset$ can assume $V_{1} \subseteq U_{1}$

Finish
\end{proof}

\begin{lemma}[Uhryson's lemma]
	If $K_{compact} \subseteq U_{open}$, then there is  $f \in C_{c} (X)$ with $\1_k \leq f \leq \1_U$
\end{lemma}

\begin{remark} dyadic rationals
		A dyadic rational is a number $q = \frac{k}{2^{n}}$ with $k,n \in \mathbb{N}$
\end{remark}

\begin{proof}
	Choose for dyadic rationals $q \in [0,1]$,
	\begin{align*}
		K \subseteq  {\underset {open} {U_{q}}} \subseteq {\underset {compact} {\bar{U}_{q}}} \subseteq U \quad \text{ s.t } 
		q \leq q' \implies \bar{U}_{q'} \subseteq U_{q}	\\
	\end{align*} 

	Now construct by induction on denominator: choose
	\begin{align*}
		U_{\frac{2^{k+1}}{2^{n+1}}} \quad {\text s.t. }\quad 
		\bar{U}_{\frac{k+1}{2^{n}}} \subseteq V_{\frac{2^{k} + 1}{2^{n+1}}} \subseteq \bar{U}_{\frac{2^{k+1}}{2^{n+1}}} \subseteq U_{\frac{k}{2^{n}}}
	\end{align*} 

	But the previous lemma does exactly this. For $x \in U_{0}$, let $f(x) = \sup\{q : q \in U_{q}\}$
	Otherwise, let $f(x) = 0$
\end{proof}

\begin{exercise}
	Prove $f$ continuous and $\1_{\bar{U}_1} \leq f \leq \1_{U_{0}}$ so $f \in C_c(X)$
\end{exercise}


\begin{lemma}
	If $K \subseteq \underset{open}{U_{1}} \cup \ldots \cup \underset{open}{U_n}$, then there are 
	$f_1, \ldots, f_n \in C_c(X)$ s.t. $f_k \subseteq \1_{U_{k}}$ for $k = 1, \ldots ..., n$ and
	$\1_k \leq f_1 + \ldots+ f_n \leq 1$
\end{lemma}

[insert image]

\begin{proof}
	Step1: for any $x \in K$, there is $K$ and $x \in \underset{open}{V_x} \subseteq \underset{compact}{\bar{V}_x} \subseteq U_x$
	Since $K$ compact, $K \subseteq V_{x_{1}} \cup \ldots \cup V_{x_N}$

	Group $V_{x_j}$ according to $x_j \in U_{k}$. Obtain compact sets
	\begin{align*}
		K_k = \bigcup_{\bar{V}_{x_j} \subseteq U_k} \bar{V}_{x_j} \subseteq U_k
	\end{align*}
	with $K \subseteq K_1 \cup \ldots \cup K_n$


Choose $g_k \in C_{c} (X)$ s.t. $\1_{K_k} \leq g_K \leq \1_{U_{K}}$. We have $g_1 + \ldots + g_n \geq \1_{K}$. We need to limit to $1$.
Let $f_1 = g_1$, $f_{k+1} = (1-g_1) \cdots (1 - g_{k}) g_{k+1} \leq \1_{U_{k+1}}$

\underline{Compute}:
\begin{align*}
\1_K \leq f_1 + \ldots + f_n = 1 - (1 - g_1) \cdots (1 - g_n) \leq 1
\end{align*} 
\end{proof}



