\section{November 16th, 2022}

\subsection{Absolute Continuity of Functions}

\begin{theorem}[Absolute Continuity of Functions]
A $f : [a,b] \to \R$  is \underline{absolutely} \underline{continuous} if and only if for $\varepsilon > 0$, there is $\delta > 0$ so that
	\begin{align*}
		\sum_{k=1} | f (b_k) - f(a_k) | < \varepsilon \quad \text{whenever} \\
		a \leq a_k \leq b_k \leq k_{k+1} \leq b_{k+1} \leq b
	\end{align*} and
	\begin{align*}
		\sum_{k=1}^{n} | b_k - a_k | < \delta
	\end{align*}
\end{theorem}

\begin{lemma}
	Absolutely continuous implies continuous.
\end{lemma}

\begin{remark}
	Note the reverse is not true.
\end{remark}

\begin{example}
	If $C \subseteq [0,1]$ is he $1/3$ Cantor set and
	\begin{align*}
		F(x) = \int \1_{[0,x] \cap C} \, dH^{\frac{\log{2}}{\log{3}}}
	\end{align*} then $F$ is continuous but NOT absolutely continuous.
\end{example}

\begin{theorem}
	$F : [a,b] \to \R$ is absolutely continuous if and only if
	\begin{align*}
		F(x) - F(0) = \int \1_{[0,x]} f \, dL \quad \text{for some } f \in L^{1} ([0,1])
	\end{align*}
\end{theorem}

\subsection{Three Lemmas}
\epigraph{\textit{After making a critical mistake} \\ ``This is just a sign mistake. It all depends on your character encoding scheme''}{Charlie}
\noindent To prove this we need three lemmas:

\begin{lemma}
	If $f : [a,b] \to \R$ is absolutely continuous and
	\begin{align*}
		\bar{f} (x) = \sup \left\{ \sum_{k=1}^{n} | f (y_k) - f (y_{k-1}) | : a = y_0 \leq y_1 \leq \ldots \leq y_n = x \right\}
	\end{align*}
	Then $f_{\pm} - \frac{1}{2} (\bar{f} \pm f)$ are absolutely continuous and non-decreasing.
\end{lemma}

\begin{notation}
	Call $\bar{f}$ the total variation of $f$.
\end{notation}

\begin{example}
	If $f$ is $C^1$, then
	\[
  	\bar{f}(x) = \int_0^x |f'(t)| \, dt.
	\]
\end{example}

\begin{proof}
	Adding a new point, $\tilde{y}$, to $\{y_{k}\}$, then
	\begin{align*}
		y_0 \leq \ldots \leq y_k \leq \tilde{y} \leq y_{k+1} \leq \ldots \leq y_n
	\end{align*}
	Then adding this point will only increase
	\begin{align*}
		\sum_{k=1}^n | f( y_k) - f(y_{k-1}) |
	\end{align*}
	If $a \leq x \leq y \leq b$, then
	\begin{align*}
		\bar{f} (y) - \bar{f}(x) &= \sup \left\{ \sum_{k=1}^{n} |f (z_k) - f (z_{k - 1}) | : x = z_0 \leq \ldots \leq z_n = y \right\} \\
								 &\geq | f(y) - f(x) |
	\end{align*}
	If $a \leq x \leq y \leq b$ then
	 \begin{align*}
		 f_{\pm} (y) - f_{\pm} (x) &= \bar{f} (y) - \bar{f} (x) \mp | f(y) - f(x) | \\
								   &\geq |f(y) - f(x) | - | f(y) - f(x)| \\
								   &= 0
	\end{align*}
	Fix $\varepsilon > 0$, since $f$ is absolutely continuous, there is a $\delta > 0$ so
	\begin{align*}
		\sum_{k=1}^n | f (b_{k}) - f (a_k) | < \varepsilon/2
	\end{align*} when
	\begin{align*}
		\sum_{k=1}^n (b_k - a_k) < \delta \quad \text{and} \quad
		a \leq a_k \leq b_k \leq a_{k+1} \leq b_{k+1} \leq b.
	\end{align*}
	Now suppose $a \leq a_k \leq b_k \leq a_{k+1} \leq b_{k+1} \leq b$
	and
	\begin{align*}
		\sum_{k+1}^n (b_k - a_k) < \delta
	\end{align*}
	For each $k$, choose
	\begin{align*}
		a_k = y_{k,1} \leq \ldots \leq y_{k, m} = b_k
	\end{align*} so that
	\begin{align*}
		0 \leq \bar{f}(b_k) - \bar{f} (a_k) \leq \frac{\varepsilon}{2^{n}} + \sum_{j=1}^m |f(y_{k,j}) - f(y_{k, j-1}) |
	\end{align*}
	since
	\[
		\sum_{k,j} |y_{k,j} - y_{k,j+2} | < \delta \quad \text{and} \quad
		(y_{k, j-1}, y_{k,j}) \text{ disjoint.}
	\]

	\noindent Now by absolute continuity:
	\begin{align*}
		\sum_{k=1}^{n} | \bar{f}(b_k) - \bar{f}(a_k) | &\leq \frac{\varepsilon}{2} + \sum_{k,j} |f (y_{k,j}) - f(y_{k,j-1}) | \\
													   &\leq \frac{\varepsilon}{2} + \frac{\varepsilon}{2}
	\end{align*}

\end{proof}

\begin{example}[Counter-example]
	Consider $f(x) = x^{2} \sin(\frac{1}{x^2})$ and let $f(0) = 0$.
	Observe $f$ is everywhere differentiable and
	\[
		f'(x) = 2x \sin(\frac{1}{x^{2}}) - \frac{2}{x} \cos(\frac{1}{x^2})
	\] (the second term dominates near zero).

	For $x \neq 0$ and $f'(0) = 0$
	\[
		\int | f'(x) | \1_{[-1, 1]} \, dL = \infty
	\]
	it's impossible to write $f = f_{+} - f_{-}$ with $f_{\pm}$ continuous and non-decreasing.
\end{example}


\begin{lemma}
	If $f : [a,b] \to \R$ is absolutely continuous and non-decreasing, then
	\begin{align*}
		F(x) - F(0) = \int \1_{[0,x)} f \, dL \quad \text{where} \quad f \in L^1 ([0,1])
	\end{align*}
\end{lemma}

\begin{proof}
	Assume $[a,b] = [0,1]$.
	Let
	\[
		G(x) = \frac{F(x) - F(0) + x}{F(1) - F(0) + 1}.
	\]
	Observe $G$ is absolutely continuous and $G$ is an increasing bijection $G: [0,1] \to [0,1]$ (check).

	It follows that $G^{-1}$ is continuous and increasing and $A \subseteq [0,1]$ is Borel if and only if $G(A)$, is Borel.
	Define a Borel measure $m$ on $[0,1]$ by
	\[
		m(A) = L(G(A)).
	\]
	Since $G$ is a bijection, $m$ is a measure. (in particular, $m$ is countably additive).

	\begin{claim}
		$m \ll L$
	\end{claim}

	Suppose $L(A) = 0$, then choose $U \contains A$ open with $L(U) < \delta$.
	$U$ is union of disjoint intervals, since we are working on $\R.$
	Since $G$ is absolutely continuous
	\[
		L(G(U)) < \varepsilon \delta
	\] where $\varepsilon \delta \to 0$ as $\delta \to 0$.
	So
	\[
		m(A) = L(G(A)) \leq L(G(U)) \leq \varepsilon \delta \to 0
	\] as $\delta \to 0$ and so $m(A) = 0$.

	By Radon-Nikodym,
	\begin{align*}
		m(A) &= \int \1_{A} g \, dL  \quad \text{with} \quad g \in L^1 ([0,1]) \\
		\int \1_{[0,x]} g \, dL &= m([0,x]) = L(G([0,x])) = L([G(0), G(x)]) = G(x)
	\end{align*}
	Let
	\[
		f = (F(1) - F(0) + 1 ) g - 1.
	\]then
	\[
		f \in L^1([0,1])
	\] and
	\[
		\int \1_{[0,x]} f \, dL = F(x) - F(0)
	\]
\end{proof}

\begin{lemma}
	If $m_1, m_2 : S \to [0, \infty)$ measures and
	 $m_1 \ll m_2$, then for $\varepsilon > 0$, there is $\delta > 0$ so
	 \[
	 	m_2(A) < \delta \implies m_1 (A) < \varepsilon
	 \]
\end{lemma}

\begin{proof}
	Suppose for contradiction that the conclusion fails,
	then there is $\varepsilon > 0 $ and $A_n \in S$ with
	\[
		m_1 (A_{n}) \geq \varepsilon \quad \text{and} \quad m_2 (A_n) \leq 2^{-n}
	\]
	Let
	\[
		B = \bigcap_{n \geq 1} \bigcup_{m \geq n} A_m
	\]

	\noindent Compute:
	\[
		m(B) = \lim_{n \to \infty} m_1 \bigg( \bigcup_{m \geq n} A_m \bigg) \geq \varepsilon
	\] and
	\begin{align*}
		m_2 (B) &= \lim_{n \to \infty} m_2 \bigg(\bigcup_{m \geq n} A_M \bigg) \\
				&= \lim_{n \to \infty} 2^{1 - n}
	\end{align*} so $m_1 \not \ll m_2 = 0$.
\end{proof}

\subsection{Proof of Theorem}

\noindent Finally we can proceed to the proof of the original theorem:

\begin{proof} We follow several steps:
	\begin{steps}
		\item
			Suppose $F : [a,b] \to \R$ is absolutely continuous.

			\noindent By Lemma 1, $F = f_{+} - F_{-}$ where
			$F_{\pm} : [a,b] \to \R$ are absolutely continuous and non-decreasing.

			\noindent By Lemma 2,
			\[
				F_{\pm}(x) - F_{\pm} (0) = \int \1_{[0,x]} f_{\pm} \, dL \quad \text{with} \quad f_{\pm} \in L^1([0,1]).
			\]
			Now,
			\[
				F(x) - F(0) = \int \1_{[0,x]} (F_{+} - f_{-} ) \, dL
			\]
			and
			\[
				f_{+} - f_{-} \in L^1([0,1]).
			\]

		\item
			Suppose
			\[
					F(x) - F(0) = \int \1_{[0,x]} f \, dL \quad \text{with} \quad f \in L^1([0,1]).
			\]
			Let $f_{\pm} = \max\{0, \pm f\}$.
			Define $m_{\pm}$ by $dm_{\pm} = f_{\pm} \, dL$.
			Since $m_{\pm} \ll L$, given $\varepsilon > 0,$ there is $\delta$ so
			\[
				L(A) < \delta \implies m_{\pm}(A) < \varepsilon.
			\]
			This implies
			\[
				F_{\pm}(x) = \int \1_{[0,1]} f_{\pm} \, dL
			\]
			are absolutely continuous.

			If
			\begin{align*}
				\sum (b_k - a_k) < \delta \quad\text{and} \quad
				a_{k} \leq b_k \leq a_{k+1} \leq b_{k+1}
			\end{align*}
			then
			 \[
				\sum | f_{\pm}(b_k) - F_{\pm}(a_k) | = m_{\pm} ( \bigcup_{k} (a_k, b_k) < \varepsilon
			\]
			and since $F(x) - F(0) = F^{+} (x) - F^{-} (x)$,
			$F$ is also absolutely continuous.
	\end{steps}

\end{proof}

Next time we will cover product measures and Fubini's Theorem.

We finished lecture by discussing Midterm 2.
