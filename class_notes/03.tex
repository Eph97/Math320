\section{2022-09-07}

\subsection{Riemann Integration and Measures}

Readings: Chapter 1 of Rudin.

sketched Lebesgue integration cartoon.

We want to be able to "measure" all of the "constructable subsets of $\R^d$

\begin{enumerate}
	\item constructable? Rectangles and unions and intersections of rectangles.
	\item so staart with all rectangles.
	\item If $A_1, A_2, \ldots$ constructable, then
		$\bigcup_{k=1}^{\infty}A_k$ and $\intersection_{k=1}^{\infty}A_k$ constructable.
	\item Lebesgue measure assigns volume to all such sets.
\end{enumerate}
 
\subsection{Metric Spaces}

A function $d : X \cross X \to \mathbb{R}$ on he cartesion product of a set  $X$ us a metric if

\begin{enumerate}
	\item $d(x,y) \geq 0$
	\item $d(x,y) = 0 \iff x = y$
	\item $d(x,y) = d(y,x)$
	\item $d(x,y) \leq d(x,y) + d(y,z)$
\end{enumerate}

We call the pair $(X, d)$ a \underline{metric space.}

\begin{definition}
A function $f : X \to Y$ between metric spaces is continuous if, $\forall x \in X$ and $\varepsilon >0 $ there exists a $\delta >0$ such that for all $y \in X$ we have $d_X (x, y) < \delta \implies d_Y(f(x), f(y)) < \varepsilon$	
\end{definition}

A subset $V \subseteq X$ of a metric space $X$ is open if for ever $x \in V$ there is a $\delta >0 $ such that $B(x,\delta) = \{y \in X: d(x,y) < \delta\} \subseteq U$ 

\begin{theorem}
	A map $f:X \to Y$ between metric spaces is continuous iff  $w \subseteq Y$ open implies $f^{-1}(w) \subseteq X$ is also open.
\end{theorem}

\begin{proof}
	(exercise)
\end{proof}

definition fun! \\
\begin{definition}:
\begin{enumerate}
	\item A \emph{collection} is a set of subsets. 
	\item A \emph{family} of collections is a set of collections.
\end{enumerate}
\end{definition}

\begin{definition}
	A collection $T$ of subsets of a set $X$ is a topology if:
	\begin{enumerate}
		\item $\emptyset, X \in T$ 
		\item $T^{'} \subseteq T$ is finite, then $\bigcap\limits_{V \in T^{'}} V \in T$ 
		\item If $T^{'} \subseteq T,$ then $\bigcup\limits_{V \in T^{'}} V \in T$
	\end{enumerate}

	We call the pair $(X, T)$ a topological space.
\end{definition}


A function $f:X \to Y$ between topological spaces is continuous if $V \in T_y \implies f^{-1}(V) \in T_X$

\begin{claim*}
	A topology is a way of designating some subsets as open.	
\end{claim*}

\subsection{Sigma Algebras}

\begin{definition}[Algebra]
	A Collection $A$ of subsets of a set $X$ is an \emph{algebra} if 
	\begin{enumerate}
		\item $\emptyset, X \in A$
		\item If $B_1, B_2 \in A,$ then
			\begin{enumerate}
				\item $B_1 \cap B_2 \in A$
				\item  $B_1 \cup B_2 \in A$
				\item  $B_1^{c} = X \setminus B_1 \in A$
			\end{enumerate}
	\end{enumerate}
\end{definition}

\begin{definition}[Sigma Algebra]
	A Collection $S$ of subsets of a set $X$ is a \emph{$\sigma$-algebra} if 
	\begin{enumerate}
		\item $\emptyset, X \in S$
		\item If $B\in S \implies B^{c} \in S$
		\item If $S^{'} \subseteq S$ is countable then 
			$\bigcup\limits_{B \in S^{'}} B \in S$ and $\bigcap\limits_{B \in S^{'}} B \in S$
	\end{enumerate}
note: $\bigcap\limits_{B \in S^{'}} = (\bigcup\limits_{B \in S^{'}} )^{c}$

We call the pair $(X, S)$ a measurable space.
\end{definition}

\begin{definition}
	A function $f : X \to Y$ between measurable spaces is measurable if $B \in S_Y \implies f^{-1}(B) \in S_X$
\end{definition}

\begin{claim*}
	Not all sigmal algebras are topologies but all important ones are.
\end{claim*}

\begin{example}
	\begin{enumerate}
		\item The power set $\mathcal{P} (X) = \{y : y \subseteq X\}$ is both a topology and a $\sigma$-algebra
		\item $\{\emptyset, X\}$ is both a topology and a $\sigma$-algebra
		\item $\{\emptyset, B, B^{c}m X\}$ is both a topology and a $\sigma$-algebra
	\end{enumerate}
\end{example}

\begin{theorem}
	For a collection $C$ of subsets of $X$ there is a smallest topology $T$ and $\sigma$-algebra $S$ with $T \supseteq C$ and $S \supseteq C$.
	Denote these by $\tau(C)$ and $\sigma(C)$ respectively.
\end{theorem}

\begin{proof}
	Let $\sigma(C) = \bigcap\limits_{S \supseteq C} S$.

	Note that $\mathcal{P}(X) \supseteq C$, the intersection is non-trivial.

	\underline{Check the Axioms}:
	\begin{enumerate}
		\item If $S \supseteq C$ is $\sigma$-algebra, then $\emptyset,X \in S$.

			Therefore  $\emptyset, X \in \bigcap \limits_{S \supseteq C \text{ is a $\sigma$-algebra}}S $
		\item $X \in \sigma(C)$ thus
			 \begin{align*}
			&\implies X \in S \text{ for } S \subseteq C \text{ a } \sigma\text{-algebra} \\
			&\implies X^{c} \in S \text{ for } S \subseteq C \text{ a } \sigma\text{-algebra} \\
			&\implies X^{c} \in \sigma(C)
			\end{align*} 

	\item $S^{'} \subseteq \sigma (C)$ countable
		 \begin{align*}
		&\implies S^{'} \subseteq S \text{ for } S \supseteq C \text{ a $\sigma$-algebra} \\
		&\implies \bigcup_{B \in S^{'}} B \in S \text{ for } S \supseteq C \text{ a $\sigma$-algebra} \\
		&\implies \bigcup_{B \in S^{'}} B \in \sigma(C)
		\end{align*} 
	\end{enumerate}

	$\sigma(C)$ is the smallest because it is contained in every $\sigma$-algebra $S \supseteq C$.

	 $\tau(C)$ is defined similarly.
\end{proof}

\begin{definition}
	We call a function $f:X \to Y$ from a measurable space to a topological space measurable if $W \in T_Y$ implies $f^{-1}(W) \in S_X$
\end{definition}

\begin{theorem}
	This is equivaet to $f$ being a measurable function from $(X, S_X)$ to $(S, \sigma(T_Y))$
\end{theorem}

\begin{example}
	If $(X,T)$ is a topological space, then call elements of $\sigma(T)$ the \emph{Borel} subsets of $(X,T)$	
\end{example}

The Borel subsets of $\mathbb{R}^d$ and the "constructable" sets from before.

\begin{theorem}
	if $(Y, S_Y)$ is a measurable space and $f: X \to Y$ then there is a smallest $\sigma$-algebra on $X$ that makes $f$ measurable.
\end{theorem}
\begin{proof}
	Check $S_X = \{f^{-1}(B) : B \in S_Y\}$ is a $\sigma$-algebra.
\end{proof}

\begin{theorem}
	If $(X, S_X)$ is a measurable space and $f:X \to Y$ then there is a largest $\sigma$-algebra on $Y$ that makes $f$ measurable.
\end{theorem}

\begin{proof}
	Check $S_Y = \{B \subseteq Y: f^{-1}(B) \in S_X \}$ is a $\sigma$-algebra.
\end{proof}

\begin{definition}
	Basic Borel sets are open or closed.
	\begin{enumerate}
		\item $B_0$ sets are open or closed subsets of $(X,T)$
		\item $B_{K+1}$ sets of countable unions of countable intersections of sets in $B_k$
	\end{enumerate}
\end{definition}

\begin{example}
	$E \in B_4$ then

	 \begin{align*}
		 &\bigcap_i \bigcup_j \bigcap_k F_{i,j,k} \quad \text{\tiny (F open)} \qquad \text{Or} \\
		 &\bigcup_i \bigcap_j \bigcup_k G_{i,j,k} \quad \text{\tiny (G closed)}
	\end{align*}
	
\end{example}

