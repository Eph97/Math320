\section{2022-11-30}

\subsection{Completness of Measures}

We arrived at the the lebesgue integral by extending Riemann integral so:

If $f_n$ integrable and $f_n \geq f_{n+1} \geq 0$, and $f = \lim_{n\to \infty} f_n$ integrable then $\int f = \lim_{n \to \infty} f_n$

\begin{example}
	$C = \cap C_n$.
	 \[
		 \int_{[0,1]} \1_{C_n} \to \frac{1}{2} \quad \text{as } n \to \infty.
	\]
	but $1_{C}$ not Riemann integrable.
\end{example}

However, we also get an additional completeness property for free from construction:


(**) If $f : \R^d \to [0,\infty]$ lebesgue measurable, $\int f dL = 0$ and $0 \leq g \leq f$, then $g$ lebesgue measurable and $\int g dL = 0$


 \begin{lemma}
	If $A \subseteq \R^d$ lebesgue measurable, $m(A) = 0$ and $B \subseteq  A$, then $B$ is lebesgue measurable.
\end{lemma}

\begin{proof}
	Recall $A \subseteq \R^d$ is lebesgue measurable if and only if
	\[
		L^*(E) = L^**E \cap A) + L^* (E \cap A^c) \quad \text{for all } E \subseteq \R^d
	\]
	If $B \subseteq A$ and $L(A) = 0$, then 
	\begin{align*}
		L^*(E) &\leq L^*(E \cap B) + L^*(E \cap B^c) \\
			   &\leq L^*(E \cap B) + L^*(E \cap A^c) + L^*(E \cap A \cap B^c) \\
			   &\leq 2 L^*(A) + L^*(E \cap A^c) \\
			   &\leq L^*(E \cap A^c) + L^*(E \cap A) \\
			   &= L^*(E)
	\end{align*} 
\end{proof}

\begin{remark}
	If we restrict $L$ to Borel sets $\B(\R^d)$ then we lose property (**)
\end{remark}

\begin{example}
	The $\frac{1}{2}$- cantor set $C$ contains non-borel sets.
	Consider your favorite staircase function.

	\[
		f(x) = \int \1_{[0,1]}\1_{C} dH^{\frac{\log{2}}{\log{3}}}
	\] 
	is continuous and non-decreasing and $f(c) = [0,1]$
\end{example}

Choose $A \subseteq [0,1]$ that is not Borel. (E.g. the vitali set which is not Lebesgue measurable).

$f^{-1} (A)$ is not Borel, since image of Borel set through continuous non-decreasing function is Borel.


\begin{remark}
	A general measure space $(X, S, m)$ will always satisfy (*) but may not satisfy  (**).
\end{remark}

\begin{theorem}
	Suppose $(X,S, m)$ measure space, let $S^*$ be subset $E \subseteq X$ for which there are $A \subseteq E \subseteq B$.
	wich $A,B \in S$ and $m(B \setminus A) = 0$ and we define  $m^*(E) = m(A) = m(B)$, then we have $(X,S^*, m^*)$ complete.
\end{theorem}

\begin{remark}
	This is the smallest possible completion.
\end{remark}

\begin{proof}
	$S^* \contains S$. If $E \in S^*$, then choose $A \subseteq E \subseteq B$ with $A,B \in S$ and $m(B \setminus A) = 0$
\end{proof}

\begin{theorem}
	Suppose $(X,s, m)$ measure space, let $S^*$ be subset $E \subseteq X$ for which there are $A \subseteq E \subseteq B$ with 
	$A, B \in S$ and $m(B \setminus A) = 0$ and we define  $m^*(E) = m(A) = m(B)$, then have $(X,S^*, m^*)$ complete.
\end{theorem}

\begin{remark}
	This is the smallest possible completion
\end{remark}

\begin{proof}
	$S^* \contains S_0$ if $E \in S^*$, then choose $A \subseteq E \subseteq B$ with $A,B \in S$ and  $m(B \setminus A) = 0$.

	Observe  $B^c \subseteq E^c \subseteq A^c$ and $A^c \setminus B^c = B \setminus A_0$. So
	 $E^c \in S^*$. If $E_k \in S^*$ and $A_k \subseteq E \subseteq B_k$ with $A_k, B_k \in S$ and $m(B_k \setminus A_k) = 0$, 
	 then
	 \[
		 A = \bigcup_{k} A_k \subseteq 
		 E = \bigcup_{k} E_k \subseteq 
		 B = \bigcup_{k} B_k
	 \] 
	 $A,B \in S$, and
	 \begin{align*}
		 m(B \setminus A) \leq \sum_{k} m(B_{k} \setminus A) \leq \sum_{k} m(B_{k} \setminus A_k) = 0
	 \end{align*}
\end{proof}

Proof that $m^*$ 

\begin{corollary}
	$L$ is completion of its restriction to Borel $\sigma$-algebra.
\end{corollary}

\begin{theorem}
	If $m : \mathcal{P}(X) \to [0,\infty]$ is an outer measure, then $(X,S_m, m)$ is complete.
\end{theorem}

\begin{proof}
	Same proof as lemma for lebesgue measure from Before.

	Completition is (muildly) incompatible with products.
\end{proof}

\begin{example}
	Suppose $(X_1, S_1, m_1)$ and  $(X_2, S_2, m_2)$ are complete, 
	$(X_1 \times X_2, S_1 \times S_2, m_1 \times m_2)$ usually \underline{not} compute.
\end{example}

Suppose $A_1 \in S$, $A_1 \neq \emptyset$,  $m_1(A_1) = 0$,  $A_2 \subseteq X_2$, $A_2 \notin S_2$
then
\begin{align*}
	A_1 \times A_2 \notin S_1 \times S_2
\end{align*}

\begin{align*}
	A_1 \times A_2 \subseteq A_1 \times X_2, \quad A_1 \times A_2 \in S_1 \times S_2, \quad \text{and} \\
	(m_1 \times m_2) (A_1 \times X_2) = 0
\end{align*}

\begin{lemma}
	If $X_1$ and $X_2$ are topological spaces, then $\B(X_1 \times X_2) = \B(X_1) \times \B(X_2)$
\end{lemma}

\begin{proof}
	The topology on $X_1 \times X_2$ is generated by $U_1 \times U_2$ for $U_1 \subseteq X_1$ and $U_2 \subseteq X_2$ open.

	$\B(X)$ generated by open $U \subseteq X$, $\B(X_1) \times \B(X_2)$ generated by $A_1 \times A_1$ for 
	$A_1 \in \B(X_1)$ and $A_2 \B(X_2)$

	Combining these, obtain
	 \[
		\B(X_1 \times X_2) \subseteq \B(X_1) \times \B(X_2)
	\] 
	Consider $\{U_1 \times X_2 : U_1 \subseteq X_1 \text{ open} \}$
	and observe it generates 
	\[
		\{A_1 \times X_2 : A_1 \subseteq X_1 \text{ Borel } \}
	\] 

	\underline{Similar}: $\{ X_1 \times U_2 : U_2 \subseteq X_2 \text{ open }\}$ 
	generates $\{ X_1 \times A_2 : A_2 \subseteq X_2 \text{Borel} \}$

	\[
		A_1 \times A_2 = (A_1 \times X_2 ) \cap (X_1 \times A_2)
	\] So
	\[
		\B(X_1 \times X_2 ) \contains \B(X_1) \times \B(X_2)
	\] 
\end{proof}

\begin{corollary}
	$L^{d_1 + d_2}$ is the completion of 
	
\end{corollary}


