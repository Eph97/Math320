\section{October 31st, 2022}

\epigraph{\textit{angrily pushed a table to make more space}}{Charlie}

\subsection{Chapter 7 -- Differentiation}

From calculus:
\[
  F(t) = \int_{t}^{0} f(s) \,ds \iff F'(t) = f(t)
\]
For continuous $f$. We want an analogy where $m = f \,ds$ is a measure.

\subsection*{Warm-Up}

\begin{theorem}
  For $m$ Borel measure on $\R$, the following are equivalent
  \begin{enumerate}
    \item $f(x) = m((-\infty, x))$ is differentiable at $x_0$ and $f'(x_0) = \alpha$.
    \item For every $\varepsilon > 0$, there is $\delta > 0$ such that
      \[
      a < x_0 < b, \; b - a < \delta \implies \left| \frac{m([a,b))}{L([a,b))} - \alpha \right| < \varepsilon.
      \]
  \end{enumerate}
\end{theorem}

\begin{proof}
  \[
    \frac{m([a,b))}{L([a,b))} = \frac{f(b) - f(a)}{b-a}.
  \]
\end{proof}

\subsection{Covering Lemmas}

\begin{lemma}[Finite Vitali]
  If $\mathcal{B}$ is a finite collection of balls in $\R^d$ and $X \subseteq \bigcup_{B \in \mathcal{B}} B$, then there is $\mathcal{B} \subseteq B$ so
  \begin{enumerate}
    \item balls in $\mathcal{B}'$ disjoint \label{lma:1}
    \item $X \subseteq \bigcup_{B \in \mathcal{B}} 3B$ where $3B(x,r) = B(x, 3r)$. \label{lma:2}
  \end{enumerate}
\end{lemma}

\begin{proof}
  Write $\mathcal{B} = \{B_1, B_2, \ldots, B_n \}$ where $\operatorname{rad}(B_k) \leq \operatorname{rad}(B_{k+1})$. Build $\mathcal{B}'$ by a greedy algorithm.
  \begin{itemize}
    \item Start with $\mathcal{B}' = \{ B_1 \}$.
    \item For $k = 2, \ldots, N$:
      \begin{itemize}
        \item if $B_k$ disjoint from balls of current $\mathcal{B}'$
          \begin{itemize}
            \item Set $\mathcal{B}'$ to $\mathcal{B}' \cup \{ B_k \}$
          \end{itemize}
        \item Condition \ref{lma:1} immediately satisfied
      \end{itemize}
    \item To check \ref{lma:2}, choose $k$ so $x \in B_k$
    \item If $B_k \in \mathcal{B}'$, then $x \in \bigcup_{B \in \mathcal{B}'} 3B$
    \item $B_k \notin B'$, then $B_k \cap B_j \ne \emptyset$ for $j < k$ and $B_j \in \mathcal{B}$.
      \begin{itemize}
        \item Observe since $\operatorname{rad}(B_j) \geq \operatorname{rad}(B_k)$ and $B_j \cap B_k \ne \emptyset$, $B_k \subseteq 3B_j$.
          \[
            w \in B(x, r) \implies |x - y| \leq s + r \leq 3s.
          \]
        \item in particular, $x \in 3B_j \subseteq \bigcup_{B \in \mathcal{B}'} 3B$.
      \end{itemize}
  \end{itemize}
\end{proof}

\begin{lemma}[Infinite Vitali]
  If $\mathcal{B}$ is a collection of balls in $\R^d$ or radius at most 1 and $x \in \bigcup_{B \in \mathcal{B}}$, then there is $\mathcal{B}' \subseteq B$ such that
  \begin{enumerate}
    \item balls in $\mathcal{B}'$ disjoint
    \item $X \subseteq \bigcup_{B \in \mathcal{B}'} 5B$
  \end{enumerate}
\end{lemma}

\begin{proof}
  (formatting can be improved for this section)
  Use upper bound to start a greedy algorithm. Let $B_k = \{ B \in \mathcal{B} : 2^{-k-1} < \operatorname{rad}(B) \leq 2^{-k}$ for $k = 0, 1, \ldots$
    Recursively choose $\mathcal{B}_k' \subseteq \mathcal{B}_k$ to be maximal subject to constraint that balls in $\mathcal{B}_0' \cup \ldots \cup \mathcal{B}_k'$ disjoint.
    Let $B' = \mathcal{B}_0', \mathcal{B}_1', \ldots$


    Again, 1 automatic.
    For 2, suppose $x \in B \in \mathcal{B}_k$

    If $B \in \mathcal{B}'$, then $x \in \bigcup_{B \in \mathcal{B}'} 3B$.
    If $B \notin \mathcal{B}'$, then there is $\tilde{B} \in \mathcal{B}_0', \ldots, \mathcal{B}_k'$ such that $B \cap \tilde{B} \ne \emptyset$.
    Since $\operatorname{rad}(\tilde{B}) \geq \frac{1}{2} \operatorname{rad}(B)$, obtain $x \in B \subseteq 5\tilde{B} \subseteq \bigcup_{B \in \mathcal{B}'} 5B$.
\end{proof}

Both version of Vitali rely on geometric lemma

\begin{lemma}
  If $B(x,r) \cap B(y,s) \ne \emptyset$ and $z \in B(x,r)$, then $|z - y| < s + 2r$.
\end{lemma}

\begin{proof}
  Choose $w \in B(x,r) \cap B(y,s)$ and compute ...
  \begin{align*}
    |z - y| & \leq |z - x| + |x - w| + |w - y| \\
            & < r + r + s
  \end{align*}
\end{proof}

\begin{remark}
  \begin{itemize}
    \item 3 is optimal for finite Vitali
    \item 5 is \underline{not} optimal for infinite Vitali. Any constant $>3$ works. \\
      Replace $2$ with $1 + \varepsilon$ in the proof gives constant $3 + 2\varepsilon$ (portion where radius is $(2^k, 2^{k+1}]$)
    \item Need upper bound for infinite Vitali at least for unbounded $X$. An example:
      \[
        \mathcal{B} = \{B(0,r) : r > 0 \} \quad \text{where } X = R^d
      \]
    \item For $X$ bounded, do not need upper bound in infinite Vitali for(?) $X \subseteq B(0,R)$. May assume $X \cap B \ne \emptyset$ for $B \in \mathcal{B}$.
      If $B \in \mathcal{B}$ with $\operatorname{rad}(B) \geq R$, then $X \subseteq 3B$.
    \item Both Vitali lemmas are true in any metric space.
  \end{itemize}
\end{remark}

\subsection{Maximal Inequality}

\begin{definition}[Maximal Function]
  Suppose $m$ is Borel measure on $\R^d$. The maximal function of $m$ is
  \[
    (Mm)(x) = \sup_{r > 0} \frac{m(B(x,r))}{L(B(x,r))}.
  \]
\end{definition}

\begin{lemma}
  $Mm$ is linear semicontinuous, that is, $(Mm)^{-1} ((a, \infty])$ is open for all $a \in \R$ and is therefore Borel measurable.
\end{lemma}

\begin{proof}
  Suppose $(Mm)(x) > a$,
  Choose $r > 0$ and $\varepsilon > 0$ so
  \[
    \frac{m(B(x,r))}{L(B(x,r))} > a + \varepsilon.
  \]
  For $y \in B(x, \delta)$, compute
  \begin{align*}
    \frac{m(B(x,r+\delta))}{L(B(x,r+\delta))} & \geq \frac{m(B(x,r+\delta))}{L(B(x,r+\delta))} \\
                                              & \geq \frac{m(B(x,r))}{L(B(x,r+\delta))} \\
                                              & \geq \frac{m(B(x,r))}{L(B(x,r))} \frac{r^d}{(r + \delta)^d} \\
                                              & > (a + \varepsilon) \frac{r^d}{(r + \delta)^d} \\
                                              & > \left(a + \frac{\varepsilon}{2}\right) \quad \textrm{for $\delta > 0$ small}
  \end{align*}
  So $B(x, \delta) \subseteq (Mm)^{-1}((a, \infty])$
\end{proof}


\begin{theorem}
  If $m$ Borel measure on $\R^d$ and $\lambda > 0$, then
  \[
    L(\{Mm > \lambda\}) \leq \frac{3^d}{\lambda} m(\R^d).
  \]
\end{theorem}

\begin{proof}
  Assume $m(\R^d) < \infty$. By definition of $Mm$,
  \[
    \{ Mm > \lambda \} \subseteq \bigcup_{B \in \mathcal{B}} B
  \]
  where $\mathcal{B} = \{B : m(B) > \lambda L(B) \}$.
  Choose $K \subseteq\{Mm \geq \lambda\}$ compact. Since $L(A) = \sup_{K \subseteq A} L(K)$, where $K$ is compact, it is enough to bound $L(K)$.
  Choose $\mathcal{B}' \subseteq \mathcal{B}$ finite so $K \subseteq \bigcup_{B \in \mathcal{B}} B$.
  By Vitali, choose $\mathcal{B}'' \subseteq \mathcal{B}$ so balls of $\mathcal{B}'$ disjoint and $K \subseteq \bigcup_{B \in \mathcal{B}} 3B$.
  Compute
  \begin{align*}
    L(K) & \leq \sum_{B \in \mathcal{B}} L(3B) = 3^d \sum_{B \in \mathcal{B}} L(B) \\
         & < 3^d \sum_{B \in \mathcal{B}} \frac{m(B)}{\lambda} = \frac{3^d}{\lambda} m \left(\bigcup_{B \in \mathcal{B}''} B\right) \\
         & \leq \frac{3^d}{\lambda} m \left( \R^d \right)
  \end{align*}
\end{proof}

\subsection{Lebesgue Differentiation}

\begin{corollary}
  If $f \in L^1 (\R^d)$ and $(Mf)(x) = \sup_{r > 0} \frac{\int_{B(x,r)} f \, dL}{L(B(x,r))}$, then
  \[
    L(\{Mf > \lambda \}) \leq \frac{3^d}{\lambda} \| f \| L^1(\R^d).
  \]
\end{corollary}

\begin{proof}
  Apply maximal inequality to measure
  \[
    m(A) = \int \1_A |f| \, dL.
  \]
\end{proof}

\begin{theorem}[Lebesgue Differentiation]
  If $f \in L^1(\R^d)$, then
  \[
    \lim_{r \to 0} \frac{\int_{B(x,r)}|f - f(x)| \, dL}{L(B(x,r))} = 0
  \]
  for $L$-almost every $x$.
\end{theorem}

\begin{proof} \phantom{=} \hfill
  \begin{steps}
    \item If $f$ is continuous, then
      \[
        \frac{\int_{B(x,r)}|f - f(x)| \, dL}{L(B(x,r))} \leq \sup_{y \in B(x,r)} |f(y) - f(x)| \to 0 \quad \textrm{as } r \to 0
      \]
    \item In general, fix $\varepsilon > 0$ and choose $g \in C_c (\R^d)$ section
      \[
        \| f - g \| L^1(\R^d) < \varepsilon
      \]
      by the theorem from the previous lecture.
  \end{steps}
  Compute:
  \begin{align*}
    F & = \limsup_{r \to 0} \frac{\int_{B(x,r)}|f - f(x)| \, dL}{L(B(x,r))} \\
      & = \limsup_{r \to 0} \frac{\int_{B(x,r)}|h - h(x)| \, dL}{L(B(x,r))} \quad \textrm{where } h = f - g \\
      & = |h(x)| + \limsup_{r \to 0} \frac{\int_{B(x,r)}|h| \, dL}{L(B(x,r))} \\
      & = |h(x)| + Mh(x)
  \end{align*}
  \begin{align*}
    L(\{F > \lambda \}) & \leq L(\{ |h| + Mh > \lambda \}) \\
                        & \leq L(\{ |h| > \frac{\lambda}{2} \}) + L(\{ Mh > \frac{\lambda}{2} \}) \\
                        & \frac{2 \cdot 3^d \cdot \varepsilon}{\lambda}.
  \end{align*}
  Send $\varepsilon \to 0$ to obtain
  \[
    L(\{F < \lambda\}) = 0.
  \]
  Now send $\lambda \to 0$.
\end{proof}




