\section{2022-12-06}

\subsection{Applications in Dynamics}

\begin{definition}[Arnold Cat Map]
	$T : \R^2 / \Z^2 \to \R^2 / \Z^2$ (tarus) \\
	given by
	\[
		T(x,y) = (2x + y, x+y) \mod{\Z^2}
	\] 
\end{definition}

suppose $x_0 \in \R^2 / \Z^2$ and $T_{k+1} = T x_{k}$. How are $\{x_k\}$ distributed?


First we need some new concepts

\begin{definition}[Convergence in Measures]
	Let $K$ be a compact metric space.
	\[
		\mathcal{M} = \{ m \text{ Borel measure on $K$ with } m(K) = 1 \}
	\] 
\end{definition}

\begin{theorem}
	For every sequence of measure $\{m K \} \subseteq \mathcal{M}$ there is a subsequence
	$\{ \tilde{m}_{j}\}$ of $\{ m_k\}$ and $\bar{m} \in \mathcal{M}$ so for every $f \in C(K)$
	 \[
		 \lim_{j \to \infty} \int f d\tilde{m}_{j} = \int f d \bar{m}
	\] 
\end{theorem}

\begin{proof}
	Choose $\{ f_l : l \in \mathbb{N} \} \subseteq C(K)$ dense in $C^0$ norm.

	For each $l$, $k \mapsto \int f_{l} dm_k$ is a bounded subsequence in $\R$ 

	Diagonalize to choose subsequence $\{\tilde{m}_{j} \}$ of $\{m_k \}$ so
	\[
		\bar{m}_{l} = \lim_{j \to \infty} \int f_{l} d \bar{m}_j \text{ exists } \forall l
	\] 
\end{proof}

\begin{exercise}
	Since $\{f_l \}$ dense, the map $f_{l} \mapsto \bar{m}_l$ extends to positive linear $\Gamma : C(K) \to \R$

	By Riesz,  $\Gamma f = \int f d \bar{m}$ for unique Borel measure $\bar{m}$. Conclude using density of $\{f_l\}$
\end{exercise}

\begin{remark}
	If $\{ f_l \} \subseteq C(K)$ dense, then every Borel measure $m$ determined by real numbers 
	$\{ \int f_l dm\}$
\end{remark}

\begin{definition}[Weak Convergence]
	$\{m_k \} \subseteq \mathcal{M}$ converges weakly to $m \in \mathcal{M}$ if and only if 
	\[
		\lim_{k \to \infty} \int f dm_k = \int f dm \forall f \in C(K)
	\] 
	weak $m_k \xrightarrow[]{*} m$.
\end{definition}

\begin{theorem}
	Suppose $\tau$ is coarsest topology of $\mathcal{M}$. So, for all $f \in C(K)$  $m \mapsto \int f dm$ is continuous
	 $(\mathcal{M}, \tau)$ is compact and metrizable.
\end{theorem}
\begin{proof}
	Let $\{f_l\} \subseteq C(K)$ be dense.
	\[
		d_{\mathcal{M}} (m_1, m_2) = \sum_{k} 2^{-k} \frac{| \int f_k dm_1 - \int f_k dm_2 |}{\| f_k \|_{C^0}}
	\] 

\begin{exercise}
This generates the coarsest topology (By desity of $\{f_l\}$)
\end{exercise}

Previous theorem gives compactness
\end{proof}

\begin{example}
	Let $K = [0,1]$ 
	\begin{align*}
		&dm_{k} = K \1_{[0,\frac{1}{K})} dL \\
		&m_K \xrightarrow[]{*} \delta_0
	\end{align*} 
\end{example}

\begin{example}
	\[
		m_k = \frac{\delta_0 + \delta_{1 - \frac{1}{k}} + \cdots + \delta_{k - \frac{1}{k}}}{k}
	\] 
	$m_k \rightarrow[]{*} L$
\end{example}

\begin{definition}
	A sequence $\{X_k\} \subseteq K$ \underline{equidistributes} to $m \in \mathcal{M}$ if and only if
	 \[
		 \frac{\delta_{x_0} + \cdots + \delta_{x_{n} - 1}}{n} \xrightarrow[]{*} m
	\] 
\end{definition}

\begin{example}
	Suppose $1, \alpha_1, \ldots , \alpha_d \in \R$ are $\Q$-independent
\end{example}

\begin{definition}
	define $T : \R^d / \Z^d \to \R^d/ \Z^d $
\begin{align*}
	T x = x + \alpha \mod{\Z^d}
\end{align*} 
[insert picture]
\begin{claim}
If $X_{k+1} = T X_{k}$, then $\{X_k\}$ equidistributes to  $L$.
\end{claim}

Consider $f \in C(K)$ given by 
\[
	f(x) = e^{2 \pi i \eta \cdot x} \quad \text{for some } \eta \in \Z^d
\] 
\end{definition}

\begin{exercise}
	These are dense in $C(K)$.
\end{exercise}

observe
\[
	\int f dL = 
	\begin{cases}
		1 & \eta = 0 \\
		0 & \eta \neq 0
	\end{cases}
\] 

then observe
\begin{align*}
	&X_k = T^{k} x_0 = (x_0 + K\alpha) \mod{\Z^d} \\
	&f(x_k) = f(x_0) e^{2 \pi i \eta \cdot k \alpha} \\
	&\int f d \frac{\delta_{x_0} + \cdots + \delta_{x_n -1}}{n} \\
	&\frac{1}{n} \sum_{k=0}^{n-1} f(x_k) = \frac{f(x_0)}{n} \sum_{k=0}^{n-1} e^{2\pi i \eta \cdot k\alpha} \quad \text{ assume } \eta \neq 0 \\
	&= \frac{f(x_0)}{n} \frac{e^{2\pi i \eta \cdot n\alpha} -1}{e^{2\pi i \eta \cdot \alpha } -1} \quad \text{ by hypothesis}
\end{align*} 
$\eta \cdot \alpha \notin \Z$ so $e^{2\pi i \eta \cdot \alpha} - 1 \neq 0 \to 0$ as $n \to \infty$.


when $\eta =0$, $f \equiv 1$ and
\[
	\int f d \frac{\delta_{x_0} + \cdots + \delta_{x_n -1}}{n} =1
\] 

\begin{definition}[Invariant measures]
	Suppose $T:K \to K$ continuous map on compact metric space $K$. Call Borel measure $m$ on $K$ $T$-invariant if and only if
	\[
		m(T^{-1}(B)) = m(B) \quad \text{for all } B \subseteq K \text{ Borel}
	\] 
\end{definition}


\begin{example}
	$m$ is $T$-invariant if and only if 
	\[
		\int f dm = \int f \circ T dm \quad \text{for all } f \in C(K)
	\] 
\end{example}

\begin{theorem}
	$M_{T} = \{ m \in \mathcal{M} : m\; T \text{-invariant} \}$ is convex, (weakly) closed, and non-empty.
\end{theorem}

\begin{proof}
	If $m_1, m_2 \in \mathcal{M}_t$ and $t \in [0,1]$, then
	$t m_1 + (1-t) m_2 \in \mathcal{M}_t$
	Follows from definition
	$m_k \in \mathcal{M}_t$ and $m_k \xrightarrow[]{*} m$, then 
	\begin{align*}
		]int f dm \leftarrow[k \to \infty] &\int f dm_k \\
		= &\int f \circ T dm_k \\
		\xrightarrow[k \to \infty] \int f \circ T dm
	\end{align*} 
	and so $m \in \mathcal{M}_t$

	To show $\mathcal{M}_T \neq 0$, choose $x_0 \in K$ and let $X_{k+1} = T(X_k),$ 

	let  \[
		m_k = \frac{\delta_x + \cdots + \delta_{x_{k} - 1}}{k}
	\] 
	Choose $K_j \to \infty$ so $m_{k_j} \xrightarrow[]{*} m$ for $f \in C(K)$, compute 
	\begin{align*}
		\int f \circ T dm &\leftarrow \int f \circ T dm_{k_j} \\
						  &= \frac{1}{k_j} \sum_{l = 0}^{n-1} f(T^{l+1} x_0 )
						  &= \frac{1}{k_j} \sum_{l}^{k_j - 1} f (T^{l} x_0) + \frac{f (T^{k_j} x_0 ) - f(x_0)}{k_j} \\
						  &= \int f dm_j + \frac{f (T^{k_j} x_0 ) - f(x_0)}{k_j} \\
						  &\rightarrow \int f dm
	\end{align*}
\end{proof}


\[
	\mathcal{M}_T \subseteq \mathcal{M}
\] convex, closed (compact) and non-empty.

\begin{exercise}
	If $T x = x$, then $\mathcal{M} = \mathcal{M}_T$
\end{exercise}

\begin{exercise}
	If $T x = x_0$ for some fixed $x_0 \in K$, then $\mathcal{M}_T = \{ \delta_{x_0} \}$
\end{exercise}


\begin{definition}
	Call $m \in \mathcal{M}_T$ \underline{extremal} if and only if 
	$m = tm_1 + (1-t) m_2$, $m_1, m_2 \in \mathcal{M}_T$ and $t \in (0,1)$, then
	\begin{align}
		m_1 = m_2 = m
	\end{align} 
	(1) By \underline{Krein-Milman} theorem from functional analysis, every element of $\mathcal{M}_T$ is a linear combination of extremal elements.
\end{definition}


\begin{definition}[Ergodic measure]
	Call $m \in \mathcal{M}_T$ \underline{ergodic} if and only if $m(B) \in \{0,1\}$ for all 
	$B \subseteq K$ Borel with $M(B \triangle T^{-1}(B)) = 0$.
\end{definition}

\begin{theorem}
	$m \in \mathcal{M}_T$ extremal if and only if egodic.
\end{theorem}


\begin{proof}
	Suppose $m \in \mathcal{M}_T$ \underline{not} ergodic.
	Choose $B \subseteq K$ borel with $m(B \triangle T^{-1}(B)) = 0$ and 
	\begin{align*}
		0 < &m(B) < 1 \\
		m_1 (A) &= \frac{m(A \cap B)}{ m(B)} \\
		m_2 (A) &= \frac{m(A \cap B^{c})}{ m(B_{c})} \\
	\end{align*}
\end{proof}

\begin{exercise}
	Since $m (B \triangle T^{-1}(B)) = 0$, both $m_1, m_2 \in \mathcal{M}_T$.
\end{exercise}

\begin{remark}
	Not ergodic means can split system $(K, T, m)$ into two independent non-trivial pieces  $(B,T, m|_{B})$ and $(B^c, T, m|_{B^c})$
\end{remark}


suppose $m \in \mathcal{M}_T$ \underline{not} extremal.
\begin{align*}
	m &= t m_1 + (1-t) m_2 \\
	m_1, m_2 &\in \mathcal{M}_T, t \in (0,1), m_1 \neq m \\
	m_1 &<< m \\
	dm_1 &= f_1 dm, \; f_1 \in L^1 (m)
\end{align*}  

\begin{claim}
	$f_1 = f_1 \circ T$ $m$-a.e.
	\begin{align*}
		m_1 (B) &= \int \1_{B} f_1 dm \\
				&= \int (\1_{B} f) \circ T dm \\
				&= \int \1_{T^{-1}(B)} (f \circ T) \circ T dm \\
				= m_1 (T^{-1}(B)) = \int \1_{T^{-1}(B)} f dm
	\end{align*} 
\end{claim}

\underline{cheating assumption} If $T$ has continuous inverse, then every Borel set $B \subseteq K$ can be written
$B = T^{-1}(T(B))$.
Obtain
\[
	\int \1_{B} (f \circ T ) dm = \int \1_{B} f dm
\] 

\begin{exercise}
	Avoid cheating
	\footnote{harder than it sounds so see professor in office hours.}
\end{exercise}

$m_1 \neq m$ implies $\not (f = 1 m\text{-a.e})$ implies $B = f^{-1}([0,1])$ has
$0 < m(B) < 1$ and $m (B \triangle T^{-1}(B)) = 0$
