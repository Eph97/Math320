\section{ 2022-09-28 }


Reviewing: $(X,T)$ nice $\Lambda : C_c (X) \to \mathbb{R}$ positive linear

\begin{align*}
	m(U) &= \sup\{\Lambda(f) : 0 \leq f \leq \1_c\} \\
	m(A) &= \inf\{m(U) : U \contains  A\}
\end{align*} 

\begin{lemma}
	$\mu$, outer measure in  $(X,\mathcal{P}(X))$ and Borel $\subseteq$ Caratheodory.
\end{lemma}

\begin{lemma}
	restric $m$ to Borel $\sigma$-algebra $\mathbb{B}$ so $(X, \mathcal{B}, m)$ is a positive measure space
\end{lemma}

\begin{lemma}
	For $f \in C_c (X)$, $\Lambda(f) = \int f dm$
\end{lemma}

\begin{proof}
	Use definition of $m$ and partition of unity

	Using the linearity of $\Lambda$ and  $\int dm$, we may assume $0 \leq f \leq 1.$

	Construct 
	 \begin{align*}
	    \Lambda (f) = ( \max f^+ ) \Lambda \left(\frac{f^+}{\max f^+}\right) - (\max f^- ) \Lambda \left(\frac{f^-}{max f^-} \right)
	\end{align*} 
	and same for $\int dm$
\end{proof}




\begin{theorem}[Riesz-representation theorem]
	If $(X,T)$ is nice and  $\Lambda : C_c (X) \to \mathbb{R}$ positive linear, then there is a \underline{unique}. positive measure $\mu$ on $(X,\mathcal{B})$ such that  $\Lambda(f) = \int f dm$
\end{theorem}

\begin{lemma}
	If $(X,T)$ nice,  $(X, \mathcal{B}, m)$ posiive measure space, and $m(X) < \infty$, then m is \underline{regular}, that is
	\begin{align}
		m(A) &= \inf\{m(U) : \underset{open}{U} \contains A\} \\
			 &= \sup\{m(K) : \underset{compact}{K} \subseteq A\}
	\end{align} 
\end{lemma}

\begin{proof}
	let $S_{*} = \{A \in S: (1) \text{ holds for $A$} \}$
It is enough to show $S_{*}$ contains open sets and is closed under complement and countable union.

Since $\mathcal{B}$ is the smallest $\sigma$-algebra containing open sets, this implies $S \contains \mathcal{B}$

\begin{enumerate}
	\item[Step 1:] Suppose $U$ open, from monotonicity we have $m(U) = \inf\{m(V) : V \contains U \}$
		write: $U = \bigcup_j K_j$ with  $K_j \subseteq K_{j+1}$ compact.
		\begin{align*}
			m(U) = m(\bigcup_j K_j = \lim_{j \to \infty} m(K_j)
		\end{align*} .

		Conclude
		\begin{align*}
			m(U) = \sup\{m(K) : \underset{compact}{K} \subseteq U \}
		\end{align*}

	\item[Step 2:] Suppose $A \in S$, choose $\underset{open}{U} \contains A \underset{compact}{K}$
		so $U^{c} \subseteq A^{c} \subseteq K^{c}$ and $m(U \setminus K) = m(U \setminus A) +m (A \setminus K) < \varepsilon$

		Then
		\begin{align*}
			m(\underset{open}{K^{c}} \setminus \underset{}{U^{c}}) < \varepsilon
		\end{align*}
		Choose $\underset{compact}{G} \subseteq K^{c}$ so $m(^{c} \setminus G) < \varepsilon$. Then 
	\begin{align*}
		g \bigcap U^{c} \subseteq A^{c} &\subseteq K^{c} \quad \text{and} \\
	m(K^{c} \setminus (G \cap U^{c})) &\leq m(k^{c} \setminus V^{c}) + m(K^{c} \setminus G) \\
									 &< 2\varepsilon
	\end{align*} 	
\item[Step 3:]
	Finite unions (automatic):
	If \begin{align*}
		K_1 &\subseteq A_1 \subseteq U_1 \\
		K_2 &\subseteq A_2 \subseteq U_2, \quad \text{then} \\
		K_1 \cup K_2 &\subseteq A_1 \cup A_2 \subseteq U_1 \cup U_2
	\end{align*} 
\item[step 4:]
	Suppose $A_j \in S$, wts $\bigcup_j A_j \in S$. By steps 2 and 3, we may assume  $A_j$ disjoint.

	Next choose $K_j \subseteq A_j \subseteq U_j$ with  $m(U_j \setminus K_j) \leq \frac{\varepsilon}{2j}$ 

	Have $\cup_j K_j \subseteq \cup_j A_j \subseteq \cup_j U_j$ and  $m(\cup U_j \setminus \cup_j K_j ) \leq 2 \varepsilon$ 

	Then 
	\begin{align*}
		m( \cup)j U_j \setminus \cup_j K_j) &= \lim_{n \to \infty} m ( \cup_j U_j \setminus \cup K_j) \quad \text{choose finite $n$ so} \\
		K_1 \cup \ldots \cup K_n &\subseteq \cup_j A_j \subseteq \cup_j U_J \quad \text{and} \\
		m(\cup_j U_j &\setminus K_1 \cup \ldots \cup K_n )) \leq 3\varepsilon
	\end{align*} 
\end{enumerate}

\end{proof}

\begin{lemma}
	If $(X, T)$ nice, and $(X,\mathcal{B}, m)$ is a positive measure space, then
	\begin{align*}
		M(A) &= \inf\{m(U) : U \contains A\} \\
			 &= \sup \{m(K) : K \subseteq A \} \quad \text{whenever} \, A \in \mathcal{B},
			 \, A \subseteq U, \, m(U) < \infty
	\end{align*}
\end{lemma}

\begin{proof}
	Restrict to subspace $(U,T)$ and apply previous lemma.
\end{proof}

It remains to show the uniqueness of the Riesz measure: suppose $m_1$ and $m_2$ are positive measures on $(X,\mathcal{B})$ and $\int f dm_1 = \int f dm_2$ for some  $f \in C_c(X)$.


We want to show that it must be  $m_1 = m_2$

\subsection*{Midterm 1 Review}
Know basic definitions, Basix examples and Basic theorems for:
\begin{enumerate}
	\item Riemann int
\end{enumerate}
