\section{September 19, 2022}

\epigraph{``Fill In Quote''}{Charlie}

\underline{Last Time}:

$(X,S, m)$ positive measure space $f : X \to [0,\infty]$ measurable, then
 \begin{align*}
\int f dm = \sup_{g \leq f} \int g dm
\end{align*}


\begin{theorem}[Montone Convergence]
	If $f_n(x) \leq f_{n+1}(x) \to f(x)$ for ever $x$, then
	\begin{align*}
		\int f_n \xrightarrow[n \to \infty]{} \int f dm
	\end{align*}
\end{theorem}

\begin{corollary}[New Measure from Old]
	If we have $f: X \to [0,\infty]$ and $\tilde{m} : S \to [0,\infty]$ given by $\tilde{m} = \int \1_A fdm$, then $\tilde{m}$ is a measure.

	\underline{Moreover}
	\begin{align*}
		\int g d\tilde{m} = \int g f dm
	\end{align*} for all $g :X \to [0,\infty]$ measurable.

\end{corollary}

\begin{notation}
	$d \tilde{m} = f dm$ (equality in distributional sense)
\end{notation}

\begin{proof}
	Check countably additivity:
	Suppose $A_1, A_2, \ldots \in S$ disjoint, let  $A = \bigcup_{k=1}^{\infty} A_k$, compute
	\begin{align*}
		\tilde{m}(A) &= \int \1_{A} f dm
					 = \int \sum_{k=1}^{\infty} \1_{A_k} f dm \\
					 &= \sum_{k=1}^{\infty} \int \1_{A_k} f dm \quad \text{By Monotone Convergence} \\
					 &=  \sum_{k=1}^{\infty} \tilde{m}( A_k )
	\end{align*}
\end{proof}

\begin{definition}
	If $f: X \to [ - \infty, \infty]$ measurable, then  $\int f dm = f_{+} dm - \int f_{-} dm$ where
	$f_{\pm} = \max\{0, \pm f\}$
\begin{remark}
	$\int f dm$ defined only when $\int |f| dm < \infty$. Moreover,  $|\int f dm | \leq \int |f| dm$
\end{remark}
\end{definition}

Let $L^{1}(X,S, m)$ denote measurable $f: X \to [-\infty, \infty]$ with $\int | f| dm < \infty$.

We want $\|f\|_{L^{1}(X,S, m)} = \int |f| dm$ to be a norm.

Problem: If $m(\{x : f(x) \neq 0 \}) = 0$, then $\int |f| dm = 0$

\begin{remark}
	Notation: Identify a property $\mathbb{P}(x)$ ofa point $x \in X$ with the set $\{x \in X : \mathbb{P}(x)\}$
\end{remark}

For example, the set of $X$ where two functions  $f,g : X \to [-\infty, \infty]$ agree.

A property $\mathbb{P}$ holds $m$- almost everywhere if $m(\mathbb{P}^{c}) = 0$

\begin{example}
	If $f : X \to (-\infty, \infty)$ is measurable and $f$ vanishes almost everywere, then $\int f dm = 0$
\end{example}


For $f,g \in L^{1}(X,S, m)$, write $f \sim g \iff f=g$ (almost everywhere)

Observe: $\sim$ is an equivalence relation and $L^{1}(X,S, m) / \sim$ is a normed vector space.

\begin{observation}
  $\sim$ is an equivalence relation and $L^{1}(X,S, m) / \sim$ is a normed vector space.
\end{observation}

\subsection{Dominated Convergence}

\begin{definition}[Dominated Convergence]
	If
	\begin{align*}
		f_n : X &\to [-\infty, \infty] \text{ measurable }, \\
		g : X &\to [0, \infty] \text{ measurable }, \\
		\int g dm &< \infty
	\end{align*}
	suppose $f(x) = \lim_{n \to \infty} f_n(x)\, \forall x \in X$, and $|f_n (x) | \leq g(x)$ so
	$x \in X$ and $n \geq 0,$

	then  $\lim_{n\to \infty} \int | f - f_n | dm = 0$ and \\
	$\lim_{n\to \infty} \int f_n dm = \int f dm$

\end{definition}

\begin{example}
	If $f_n : [0,1] \to [0, \infty]$ is defined by
	\begin{align*}
		f_n (x)
		\begin{cases}
			2^n & 2^{-n-1} < x < 2^{-n} \\
			0 & otherwise
		\end{cases}
	\end{align*}
	then $lim_{n \to \infty} f_n (x) = 0, \, \forall x \in [0,1]$ and  $\int f_{n} dx = \frac{1}{2} ,\, \forall n \geq 0$
	insert image
\end{example}

We will see later that the $f_{n}$ are converging (in some sense) to $\frac{1}{2}$Dirac measure at $0$. The dominating $g$ prevents this behaviour.

\begin{proof}
	\begin{align*}
		\int 2g dm &= \int \left( \lim_{n \to \infty} 2g - | f - f_{n}| \right) dm \\
				   &\leq \liminf_{n \to \infty} \int 2g - |f - f_{n}|dm \quad \text{ (By Fatous Lemma)} \\
				   &= \int 2g dm - \limsup_{n \to \infty} \int | f - f_{n}| dm
	\end{align*}
	Since, $\int 2gdm < \infty$, conclude
	\begin{align*}
		&\limsup{n \to \infty} \int |f - f_n | dm = 0 \\ \text{finally \quad}
		& \left| \int f_n dm - \int f dm \right| \leq \int | f - f_n | dm
	\end{align*}
\end{proof}

\begin{corollary}
	If $f_{n} : X \to [-\infty, \infty]$ measurable and $\sum_{n=1}^{\infty} | f_n | dm < \infty$,
	then $f(x) = \sum_{n=1}^{\infty}f_n (x)$ exists for almost every $x \in X$ and $\int f dm = \sum_{n=1}^{\infty} \int f_{n}dm$
\end{corollary}

\begin{proof}
	Use dominated convergence. Let $g = \sum_{n=1}^{\infty} | f_n | : X \to [0,\infty]$.
	By hypothesis and monotone convergence, $\int gdm < \infty$

	Since $\int g dm < \infty$ and $g < \infty$ almost everywhere, if $x \in X$ and $g(x) < \infty$,
	then $\sum_{n=1}^{\infty} f_{n} (x)$ is absolutely convergent. That is $g(x) < \infty \implies f(x) = \sum_{n=1}^{\infty} f_{n}(x)$ converges.

	The partial sums $F_n (x) = \sum_{k=1}^{\infty} f_{n}(x)$ have $F_n \to f$ almost everywhere and $|F_{n}| \leq |g|$ a.e.

	By dominated convergence,
	\begin{align*}
	\sum_{k=1}^{\infty}f_{k} dm = \lim_{n \to \infty} \int F_n dm = \int f dm
	\end{align*}
\end{proof}

\begin{corollary}
	$L^{1}(X,S, m) / \sim$ is a \underline{complete} normed vector space.
\end{corollary}

\begin{proof}
	Suppose $\{f_n\} \subseteq L^{1}(X,S, m) / \sim$ is cauchy.

	Refining the sequence, we may assume
	\begin{align*}
	&\int_{\infty}|f_{n+1} - f_{n} | dm < 2^{-1} \\ \text{s.t. \quad}
		\sum_{n=1}^{\infty} &\int | f_{n+1} - f_{n} | dm < \infty
	\end{align*}
	Apply corollary to conclude $f(x) = \lim_{n \to \infty} f_{n}(x)$ converges for almost every $x$ and
	\begin{align*}
		\int |f - f_{n} | dm \to 0 \text{ as } n \to \infty
	\end{align*}
\end{proof}


\subsection{Chapter 2 - Borel Measure}

insert Diagram


generalize to:

insert diagram

\begin{definition}[Positive Linear Functional]
	Suppose $K$ is a compact metric space. A positive linear functional on $C(K)$ is a map $\Lambda : C(K) \to \mathbb{R}$ s.t.
	$\Lambda(\alpha f + \beta g) = \alpha \Lambda(f) + \beta\Lambda(g)$ for any
	$g \geq 0 \implies \lambda (g) \geq 0$
\end{definition}

\begin{example}
	\begin{enumerate}
		\item $K = [0,1]$, $\Lambda (f) = \int_{Riemann} f$
		\item $x \in X$, \, $\Lambda(f) = f(x)$
	\end{enumerate}
\end{example}

A finite positive Borel measure on $K$ is a positive measure on the Borel $\sigma$-algebra of $K$ with $m(X) < \infty$


\begin{theorem}
	There is a natural bijection between positive linear functionals on compact metric space $K$ and the finite positive Borlemeasures on $K$ and it is characterized by $\Lambda \leftrightarrow m$
	$\iff$
	 \begin{align*}
	\Lambda(f) = \int f dm \quad \forall f \in C(K)
	\end{align*}
\end{theorem}

\begin{remark}
	It is easier to prove a mor general version that uses only \underline{topological}-properties of the underlying space.
\end{remark}

\begin{definition}["Nice" topological space]
	Call a topological space $X$ nice if the following hold:
	\begin{enumerate}
		\item $X$ is Hausdorff: if $x,y \in X$ distinct, then there are open $u,v \subseteq X$ with  $x \in U$, $y \in V$ and $U \cap V  = \emptyset$
		\item $X$ is locally compact:
			if $x \in X$, then there are $K, V \subseteq X$ with $x \in V \subseteq K $,
			$V$ open, $K$ compact.

		\item $X$ is "really" $\sigma$-compact: if $V \subseteq X$ open, then there are compact $K_n \subseteq X$ with $V = \bigcup_{n=1}^{\infty} K_n$
	\end{enumerate}
\end{definition}

