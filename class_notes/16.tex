\section{2022-11-02}

Last time

\begin{theorem}[Vitali]
	Ig $B$ is a collection of open balls covering $X$, then there is a disjoint subcollection $B' \subseteq B$ with 
\end{theorem}


\subsection{Absolute Continuity + Radon-Nikodym}

fix $\sigma$-algebra $S$ on a set $X$.


\begin{definition}[Absolutely Continuous]
	Suppose $m_1$ and $m_{2}$ are measures on $S$, then $m_1$ is absolutely continuous with respect to $m_{2}$ if and only if 
	\begin{align*}
		m_{2} (A) = 0 \implies m_1 (A) = 0
	\end{align*} in which case we write $m_1 << m_2$
\end{definition}

\begin{definition}
	A measure $m$ on $S$ concentrates on a set $A \in S$ $\iff $ $m(A^{c}) = 0$
\end{definition}

\begin{remark}
	If $f \in L^{1} (m)$ , then $d\tilde{m} = f dm \iff \tilde{m} (A) = \int \1_{A}f dm$
\end{remark}


\begin{example}
	$U, V \subseteq  \R^{d}$ open bounded and non-empty.
	\begin{align*}
		d m_{U} = \1_{U} dL, \quad d m_{V} = \1_{V} dL
	\end{align*} if $U \subseteq V$, then $m_{U} << m_{V}$.
	if $U \cap V = \emptyset$, then $M_{U} \perp m_{V}$
\end{example}

\begin{theorem}
	If $m$ and $\tilde m$ are measure on  $S$ and  $m(X) < \infty,$ and $\tilde m (X) < \infty$, then there are unique measure $m_{a}$ and $m_{s}$ on $S$ so $m = m_{a} + m_{s}$, $m_{a} << \tilde m$ and $m_{s} \perp \tilde m$
\end{theorem}

\begin{proof}
	\begin{enumerate}
		\item[step 1] Uniqueness: let $m = m_{a}' + m_{s}'$.
			Since $m_{s} \perp \tilde m$ and $m_{s}' \perp \tilde m$, can choose
			$A , A'$ so that 
			\begin{align*}
				&m_s (A) = 0 \quad \tilde m (A^{c}) = 0, \\
				&m_s' (A') = 0, \tilde m ( (A')^c) = 0
			\end{align*} 
			Now, let $B = A \cap A'$, then  $m_s (B) = 0$, $m_s (B) = 0$, and $\tilde m(B^c) = 0$

			Now for arbitrary $E \in S$, we have
			\begin{align*}
				m_s (E) &= m_s ( E \cap B^c) \\
						&= m_s (E \cap B^c) + m)a (E \cap B^c) \\
						&= m (E \cap B^c) \\
						&= m_a' (E \cap B^c) + m_a' (E \cap B^c) \\
						&= m_s' (E \cap B^c) = m_s'(E)
			\end{align*} 
			Now, since $m (X) < \infty$, $m_a = m_a'$
		\item[step 2] Existence:

			The idea is to build a set on which $m_s$ concentrates by maximizing $m$ measure amon sets and $\tilde m$ is a null sets

			Choose $A_{n} \in S$ s.t. $ \tilde m (A_{n})$
			\begin{align*}
				m(A_{n}) \geq -\frac{1}{n} + \sup_{\tilde{m} (A) = 0} m(A)
			\end{align*} Possible because $m(X) < \infty$


			Now let  $B = \bigcup_{n} A_{n}$. Since $\tilde m (B) = 0$, we have
			\begin{align}
				m(B) = \sup_{\tilde m (A) = 0} m (A)
			\end{align} 

			Set $m_s (E) = m (E \cap B)$ and $m_a (f) = m (E \cap B^c)$.
			Since  $m_s (B^c) = m( \emptyset) = 0$, we have $\tilde m \perp m_s$.


			Suppose for Contradiction, $m_a < \nless \tilde m$.

			Choose $E$, so $\tilde m (E) = 0$ and $m_a (E) > 0$.

			Then $m(E \cup B^c) > 0$ and $\tilde m (E \cap B^c) = 0$, then 
			$m(E \cup B) > m (B)$
			so  $\tilde m (E \cup B) = 0$ contradicting (1).
	\end{enumerate} 
\end{proof}

\subsection{Radon-Nikodym}

\begin{theorem}
	If $m, \tilde{m}$ are measure on $S$, $m(X) < \infty$, and $m << \tilde{m}$, then there is $f \ in L^{1} ( \quad{m})$ such that
	$dm = f d\tilde{m}$
\end{theorem}

\begin{remark}
	\begin{enumerate}
		\item This is another form of Riesz-Representation theorem.
		\item Use to upgrade decomposition
			\begin{align*}
				dm &= dm_{a} + dm_s \\
				   &= dm_a + dm_s \\
				   &= f d\tilde{m} + d m_{s}
			\end{align*} where $f \in L^{1} (\tilde m)$
		\item Converse is trivial if $dm = f d \tilde{m}$ and
			$f \in L^1 (\tilde{m} )$, then $m << \tilde{m}$
	\end{enumerate}
\end{remark}

There are two proofs:
\begin{enumerate}
	\item A measure theoretic proof
	\item An (elementary) Functional Analysis Proof (Rudin)
\end{enumerate}

Well start with the first proof (1):

\begin{theorem}[Hahn Decomposition] If $m_1$ and $m_2$ are measures
\end{theorem}


\begin{remark}
	\begin{enumerate}
		\item If we think of $m = m_1 - m_2$ as a signed measure, then $P$ is "positive" set and $P^c$ is a "negative" set.
		\item If $X = \{1, \ldots, N\}$, and $S = \mathcal{P} (X)$, then
			\begin{align*}
				P = \{x : m_{1} (\{x\}) \geq m_2 (\{ x \})\}
			\end{align*} 
	\end{enumerate}
\end{remark}

