\section{November 11, 2022}

\subsection*{Last time}

\begin{theorem}[Vitali]
	If $B$ is a collection of open balls covering $X$, then there is a disjoint subcollection $B' \subseteq B$ with
	\begin{align*}
		X \subseteq \bigcup_{B \in \mathcal{B}} 5 B
	\end{align*}
\end{theorem}

\begin{theorem}[Maximal Inequality]
	Let $m$ be a Borel measure on $\R^{d}$ and
	\begin{align*}
		(M m )(x) = \sup_{r > 0} \frac{m (B(x, r))}{L (B(x,r))},
	\end{align*}
  then
  \[
		L\left( \{ M m > \lambda \} \right) \leq 3^{d} m (\R^d) \lambda^{-1}.
  \]
\end{theorem}

\begin{definition}[Lebesgue Point]
	$X \in \R^{d}$ is a Lebesgue point for $f \in L^{1} ( \R^{d} )$ if and only if
	\begin{align*}
		\lim_{r \to 0} \frac{\int_{B(x, r)} | f - f(x) | \, dL}{L (B(x, r))} = 0
	\end{align*}
\end{definition}

\begin{definition}[Lebesgue Differentiation]
	If $f \in L^{1}( \R^{d})$ and
	\begin{align*}
		&E = \{ x \in \R^{d} : x \text{ is a Lebesgue point of } f \},
	\end{align*}
	then $L(E) = 0$.
\end{definition}


\subsection{Absolute Continuity \& Radon-Nikodym}

Fix $\sigma$-algebra $S$ on a set $X$.

\begin{definition}[Absolutely Continuous]
	Suppose $m_1$ and $m_{2}$ are measures on $S$. Then $m_1$ is absolutely continuous with respect to $m_{2}$ if and only if
	\begin{align*}
		m_{2} (A) = 0 \implies m_1 (A) = 0
	\end{align*} in which case we write $m_1 \ll m_2$.
\end{definition}
\begin{definition}[Mutually Singular]
	Suppose $m_1$ and $m_2$ are measures on $S$. $m_1$ and $m_2$ are mutually singular if and only if there is $A \in S$ such that $m_1(A) = 0$ and $m_2(A^c) = 0$.
\end{definition}

\begin{definition}[Measure Concentration]
	A measure $m$ on $S$ concentrates on a set $A \in S$ if and only if $m(A^{c}) = 0$.
\end{definition}

\begin{notation}
	If $f \in L^{1} (m)$ , then $d\tilde{m} = f dm \iff \tilde{m} (A) = \int \1_{A}f dm$
\end{notation}


\begin{example}
	Let $U, V \subseteq  \R^{d}$ open bounded and non-empty sets.
	\begin{align*}
		d m_{U} = \1_{U} \, dL, \quad d m_{V} = \1_{V} \, dL
	\end{align*}
	\begin{itemize}
		\item If $U \subseteq V$, then $m_{U} \ll m_{V}$.
		\item If $U \cap V = \emptyset$, then $m_{U} \perp m_{V}$
	\end{itemize}
\end{example}

\begin{theorem}
	If $m$ and $\tilde m$ are measures on  $S$ and  $m(X) < \infty$ and $\tilde m (X) < \infty$, then there are unique measures $m_{a}$ and $m_{s}$ on $S$ so $m = m_{a} + m_{s}$, $m_{a} \ll \tilde m$ and $m_{s} \perp \tilde m$
\end{theorem}

\begin{proof} \phantom{.} \hfill
	\begin{steps}
	\item \textsc{Uniqueness:} \par
			Let $m = m_{a}' + m_{s}'$.
			Since $m_{s} \perp \tilde m$ and $m_{s}' \perp \tilde m$, we can choose
			$A , A'$ so that
      % TODO: fix these equations at some point
      \begin{align*}
        m_s (A) = 0    && \tilde m (A^{c}) = 0, \\
        m_s' (A') = 0  && \tilde m ( (A')^c) = 0
			\end{align*}
			Now, let $B = A \cap A'$, then  $m_s (B) = 0$, $m_s (B) = 0$, and $\tilde m(B^c) = 0$

			Now for arbitrary $E \in S$, we have
			\begin{align*}
				m_s (E) &= m_s ( E \cap B^c) \\
						&= m_s (E \cap B^c) + m_a (E \cap B^c) \\
						&= m (E \cap B^c) \\
						&= m_s' (E \cap B^c) + m_a' (E \cap B^c) \\
						&= m_s' (E \cap B^c) = m_s'(E)
			\end{align*}
			Now, since $m (X) < \infty$, $m_a = m_a'$.
		\item \textsc{Existence:}

			The idea is to build a set on which $m_s$ concentrates by maximizing $m$ measure among sets of $\tilde m$ null sets.

			Choose $A_{n} \in S$ s.t. $ \tilde m (A_{n})$
			\begin{align*}
        m(A_{n}) \geq -\frac{1}{n} + \sup_{\tilde{m} (A) = 0} m(A)\note{possible because $m(X) < \infty$}
			\end{align*} 


			Now let  $B = \bigcup_{n} A_{n}$. Since $\tilde m (B) = 0$, we have
      \begin{align}\label{eqn:16.1}
				m(B) = \sup_{\tilde m (A) = 0} m (A)
			\end{align}

			Set $m_s (E) = m (E \cap B)$ and $m_a (f) = m (E \cap B^c)$.
			Since  $m_s (B^c) = m( \emptyset) = 0$, we have $\tilde m \perp m_s$.


			Suppose for Contradiction, $m_a \not \ll \tilde m$.

			Choose $E$, so $\tilde m (E) = 0$ and $m_a (E) > 0$.

			Then $m(E \cup B^c) > 0$ and $\tilde m (E \cap B^c) = 0$, then
			$m(E \cup B) > m (B)$
      so  $\tilde m (E \cup B) = 0$ contradicting \ref{eqn:16.1}.
	\end{steps}
\end{proof}

\subsection{Radon-Nikodym}

\begin{theorem}
	If $m, \tilde{m}$ are measure on $S$, $m(X) < \infty$, and $m \ll \tilde{m}$, then there is $f \ in L^{1} ( \quad{m})$ such that
	$dm = f d\tilde{m}$
\end{theorem}

\begin{remark}
	\begin{enumerate}
		\item This is another form of Riesz-Representation theorem.
		\item Use to upgrade decomposition
			\begin{align*}
				dm &= dm_{a} + dm_s \\
				   &= dm_a + dm_s \\
				   &= f d\tilde{m} + d m_{s}
			\end{align*} where $f \in L^{1} (\tilde m)$
		\item Converse is trivial if $dm = f d \tilde{m}$ and
			$f \in L^1 (\tilde{m} )$, then $m \ll \tilde{m}$
	\end{enumerate}
\end{remark}

There are two proofs:
\begin{enumerate}
	\item A measure theoretic proof
	\item An (elementary) Functional Analysis Proof (Rudin)
\end{enumerate}

Well start with the first proof (1):

\begin{theorem}[Hahn Decomposition] If $m_1$ and $m_2$ are measures on $S$ and $m_1 (x)$, $m_2 (x) < \infty$, then there is $P \in S$ s.t.
	\begin{enumerate}
		\item $A \in S$ and $A \subseteq P \implies m_1 (A) - m_2 (A) \geq 0$
		\item $A \in S$ and $A \subseteq P^c \implies m_1 (A) - m_2 (A) \leq 0$
	\end{enumerate}
\end{theorem}


\begin{remark}
	\begin{enumerate}
		\item If we think of $m = m_1 - m_2$ as a signed measure, then $P$ is "positive" set and $P^c$ is a "negative" set.
		\item If $X = \{1, \ldots, N\}$, and $S = \mathcal{P} (X)$, then
			\begin{align*}
				P = \{x : m_{1} (\{x\}) \geq m_2 (\{ x \})\}
			\end{align*}
	\end{enumerate}
\end{remark}

\begin{proof}
	Call $A \in S$ \underline{positive} if $m_1 (B) - m_2 (B) \geq 0$ for all $B \subseteq A$.

	 \begin{claim}
		 If $m_1 (A) - m_{2} (A) > 0$, then there is positive $B \subseteq A$ with
		 \begin{align*}
			 m_{1} (B) - m_{2} (B) \geq m_1 (A) - m_2 (A)
		 \end{align*}.
	\end{claim}
	Construct $B$ by removing negative mass. Recursively define $A = A_{0} \contains A_1 \contains \ldots$
	Choose $A_{n+1 } \subseteq A_{n}$ such that
	\begin{align*}
		m_{i} (A_{n +1} - m_{2} (A_{n+1}) =
	\end{align*} [check]

	Let $B = \bigcup_{n} A_{n}$
	\begin{align*}
		m_{1} (B) - m_{2} (B) &= \lim_{n \to \infty} m_{1} (A_{n}) - m_{2} (A_{n}) \\
							  &\geq m_{1} (A) - m_{2} (A)
	\end{align*}
	To see that $B$ is positive, suppose for contradiction $E \subseteq B$ and $m_{1} (E) - m_{2} (E) < 0$.
	For large $n$, have $m_1 (E) - m_{2} (E) < -\frac{1}{n}$.

	We can imporive our choice of $A_{n +1}$ by replacing it with $A_{n +1 } \setminus E$ giving increase greater than $\frac{1}{n}$ which contradicts our construction.
	(The claim is now proved)

	Let $B_{n}$ be a sequence of positive sets with
	\begin{align*}
		m_{1} (B_{n}) - m_{2} (B_{n}) \geq \frac{-1}{n} + \sup_{B \text{ pos}} (m_{1} (B) - m_{2} (B))
	\end{align*}
	Let $P = \bigcup_{n} B_{n}$. to see that $P$ is positive:

	Let $E \subseteq  P$, then
	\begin{align*}
		E = ( E \cap B_{1} ) \cup ( E \cap B_{2} \cup B_{1}^c) \cup ( E \cap B_{3} \cap B_{2}^{c} \cap B_{1}^c ) \cup \ldots
	\end{align*}
	And all disjoint pieces have $m_{1} - m_{2} \geq 0$.

	If $E \subseteq P^c$ and $m_{1} (E) - m_{2} (E) > 0$, then by claim, there is positive $B \subseteq E$
	with $m_{1} (B) - m_{2} (B) \geq \varepsilon > 0$.

	Now $B_{n} \cup B$ would be positive with
	\begin{align*}
		m_{1} (B_{n} \cup B) - m_{2} (B_{n} \cup B) &= m_{1} (B_{n}) - m_{2} (B_{n} ) + m_{1} (B) - m_{2} (B) \\
													&\geq m_{1} (B_{n}) - m_{2} (B_{n} ) + \varepsilon
	\end{align*}
	contradicting (star) for large $n$.
\end{proof}

