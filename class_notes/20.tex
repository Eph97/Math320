\section{2022-11-28}

\epigraph{\textit{hello world}}{}

\subsection{Product Measures}

Suppose we have finite measure spaces $(x_1, S_1, m_1),$ $(x_2, S_2, m_2)$. We want to build a ``natural'' product
$(x_1 \times x_2, S_1 \times S_2, m_1 \times m_2 )$ where
\[
	x_1 \times x_2 = \{ (x_1, x_2) : x_1 \in X_1 \text{ and } x_2 \in X_2 \}
\] is the usual Cartesian product.

A \underline{measurable rectangle} is a set $A_1 \times A_2 \subseteq X_1 \times X_2$ where $A_1 \in S_1$ and $A_2 \in S_2$.

\begin{definition}
	\[
		S_1 \times S_2 = \sigma(\{ A_1 \times A_2 : A_1 \in S_1, \text{ and } A_2 \in S_2 \} )
	\] that is, the $\sigma$-algebra generated by measurable rectangles.
\end{definition}

\begin{lemma}
	If $A \in S_1 \times S_2 $, then for every $x_1 \in X_1$ and $x_2 \in X_2$ the sections
	\begin{align*}
		A_{x_1, \cdot } &= \{ y_2 \in X_2 : (x_1 , y_2 ) \in A \} \quad \text{and} \\
		A_{\cdot, x_2 } &= \{ y_1 \in X_2 : (y_1 , x_2 ) \in A \}
	\end{align*} satisfy $A_{x_1 , \cdot} \in S_2$ and $A_{\cdot, x_2} \in S_1$.
\end{lemma}

\begin{proof}
	Use induction over construction of $S_1 \times S_2$.
	Let $\Omega \subseteq S_1 \times S_2$ be subcollection of $A \in S_1 \times S_2$ for which conclusion holds.

	 \begin{enumerate}
		 \item[Claim 1] If $R$ rectangle, then $R \in \Omega$. Suppose  $R = A_1 \times A_2$, $A_1 \in S_1$ and $A_2 \in S_2$

			 \begin{align*}
				 (A_1 \times A_2)_{(X_1, \cdot)} =
				 \begin{cases}
					 \emptyset & X_1 \notin A_1 \\
					 A_2 & X_1 \in A_1
				 \end{cases}
			 \end{align*}
			 $(A_1 \times A_2)_{(\cdot, X_2)}$ can be shown similarly.
		 \item[Claim 2] If $A \in \Omega$, then  $A^c \in \Omega$.
			 \begin{align*}
				 (A^c)_{(X_1, \cdot)} &= (A_{(X_1, \cdot)})^c \quad \text{and} \\
				 (A^c)_{(\cdot, X_2)} &= (A_{(\cdot, X_2)})^c
			 \end{align*}

		 \item[Claim 3] If $A_k \in \Omega$, then  $\bigcup_{k \in \N} A_k \in \Omega$
			 \begin{align*}
				 \big(\bigcup_{k}A_k\big)_{(x_1, \cdot)} &= \bigcup_{k}(A_k)_{(x_1, \cdot)} \quad \text{and} \\
				 \big(\bigcup_{k}A_k\big)_{(\cdot, x_2)} &= \bigcup_{k}(A_k)_{(\cdot, x_2)}
			 \end{align*}

	\end{enumerate}
\end{proof}

\begin{exercise}
	Is the converse true?
\end{exercise}

\begin{definition}
	Call $A \subseteq S_1 \times S_2$ \underline{elementary} if it is a finite disjoint union of measurable rectangles.
\end{definition}

\begin{lemma}
	Elementary sets form an Algebra.
\end{lemma}
\begin{proof}
	Exercise! (Hint: $\bigcap$ of any two measurable rectangles is measurable.
	$\bigcup$ of any two measurable rectangles can be written as a intersection of three disjoint measurable rectangles.)
\end{proof}

\begin{lemma}
	$S_1 \times S_2$ is minimal collection of subset of $x_1 \times x_2$ that contains elementary set and is monotone.
	\begin{align*}
		&A_k \in S_1 \times S_2, A_k \subseteq A_{k+1} \implies \bigcup_{k}A_{k} \in S_1 \times S_2 \\
		&B_k \in S_1 \times S_2, B_k \contains B_{k+1} \implies \bigcap_{k}B_{k} \in S_1 \times S_2
	\end{align*}
\end{lemma}

\begin{remark}
	We already know a monotone algebra is a $\sigma$-algebra.
\end{remark}

\begin{proof}
	Let $S \subseteq \mathcal{P}(X_1 \times X_2)$ be minimal collection that contains elementary sets and is monotone.
	We know $S \subseteq S_1 \times S_2.$ \\
	\underline{It's enough to show $S$ is algebra.} The argument is subtle:

	\noindent For $A \in S$ define
	\[
		\Omega (A) = \{ B \in S : A \cap B, A \cup B, A \cap B^c, A^c \cap \in S\}
	\]
	\begin{enumerate}
		\item[Claim 1] $B \in \Omega (A) \implies A \in \Omega (B)$.
		\item[Claim 2]  If $A$ is elementary implies $\Omega (A) \contains$ \big\{ elementary \big\}.
		\item[Claim 3] $\Omega (A)$ monotone.

			$B_{k} \in \Omega (A) $ monotone implies
			\[
				B_{k} \cup A, B_k \cap A, B_{k}^c \cap A, B_k \cap A^c \in S
			\] and monotone $(\cdot, k)$ implies
			\footnote{Since $S$ monotone}
			\[
				\bar{B} \cup A, \bar{B} \cap A, \bar{B}^c \cap A, \bar{B} \cap A^c \in S
			\] where $\bar{B}$ either $\bigcup_{k} B_k$ or $\bigcap_{k} B_k$

		\item[Claim 4] $A$ elementary implies $\Omega(A) = S$ by claim 2 and 3, $\Omega (A)$ contains elementary and is monotone so $\Omega (A) \contains S$ by minimality.
		\item[Claim 5] $A \in S \implies \Omega (A) = S$.

			Suppose $A \in S$. If $B$ elementary, $A \in \Omega (B)$ by claim 4. So $B \in \Omega (A)$ by claim 1. So $\Omega (A)$ contains all elementary and is monotone (By claim 3).

			So again, using minimality, $\Omega (A) = S$.
		\item[Claim 6] $S$ an $\sigma$-algebra by Claim 5 and definition of $\Omega$. In addition we know monotone algebra implies $\sigma$-algebra.
 	\end{enumerate}
\end{proof}

\begin{lemma}
	If $f : X_1 \times X_2 \to [0,\infty]$ is $(S_1 \times S_2)$ measurable, then
	\begin{align*}
		f(x_1, \cdot) &: X_2 \to [0,\infty] \quad \text{and} \\
		f(\cdot, x_2) &: X_1 \to [0,\infty]
	\end{align*}
	are $S_2$ and $S_1$ measurable
\end{lemma}

\begin{proof}
	\[
		f_{(X_1, \cdot)}^{-1}((a,b)) = \big(f^{-1}(\, (a,b) \,)\big)_{(X_1, \cdot)}
	\]
\end{proof}

\subsection{Fubini's Theorem}

\begin{lemma}\label{lem:f1f2}
	If $A \in S_1 \times S_2$, $f_1 (X_1) = m_2 (A_{X_1, \cdot})$, $f_2(X_2) = m_1 (A_{\cdot, X_2})$, then
	$f_1$ is $S_1$-measurable, $f_2$ is $S_2$-measurable, and 
	\[
		\int f_1 \, dm_1 = \int f_2 \, dm_2.
	\]
\end{lemma}

\begin{remark}
	We are proving
	\begin{align*}
		\int \left[ \int \1_A (x_1, x_2) \, dm_1 (x_1)\right] \, dm_2 (x_2) = \int \left[ \int \1_A (x_1, x_2) \, dm_2(x_2) \right] \, dm_1 (x_1)
	\end{align*}
\end{remark}

\begin{proof}
	Let $\Omega \subseteq S_1 \times S_2$ be collection of $A$ where conclusion holds.
	\begin{enumerate}
		\item[Claim 1] If $\R$ rectangle, then $R \in \Omega$.
			 \begin{gather*}
				 R = A_1 \times A_2, A_1 \in S_1, A_2 \in S_2 \\
				 f_1 = m_2 (A_2) \1_{A_1}, \\
				 f_2 = m_1 (A_1) \1_{A_2}, \\
				 \int f_1 \, dm_1 = m_1 (A_1) m_2(A_2) = \int f_2 \, dm_2
			\end{gather*}
		\item[Claim 2] If $A$ elementary, then $A \in \Omega$. Conclusion commutes with finite sums.
		\item[Claim 3] $A$ monotone. Use monotone convergence theorem.
	\end{enumerate}
\end{proof}

\begin{definition}\label{def:m1xm2}
	If $A \in S_1 \times S_2$, then
	\begin{align*}
		(m_1 \times m_2) (A) &= \iint \1_{A} (x_1, x_2) \, dm_1(x_1) \, dm_2(x_2) \\
							&= \iint \1_{A} (x_1, x_2) \, dm_2(x_2) \, dm_1(x_1)
	\end{align*}
\end{definition}

\begin{theorem}
	$m_1 \times m_2$ is a measure.
\end{theorem}

\begin{proof}
	Use Monotone Convergence.
\end{proof}

\begin{theorem}[Fubini's Theorem]
	If $f:X_1 \times X_2 \to [0, \infty]$ is  $S_1 \times S_2$-measurable, then
	\begin{align*}
		g_1 (x_1) &= \int f(x_1, x_2) \, dm_2(x_2) \quad \text{and} \\
		g_2 (x_1) &= \int f(x_1, x_2) \, dm_1(x_1)
	\end{align*}
	are $S_1$ and $S_2$ measurable and $\int g_1 \, dm_1 = \int g_2 \, dm_2 = \int f \, d(m_1 \times m_2)$
\end{theorem}

\begin{proof}
	By lemma \ref{lem:f1f2} and definition of $m_1 \times m_2$, the conclusion holds for $f = \1_A$ with $A \in S_{1} \times S_2$.
	By linearity, conclusion holds for any simple function.

	If $f: X_1 \times X_2 \to [0,\infty]$ is  $S_1 \times S_2$ measurable, then $f(x) = \lim_{n\to \infty}f_k (x)$
	with $f_k \leq f_{k+1}$ simple, Now apply monotone convergence.
\end{proof}

\subsection{Three Important Examples}

\begin{example}[First Example] Choose your favorite bump function $g: (0,1) \to [0,\infty]$ continuous with
	\[
		\int g \, dL = 1 \quad \text{and} \quad g = 0 \text{ on } \left(0, \frac{1}{2}\right]
	\]
	[insert pic]
	Let $g_n(x) = 2^n g(2^nx)$ \\
	Let $f:(0,1)^2 \to [0,\infty)	$ be
	\[
		f(x,y) = \sum_{n \geq 0} (g_n (x) - g_{n+1}(x)) g_n(y)
	\]
	\begin{align*}
		\iint f(x,y) \, dy \, dx &= 1 \\
		\iint f(x,y) \, dx \, dy &= 0
	\end{align*}
	Note
	\[
		\iint |f| \, dx \, dy = \infty
	\]
\end{example}

\begin{corollary}
	If $f: X_1 \times X_2 \to \R$ measurable, then
	\[
		\iint f(x_1, x_2) \, dm_1 (x_1) \, dm_2 (x_2) = \iint f(x_1, x_2) \, dm_2(x_2) \, dm_1(x_1)
	\]
	provided $\iint |f| \, d(m_1 \times m_2) < \infty$
\end{corollary}

\begin{example}[Second Example] Let $f$ be such that 
	\begin{align*}
		&f:[0,1]^2 \to [0,\infty] \\
		&f(x,y) =
		\begin{cases}
			1 & x=y \\
			0 & x \neq y
		\end{cases}
	\end{align*}
	\begin{align*}
		\iint f(x,y) \, dL(x) \, dH^{0} (y) &= 0 \\
		\iint f(x,y) \, dH^{0}(x) \, dL (y) &= 1
	\end{align*}
	The issue: $H^0$ not finite on $[0,1]$.
\end{example}

\begin{example}[Third Example]
	Use continuum hypothesis and axiom of choice. Choose ordering $<<$ on $[0,1]$ so, for any $x \in [0,1]$.
	\begin{align*}
		&\{ y \in [0,1] : y \ll x \} \quad \text{is countable} \\
		&A \subseteq [0,1]^2 \\
		&A = \{ (x,y) : x \ll y \} \\
		&A_{\cdot, y} \, \text{countable}\\
		&A_{x, \cdot} \, \text{co-countable\footnote{its complement is countable}}
	\end{align*}
	Both $A_{\cdot, y}$ and $A_{x, \cdot}$ are Borel.
	\begin{align*}
		&\iint \1_{A} (x_1, x_2) \, dL(x_1) \, dL(x_2) = 0 \\
		&\iint \1_{A} (x_1, x_2) \, dL(x_2) \, dL(x_1) = 1
	\end{align*}

	Fubini's Theorem does not apply since $A$ is not a Borel measurable.
\end{example}
