\section{10-17-2022}


Recall:

come back to

The $\alpha$-dimensional Hausdorff measure is the restriction of $H^{\alpha}$ to Borel sets.

\subsection{Hausdorff Dimension}

\begin{lemma}
	If $0 \leq \alpha \leq \beta$ then $H^{\beta}_r (A) \leq r^{\beta - \alpha} H^{\alpha}_{r} (A)$
\end{lemma}

\begin{proof}
	If $A \subseteq \bigcup_{k \geq 0} B_{k}$ and $\sup_{k \geq 0} diam (B_{k}) \leq r$, 
	then 
	\begin{align*}
		\sum_{k \geq 0} diam (B_{k})^{\beta} \leq r^{\beta - \alpha} \sum_{k \geq 0} diam (A_{k})^{\alpha}
	\end{align*} 
	using $ \beta - \alpha > 0$
\end{proof}

\begin{theorem}
	For any $A \subseteq \mathbb{R}^{d},$ 
	\begin{align*}
		diam_{H}(A) = \inf \{\alpha \geq 0 : H^{\alpha}(A) = 0\}
	\end{align*} 
	satisfies $diam_{H}(A) \in [0, d]$. moreover, if $\beta \in [0, \infty),$ then $H^{\beta}(A) = \infty$
\end{theorem}

\begin{proof}
	From last time, $H^{\beta}(A) = 0$ for any $\beta > d$. It follows that $\dim_{H}(A) \in [0, d]$.
	That $H^{\beta}(A) = \infty$ for $\beta \in [0, \infty)$ follows from the next lemma.
\end{proof}

\begin{lemma}
If $0 \leq \alpha < \beta$, and $H^{\alpha}(A) < \infty$, then $H^{\beta} (A) = 0$
\end{lemma}

\begin{proof}
	If $H^{\alpha}(A) < \infty$, then
	\begin{align*}
		H^{\beta} (A) &= \sup_{r > 0} H^{\beta} (A) \\
					  &= \lim_{r \to 0} H_{r}^{\beta} (A) \\
					  &\leq \limsup_{r \to 0} r^{\beta - \alpha} H_{r}^{\alpha} (A) \\
					  &\leq ( \limsup_{r \to 0} r^{\beta - \alpha} ) ( \limsup_{r \to 0} H_{r}^{\alpha} (A) ) \\
					  &\leq 0 \cdot H^{\alpha} (A) = 0
	\end{align*}
\end{proof}

For any $A \subseteq \mathbb{R}^d$ the graph of $\alpha \mapsto H^{\alpha} (A)$ looks like either 

insert graphs.

\begin{exercise}
	Find examples of fractals where each of the above graphs occurs.
\end{exercise}

\subsection{Fractals}

For self-similar fractals, find an equation that the fractal solves. Use the properties of the equation to study fractals.

\begin{example}
	For serpinski triangle $S \subseteq \mathbb{R}^2$:
	\begin{align*}
		\varphi_{j} (x, y) = (\frac{x}{2} + \cos(\frac{2 \pi}{3 j}), \frac{y}{2} + \cos(\frac{2 \pi}{3 j}) )
	\end{align*} for $j = 1,2,3, \ldots$, observe
	$S = \bigcup_{j = 1,2,3} \varphi_j (S) $
\end{example}

\begin{theorem}
	If $\varphi_{1}, \ldots, \varphi_{m} : \mathbb{R}^d \to \mathbb{R}^d$ are strict contracts meaning if there are $0 \leq r_{j} < 1$ s.t.
	\begin{align*}
		| Q_j (x) - Q_j (y) | \leq r_{j} |x - y |
	\end{align*} then there is a unique compact
	$K = \bigcup_{j=1}^{m} \varphi_{j} (k)$
\end{theorem}

\begin{proof}
	Use Banach contraction mapping. Let $(M,d)$ be the space of compact subsets of $\mathbb{R}^d$ equipped with Hausdorf measure.
	That is $M = \{K \subseteq \mathbb{R}^d\}$ where K is compact and non-empty
	and
	\begin{align*}
		d(K_1, K_2) = \max\{ \max_{x_1 \in K_1} \min_{x_2 \in K_2} |x_1 - x_2|, \max_{x_2 \in K_2} \min_{x_1 \in K_1} |x_1 - x_2| \}
	\end{align*}

	Define $\phi : M \to M$ by  $\phi (K) = \bigcup_{j=1}^{m} \phi_j (k)$
	\item[claim 1]: $\phi$ strict contraction
	\item[claim 2]: $(m,d)$ is complete.

		First, observe that $d(K_1 \cup K_2, k_3 \cup K_4) \leq \max\{d(K_1, K_3), d(K_2, K_4)\}$

	\item[Proof of claim 1]:
		\begin{align*}
			d(\phi(K_1), \phi (K_2)) &= d( \bigcup_{j=1}^{m} \phi_j(K_1),\bigcup_{j=1}^{m} \phi_j (K_2) \\
									 &\leq \max_{j} d( \phi_{j}(K_1), \phi_{j}(K_2)) \\
									 &\leq (\max_{j} r_{j}) d(K_{1}, K_2)
		\end{align*} 


	\item[Proof of claim 2]: Suppose $K_1, K_2, \ldots \in M$ cauchy, refine the sequence so that $d(K_{j+1}, K_j ) \leq 2^{-j}$

		Let $K = \bigcap_{n \geq 0} \overline{\bigcup_{m \geq n} K_m}$ where $K$ is non-empty.

		Let $x_1 \in K_1$ arbitrary, recursiveily choose $x_{j+1} \in K_{j+1}$ so $|X_{j+1} - X_j | \leq 2^{-j}$
		$\{X_j\}$ is cauchy in $\mathbb{R}^d$ so $x = \lim_{j \to \infty} x_j$.

		Since $\{x_j\} \geq$ come back to

		$x \in \overline{\bigcup_{m \geq n} K_{m}}$ and therefore $x \in K$.
		
		We know K is closed because it is the intersection of closed sets.
		
		We also know K is bounded because $K_1$ is compact so $K_{1} \subseteq \mathcal{B}_{R}$ And thus

		\begin{align*}
			d(K_1, K_2) &\leq 2^{-1} \implies K_{2} \subseteq \mathcal{B}_{R + 2^{-1}} \\
			d(K_2, K_3) &\leq 2^{-2} \implies K_{3} \subseteq \mathcal{B}_{R + 2^{-1} + 2^{-2}} \\
		\end{align*} 
		$K_{m} \subseteq \mathcal{B}_{R + 1}$ for $m \geq 1$
\end{proof}

\begin{theorem}
		If $\phi_1, \ldots, \phi_m: \mathbb{R}^d \to \mathbb{R}^d$ where $r_j \in (0,1)$ and 
		\begin{align*}
			|\phi_j (x) - \phi_j (y) | \leq r_{j} |x - y|
		\end{align*},
		then $\phi_j$ is an isometry-dilation.

		Consider $U \subseteq \mathbb{R}^d$ open, $\phi_1 (U), \ldots , \phi_m(U)$ disjoint,
		$U \contains \bigcup_{j = 1}^{m} \phi_j (U)$, then the unique compact $K$ solving $K = \bigcup_{j = 1}^{m}\phi_j (K)$ has 
		$0 < H^{\alpha}(K) < \infty$ where $\alpha$ determined by 
		$\sum_{j=1}^{m} r_{j}^{\alpha} = 1$
\end{theorem}

\begin{example}
		Let $U = (0,1)$
		\begin{align*}
			\phi_1 (x) &= \frac{x}{3} \quad \phi_{2} (x) = \frac{x}{3} + \frac{2}{3} \\
			\quad \phi_1 (U) & = (0,\frac{1}{3} ), \quad \phi_2 (U) = (\frac{2}{3}, 1) \\
		\end{align*} 
		then $2 \cdot (\frac{1}{3})^{\alpha} = 1 \iff \alpha = \frac{\log 2}{\log 3}$ 

		The open set condition guarentees $H^{\alpha}( \phi_j (K) \cap \phi_l (K)) = 0$ \,
		for $j \neq l$.
\end{example}

