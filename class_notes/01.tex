\section{August 31st, 2022}

\epigraph{``How do mathematicians measure things?''}{Charlie}

Following the universal rule for first class, we covered the syllabus and logistics. We then dove into review on the Riemann integral to motivate measure theory. Prof. Smart seemed much more comfortable talking about math than logistics.

\subsection{Course Plan}
\begin{itemize}
	\item Review Riemann Integration
	\item Abstract Measure Theory and Integration (Majority of class)
	\item Applications in Probability
	\item Applications in Dynamics
	\item Some brief applications in Geometry
\end{itemize}

\begin{problem} The Riemann Integral
\item Works for most functions of interest but notably...
	\begin{enumerate}
		\item It is not closed under important limits.
		\item It is hard to generalize to new geometries.
	\end{enumerate}
\end{problem}

\begin{definition}[Riemann Integral]
	We can define the Riemann integral as $\int_{{Q}}^{{}} {f}$ of a bounded $f : Q \to R$
	defined in closed rectangles
	$Q = [a_1,b_1] \times [a_2,b_2] \times \ldots \times [a_d,b_d] \subseteq R$. We use Darboux sums
\end{definition}

\vline
\begin{definition}[Partition]
	A \underline{partition} of $Q$ is a finite set of \emph{closed} rectangles whose \underline{interiors} are disjoint and whose union is $Q$.
\end{definition}

\vline

\begin{definition}[Darboux Sums]
	The upper and lower Darboux sums are
	\[
		U(p,f) = \sum_{R \in P} \sup_R f \cdot |R|
	\]
	and
	\[
		L(p,f) = \sum_{R \in P} \inf_R f \cdot |R|.
	\]
\end{definition}

\begin{lemma}
	For any 2 partitions $P_1$ and $P_2$ of $Q$, $L(P_1,f) \leq U(P_2, f)$.
\end{lemma}

\begin{solution}
	We have
	\begin{align*}
		L(P_1, f) &= \sum_{R \in P_1}^{}  \inf_{R} f \cdot |R| = \sum_{R_1 \in P_1}^{} \inf_{R} f
		\sum_{R_2 \in P_2}^{} |R_1 \intersection R_2 | \\
				  &= \sum_{\substack{R_1 \in P_1 \\ R_2 \in P_2 \\ |R_1 \intersection R_2 | > 0} }^{} \inf_{R_1} f \cdot |R_1 \intersection R_2 |
				  \leq \sum_{\substack{R_1 \in P_1 \\ R_2 \in P_2 \\ |R_1 \intersection R_2 | > 0} }^{} \inf_{R_1 \intersection R_2} f \cdot |R_1 \intersection R_2 | \\
				  &\leq \sum_{\substack{R_1 \in P_1 \\ R_2 \in P_2 \\ |R_1 \intersection R_2 | > 0} }^{} \sup_{R_1 \intersection R_2} f \cdot |R_1 \intersection R_2 |
				  \leq \sum_{\substack{R_1 \in P_1 \\ R_2 \in P_2 \\ |R_1 \intersection R_2 | > 0} }^{} \sup_{R_2} f \cdot |R_1 \intersection R_2 | \\
				  &= \sum_{R_2 \in P_2}^{} \sup_{R_2} f \sum_{\substack{R_1 \in P_1 \\ |R_1 \intersection R_2 | > 0}}^{} |R_1 \intersection R_2 |  = \sum_{R_2 \in P_2}^{} \sup_{R_2} f \cdot |R_2| = U(P_2, f)
	\end{align*}
\end{solution}


Note we say a bounded $f : Q \to R$ is Riemann integrable when
\[\sup_p L(p,f) = \inf_p U(p,f)\]
in which case we let  $\int_{Q} f$ denote the common value.


\begin{theorem}
	If $f : Q \to R$ is continuous, then f is Riemann integrable.
\end{theorem}

\begin{solution}
	$Q$ is compact, so  $f$ is uniformly cont. Thus, given a $\varepsilon >0 $, choose a $\delta >0$ so  $|x-y| < \delta \implies |f(x) - f(y)| > \varepsilon$. Let $p$ be any partition of $Q$ whose rectangles have diameter  $< \delta$ \newline
	Next compute
	\begin{align*}
		0 &\leq \inf_{p_1} U(p_1, f) - \sup_{p_2} L(p_2, f) \leq U(p_1, f) - L(p_1, f) \\
		  &= \sum_{R \in P}^{} \left( \sup_{R} f - \inf_{R}f \right) \cdot |R| \\
		  &\leq \sum_{R \in P}^{} \varepsilon \cdot |Q|
	\end{align*}
	but since $\varepsilon > 0$ is arbitrary,  $f$ is Riemann integrable.
\end{solution}

\begin{example}
	If $f:[0,1] \to \R$ and $f(x) =
	\begin{cases}
		1 & x \notin Q \\
		0 & x \in Q
	\end{cases}$.
	Then $f$ is not Riemann integrable.
	\begin{solution}
		For any $P$, we can see  $U(p,f) = 1$ and  $L(p,f) = 0$
	\end{solution}
\end{example}

\subsection{Measures}

\noindent We have two notions of volume.
\begin{enumerate}
	\item The outer Jordan measure of a set $x \subseteq \R^d$ is
			\[
				J^* (x)
				= \inf \left\{ \sum_{k=1}^{n} |Q_k| : n \geq 1, Q_1, \ldots, Q_n \subseteq R^n \text{ and }
				x \subseteq \bigcup_{k=1}^{b} Q_k \right\}
			\]
	\item Outer Lebesgue measure of a set $X \subseteq \R^n$ is
		\begin{align*}
			L^*(x) 
			= \inf \left\{ \sum_{k=1}^{\infty}|Q_k| : Q_1, Q_2, \ldots \subseteq R^n \text{ and }
			x \subseteq \bigcup_{k=1}^{\infty} Q_k \right\}
		\end{align*}
		where $R^n$ are closed rectangles.

		We have some important properties for these measures, namely

		\begin{enumerate}
			\item $L^*(x) \leq J^* (x)$
			\item If  $X$ is compact, then $L^*(x) = J^*(x)$
		\end{enumerate}
		\begin{solution}
			Suppose $x \subseteq \bigcup_{k=1}^{\infty} Q_k$, pick $\varepsilon > 0$ and let  $(1 + \varepsilon)Q_k$ be the $(1 + \varepsilon)$-dilations of $Q_k$. Note $(1+ \varepsilon ) Q_k$ are open cover of  $x$ so $x \subseteq \sum_{k=1}^n (1+\varepsilon) Q_k$ and
			\begin{align*}
				\sum_{k=1}^{n} |(1 + \varepsilon) Q_k| = (1+\varepsilon)^d \sum_{k=1}^{n}|Q_k| \leq (1 + \varepsilon)^d \sum_{k=1}^{\infty} | Q_k|
			\end{align*}
		\end{solution}

	\item $L^* \left(\bigcup^{\infty}_{k = 1} x_k \right) \leq \sum_{k = 1}^{\infty} L^* (X_{k})$

		\begin{solution}
			sketch by "Dovetail" (Come back to)
		\end{solution}
\end{enumerate}

\begin{example}
	The set $x = Q \intersection [0,1]$ has  $L^*(x) = 0$ and  $J^*(x) = 1$ write  $x = \{q_1, q_2, \ldots \}$ then let $Q_k = [q_k =  \frac{\varepsilon}{2^k}, q_k + \frac{\varepsilon}{2^k}]$
	and observe that $x \subseteq \bigcup_{k = 1}^{\infty} Q_k$ and $\sum_{k=1}^{\infty} |Q_k| = 2 \varepsilon$
\end{example}

\begin{example}
	Let $X \subseteq [0,1]$ be generalized Cantor set. Let $1_{\C}$ be the indicator function of $\C$.

	Observe $1_ {\C}$ is not Riemann int on $[0,1]$. Indeed, C is the set of discontinuities of $1_{\C}$ and $L^*(\C) = \frac{1}{2}$. Moreover $1_{\C}(x) = \lim_{n\to \infty} 1_{\C_n} (x)$ and
	$0 \leq 1_{\C} \leq 1_{C_{n+1}} \leq 1_{\C_{n}} \leq 1$ and
	$\lim_{n \to \infty} \int_{[0,1]} 1_{C_{n}} = \frac{1}{2} $
\end{example}
