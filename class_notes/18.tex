\section{2022-11-14}

We previously proved 

\begin{theorem}[Radon-Nikodym]
	If there are $m_1, m_2 : S \to [0, \infty)$ measures, then there are unique $m_s, m_a : S \to [0, \infty)$ and $f \in L^{1} (m_2)$ so
	\begin{align*}
		m_1 &= m_a + m_s \quad m_s \perp m_2 \\
		m_a << m_2 \quad d m_a = f d_2
	\end{align*} 

\end{theorem}


\begin{theorem}[Maximal inequality]
		if $m : \mathcal{B}(R)$
\end{theorem}

\begin{theorem}[Lebesgue differentiation]
	If $f \in L^{1} (\R^d)$, then $L ()$
\end{theorem}

\subsection{Difference Quotient}

\begin{definition}[Difference quotients]
	For $m: \mathcal{B}(\R^d) \to [0, \infty]$ measure,
	let 
	\begin{align*}
		D m (x) = \lim_{r \to 0} \frac{m ( \mathcal{B} (x, r))}{L(\mathcal{B}(x, r))}
	\end{align*} 

	Want: $D m$ is $f$ from Radon-Nikodym decomposition of $m$ with respect to $L$.
\end{definition}

\begin{theorem}
	If $m: \mathcal{B}(\R^d) \to [0, \infty)$ measure and $m << L$, then 
	\begin{align*}
		D m \in L^{1} (\R^d) \quad \text{and} \quad m(A) \int \1_A dm dL
	\end{align*} 
\end{theorem}

\begin{proof}
	By Radon-Nikodym, there is unique $f \in L^1(\R^d)$ sp $m(A) = \int \1_A f dL$.
	It is enough to show  $f = D m$ for $L$ a.e $x \in \R^d$.

	Since  by Lebesgue differentiation, $L$-a.e. $x \in \R^d$ is Lebesgue for $f,$ have

	\begin{align*}
		f(x) = \lim_{r \to 0} \frac{\int_{B(x, r)} f dL}{L(B(x,r))} = \lim_{r \to 0} \frac{m(B(x,r))}{L(B(x,r))} = D m(x)
	\end{align*} 
	for $L$-a.e. $x \in \R^d$.
\end{proof}

\begin{remark}
	the above handles absolutey continuous case. now we will  look at the singular case.
\end{remark}


\begin{theorem}
	If $m : \mathcal{B}(\R^d) \to [0,\infty)$ a measure and $m \perp L$, then
	\begin{align*}
		D m(x) = 0 \quad \text{for} \quad L \text{-a.e.} \quad x \in \R^d
	\end{align*} 
\end{theorem}

\begin{proof}
	Define 
	\begin{align*}
		\bar{D} m(x) = \limsup_{r \to 0} \frac{m(B(x,r))}{L(B(x,r))}
	\end{align*} 
we show $L( \{ \bar{D}m > \lambda \}) = 0$

Since $m \perp L$, there is Borel $E \subseteq \R^d$. So $L(E) = 0$ and $m(E^c) = 0$.
Since  $m$ is a borel measure and $\R^d$ nice, we can choose $K \subseteq  E$ compact with
\begin{align*}
	m(K) \geq m(E) - \varepsilon.
\end{align*} 
Let
\begin{align*}
	m_1 (A) &= m(A \cap K) \quad \text{and} \\
	m_2 (A)	&= m( A \cap K^c)
\end{align*}

Observe: 
\begin{align*}
	&m = m_1 + m_2 \quad \text{and} \\
	&m_2(\R^d)\leq \varepsilon
\end{align*} 

So $m_1$ dives on compact sets of $L$-measure zero and $m_2$ has small total mass.
If $X \in K^c$, then 
\begin{align*}
	\bar{D}m(x) = \bar D m_2 (x) \leq M m_2 (x)
\end{align*} 
It follows that 
\begin{align*}
	&\{ \bar{D} m > \lambda \} \subseteq  K U \{ M m_2 > \lambda \}. \quad \text{compute} \\
	&L (K) \leq L(E) = 0 \quad \text{and} \\
	&L (\{ M m_2 > \lambda \}) \leq \frac{3^d}{\lambda} m_2 (\R^d) \leq \frac{3^d \varepsilon}{\lambda}
\end{align*} 
Sending $\varepsilon \to 0$ gives
\begin{align*}
	L ( \{ \bar{D} m > \lambda \} = 0
\end{align*} 
conclude
\[
	L\{ \bar{D} m > 0 \}) = 0
\]
\end{proof}


\begin{corollary}
	If $m : \mathcal{B}(\R^d) \to [0, \infty)$ a measure, then
	\begin{align*}
		dm = dm_{s} + Dm dL
	\end{align*} where $m_s \perp L$. in particular, $m \perp L \iff Dm = 0$ \\
	$L$-almost everywhere.
\end{corollary}

\begin{remark}
	if we consider $m$ instead, then we have the following theorem:
\end{remark}

\begin{theorem}
	If $m : \mathcal{B}(\R^d) \to [0, \infty)$ a measure, and $m \perp L$, then $D m(x) = \infty$ for $m$-almost every $x \in \R^d$
\end{theorem}

\begin{proof}
	Define
	\[
		\bar{D} m(X) = \limsup_{r \to 0} \frac{m(B(x,r))}{L(B(x,r))}
	\] we show 
	\[
		L(\{ \bar{D} m > \lambda \} ) = 0
	\] 
	Since $m \perp L$, there is Borel $E \subseteq \R^d$ so $L(E) = 0$ and $m(E^c) = 0$.
	And since  $m$ a borel measure and $\R^d$ nice, we can choose $K \subseteq E$ compact with
	\[
		m(K) \geq m(E) - \varepsilon
	\]
	let
	\begin{align*}
		m_1 (A) &= m( A \cap K) \quad \text{and} \\
		m_2 (A) &= m(A \cap k^c)
	\end{align*} 
	observe: $m = m_1 + m_2$ and $m_2 (\R^d) \leq \varepsilon$.
	So $m_1$ dives on compact set of $L$-measure zero and $m_2$ has small total mass.
	If $x \in K^c$, then
	\[
		\bar{D} m(x) = \bar{D}m_2 \leq M m_2 (x)
	\] 

	Now it follows that
	\[
		\{ \bar{D} > \lambda \} \subseteq K \cup \{ M m_2 > \lambda \}.
	\] compute
	\[
		L(K) \leq L(E) = 0
	\] and
	\[
		L( \{M m_2 > \lambda \}) \leq \frac{3^d}{\lambda} m_2(\R^d) \leq \frac{3^d \varepsilon}{\lambda}
	\] sending $\varepsilon \to 0$ gives
	\[
		L( \{ \bar{D} m > \lambda \}) = 0
	\] and we conclude that
	\[
	L( \bar{D} m > 0 \}) = 0
	\] 
\end{proof}


\begin{corollary}
	If $m : \mathcal{B}(R^d) \to [0,\infty)$ a measure, then
	\[
		dm = dm_s + Dm dL \quad \text{where} \quad m_s \perp L.
	\] in particular
	\[
		m \perp L \iff D m=0 \quad L\text{-a.e.}
	\] 
\end{corollary}

If we consider $m$ instead, then

\begin{theorem}
	If $m: \mathcal{B}(\R) \to [0,\infty)$ a measure and $m \perp L$, then $D m(x) = \infty$ for  $m$-a.e. $x\in \R^d$
\end{theorem}

\begin{proof}
Let 
\[
	\underline{D} m(x) = \liminf_{r \to 0} \frac{m(B(x,r))}{m(L(x,r))}
\]
we show $m(\{ \underline{D} m < \alpha \}) = 0$.

Since $m \perp L$, there is $E \subseteq \R^d$ borel with $L(E) = 0$ and $m(E^c) = 0$. 

Choose $V \contains E$ open with $L(V) \leq \varepsilon$. Let $w = \{ \underline{D} m < \alpha \} \cap E$.

If  $x \in W$, then there are arbitrary small  $r > 0$ so
\[
	m(B(x,r)) < \alpha L(B(x,r))
\]. 
Let $B$ be a collection of balls $B(x,r)$ so 
\begin{align*}
	&x \in W \\
	&B(x,5r) \subseteq V \quad \text{and} \\
	&m(B(x,5r)) < \alpha L(B(x,5r)).
\end{align*}
Note $W \subseteq \bigcup_{B \in \mathcal{B}} B$, By Vitali there are disjoint
$\mathcal{B}_0 \subseteq \B$ with $W \subseteq \bigcup_{B \in \B_{0}} 5B$.
Compute
\begin{align*}
	m(W) &\leq m( \bigcup_{B \in \B_{0}} 5B) \\
		 &\leq \sum_{B \in \B_0} m(5B) \\
		 &< \sum_{B \in \B_{0}} \alpha L(5B) \\
		 &= \sum_{B \in \B_0} 5^d \alpha L(B) = 5^d \alpha L(\bigcup_{B \in \B_0} B) \\
		 &\leq 5^d \alpha L(V) \leq 5^d \alpha \varepsilon.
\end{align*} 

Sending $\varepsilon \to 0$ gives $m(\{ \underline{D} m < \alpha \} ) = 0$.
Conclude that
 \[
	 m(\{ \underline{D} m < \infty\}) \leq \sum_{n \geq 0} m (\{ \underline{D} m < n \} ) = 0
\] 
\end{proof}
[finish]

\begin{remark}
	If $m \perp L$, then 
	\[
		\frac{m(B(x,r))}{L(B(x,r))} 
	\] 
\end{remark}
[comeback to]

\begin{example}
	$l \subseteq \R^d$, where $l$ is a finite line segment. $m(A) = H^1 (A \cap l)$.
	then  $m \perp L$
\end{example}

\begin{remark}
	$L$-a.e. $x$ satisfies $x \in l^c$, and $m$-a.e. $x$ satisfies $x \in l.$
\end{remark}

Rudin generalizes to not need finite $l$.

\subsection{Metric Density}

Suppose $A \subseteq \R^d$ is borel and bounded.

\begin{remark}
	 \[
		\1_a \in L^1 (\R^d)
	\] 
\end{remark}

By lebesgue differentiation
\begin{align*}
	\lim_{r \to 0} \frac{\int_{B(x,r) |\1_A - \1_A(x)| dL}}{L(B(x,r))} = 0
\end{align*} for $L$-a.e. $x \in \R^d$

\underline{Unrolling Definitions}

\[
	\lim_{r \to 0}  \frac{L(B(,x,r) \cap A)}{L(B(x,r))} = \1_{A} (x) 
	\quad \text{for } L \text{-a.e. } x \in \R^d
\] 

Density is $L$-a.e. either $0$ or $1$. It is impossible to build a Borel set $A$ so
\[
	0 < \varepsilon < \frac{L(A \cap B(x,r))}{L(B(x,r))} < 1 - \varepsilon
\] 
for all $B(x,r)$.

\begin{remark}
	The measure theoretic boundary of a borel set $A \subseteq \R^d$ is
	\[
\partial_{L}(A) = \left\{x \in \R^d : \lim_{r \to 0}  \frac{L(A \cap B(x,r))}{L(B(x,r))} \notin \{0,1\} \right\}
	\] 
	and always satisfies $L(\partial_{L} A) = 0$
\end{remark}


\begin{remark}
	this is false for topologica boundaries in general.
\end{remark}



\subsection{Fundamental Theorem of Calculus}

\begin{lemma}
	If $x \in \R^d$ is lebesgue for $f \in L^1 (\R^d)$, then
	\[
		f(x) = \lim_{k \to \infty}  \frac{\int | f - f(k) |\1_{A_k} dL}{l(A_{k})}
	\] whenever 
	\[
		A_k \subseteq B(x,r_{k}), \quad \inf_{k} \frac{L(A_k)}{L(B(x,r_k))} > 0 \quad \text{and }\quad r_k \to 0
	\]
\end{lemma}

\begin{proof}
	Let
	\begin{align*}
		\varepsilon = \inf_{k} \frac{L(A_k)}{L(B(x,r_k))} > 0
	\end{align*}. Compute
	\begin{align*}
		\frac{\int | f - f(x) | \1_{A_k} dL}{L(A_k)} \leq \frac{\int |f - f(x) | \1_{B(x,r_k) dL}}{\varepsilon L(B(x,r))} 
		\longrightarrow 0 \quad \text{as} \quad r_k \to 0
	\end{align*} 
	for $x$ lebesgue.
\end{proof}


\begin{corollary}
	If $f \in L^1(\R)$ and
	\[
		F(x) = \int f \1_{(-\infty, x)} dL = \int_{-\infty}^{x} f dL
	\] then
	$F' = f$ at $L$-a.e. point.
\end{corollary}

\begin{proof}
	for $h > 0$,
	 \[
		\frac{F(x +h) - F(x)}{h} = \frac{\int f \1_{[x,x+h)} dL}{L([x,x+h))}
	\] 
	apply lemma to $A_k = [x, x+h) \subseteq (x-h, x+h)$ to conclude $F' = f$ at all lebesgue points of f.
\end{proof}




\begin{definition}
	$f: [a,b] \to \R$ is absolutely continuous if and only if for every $\varepsilon > 0$, there is $\delta > 0$ so
	\[
		\sum_{k=1}^n |f(b_{k}) - f(a_k) | < \varepsilon
	\]whenever
	\[
		a \leq a_1 \leq b_1 \leq a_2 \leq b_2 \leq \ldots \leq a_n \leq b_n \leq b
	\] and
	\[
		\sum_{k=1}^n (b_n - a_k) < \delta
	\] 
\end{definition}

\begin{remark}
	Uniform continuity is absolute continuity with $n$ restricted to be $1$.
\end{remark}

\begin{example}
	Devil's staircase is uniformly continuous but not absolutely continuous.
\end{example}



