\section{2022-11-14}

We previously proved 

\begin{theorem}[Radon-Nikodym]
	If there are $m_1, m_2 : S \to [0, \infty)$ measures, then there are unique $m_s, m_a : S \to [0, \infty)$ and $f \in L^{1} (m_2)$ so
	\begin{align*}
		m_1 &= m_a + m_s \quad m_s \perp m_2 \\
		m_a << m_2 \quad d m_a = f d_2
	\end{align*} 

\end{theorem}


\begin{theorem}[Maximal inequality]
		if $m : \mathcal{B}(R)$
\end{theorem}

\begin{theorem}[Lebesgue differentiation]
	If $f \in L^{1} (\R^d)$, then $L ()$
\end{theorem}

\subsection{Difference Quotient}

\begin{definition}[Difference quotients]
	For $m: \mathcal{B}(\R^d) \to [0, \infty]$ measure,
	let 
	\begin{align*}
		D m (x) = \lim_{r \to 0} \frac{m ( \mathcal{B} (x, r))}{L(\mathcal{B}(x, r))}
	\end{align*} 

	Want: $D m$ is $f$ from Radon-Nikodym decomposition of $m$ with respect to $L$.
\end{definition}

\begin{theorem}
	If $m: \mathcal{B}(\R^d) \to [0, \infty)$ measure and $m << L$, then 
	\begin{align*}
		D m \in L^{1} (\R^d) \quad \text{and} \quad m(A) \int \1_A dm dL
	\end{align*} 
\end{theorem}

\begin{proof}
	By Radon-Nikodym, there is unique $f \in L^1(\R^d)$ sp $m(A) = \int \1_A f dL$.
	It is enough to show  $f = D m$ for $L$ a.e $x \in \R^d$. \\

	Since  by Lebesgue differentiation, $L$-a.e. $x \in \R^d$ is lebesgue for $f,$ have

	\begin{align*}
		f(x) = \lim_{r \to 0} \frac{\int_{B(x, r)} f dL}{L(B(x,r))} = \lim_{r \to 0} \frac{m(B(x,r))}{L(B(x,r))} = D m(x)
	\end{align*} 
	for $L$-a.e. $x \in \R^d$.
\end{proof}

\begin{remark}
	the above handles absolutey continuous case. now we will  look at the singular case.
\end{remark}


\begin{theorem}
	If $m : \mathcal{B}(\R^d) \to [0,\infty)$ a measure and $m \perp L$, then
	\begin{align*}
		D m(x) = 0 \quad \text{for} \quad L \text{-a.e.} \quad x \in \R^d
	\end{align*} 
\end{theorem}

\begin{proof}
	Define 
	\begin{align*}
		\bar{D} m(x) = \limsup_{r \to 0} \frac{m(B(x,r))}{L(B(x,r))}
	\end{align*} 
we show $L( \{ \bar{D}m > \lambda \}) = 0$

Since $m \perp L$, there is Borel $E \subseteq \R^d$. So $L(E) = 0$ and $m(E^c) = 0$.
Since  $m$ is a borel measure and $\R^d$ nice, we can choose $K \subseteq  E$ compact with
\begin{align*}
	m(K) \geq m(E) - \varepsilon.
\end{align*} 
Let
\begin{align*}
	m_1 (A) &= m(A \cap K) \quad \text{and} \\
	m_2 (A)	&= m( A \cap K^c)
\end{align*}

Observe: 
\begin{align*}
	&m = m_1 + m_2 \quad \text{and} \\
	&m_2(\R^d)\leq \varepsilon
\end{align*} 

So $m_1$ dives on compact sets of $L$-measure zero and $m_2$ has small total mass.
If $X \in K^c$, then 
\begin{align*}
	\bar{D}m(x) = \bar D m_2 (x) \leq M m_2 (x)
\end{align*} 
It follows that 
\begin{align*}
	&\{ \bar{D} m > \lambda \} \subseteq  K U \{ M m_2 > \lambda \}. \quad \text{compute} \\
	&L (K) \leq L(E) = 0 \quad \text{and} \\
	&L (\{ M m_2 > \lambda \}) \leq \frac{3^d}{\lambda} m_2 (\R^d) \leq \frac{3^d \varepsilon}{\lambda}
\end{align*} 
Sending $\varepsilon \to 0$ gives
\begin{align*}
	L ( \{ \bar{D} m > \lambda \} = 0
\end{align*} 
conclude
\[
	L\{ \bar{D} m > 0 \}) = 0
\]
\end{proof}


\begin{corollary}
	If $m : \mathcal{B}(\R^d) \to [0, \infty)$ a measure, then
	\begin{align*}
		dm = dm_{s} + Dm dL
	\end{align*} where $m_s \perp L$. in particular, $m \perp L \iff Dm = 0$ \\
	$L$-almost everywhere.
\end{corollary}

\begin{remark}
	if we consider $m$ instead, then we have the following theorem:
\end{remark}

\begin{theorem}
	If $m : \mathcal{B}(\R^d) \to [0, \infty)$ a measure, and $m \perp L$, then $D m(x) = \infty$ for $m$-almost every $x \in \R^d$
\end{theorem}

