\documentclass{lnotes}
\usepackage{amsfonts}
% Bibliography
\usepackage[
	style = alphabetic ,
]{biblatex}
\addbibresource{references.bib}

% Packages
\usepackage{epigraph}
\usepackage{wasysym}
\usepackage[osf]{libertineRoman}
\usepackage{tikz-cd}

% Commands
\DeclareMathOperator*{\osc}{osc}


% Document Information
\title{Measure Theory and~Integration}
\course{\textsc{math} 320}
\place{Yale University}
\term{Fall}
\year{2022}

\blurb{
	These are lecture notes for \textsc{math} 320, ``Measure Theory and Integration'' taught by Charlie Smart at Yale University during the fall of 2022.
	These notes are not official, and have not been proofread by the instructor for the course.
	These notes live in my lecture notes respository at 
	\[\text{\url{https://github.com/Eph97/Eph97/Math320}.}\]
	If you find any errors, please open a bug report describing the error, and label it with the course identifier, or open a pull request so I can correct it.
}

\begin{document}

\section*{Syllabus}

\begin{center}
\begin{tabular}{@{}rp{10cm}@{}}
\toprule 
\textbf{Instructor} &  Prof. Charlie Smart, \url{charlie.smart@yale.edu} \\
\textbf{Lecture} & MW 11:35 \textsc{am}--12:50 \textsc{pm}, Humanities Quadrangle 207 \\
\textbf{OH} & TBD \\
% \textbf{Recitation} & TBA \\
\textbf{Textbook} & \fullcite{Rudin} \\
\textbf{Midterms} & Mon Oct 3, 2022 (In-Class) \\
				  & Mon Nov 7, 2022 (In-Class) \\
\textbf{Final} & Tue Dec 20, 2022 at 2pm (Location: TBD)\\
\bottomrule 
\end{tabular} \\[3ex]
\end{center}

\subsection*{Course Description}
This course will provide a comprehensive introduction to measure theory.  We will start from real analysis and build from there.  The following topics will be covered.
\begin{itemize}
  \item Measure spaces
  \item Integrals on general measure spaces
  \item Product measures
  \item Integral representation of absolutely continuous measures
  \item Integral representation of linear functions
  \item Lebesgue measure
  \item Hausdorff measure and fractals
  \item Applications in dynamics
  \item Applications in probability theory
\end{itemize}

\subsection*{Course Format}
Approximately 2/3 of the material with be presented in lectures while the remaining 1/3 will be covered in the weekly homework assignments.

\subsection*{Prerequisites}
Math 305 or equivalent.

\subsection*{Assessments and Grading}
There will be ten homework assignments, two in-class midterms, and an in-person final, comprising 30\%, 40\%, and 30\%, respectively, of the total grade.


\section{August 31st, 2022}

\epigraph{``How do mathematicians measure things?''}{Charlie}

Following the universal rule for first class, we covered the syllabus and logistics. We then dove into review on the Riemann integral to motivate measure theory. Prof. Smart seemed much more comfortable talking about math than logistics.

\subsection{Course Plan}
\begin{itemize}
	\item Review Riemann Integration
	\item Abstract Measure Theory and Integration (Majority of class)
	\item Applications in Probability
	\item Applications in Dynamics
	\item Some brief applications in Geometry
\end{itemize}

\begin{problem} The Riemann Integral
\item Works for most functions of interest but notably...
	\begin{enumerate}
		\item It is not closed under important limits. 
		\item It is hard to generalize to new geometries.
	\end{enumerate}
\end{problem}

\begin{definition}[Riemann Integral]
	We can define the Riemann integral as $\int_{{\Q}}^{{}} {f}$ of a bounded $f : \Q \to \R$
	defined in closed rectangles
	$\Q = [a_1,b_1] \times [a_2,b_2] \times \ldots \times [a_d,b_d] \subseteq \R$. We use Darboux sums
\end{definition}

\vline
\begin{definition}[Partition]
	A \underline{partition} of $\Q$	is a finite set of \emph{closed} rectangles whose \underline{interiors} are disjoint and whose union is $\Q$.
\end{definition}

\vline

\begin{definition}[Darboux Sums]
	The upper and lower Darboux sums are 
	\[
		U(p,f) = \sum_{R \in P} \sup_R f \cdot |R|
	\] 
	and
	\[
		L(p,f) = \sum_{R \in P} \inf_R f \cdot |R|.
	\]	
\end{definition}

\begin{lemma}
	For any 2 partitions $P_1$ and $P_2$ of  $\Q$,  $L(P_1,f) \leq U(P_2, f)$
\end{lemma}

\begin{solution}
	We have
	\begin{align*}
		L(P_1, f) &= \sum_{R \in P_1}^{}  \inf_{R} f \cdot |R| = \sum_{R_1 \in P_1}^{} \inf f
		\sum_{R_2 \in P_2}^{} |R_1 \intersection R_2 | \\
				  &= \sum_{\substack{R_1 \in P_1 \\ R_2 \in P_2 \\ |R_1 \intersection R_2 | > 0} }^{} \inf_{R_1} f \cdot |R_1 \intersection R_2 |
				  \leq \sum_{\substack{R_1 \in P_1 \\ R_2 \in P_2 \\ |R_1 \intersection R_2 | > 0} }^{} \inf_{R_1 \intersection R_2} f \cdot |R_1 \intersection R_2 | \\
				  &\leq \sum_{\substack{R_1 \in P_1 \\ R_2 \in P_2 \\ |R_1 \intersection R_2 | > 0} }^{} \sup_{R_1 \intersection R_2} f \cdot |R_1 \intersection R_2 |
				  \leq \sum_{\substack{R_1 \in P_1 \\ R_2 \in P_2 \\ |R_1 \intersection R_2 | > 0} }^{} \sup_{R_2} f \cdot |R_1 \intersection R_2 | \\
				  &= \sum_{R_2 \in P_2}^{} \sup_{R_2} f \sum_{\substack{R_1 \in P_1 \\ |R_1 \intersection R_2 | > 0}}^{} |R_1 \intersection R_2 |  = \sum_{R_2 \in P_2}^{} \sup_{R_2} f \cdot |R_2| = U(P_2, f)
	\end{align*}
\end{solution}


Note we say a bounded $f : \Q \to \R$ is Riemann integrable when  
\[\sup_p L(p,f) = \inf_p U(p,f)\]
in which case we let  $\int_{\Q} f$ denote the common value.


\begin{theorem}
	If $f : \Q \to \R$ is continuous, then f is Riemann integrable.
\end{theorem}

\begin{solution}
	$\Q$ is compact, so  $f$ is uniformly cont. Thus, given a $\varepsilon >0 $, choose a $\delta >0$ so  $|x-y| < \delta \implies |f(x) - f(y)| > \varepsilon$. Let $p$ be any partition of  $\Q$ whose rectangles have diameter  $< \delta$ \newline
	Next compute
	\begin{align*}
		0 &\leq \inf_{p_1} U(p_1, f) - \sup_{p_2} L(p_2, f) \leq U(p_1, f) - L(p_1, f) \\
		  &= \sum_{R \in P}^{} \left( \sup_{R} f - \inf_{R}f \right) \cdot |R| \\
		  &\leq \sum_{R \in P}^{} \varepsilon \cdot |Q|
	\end{align*}
	but since $\varepsilon > 0$ is arbitrary,  $f$ is Riemann integrable.
\end{solution}

\begin{example}
	If $f:[0,1] \to \R$ and $f(x) =
	\begin{cases}
		1 & x\notin \Q \\
		0 & x\in \Q
	\end{cases}$.
	Then $f$ is not Riemann integrable.
	\begin{solution}
		For any $P$, we can see  $U(p,f) = 1$ and  $L(p,f) = 0$
	\end{solution}
\end{example}

\subsection{Measures}

We have two notions of volume
\begin{enumerate}
	\item The outer Jordan measure of a set $x \subseteq \R^d$ is
			\[
				J^* (x)
				= \inf \left\{ \sum_{k=1}^{n} |Q_k| : n \geq 1, Q_1, \ldots, Q_n \subseteq R^n
					\text{ closed rectangles and }
				x \subseteq \union_{k=1}^{b} Q_k \right\}
			\]
	\item Outer Lebesgue measure of a set $X \subseteq \R^n$ is
		\begin{align*}
			L^*(x) = \inf \left\{ \sum_{k=1}^{\infty}|Q_k| : Q_1, Q_2, \ldots \subseteq R^n  \text{ closed rectangles and }
			x \subseteq \union_{k=1}^{\infty} Q_k \right\}
		\end{align*}

		We have some important properties for these measures, namely

		\begin{enumerate}
			\item $L^*(x) \leq J^* (x)$
			\item If  $X$ is compact, then $L^*(x) = J^*(x)$
		\end{enumerate}
		\begin{solution}
			Suppose $x \subseteq \union_{k=1}^{\infty} Q_k$, pick $\varepsilon > 0$ and let  $(1 + \varepsilon)Q_k$ be the $(1 + \varepsilon)$-dilations of $Q_k$. Note $(1+ \varepsilon ) Q_k$ are open cover of  $x$ so $x \subseteq \sum_{k=1}^n (1+\varepsilon) Q_k$ and
			\begin{align*}
				\sum_{k=1}^{n} |(1 + \varepsilon) Q_k| = (1+\varepsilon)^d \sum_{k=1}^{n}|Q_k| \leq (1 + \varepsilon)^d \sum_{k=1}^{\infty} | Q_k|
			\end{align*}
		\end{solution}

	\item $L^* (\union^{\infty} x_k ) \leq \sum_{}^{\infty} L^* (X_{k})$

		\begin{solution}
			sketch by "Dovetail" (Come back to)
		\end{solution}
\end{enumerate}

\begin{example}
	The set $x = \Q \intersection [0,1]$ has  $L^*(x) = 0$ and  $J^*(x) = 1$ write  $x = \{q_1, q_2, \ldots \}$ then let $Q_k = [q_k =  \frac{\varepsilon}{2^k}, q_k + \frac{\varepsilon}{2^k}]$
	and observe that $x \subseteq \union_{1}^{\infty} Q_k$ and $\sum_{k=1}^{\infty} |Q_k| = 2 \varepsilon$
\end{example}

\begin{example}
	Let $X \subseteq [0,1]$ be generalized Cantor set. Let $1_{\C}$ be the indicator function of $\C$.

	Observe $1_ {\C}$ is not Riemann int on $[0,1]$. Indeed, C is the set of discontinuities of $1_{\C}$ and $L^*(\C) = \frac{1}{2}$. Moreover $1_{\C}(x) = \lim_{n\to \infty} 1_{\C_n} (x)$ and
	$0 \leq 1_{\C} \leq 1_{C_{n+1}} \leq 1_{\C_{n}} \leq 1$ and
	$\lim_{n \to \infty} \int_{[0,1]} 1_{C_{n}} = \frac{1}{2} $
\end{example}

\section{2022-09-02}

\subsection{Riemann Integration and Measures}
Last time: Definition of a Riemann Integral $\int_Q f$ of bounded $f : Q \to R$ on closed rectangle $Q
\subseteq \R^d$

\begin{theorem}
  A bounded $f : Q \to R$ is Riemann integrable if and only if the set $Z = \{ x \in Q \mid f
  \text{ not continuous of } x \}$ has $L^* (z) = 0$.
\end{theorem}

\begin{definition}[oscillation]
  The oscillation of $f : Q \to R$ at $x \in Q$ is
\[\osc(f,x) = \inf_{\delta > 0} \sup_{\substack{y, z \in Q \\ | y - x | < \delta \\ |z - x| < \delta}} |f(y) - f(z)| \]
\end{definition}

\begin{lemma}
  $f$ is continuous at $X$ if and only if $\osc (f,x) = 0$.
\end{lemma}

\begin{lemma}
  The set $z_\varepsilon = \{ x \in Q : \osc(f,x) \geq \varepsilon \}$ is closed (compact) for $\varepsilon > 0$.
\end{lemma}

\begin{proof}
  Since $\R^d$ is a metric space and closed, it is also sequentially closed.
  Suppose $x_1, x_2, \ldots \in Z_\varepsilon$ and $x = \lim_{n \to \infty} x_n$.
  Since $x_n \in Z_\varepsilon$, we can choose $y_n, z_n$ so that
  \[|y_n - x_n| < \frac{1}{n}, |z_n - x_n| < \frac{1}{n}, \text{ and } |f(y_n) - f(z_n)| \geq \varepsilon - \frac{1}{n}\]
  For any $\delta > 0$, choose $n > 0$ such that $\frac{1}{n} > \frac{\delta}{2}$ and $|x_n - x| < \frac{\delta}{2}$.
  We now have
  \[|y_n - x_n| < \delta, |z_n - x_n| < \delta, \text{ and } |f(y_n) - f(z_n)| \geq \varepsilon - \frac{\delta}{2}\]
  We now conclude that $x \in Z_\varepsilon$.
\end{proof}

Note on notation: The set of discontinuities $Z$ can be the countable union of compact sets, which is represented by,
\[ Z = \bigcup_{n \geq 1} Z_{\frac{1}{n}} \]

\begin{theorem}
  A bounded $f : Q \to R$ is Riemann integrable if and only if $Z = \{ x \in Q : \osc (f,x) > 0 \}$ has $L^*(x) = 0$.
\end{theorem}

\begin{proof}
  Assume $|f| \leq B$. First, suppose $L^*(z) = 0$.
  Let $Z_\varepsilon = \{ x \in Q : \osc (f,x) \geq \varepsilon\}$.
  Since $Z_\varepsilon \subseteq Z$, $L^*(Z_\varepsilon) = 0$.
  Since $Z_\varepsilon$ is compact, $J^*(Z_\varepsilon) = 0$.
  Choose a finite set of rectangles $S$ such that $Z_\varepsilon \subseteq \bigcup_{R_1 \in S_1} \mathring{R_1}$ and $\sum_{R_1 \in S_1} |R_1| < \varepsilon$.

  Let $W_\varepsilon = Q \setminus \bigcup_{R_1 \in S_1} \mathring{R}$ and observe that $W_\varepsilon$ is compact and $\osc(f,x) < \varepsilon$ for all $x \in W_\varepsilon$.
  \footnote{Note: this helps cover the boundary condition, which Spivak incorrectly omits in his book.}
  For every $x \in W_\varepsilon$, we can choose rectangle $R$ so $x \in \mathring{R}$ and $\sup_R f - \inf_R f < \varepsilon$.
  Since $W_\varepsilon$ is compact, they are a finite set of rectangles $S_2$ such that
  \[ W_\varepsilon \subseteq \bigcup_{R_2 \in S_2} \mathring{R}_2 \quad \text{and} \quad \sup_{R_2} f - \inf_{R_2} f < \varepsilon \quad \forall R_2 \in S \]

  Let $P$ be a partition of $Q$ so that for $R \in P$ and $R_k \in S_k$, where $k = 1,2$ have $\mathring{R} \cup \mathring{R}_k \ne \emptyset \implies R \subseteq R_k$
  (Let $P$ be a set of minimal non-trivial intersections of rectangles from $R_1 \cup R_2 \cup \{ Q \}$ that are inside $Q$.)

  \begin{lemma}
  If $R_1, \ldots, R_n$ have disjoint interiors and $\bigcup^n_{k=1} R_k \subseteq \bigcup^m_{k=1} \tilde{R}_k$, then $\sum_{k=1}^{n} |R_k| \leq \sum_{k=1}^{m} |\tilde{R}_k|$.
  \end{lemma}

  If $R \in P$ and $R \in Z_\varepsilon \ne \emptyset$, then $R \subseteq R_1 \in S_1$.
  We have a partition $P$ such that.

  \begin{align*}
    U(P,f) - L(P,f) & = \sum_{\substack{R \in P \\ R \cap Z_\varepsilon \ne \emptyset}} \left( \sup_R f - \inf_R f \right) |R| + \sum_{\substack{R \in P \\ R \cap Z_\varepsilon \ne \emptyset}} \left( \sup_R f - \inf_R f \right) |R| \\
                    & \leq \sum_{\substack{R \in P \\ R \cap Z_\varepsilon \ne \emptyset}} 2B \cdot |R| + \sum_{\substack{R \in P \\ R \cap Z_\varepsilon \ne \emptyset}} \varepsilon |R| \\
                    & \leq 2B \cdot \varepsilon + \varepsilon |Q|
  \end{align*}
  Since the choice of $\varepsilon > 0$ is arbitrary and $B, |Q|$ is fixed, we conclude that $f$ is Riemann integrable.

  Now suppose $f$ is Riemann integrable. Since $z = \bigcup_{n \geq 1} Z_{1/n}$, it is enough to show $L^*(Z_\varepsilon) = 0$ for $\varepsilon > 0$.\footnote{this is why we need Jordan outer measure}

  Given $\delta > 0$, choose a partition $P$ of $Q$ so that $U(P,f) - L(P,f) < \delta$.
  \begin{align*}
    U(P,f) - L(P,f) & = \sum_{R \in P} \left( \sup_R f - \inf_R f \right) |R| \\
                    & \geq \sum_{\substack{R \in P \\ \mathring{R} \cup Z_\varepsilon \ne \emptyset}} \left( \sup_R f - \inf_R f \right) |R| \\
                    & \geq \sum_{\substack{R \in P \\ \mathring{R} \cup Z_\varepsilon \ne \emptyset}} \varepsilon |R|
  \end{align*}
  and after rearranging, we obtain
  \[ \sum_{\substack{R \in P \\ \mathring{R} \cup Z_\varepsilon \ne \emptyset}} |R| \leq \frac{\delta}{\varepsilon}\]
  Therefore, we can cover $Z_\varepsilon$ with an arbitrary small volume of rectangles plus some boundaries of rectangles.
\end{proof}

\begin{exercise}
  Find a Riemann integrable $f : Q \to R$ with $J^* (z) > 0$. 
\end{exercise}

\subsection{Riemann v. Lebesgue}

[There is a missing graphic comparing the two]

\begin{theorem}
  For any metric space $M$, there is a complete metric space $M$ and isometry $i : M \to \bar{M}$ such that for continuous $f : M \to \bar{N}$, there is a unique $\bar{f} : \bar{M} \to \bar{N}$ with $f = \bar{f} \cdot i$. 
\end{theorem}

\begin{proof}[Sketch Proof]
  $\bar{M}$ is equivalence classes of Cauchy sequences in $M$.
\end{proof}

\begin{corollary}
  $\bar{M}$ is unique. 
\end{corollary}

$M = (C(Q), \| \cdot \|_{C^\circ})$ is a complete metric space and $\int_{\substack{Q \\ \text{Riemann}}} \cdot M \to R$. 

Consider isn



\section{September 7th, 2022}

\epigraph{``Fish fish fish eat eat eat''}{Charlie}

Readings: Chapter 1 of Rudin.

\subsection{Lebesgue Integration and Motivation}

[sketched Lebesgue integration cartoon.]

It can be defined as 
\begin{align*}
	\int f \approx \sum_{k=1}^{n-1} S_k L^* (\{x : S_k \leq f(x) \leq S_{k+1}\})
\end{align*} 

[\underline{missing sketch}]

We want to be able to "measure" all of the "constructable subsets of $\R^d$

We should be able to handle $1_C$ of $\frac{1}{2^{k+1} -1 }$ cantor set ($C \in [0,1]$) and similar functions

\begin{enumerate}
	\item constructable? Rectangles and unions and intersections of rectangles.
	\item Start with all \sout{rectangles} open and closed sets.
	\item If $A_1, A_2, \ldots$ constructable, then
		$\bigcup_{k=1}^{\infty}A_k$ and $\intersection_{k=1}^{\infty}A_k$ constructable.
	\item Lebesgue measure assigns volume to all such sets.
\end{enumerate}
 
\subsection{Metric Spaces}

\begin{definition}[Metric \& Metric Spaces]
	A function $d : X \times X \to \mathbb{R}$ on the Cartesian product of a set $X$ us a \underline{metric} if

\begin{enumerate}
	\item $d(x,y) \geq 0$
	\item $d(x,y) = 0 \iff x = y$
	\item $d(x,y) = d(y,x)$
	\item $d(x,y) \leq d(x,y) + d(y,z)$
\end{enumerate}

We call the pair $(X, d)$ a \underline{metric space.}
\end{definition}


\begin{definition}[Continuous Functions]
A function $f : X \to Y$ between metric spaces is continuous if, $\forall x \in X$ and $\varepsilon >0 $ there exists a $\delta >0$ such that for all $y \in X$ we have $d_X (x, y) < \delta \implies d_Y(f(x), f(y)) < \varepsilon$	
\end{definition}

A subset $V \subseteq X$ of a metric space $X$ is open if for ever $x \in V$ there is a $\delta >0 $ such that $B(x,\delta) = \{y \in X: d(x,y) < \delta\} \subseteq U$ 

\begin{theorem}
	A map $f:X \to Y$ between metric spaces is continuous if and only if $w \subseteq Y$ open implies $f^{-1}(w) \subseteq X$ is also open.
\end{theorem}

\begin{proof}
	(exercise) definition fun!
\end{proof}

\begin{definition}[Collections and Families] \phantom{text} \hfill
\begin{enumerate}
	\item A \emph{collection} is a set of subsets. 
	\item A \emph{family} of collections is a set of collections.
\end{enumerate}
\end{definition}

\begin{definition}[Topology]
	A collection $T$ of subsets of a set $X$ is a \textit{topology} if:
	\begin{enumerate}
		\item $\emptyset, X \in T$ 
		\item $T^{'} \subseteq T$ is finite, then $\bigcap\limits_{V \in T^{'}} V \in T$ 
		\item If $T^{'} \subseteq T,$ then $\bigcup\limits_{V \in T^{'}} V \in T$
	\end{enumerate}

	We call the pair $(X, T)$ a topological space.
\end{definition}


A function $f:X \to Y$ between topological spaces is continuous if $V \in T_y \implies f^{-1}(V) \in T_X$

\begin{claim*}
	A topology is a way of designating some subsets as open.	
\end{claim*}

\subsection{Sigma Algebras}

\begin{definition}[Algebra]
	A Collection $A$ of subsets of a set $X$ is an \emph{algebra} if 
	\begin{enumerate}
		\item $\emptyset, X \in A$
		\item If $B_1, B_2 \in A,$ then
			\begin{enumerate}
				\item $B_1 \cap B_2 \in A$
				\item  $B_1 \cup B_2 \in A$
				\item  $B_1^{c} = X \setminus B_1 \in A$
			\end{enumerate}
	\end{enumerate}
\end{definition}

\begin{definition}[Sigma Algebra]
	A Collection $S$ of subsets of a set $X$ is a \emph{$\sigma$-algebra} if 
	\begin{enumerate}
		\item $\emptyset, X \in S$
		\item If $B\in S \implies B^{c} \in S$
		\item If $S^{'} \subseteq S$ is countable then 
			$\bigcup\limits_{B \in S^{'}} B \in S$ and $\bigcap\limits_{B \in S^{'}} B \in S$
	\end{enumerate}
note: $\bigcap\limits_{B \in S^{'}} = (\bigcup\limits_{B \in S^{'}} )^{c}$

We call the pair $(X, S)$ a measurable space.
\end{definition}

\begin{definition}
	A function $f : X \to Y$ between measurable spaces is measurable if $B \in S_Y \implies f^{-1}(B) \in S_X$
\end{definition}

\begin{claim*}
	Not all $\sigma$ algebras are topologies but all important ones are.
\end{claim*}

\begin{example}
	\begin{enumerate}
		\item The power set $\mathcal{P} (X) = \{y : y \subseteq X\}$ is both a topology and a $\sigma$-algebra
		\item $\{\emptyset, X\}$ is both a topology and a $\sigma$-algebra
		\item $\{\emptyset, B, B^{c}, X\}$ is both a topology and a $\sigma$-algebra
	\end{enumerate}
\end{example}

\begin{theorem}
	For a collection $C$ of subsets of $X$ there is a smallest topology $T$ and $\sigma$-algebra $S$ with $T \supseteq C$ and $S \supseteq C$.
	Denote these by $\tau(C)$ and $\sigma(C)$ respectively.
\end{theorem}

\begin{proof}
	Let $\sigma(C) = \bigcap\limits_{S \supseteq C} S$.

	Note that $\mathcal{P}(X) \supseteq C$, the intersection is non-trivial.

	\underline{Check the Axioms}:
	\begin{enumerate}
		\item If $S \supseteq C$ is $\sigma$-algebra, then $\emptyset,X \in S$.

			Therefore  $\emptyset, X \in \bigcap \limits_{S \supseteq C \text{ is a $\sigma$-algebra}}S $
		\item $X \in \sigma(C)$ thus
			 \begin{align*}
			&\implies X \in S \text{ for } S \subseteq C \text{ a } \sigma\text{-algebra} \\
			&\implies X^{c} \in S \text{ for } S \subseteq C \text{ a } \sigma\text{-algebra} \\
			&\implies X^{c} \in \sigma(C)
			\end{align*} 

	\item $S^{'} \subseteq \sigma (C)$ countable
		 \begin{align*}
		&\implies S^{'} \subseteq S \text{ for } S \supseteq C \text{ a $\sigma$-algebra} \\
		&\implies \bigcup_{B \in S^{'}} B \in S \text{ for } S \supseteq C \text{ a $\sigma$-algebra} \\
		&\implies \bigcup_{B \in S^{'}} B \in \sigma(C)
		\end{align*} 
	\end{enumerate}

	$\sigma(C)$ is the smallest because it is contained in every $\sigma$-algebra $S \supseteq C$.

	 $\tau(C)$ is defined similarly.
\end{proof}

\begin{definition}[Measurable Function]
	We call a function $f:X \to Y$ from a measurable space to a topological space \textit{measurable} if $W \in T_Y$ implies $f^{-1}(W) \in S_X$
\end{definition}

\begin{theorem}
	This is equivalent to $f$ being a measurable function from $(X, S_X)$ to $(S, \sigma(T_Y))$
\end{theorem}

\begin{example}
	If $(X,T)$ is a topological space, then call elements of $\sigma(T)$ the \emph{Borel} subsets of $(X,T)$	
\end{example}

The Borel subsets of $\mathbb{R}^d$ and the "constructable" sets from before.

\begin{theorem}
	if $(Y, S_Y)$ is a measurable space and $f: X \to Y$ then there is a smallest $\sigma$-algebra on $X$ that makes $f$ measurable.
\end{theorem}
\begin{proof}
	Check $S_X = \{f^{-1}(B) : B \in S_Y\}$ is a $\sigma$-algebra.
\end{proof}

\begin{theorem}
	If $(X, S_X)$ is a measurable space and $f:X \to Y$ then there is a largest $\sigma$-algebra on $Y$ that makes $f$ measurable.
\end{theorem}

\begin{proof}
	Check $S_Y = \{B \subseteq Y: f^{-1}(B) \in S_X \}$ is a $\sigma$-algebra.
\end{proof}

\begin{definition}
	Basic Borel sets are open or closed.
	\begin{enumerate}
		\item $B_0$ sets are open or closed subsets of $(X,T)$
		\item $B_{K+1}$ sets of countable unions of countable intersections of sets in $B_k$
	\end{enumerate}
\end{definition}

\begin{example}
	$E \in B_4$ then

	 \begin{align*}
		 &\bigcap_i \bigcup_j \bigcap_k F_{i,j,k} \quad \text{\tiny (F open)} \qquad \text{Or} \\
		 &\bigcup_i \bigcap_j \bigcup_k G_{i,j,k} \quad \text{\tiny (G closed)}
	\end{align*}
	
\end{example}


\section{September 12th, 2022}

\epigraph{Shirt of the day: ``I have eated all the dinner''}{Charlie}

Last Time: Basic results for measurable spaces $(X, S)$.

\subsection{Measurable Space}

\begin{theorem}
  There is a smallest $\sigma$-algebra $\sigma(c)$ containing any collection $C$ of subsets of a set $X$.
\end{theorem}

\begin{lemma}
	If $(X,S_x), (Y,S_y)$ are measurable spaces, $f: X \to Y$ is $(S_x, S_y)$ measurable, $S_x \supseteq S_y$ and $S_y\supseteq S_y$ s.t. $S_x' \subseteq S_X$ and $S_Y' \subseteq S_Y$, then $f$ is $(S_x', S_y')$ measurable.
\end{lemma}

\begin{theorem}
  If $(X, S)$ is a measurable space and $f: X \to Y$, then
	\[
		f^*(s) = \left\{A \subseteq Y \mid f^{-1}(A) \in S \right\}.
	\]
	is the largest $\sigma$-algebra such that $f$ is $(S, f^*(S))$-measurable
\end{theorem}

\begin{theorem}
	If $(Y,S)$ is a measurable space and $f : X \to Y$, then
	 \begin{align*}
		 f^*(S) &= \{f^{-1}(A) : A \in S\}
	\end{align*}
	is the smallest $\sigma$-algebra such that $f$ is $(f^*(S), S)$- measurable.
\end{theorem}

\begin{theorem}
	If $(X, T_X)$ $(Y,T_Y)$ are topological spaces, and  $f : X \to Y$ is $T_X, T_Y$-continuous, then $f$ is also $(\sigma(T_X), \sigma(T_Y))$-measurable.
\end{theorem}

\begin{proof}
	$f$ is $(T_X,T_Y)-continuous$
	\begin{align*}
		&\iff \{f^{-1}(A) : A \in T_Y\} \subseteq T_X \\
		&\iff \{f^{-1}(A) : A \in \sigma(T_Y)\} \subseteq \sigma(T_X) \\
		&\iff f \text{ is } (\sigma(T_X), \sigma(T_Y)) \text{ -- measurable.}
	\end{align*}
\end{proof}

% Abuse of notation:
\begin{abuse}
If $(X, S_X)$ is a measurable space and  $(Y, T_Y)$ is a topological space, then call $f : X \to Y$
$(S_X, T_Y)$-measurable $\iff f$ is $(S_X, \sigma(T_Y))$- measurable.
\end{abuse}

\begin{theorem}
	$f : X \to Y$ is $(S_X, \sigma(T_Y))$- measurable $\iff \forall A \in T_Y$ we have $f^{-1}(A) \in S_X$.
\end{theorem}

\begin{proof}
	Since $T_Y \subseteq \sigma(T_Y)$
	then $\implies$ directions trivial.

	For the $\impliedby$ direction, observe that
	\begin{align*}
		&\{f^{-1}(A) : A \in T_Y\} \subseteq S_X \\
		&\implies \sigma(\{f^{-1}(A) : A \in T_Y\}) \subseteq \sigma(S_X) = S_X \\
		&\iff \sigma(\{f^{-1}(A) : A \in \sigma(T_Y)\}) \subseteq  S_X \text{--measurable}
	\end{align*}
\end{proof}

If $(X,T)$ is a topological space, then call $\sigma(T)$ the Borel subsets of  $X.$

If $(X, T_X)$ and $(Y,T_Y)$ are topological spaces, then $f:X \to Y$ is measurable.

If it is  $(\sigma(T_X), \sigma(T))$-measurable.

comeback too

\begin{definition}
	The extended Reals.
	$[-\infty,\infty]$ is $\mathbb{R} \cup \{-\infty, +\infty\}$.

	Add new open sets $[-\infty, a)$ and $(a, +\infty]$
	and the usual arithmetic rules
	\begin{align*}
		a + \infty = \infty, \quad a\cdot \infty = \infty, \quad a > 0
	\end{align*} etc defined by demanding $+$ and $\cdot$ extend continuously to
	\begin{align*}
		&[-\infty, \infty]^2 \setminus \{(-\infty, \infty), (\infty, -\infty)\} \text{ and } \\
		&[-\infty, \infty]^2 \setminus \{(0, \pm \infty), (\pm \infty, 0)\}
	\end{align*}
\end{definition}

\begin{corollary}
	If $(X,S)$ measurable, then $f : X \to [-\infty, \infty]$ is measurable
	$iff$ $f^{-1}((a,\infty]) \in S$ for all $a \in [-\infty,\infty]$
\end{corollary}

\begin{proof}
	The main point is
	\begin{align*}
		&\sigma(\{(a,\infty] : a \in [-\infty, a)\}) \\
		= &\sigma(\{ A \subseteq [-\infty, \infty] : A \text{ open} \})
	\end{align*}
\end{proof}

\begin{theorem}
	If $(X,S)$ is measurable space,
	$f_1, \ldots, f_d : X \to \mathbb{R}$ are measurable and $g:\mathbb{R}^d \to \mathbb{R}$ continuous, then $g \circ (f_1, \ldots, f_d) : X \to \mathbb{R}$ is measurable.
\end{theorem}

\begin{proof}
	It is enough to show that $(f_1, \ldots, f_d) : X \to \mathbb{R}^d$ is measurable. But given $A \subseteq \mathbb{R}^d$ is open, we need to show $f^{-1}(A) \in S$.

	Write $A = \bigcup_{k=1}^{\infty} R_k$ with $R_k$ open rectangles. It is enough to show $f^{-1}(R) \in S$ for any open rectangle $\mathbb{R} \subseteq \mathbb{R}^d$.

	Write $R = (a,b) \times \ldots \times (a_d, b_d)$ and compute
	$f^{-1}(R) =  f^{-1}((a,b)) \cap \ldots \cap f^{-1}((a_d, b_d)) \in S$
	by measurability of $f_1, \ldots , f_d$
\end{proof}

\begin{corollary}
	If $(X,S)$ is a measurable space and $f,g : X \to [-\infty, \infty]$ is measurable,
	then $f+g$, $f\circ g$, $\min\{f,g\}$, $\max\{f,g\}$, and $|f|$ are all measurable (when defined).
\end{corollary}

\begin{theorem}
	If $(X,S)$ measurable space and $f_1, f_2, \ldots : X \to [-\infty, \infty]$ measurable, then $f = \sup_k f_k$ is measurable.
\end{theorem}

\begin{proof}
	The main part is
	\begin{align*}
		f^{-1}((a, \infty]) = \bigcup_{k=1}^{\infty} f_k^{-1}((a,\infty]) \in S
	\end{align*}
\end{proof}

\begin{corollary}
	\begin{align*}
		\inf_k f_k, \, \lim \sup_k f_k, \, \text{and } \, \lim \inf_k f_k
	\end{align*} are also measurable.
\end{corollary}

\begin{proof}
	\begin{align*}
		&\inf_k f_k = - \sup_{k} (-f_k) \\
		&\lim_{R} \sup f_k = \inf_{k} (\sup_{j} f_{k+j}) \\
		&\lim_{k}\inf f_k = - \lim_{k} \sup (-f_k)
	\end{align*}
\end{proof}

\begin{corollary}
	If $(X,S)$ measurable space, $f_k : X \to (-\infty, \infty]$ measurable for  $k \geq 0$
	 \begin{align*}
	    f (x) = \lim_{k \to  \infty} f_k (x) \, \forall x \in X
	\end{align*} then $f$ is measurable.
\end{corollary}

\begin{proof}
	 \begin{align*}
	f = \lim_{k} f_k \implies f = \lim_{k} \sup f_k
	\end{align*}
\end{proof}

\underline{Review:} From Real analysis

check
\begin{enumerate}
	\item Locally uniform limits of continuous functions are continuous.
	\item Pointwise limits of continuous functions need not be continuous.
\end{enumerate}

\subsection{Positive Measure}

\begin{definition}[Positive Measure]
	A positive measure is a function $m : S \to [0, \infty]$ on a $\sigma$-algebra $S$ on a set $X$ such that $m(\emptyset) = 0$
	and $m( \bigcup_{k=1}^{\infty} A_k) = \sum_{k=1}^{\infty} m(A_k)$ when
	$A_1, A_2, \ldots \in S$ disjoint.

	{\tiny note $m(\emptyset) = 0$ allows us to exclude $m$ which gives everything $+\infty$ measure.}

	Call the triple $(X,S,m)$ a positive measure space.
\end{definition}

\begin{example}
	\begin{enumerate}
		\item $X = \{x_1, \ldots, x_n\}$,
			\begin{align*}
				S = \mathcal{P}(X) \text{ and } m(A) = \#A \quad \text{\tiny the counting measure}
			\end{align*}
		\item $X$, arbitrary
			\begin{align*}
				S = \mathcal{P}(X) \text{ and } m(A) =
				\begin{cases}
					\#A & A\text{-finite} \\
					+\infty & A\text{-infinite} \\
				\end{cases}
			\end{align*}

		\item The Dirac Measure:
			$X$ arbitrary, $S$ arbitrary, and $X_0 \in X$ and
			\begin{align*}
				m(A) =
				\begin{cases}
					1 & x_0 \in A \\
					0 & x_0 \notin A
				\end{cases}
			\end{align*}
	\end{enumerate}
\end{example}


\begin{definition}[Basic Properties]
	\begin{enumerate}
		\item If $A,B \in S$  and $A \subseteq B$, then $m(A) \leq m(B)$
		\item If  $A_1, \ldots, A_n \in S$ disjoint, then
			$m(A_1 \cup A_2 \cup \ldots \cup A_n) = m(A_1) + \ldots + m(A_n)$
		\item If  $A_k \in S$ and $A_1 \subseteq A_2 \subseteq \ldots$, then
			\begin{align*}
				m(\bigcup_{k=1}^{\infty} A_k ) = \lim_{k \to \infty} m(A_k)
			\end{align*}
		\item If $A_k \in S$, and $A_1 \supseteq A_2 \supseteq \ldots$ and $m(A_1) < \infty$,
			then
			 \begin{align*}
				m(\bigcap_{k=1}^{\infty} A_k) = \lim_{k \to \infty}  m(A_k)
			\end{align*}
	\end{enumerate}
\end{definition}

\begin{proof}(of properties)
	\begin{enumerate}
		\item[2] let $A_k = \emptyset$ for $k >n$. Now $A_1, A_2, \ldots$ disjoint and
			\begin{align*}
				m(A_1 \cup \ldots \cup A_n) &= m(\bigcup_{k=1}^{\infty} A_k ) \\
				= \sum_{k=1}^{\infty} m(A_k) &= m(A_1) + \cdots + m(A_n)
			\end{align*}
		\item[1] Write $B = A \cup (B \setminus A)$, and $B \setminus A \in S$ since $S$ is a $\sigma$-algebra
			\begin{align*}
				m(B) = m(A) + m(B \setminus A) \geq m(A)
			\end{align*}
		\item[3] Compute (Check)
			\begin{align*}
				m(\bigcup_{k=1}^{\infty} A_k) &= m(A_1 \cup \bigcup_{k=1}^{\infty} (A_{k+1} \setminus A_k) \\
											  &= m(A_1) + \sum_{k=1}^{\infty} m(A_{k+1} \setminus A_k) \\
											  &= m(A_1) + \lim_{n \to \infty}  \sum_{k=1}^{n} m(A_{k+1} \setminus A_k) \\
											  &= \lim_{n \to \infty} \left(m(A_1) + \sum_{k=1}^{n} m(A_{k+1} \setminus A_k) \right) \\
											  &= \lim_{n \to \infty} m \left(A_1 \cup \bigcup_{k=1}^{\infty} (A_{k+1} \setminus A_k) \right) \\
											  &= \lim_{n \to \infty} m (A_{n+1})  \\
			\end{align*}
		\item[4] Let $B_k = A_1 \setminus A_k \in S$. Observe: $B_1 \subseteq B_2 \subseteq \ldots$

			\underline{Conclude}:
			\begin{align*}
				m(\bigcup_{k=1}^{\infty} B_k) = \lim_{n \to \infty} m(B_n)
			\end{align*}
			Rewrite as:
			\[
				\quad m(A_1 \setminus \bigcap_{k=1}^{\infty} A_k) =
				\lim_{n \to \infty} m(A_1 \setminus A_n).
			\]
			Rewrite as:
			\begin{align*}
				m(A_1) - m(\bigcap_{k=1}^{\infty} A_k) = m(A_1) - \lim_{n \to \infty} m(A_n)
			\end{align*}
			Since $m(A_1) < \infty$ we can cancel.
	\end{enumerate}
\end{proof}

\begin{example}
	Let $m$ be the counting measure on  $\mathbb{N}$ and consider $A_n = \{n, n+2, n+2, \ldots \}$.
	Then $A_1 \supseteq A_2 \supseteq \ldots$ , $m(A_k) = \infty$ and
	\begin{align*}
		m(\bigcap_{k=1}^{\infty} A_k ) = m(\emptyset) = 0
	\end{align*}
\end{example}




\end{document}
