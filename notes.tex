\documentclass{lnotes}
\usepackage{amsfonts}
% Bibliography
\usepackage[
	style = alphabetic ,
]{biblatex}
\addbibresource{references.bib}

% Packages
\usepackage{epigraph}
\usepackage{wasysym}
\usepackage[osf]{libertineRoman}
\usepackage{tikz-cd}

% Commands
\DeclareMathOperator*{\osc}{osc}


% Document Information
\title{Measure Theory and~Integration}
\course{\textsc{math} 320}
\place{Yale University}
\term{Fall}
\year{2022}

\blurb{
	These are lecture notes for \textsc{math} 320, ``Measure Theory and Integration'' taught by Charlie Smart at Yale University during the fall of 2022.
	These notes are not official, and have not been proofread by the instructor for the course.
	These notes live in my lecture notes respository at 
	\[\text{\url{https://github.com/Eph97/Eph97/Math320}.}\]
	If you find any errors, please open a bug report describing the error, and label it with the course identifier, or open a pull request so I can correct it.
}

\begin{document}

\section*{Syllabus}

\begin{center}
\begin{tabular}{@{}rp{10cm}@{}}
\toprule 
\textbf{Instructor} &  Prof. Charlie Smart, \url{charlie.smart@yale.edu} \\
\textbf{Lecture} & MW 11:35 \textsc{am}--12:50 \textsc{pm}, Humanities Quadrangle 207 \\
\textbf{OH} & TBD \\
% \textbf{Recitation} & TBA \\
\textbf{Textbook} & \fullcite{Rudin} \\
\textbf{Midterms} & Mon Oct 3, 2022 (In-Class) \\
				  & Mon Nov 7, 2022 (In-Class) \\
\textbf{Final} & Tue Dec 20, 2022 at 2pm (Location: TBD)\\
\bottomrule 
\end{tabular} \\[3ex]
\end{center}

\subsection*{Course Description}
This course will provide a comprehensive introduction to measure theory.  We will start from real analysis and build from there.  The following topics will be covered.
\begin{itemize}
  \item Measure spaces
  \item Integrals on general measure spaces
  \item Product measures
  \item Integral representation of absolutely continuous measures
  \item Integral representation of linear functions
  \item Lebesgue measure
  \item Hausdorff measure and fractals
  \item Applications in dynamics
  \item Applications in probability theory
\end{itemize}

\subsection*{Course Format}
Approximately 2/3 of the material with be presented in lectures while the remaining 1/3 will be covered in the weekly homework assignments.

\subsection*{Prerequisites}
Math 305 or equivalent.

\subsection*{Assessments and Grading}
There will be ten homework assignments, two in-class midterms, and an in-person final, comprising 30\%, 40\%, and 30\%, respectively, of the total grade.


\section{August 31st, 2022}

\epigraph{``How do mathematicians measure things?''}{Charlie}

Following the universal rule for first class, we covered the syllabus and logistics. We then dove into review on the Riemann integral to motivate measure theory. Prof. Smart seemed much more comfortable talking about math than logistics.

\subsection{Course Plan}
\begin{itemize}
	\item Review Riemann Integration
	\item Abstract Measure Theory and Integration (Majority of class)
	\item Applications in Probability
	\item Applications in Dynamics
	\item Some brief applications in Geometry
\end{itemize}

\begin{problem} The Riemann Integral
\item Works for most functions of interest but notably...
	\begin{enumerate}
		\item It is not closed under important limits. 
		\item It is hard to generalize to new geometries.
	\end{enumerate}
\end{problem}

\begin{definition}[Riemann Integral]
	We can define the Riemann integral as $\int_{{\Q}}^{{}} {f}$ of a bounded $f : \Q \to \R$
	defined in closed rectangles
	$\Q = [a_1,b_1] \times [a_2,b_2] \times \ldots \times [a_d,b_d] \subseteq \R$. We use Darboux sums
\end{definition}

\vline
\begin{definition}[Partition]
	A \underline{partition} of $\Q$	is a finite set of \emph{closed} rectangles whose \underline{interiors} are disjoint and whose union is $\Q$.
\end{definition}

\vline

\begin{definition}[Darboux Sums]
	The upper and lower Darboux sums are 
	\[
		U(p,f) = \sum_{R \in P} \sup_R f \cdot |R|
	\] 
	and
	\[
		L(p,f) = \sum_{R \in P} \inf_R f \cdot |R|.
	\]	
\end{definition}

\begin{lemma}
	For any 2 partitions $P_1$ and $P_2$ of  $\Q$,  $L(P_1,f) \leq U(P_2, f)$
\end{lemma}

\begin{solution}
	We have
	\begin{align*}
		L(P_1, f) &= \sum_{R \in P_1}^{}  \inf_{R} f \cdot |R| = \sum_{R_1 \in P_1}^{} \inf f
		\sum_{R_2 \in P_2}^{} |R_1 \intersection R_2 | \\
				  &= \sum_{\substack{R_1 \in P_1 \\ R_2 \in P_2 \\ |R_1 \intersection R_2 | > 0} }^{} \inf_{R_1} f \cdot |R_1 \intersection R_2 |
				  \leq \sum_{\substack{R_1 \in P_1 \\ R_2 \in P_2 \\ |R_1 \intersection R_2 | > 0} }^{} \inf_{R_1 \intersection R_2} f \cdot |R_1 \intersection R_2 | \\
				  &\leq \sum_{\substack{R_1 \in P_1 \\ R_2 \in P_2 \\ |R_1 \intersection R_2 | > 0} }^{} \sup_{R_1 \intersection R_2} f \cdot |R_1 \intersection R_2 |
				  \leq \sum_{\substack{R_1 \in P_1 \\ R_2 \in P_2 \\ |R_1 \intersection R_2 | > 0} }^{} \sup_{R_2} f \cdot |R_1 \intersection R_2 | \\
				  &= \sum_{R_2 \in P_2}^{} \sup_{R_2} f \sum_{\substack{R_1 \in P_1 \\ |R_1 \intersection R_2 | > 0}}^{} |R_1 \intersection R_2 |  = \sum_{R_2 \in P_2}^{} \sup_{R_2} f \cdot |R_2| = U(P_2, f)
	\end{align*}
\end{solution}


Note we say a bounded $f : \Q \to \R$ is Riemann integrable when  
\[\sup_p L(p,f) = \inf_p U(p,f)\]
in which case we let  $\int_{\Q} f$ denote the common value.


\begin{theorem}
	If $f : \Q \to \R$ is continuous, then f is Riemann integrable.
\end{theorem}

\begin{solution}
	$\Q$ is compact, so  $f$ is uniformly cont. Thus, given a $\varepsilon >0 $, choose a $\delta >0$ so  $|x-y| < \delta \implies |f(x) - f(y)| > \varepsilon$. Let $p$ be any partition of  $\Q$ whose rectangles have diameter  $< \delta$ \newline
	Next compute
	\begin{align*}
		0 &\leq \inf_{p_1} U(p_1, f) - \sup_{p_2} L(p_2, f) \leq U(p_1, f) - L(p_1, f) \\
		  &= \sum_{R \in P}^{} \left( \sup_{R} f - \inf_{R}f \right) \cdot |R| \\
		  &\leq \sum_{R \in P}^{} \varepsilon \cdot |Q|
	\end{align*}
	but since $\varepsilon > 0$ is arbitrary,  $f$ is Riemann integrable.
\end{solution}

\begin{example}
	If $f:[0,1] \to \R$ and $f(x) =
	\begin{cases}
		1 & x\notin \Q \\
		0 & x\in \Q
	\end{cases}$.
	Then $f$ is not Riemann integrable.
	\begin{solution}
		For any $P$, we can see  $U(p,f) = 1$ and  $L(p,f) = 0$
	\end{solution}
\end{example}

\subsection{Measures}

We have two notions of volume
\begin{enumerate}
	\item The outer Jordan measure of a set $x \subseteq \R^d$ is
			\[
				J^* (x)
				= \inf \left\{ \sum_{k=1}^{n} |Q_k| : n \geq 1, Q_1, \ldots, Q_n \subseteq R^n
					\text{ closed rectangles and }
				x \subseteq \union_{k=1}^{b} Q_k \right\}
			\]
	\item Outer Lebesgue measure of a set $X \subseteq \R^n$ is
		\begin{align*}
			L^*(x) = \inf \left\{ \sum_{k=1}^{\infty}|Q_k| : Q_1, Q_2, \ldots \subseteq R^n  \text{ closed rectangles and }
			x \subseteq \union_{k=1}^{\infty} Q_k \right\}
		\end{align*}

		We have some important properties for these measures, namely

		\begin{enumerate}
			\item $L^*(x) \leq J^* (x)$
			\item If  $X$ is compact, then $L^*(x) = J^*(x)$
		\end{enumerate}
		\begin{solution}
			Suppose $x \subseteq \union_{k=1}^{\infty} Q_k$, pick $\varepsilon > 0$ and let  $(1 + \varepsilon)Q_k$ be the $(1 + \varepsilon)$-dilations of $Q_k$. Note $(1+ \varepsilon ) Q_k$ are open cover of  $x$ so $x \subseteq \sum_{k=1}^n (1+\varepsilon) Q_k$ and
			\begin{align*}
				\sum_{k=1}^{n} |(1 + \varepsilon) Q_k| = (1+\varepsilon)^d \sum_{k=1}^{n}|Q_k| \leq (1 + \varepsilon)^d \sum_{k=1}^{\infty} | Q_k|
			\end{align*}
		\end{solution}

	\item $L^* (\union^{\infty} x_k ) \leq \sum_{}^{\infty} L^* (X_{k})$

		\begin{solution}
			sketch by "Dovetail" (Come back to)
		\end{solution}
\end{enumerate}

\begin{example}
	The set $x = \Q \intersection [0,1]$ has  $L^*(x) = 0$ and  $J^*(x) = 1$ write  $x = \{q_1, q_2, \ldots \}$ then let $Q_k = [q_k =  \frac{\varepsilon}{2^k}, q_k + \frac{\varepsilon}{2^k}]$
	and observe that $x \subseteq \union_{1}^{\infty} Q_k$ and $\sum_{k=1}^{\infty} |Q_k| = 2 \varepsilon$
\end{example}

\begin{example}
	Let $X \subseteq [0,1]$ be generalized Cantor set. Let $1_{\C}$ be the indicator function of $\C$.

	Observe $1_ {\C}$ is not Riemann int on $[0,1]$. Indeed, C is the set of discontinuities of $1_{\C}$ and $L^*(\C) = \frac{1}{2}$. Moreover $1_{\C}(x) = \lim_{n\to \infty} 1_{\C_n} (x)$ and
	$0 \leq 1_{\C} \leq 1_{C_{n+1}} \leq 1_{\C_{n}} \leq 1$ and
	$\lim_{n \to \infty} \int_{[0,1]} 1_{C_{n}} = \frac{1}{2} $
\end{example}

\section{2022-09-02}

\subsection{Riemann Integration and Measures}
Last time: Definition of a Riemann Integral $\int_Q f$ of bounded $f : Q \to R$ on closed rectangle $Q
\subseteq \R^d$

\begin{theorem}
  A bounded $f : Q \to R$ is Riemann integrable if and only if the set $Z = \{ x \in Q \mid f
  \text{ not continuous of } x \}$ has $L^* (z) = 0$.
\end{theorem}

\begin{definition}[oscillation]
  The oscillation of $f : Q \to R$ at $x \in Q$ is
\[\osc(f,x) = \inf_{\delta > 0} \sup_{\substack{y, z \in Q \\ | y - x | < \delta \\ |z - x| < \delta}} |f(y) - f(z)| \]
\end{definition}

\begin{lemma}
  $f$ is continuous at $X$ if and only if $\osc (f,x) = 0$.
\end{lemma}

\begin{lemma}
  The set $z_\varepsilon = \{ x \in Q : \osc(f,x) \geq \varepsilon \}$ is closed (compact) for $\varepsilon > 0$.
\end{lemma}

\begin{proof}
  Since $\R^d$ is a metric space and closed, it is also sequentially closed.
  Suppose $x_1, x_2, \ldots \in Z_\varepsilon$ and $x = \lim_{n \to \infty} x_n$.
  Since $x_n \in Z_\varepsilon$, we can choose $y_n, z_n$ so that
  \[|y_n - x_n| < \frac{1}{n}, |z_n - x_n| < \frac{1}{n}, \text{ and } |f(y_n) - f(z_n)| \geq \varepsilon - \frac{1}{n}\]
  For any $\delta > 0$, choose $n > 0$ such that $\frac{1}{n} > \frac{\delta}{2}$ and $|x_n - x| < \frac{\delta}{2}$.
  We now have
  \[|y_n - x_n| < \delta, |z_n - x_n| < \delta, \text{ and } |f(y_n) - f(z_n)| \geq \varepsilon - \frac{\delta}{2}\]
  We now conclude that $x \in Z_\varepsilon$.
\end{proof}

Note on notation: The set of discontinuities $Z$ can be the countable union of compact sets, which is represented by,
\[ Z = \bigcup_{n \geq 1} Z_{\frac{1}{n}} \]

\begin{theorem}
  A bounded $f : Q \to R$ is Riemann integrable if and only if $Z = \{ x \in Q : \osc (f,x) > 0 \}$ has $L^*(x) = 0$.
\end{theorem}

\begin{proof}
  Assume $|f| \leq B$. First, suppose $L^*(z) = 0$.
  Let $Z_\varepsilon = \{ x \in Q : \osc (f,x) \geq \varepsilon\}$.
  Since $Z_\varepsilon \subseteq Z$, $L^*(Z_\varepsilon) = 0$.
  Since $Z_\varepsilon$ is compact, $J^*(Z_\varepsilon) = 0$.
  Choose a finite set of rectangles $S$ such that $Z_\varepsilon \subseteq \bigcup_{R_1 \in S_1} \mathring{R_1}$ and $\sum_{R_1 \in S_1} |R_1| < \varepsilon$.

  Let $W_\varepsilon = Q \setminus \bigcup_{R_1 \in S_1} \mathring{R}$ and observe that $W_\varepsilon$ is compact and $\osc(f,x) < \varepsilon$ for all $x \in W_\varepsilon$.
  \footnote{Note: this helps cover the boundary condition, which Spivak incorrectly omits in his book.}
  For every $x \in W_\varepsilon$, we can choose rectangle $R$ so $x \in \mathring{R}$ and $\sup_R f - \inf_R f < \varepsilon$.
  Since $W_\varepsilon$ is compact, they are a finite set of rectangles $S_2$ such that
  \[ W_\varepsilon \subseteq \bigcup_{R_2 \in S_2} \mathring{R}_2 \quad \text{and} \quad \sup_{R_2} f - \inf_{R_2} f < \varepsilon \quad \forall R_2 \in S \]

  Let $P$ be a partition of $Q$ so that for $R \in P$ and $R_k \in S_k$, where $k = 1,2$ have $\mathring{R} \cup \mathring{R}_k \ne \emptyset \implies R \subseteq R_k$
  (Let $P$ be a set of minimal non-trivial intersections of rectangles from $R_1 \cup R_2 \cup \{ Q \}$ that are inside $Q$.)

  \begin{lemma}
  If $R_1, \ldots, R_n$ have disjoint interiors and $\bigcup^n_{k=1} R_k \subseteq \bigcup^m_{k=1} \tilde{R}_k$, then $\sum_{k=1}^{n} |R_k| \leq \sum_{k=1}^{m} |\tilde{R}_k|$.
  \end{lemma}

  If $R \in P$ and $R \in Z_\varepsilon \ne \emptyset$, then $R \subseteq R_1 \in S_1$.
  We have a partition $P$ such that.

  \begin{align*}
    U(P,f) - L(P,f) & = \sum_{\substack{R \in P \\ R \cap Z_\varepsilon \ne \emptyset}} \left( \sup_R f - \inf_R f \right) |R| + \sum_{\substack{R \in P \\ R \cap Z_\varepsilon \ne \emptyset}} \left( \sup_R f - \inf_R f \right) |R| \\
                    & \leq \sum_{\substack{R \in P \\ R \cap Z_\varepsilon \ne \emptyset}} 2B \cdot |R| + \sum_{\substack{R \in P \\ R \cap Z_\varepsilon \ne \emptyset}} \varepsilon |R| \\
                    & \leq 2B \cdot \varepsilon + \varepsilon |Q|
  \end{align*}
  Since the choice of $\varepsilon > 0$ is arbitrary and $B, |Q|$ is fixed, we conclude that $f$ is Riemann integrable.

  Now suppose $f$ is Riemann integrable. Since $z = \bigcup_{n \geq 1} Z_{1/n}$, it is enough to show $L^*(Z_\varepsilon) = 0$ for $\varepsilon > 0$.\footnote{this is why we need Jordan outer measure}

  Given $\delta > 0$, choose a partition $P$ of $Q$ so that $U(P,f) - L(P,f) < \delta$.
  \begin{align*}
    U(P,f) - L(P,f) & = \sum_{R \in P} \left( \sup_R f - \inf_R f \right) |R| \\
                    & \geq \sum_{\substack{R \in P \\ \mathring{R} \cup Z_\varepsilon \ne \emptyset}} \left( \sup_R f - \inf_R f \right) |R| \\
                    & \geq \sum_{\substack{R \in P \\ \mathring{R} \cup Z_\varepsilon \ne \emptyset}} \varepsilon |R|
  \end{align*}
  and after rearranging, we obtain
  \[ \sum_{\substack{R \in P \\ \mathring{R} \cup Z_\varepsilon \ne \emptyset}} |R| \leq \frac{\delta}{\varepsilon}\]
  Therefore, we can cover $Z_\varepsilon$ with an arbitrary small volume of rectangles plus some boundaries of rectangles.
\end{proof}

\begin{exercise}
  Find a Riemann integrable $f : Q \to R$ with $J^* (z) > 0$. 
\end{exercise}

\subsection{Riemann v. Lebesgue}

[There is a missing graphic comparing the two]

\begin{theorem}
  For any metric space $M$, there is a complete metric space $M$ and isometry $i : M \to \bar{M}$ such that for continuous $f : M \to \bar{N}$, there is a unique $\bar{f} : \bar{M} \to \bar{N}$ with $f = \bar{f} \cdot i$. 
\end{theorem}

\begin{proof}[Sketch Proof]
  $\bar{M}$ is equivalence classes of Cauchy sequences in $M$.
\end{proof}

\begin{corollary}
  $\bar{M}$ is unique. 
\end{corollary}

$M = (C(Q), \| \cdot \|_{C^\circ})$ is a complete metric space and $\int_{\substack{Q \\ \text{Riemann}}} \cdot M \to R$. 

Consider isn





\end{document}
