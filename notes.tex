\documentclass{lnotes}
\usepackage{amsfonts}

% Bibliography
\usepackage[style = alphabetic]{biblatex}
\addbibresource{references.bib}

% Packages
\usepackage{epigraph}
\usepackage[osf]{libertineRoman}
\usepackage{biolinum} % nice sans serif font
\usepackage{ulem} % for strikethrough text
% \usepackage{tikz-cd}
% \usepackage{dsfont}
% \newcommand{\1}{\mathds{1}}
% \usepackage{wasysym}

% Commands
\newcommand{\contains}{\supseteq}
\newcommand{\ts}{\textsuperscript}

% Operators
\DeclareMathOperator*{\diam}{diam}
\DeclareMathOperator*{\osc}{osc}

% Document Information
\title{Measure Theory and~Integration}
\course{\textsc{MATH} 320}
\place{Yale University}
\term{Fall}
\year{2022}

\blurb{
	These are lecture notes for \textsc{math} 320, ``Measure Theory and Integration'' taught by Charlie Smart at Yale University during the fall of 2022.
	These notes are not official, and have not been proofread by the instructor for the course.
	These notes live in my lecture notes respository at
	\[\text{\url{https://github.com/Eph97/Math320}.}\]
	If you find any errors, please open a bug report describing the error, and label it with the course identifier, or open a pull request so I can correct it.
}

\begin{document}

\section*{Syllabus}

\begin{center}
\begin{tabular}{@{}rp{10cm}@{}}
\toprule
\textbf{Instructor} &  Prof. Charlie Smart, \url{charlie.smart@yale.edu} \\
\textbf{Lecture} & MW 11:35 \textsc{am}--12:50 \textsc{pm}, Humanities Quadrangle 207 \\
\textbf{OH} & M 1:00 \textsc{pm}--3:00 \textsc{pm}, Dunham Lab 450 \\
% \textbf{Recitation} & TBA \\
\textbf{Textbook} & \fullcite{Rudin} \\
\textbf{Midterms} & Mon Oct 3, 2022 (In-Class) \\
				  & Mon Nov 7, 2022 (In-Class) \\
\textbf{Final} & Tue Dec 20, 2022 at 2pm (Location: TBD) \\
\bottomrule
\end{tabular} \\[3ex]
\end{center}

\subsection*{Course Description}
This course will provide a comprehensive introduction to measure theory.  We will start from real analysis and build from there.  The following topics will be covered.
\begin{itemize}
  \item Measure spaces
  \item Integrals on general measure spaces
  \item Product measures
  \item Integral representation of absolutely continuous measures
  \item Integral representation of linear functions
  \item Lebesgue measure
  \item Hausdorff measure and fractals
  \item Applications in dynamics
  \item Applications in probability theory
\end{itemize}

\subsection*{Course Format}
Approximately 2/3 of the material with be presented in lectures while the remaining 1/3 will be covered in the weekly homework assignments.

\subsection*{Prerequisites}
Math 305 or equivalent.

\subsection*{Assessments and Grading}
There will be ten homework assignments, two in-class midterms, and an in-person final, comprising 30\%, 40\%, and 30\%, respectively, of the total grade.

\section{August 31st, 2022}

\epigraph{``How do mathematicians measure things?''}{Charlie}

Following the universal rule for first class, we covered the syllabus and logistics. We then dove into review on the Riemann integral to motivate measure theory. Prof. Smart seemed much more comfortable talking about math than logistics.

\subsection{Course Plan}
\begin{itemize}
	\item Review Riemann Integration
	\item Abstract Measure Theory and Integration (Majority of class)
	\item Applications in Probability
	\item Applications in Dynamics
	\item Some brief applications in Geometry
\end{itemize}

\begin{problem} The Riemann Integral
\item Works for most functions of interest but notably...
	\begin{enumerate}
		\item It is not closed under important limits. 
		\item It is hard to generalize to new geometries.
	\end{enumerate}
\end{problem}

\begin{definition}[Riemann Integral]
	We can define the Riemann integral as $\int_{{\Q}}^{{}} {f}$ of a bounded $f : \Q \to \R$
	defined in closed rectangles
	$\Q = [a_1,b_1] \times [a_2,b_2] \times \ldots \times [a_d,b_d] \subseteq \R$. We use Darboux sums
\end{definition}

\vline
\begin{definition}[Partition]
	A \underline{partition} of $\Q$	is a finite set of \emph{closed} rectangles whose \underline{interiors} are disjoint and whose union is $\Q$.
\end{definition}

\vline

\begin{definition}[Darboux Sums]
	The upper and lower Darboux sums are 
	\[
		U(p,f) = \sum_{R \in P} \sup_R f \cdot |R|
	\] 
	and
	\[
		L(p,f) = \sum_{R \in P} \inf_R f \cdot |R|.
	\]	
\end{definition}

\begin{lemma}
	For any 2 partitions $P_1$ and $P_2$ of  $\Q$,  $L(P_1,f) \leq U(P_2, f)$
\end{lemma}

\begin{solution}
	We have
	\begin{align*}
		L(P_1, f) &= \sum_{R \in P_1}^{}  \inf_{R} f \cdot |R| = \sum_{R_1 \in P_1}^{} \inf f
		\sum_{R_2 \in P_2}^{} |R_1 \intersection R_2 | \\
				  &= \sum_{\substack{R_1 \in P_1 \\ R_2 \in P_2 \\ |R_1 \intersection R_2 | > 0} }^{} \inf_{R_1} f \cdot |R_1 \intersection R_2 |
				  \leq \sum_{\substack{R_1 \in P_1 \\ R_2 \in P_2 \\ |R_1 \intersection R_2 | > 0} }^{} \inf_{R_1 \intersection R_2} f \cdot |R_1 \intersection R_2 | \\
				  &\leq \sum_{\substack{R_1 \in P_1 \\ R_2 \in P_2 \\ |R_1 \intersection R_2 | > 0} }^{} \sup_{R_1 \intersection R_2} f \cdot |R_1 \intersection R_2 |
				  \leq \sum_{\substack{R_1 \in P_1 \\ R_2 \in P_2 \\ |R_1 \intersection R_2 | > 0} }^{} \sup_{R_2} f \cdot |R_1 \intersection R_2 | \\
				  &= \sum_{R_2 \in P_2}^{} \sup_{R_2} f \sum_{\substack{R_1 \in P_1 \\ |R_1 \intersection R_2 | > 0}}^{} |R_1 \intersection R_2 |  = \sum_{R_2 \in P_2}^{} \sup_{R_2} f \cdot |R_2| = U(P_2, f)
	\end{align*}
\end{solution}


Note we say a bounded $f : \Q \to \R$ is Riemann integrable when  
\[\sup_p L(p,f) = \inf_p U(p,f)\]
in which case we let  $\int_{\Q} f$ denote the common value.


\begin{theorem}
	If $f : \Q \to \R$ is continuous, then f is Riemann integrable.
\end{theorem}

\begin{solution}
	$\Q$ is compact, so  $f$ is uniformly cont. Thus, given a $\varepsilon >0 $, choose a $\delta >0$ so  $|x-y| < \delta \implies |f(x) - f(y)| > \varepsilon$. Let $p$ be any partition of  $\Q$ whose rectangles have diameter  $< \delta$ \newline
	Next compute
	\begin{align*}
		0 &\leq \inf_{p_1} U(p_1, f) - \sup_{p_2} L(p_2, f) \leq U(p_1, f) - L(p_1, f) \\
		  &= \sum_{R \in P}^{} \left( \sup_{R} f - \inf_{R}f \right) \cdot |R| \\
		  &\leq \sum_{R \in P}^{} \varepsilon \cdot |Q|
	\end{align*}
	but since $\varepsilon > 0$ is arbitrary,  $f$ is Riemann integrable.
\end{solution}

\begin{example}
	If $f:[0,1] \to \R$ and $f(x) =
	\begin{cases}
		1 & x\notin \Q \\
		0 & x\in \Q
	\end{cases}$.
	Then $f$ is not Riemann integrable.
	\begin{solution}
		For any $P$, we can see  $U(p,f) = 1$ and  $L(p,f) = 0$
	\end{solution}
\end{example}

\subsection{Measures}

We have two notions of volume
\begin{enumerate}
	\item The outer Jordan measure of a set $x \subseteq \R^d$ is
			\[
				J^* (x)
				= \inf \left\{ \sum_{k=1}^{n} |Q_k| : n \geq 1, Q_1, \ldots, Q_n \subseteq R^n
					\text{ closed rectangles and }
				x \subseteq \union_{k=1}^{b} Q_k \right\}
			\]
	\item Outer Lebesgue measure of a set $X \subseteq \R^n$ is
		\begin{align*}
			L^*(x) = \inf \left\{ \sum_{k=1}^{\infty}|Q_k| : Q_1, Q_2, \ldots \subseteq R^n  \text{ closed rectangles and }
			x \subseteq \union_{k=1}^{\infty} Q_k \right\}
		\end{align*}

		We have some important properties for these measures, namely

		\begin{enumerate}
			\item $L^*(x) \leq J^* (x)$
			\item If  $X$ is compact, then $L^*(x) = J^*(x)$
		\end{enumerate}
		\begin{solution}
			Suppose $x \subseteq \union_{k=1}^{\infty} Q_k$, pick $\varepsilon > 0$ and let  $(1 + \varepsilon)Q_k$ be the $(1 + \varepsilon)$-dilations of $Q_k$. Note $(1+ \varepsilon ) Q_k$ are open cover of  $x$ so $x \subseteq \sum_{k=1}^n (1+\varepsilon) Q_k$ and
			\begin{align*}
				\sum_{k=1}^{n} |(1 + \varepsilon) Q_k| = (1+\varepsilon)^d \sum_{k=1}^{n}|Q_k| \leq (1 + \varepsilon)^d \sum_{k=1}^{\infty} | Q_k|
			\end{align*}
		\end{solution}

	\item $L^* (\union^{\infty} x_k ) \leq \sum_{}^{\infty} L^* (X_{k})$

		\begin{solution}
			sketch by "Dovetail" (Come back to)
		\end{solution}
\end{enumerate}

\begin{example}
	The set $x = \Q \intersection [0,1]$ has  $L^*(x) = 0$ and  $J^*(x) = 1$ write  $x = \{q_1, q_2, \ldots \}$ then let $Q_k = [q_k =  \frac{\varepsilon}{2^k}, q_k + \frac{\varepsilon}{2^k}]$
	and observe that $x \subseteq \union_{1}^{\infty} Q_k$ and $\sum_{k=1}^{\infty} |Q_k| = 2 \varepsilon$
\end{example}

\begin{example}
	Let $X \subseteq [0,1]$ be generalized Cantor set. Let $1_{\C}$ be the indicator function of $\C$.

	Observe $1_ {\C}$ is not Riemann int on $[0,1]$. Indeed, C is the set of discontinuities of $1_{\C}$ and $L^*(\C) = \frac{1}{2}$. Moreover $1_{\C}(x) = \lim_{n\to \infty} 1_{\C_n} (x)$ and
	$0 \leq 1_{\C} \leq 1_{C_{n+1}} \leq 1_{\C_{n}} \leq 1$ and
	$\lim_{n \to \infty} \int_{[0,1]} 1_{C_{n}} = \frac{1}{2} $
\end{example}

\section{2022-09-02}

\subsection{Riemann Integration and Measures}
Last time: Definition of a Riemann Integral $\int_Q f$ of bounded $f : Q \to R$ on closed rectangle $Q
\subseteq \R^d$

\begin{theorem}
  A bounded $f : Q \to R$ is Riemann integrable if and only if the set $Z = \{ x \in Q \mid f
  \text{ not continuous of } x \}$ has $L^* (z) = 0$.
\end{theorem}

\begin{definition}[oscillation]
  The oscillation of $f : Q \to R$ at $x \in Q$ is
\[\osc(f,x) = \inf_{\delta > 0} \sup_{\substack{y, z \in Q \\ | y - x | < \delta \\ |z - x| < \delta}} |f(y) - f(z)| \]
\end{definition}

\begin{lemma}
  $f$ is continuous at $X$ if and only if $\osc (f,x) = 0$.
\end{lemma}

\begin{lemma}
  The set $z_\varepsilon = \{ x \in Q : \osc(f,x) \geq \varepsilon \}$ is closed (compact) for $\varepsilon > 0$.
\end{lemma}

\begin{proof}
  Since $\R^d$ is a metric space and closed, it is also sequentially closed.
  Suppose $x_1, x_2, \ldots \in Z_\varepsilon$ and $x = \lim_{n \to \infty} x_n$.
  Since $x_n \in Z_\varepsilon$, we can choose $y_n, z_n$ so that
  \[|y_n - x_n| < \frac{1}{n}, |z_n - x_n| < \frac{1}{n}, \text{ and } |f(y_n) - f(z_n)| \geq \varepsilon - \frac{1}{n}\]
  For any $\delta > 0$, choose $n > 0$ such that $\frac{1}{n} > \frac{\delta}{2}$ and $|x_n - x| < \frac{\delta}{2}$.
  We now have
  \[|y_n - x_n| < \delta, |z_n - x_n| < \delta, \text{ and } |f(y_n) - f(z_n)| \geq \varepsilon - \frac{\delta}{2}\]
  We now conclude that $x \in Z_\varepsilon$.
\end{proof}

Note on notation: The set of discontinuities $Z$ can be the countable union of compact sets, which is represented by,
\[ Z = \bigcup_{n \geq 1} Z_{\frac{1}{n}} \]

\begin{theorem}
  A bounded $f : Q \to R$ is Riemann integrable if and only if $Z = \{ x \in Q : \osc (f,x) > 0 \}$ has $L^*(x) = 0$.
\end{theorem}

\begin{proof}
  Assume $|f| \leq B$. First, suppose $L^*(z) = 0$.
  Let $Z_\varepsilon = \{ x \in Q : \osc (f,x) \geq \varepsilon\}$.
  Since $Z_\varepsilon \subseteq Z$, $L^*(Z_\varepsilon) = 0$.
  Since $Z_\varepsilon$ is compact, $J^*(Z_\varepsilon) = 0$.
  Choose a finite set of rectangles $S$ such that $Z_\varepsilon \subseteq \bigcup_{R_1 \in S_1} \mathring{R_1}$ and $\sum_{R_1 \in S_1} |R_1| < \varepsilon$.

  Let $W_\varepsilon = Q \setminus \bigcup_{R_1 \in S_1} \mathring{R}$ and observe that $W_\varepsilon$ is compact and $\osc(f,x) < \varepsilon$ for all $x \in W_\varepsilon$.
  \footnote{Note: this helps cover the boundary condition, which Spivak incorrectly omits in his book.}
  For every $x \in W_\varepsilon$, we can choose rectangle $R$ so $x \in \mathring{R}$ and $\sup_R f - \inf_R f < \varepsilon$.
  Since $W_\varepsilon$ is compact, they are a finite set of rectangles $S_2$ such that
  \[ W_\varepsilon \subseteq \bigcup_{R_2 \in S_2} \mathring{R}_2 \quad \text{and} \quad \sup_{R_2} f - \inf_{R_2} f < \varepsilon \quad \forall R_2 \in S \]

  Let $P$ be a partition of $Q$ so that for $R \in P$ and $R_k \in S_k$, where $k = 1,2$ have $\mathring{R} \cup \mathring{R}_k \ne \emptyset \implies R \subseteq R_k$
  (Let $P$ be a set of minimal non-trivial intersections of rectangles from $R_1 \cup R_2 \cup \{ Q \}$ that are inside $Q$.)

  \begin{lemma}
  If $R_1, \ldots, R_n$ have disjoint interiors and $\bigcup^n_{k=1} R_k \subseteq \bigcup^m_{k=1} \tilde{R}_k$, then $\sum_{k=1}^{n} |R_k| \leq \sum_{k=1}^{m} |\tilde{R}_k|$.
  \end{lemma}

  If $R \in P$ and $R \in Z_\varepsilon \ne \emptyset$, then $R \subseteq R_1 \in S_1$.
  We have a partition $P$ such that.

  \begin{align*}
    U(P,f) - L(P,f) & = \sum_{\substack{R \in P \\ R \cap Z_\varepsilon \ne \emptyset}} \left( \sup_R f - \inf_R f \right) |R| + \sum_{\substack{R \in P \\ R \cap Z_\varepsilon \ne \emptyset}} \left( \sup_R f - \inf_R f \right) |R| \\
                    & \leq \sum_{\substack{R \in P \\ R \cap Z_\varepsilon \ne \emptyset}} 2B \cdot |R| + \sum_{\substack{R \in P \\ R \cap Z_\varepsilon \ne \emptyset}} \varepsilon |R| \\
                    & \leq 2B \cdot \varepsilon + \varepsilon |Q|
  \end{align*}
  Since the choice of $\varepsilon > 0$ is arbitrary and $B, |Q|$ is fixed, we conclude that $f$ is Riemann integrable.

  Now suppose $f$ is Riemann integrable. Since $z = \bigcup_{n \geq 1} Z_{1/n}$, it is enough to show $L^*(Z_\varepsilon) = 0$ for $\varepsilon > 0$.\footnote{this is why we need Jordan outer measure}

  Given $\delta > 0$, choose a partition $P$ of $Q$ so that $U(P,f) - L(P,f) < \delta$.
  \begin{align*}
    U(P,f) - L(P,f) & = \sum_{R \in P} \left( \sup_R f - \inf_R f \right) |R| \\
                    & \geq \sum_{\substack{R \in P \\ \mathring{R} \cup Z_\varepsilon \ne \emptyset}} \left( \sup_R f - \inf_R f \right) |R| \\
                    & \geq \sum_{\substack{R \in P \\ \mathring{R} \cup Z_\varepsilon \ne \emptyset}} \varepsilon |R|
  \end{align*}
  and after rearranging, we obtain
  \[ \sum_{\substack{R \in P \\ \mathring{R} \cup Z_\varepsilon \ne \emptyset}} |R| \leq \frac{\delta}{\varepsilon}\]
  Therefore, we can cover $Z_\varepsilon$ with an arbitrary small volume of rectangles plus some boundaries of rectangles.
\end{proof}

\begin{exercise}
  Find a Riemann integrable $f : Q \to R$ with $J^* (z) > 0$. 
\end{exercise}

\subsection{Riemann v. Lebesgue}

[There is a missing graphic comparing the two]

\begin{theorem}
  For any metric space $M$, there is a complete metric space $M$ and isometry $i : M \to \bar{M}$ such that for continuous $f : M \to \bar{N}$, there is a unique $\bar{f} : \bar{M} \to \bar{N}$ with $f = \bar{f} \cdot i$. 
\end{theorem}

\begin{proof}[Sketch Proof]
  $\bar{M}$ is equivalence classes of Cauchy sequences in $M$.
\end{proof}

\begin{corollary}
  $\bar{M}$ is unique. 
\end{corollary}

$M = (C(Q), \| \cdot \|_{C^\circ})$ is a complete metric space and $\int_{\substack{Q \\ \text{Riemann}}} \cdot M \to R$. 

Consider isn



\section{September 7th, 2022}

\epigraph{``Fish fish fish eat eat eat''}{Charlie}

Readings: Chapter 1 of Rudin.

\subsection{Lebesgue Integration and Motivation}

[sketched Lebesgue integration cartoon.]

It can be defined as 
\begin{align*}
	\int f \approx \sum_{k=1}^{n-1} S_k L^* (\{x : S_k \leq f(x) \leq S_{k+1}\})
\end{align*} 

[\underline{missing sketch}]

We want to be able to "measure" all of the "constructable subsets of $\R^d$

We should be able to handle $1_C$ of $\frac{1}{2^{k+1} -1 }$ cantor set ($C \in [0,1]$) and similar functions

\begin{enumerate}
	\item constructable? Rectangles and unions and intersections of rectangles.
	\item Start with all \sout{rectangles} open and closed sets.
	\item If $A_1, A_2, \ldots$ constructable, then
		$\bigcup_{k=1}^{\infty}A_k$ and $\intersection_{k=1}^{\infty}A_k$ constructable.
	\item Lebesgue measure assigns volume to all such sets.
\end{enumerate}
 
\subsection{Metric Spaces}

\begin{definition}[Metric \& Metric Spaces]
	A function $d : X \times X \to \mathbb{R}$ on the Cartesian product of a set $X$ us a \underline{metric} if

\begin{enumerate}
	\item $d(x,y) \geq 0$
	\item $d(x,y) = 0 \iff x = y$
	\item $d(x,y) = d(y,x)$
	\item $d(x,y) \leq d(x,y) + d(y,z)$
\end{enumerate}

We call the pair $(X, d)$ a \underline{metric space.}
\end{definition}


\begin{definition}[Continuous Functions]
A function $f : X \to Y$ between metric spaces is continuous if, $\forall x \in X$ and $\varepsilon >0 $ there exists a $\delta >0$ such that for all $y \in X$ we have $d_X (x, y) < \delta \implies d_Y(f(x), f(y)) < \varepsilon$	
\end{definition}

A subset $V \subseteq X$ of a metric space $X$ is open if for ever $x \in V$ there is a $\delta >0 $ such that $B(x,\delta) = \{y \in X: d(x,y) < \delta\} \subseteq U$ 

\begin{theorem}
	A map $f:X \to Y$ between metric spaces is continuous if and only if $w \subseteq Y$ open implies $f^{-1}(w) \subseteq X$ is also open.
\end{theorem}

\begin{proof}
	(exercise) definition fun!
\end{proof}

\begin{definition}[Collections and Families] \phantom{text} \hfill
\begin{enumerate}
	\item A \emph{collection} is a set of subsets. 
	\item A \emph{family} of collections is a set of collections.
\end{enumerate}
\end{definition}

\begin{definition}[Topology]
	A collection $T$ of subsets of a set $X$ is a \textit{topology} if:
	\begin{enumerate}
		\item $\emptyset, X \in T$ 
		\item $T^{'} \subseteq T$ is finite, then $\bigcap\limits_{V \in T^{'}} V \in T$ 
		\item If $T^{'} \subseteq T,$ then $\bigcup\limits_{V \in T^{'}} V \in T$
	\end{enumerate}

	We call the pair $(X, T)$ a topological space.
\end{definition}


A function $f:X \to Y$ between topological spaces is continuous if $V \in T_y \implies f^{-1}(V) \in T_X$

\begin{claim*}
	A topology is a way of designating some subsets as open.	
\end{claim*}

\subsection{Sigma Algebras}

\begin{definition}[Algebra]
	A Collection $A$ of subsets of a set $X$ is an \emph{algebra} if 
	\begin{enumerate}
		\item $\emptyset, X \in A$
		\item If $B_1, B_2 \in A,$ then
			\begin{enumerate}
				\item $B_1 \cap B_2 \in A$
				\item  $B_1 \cup B_2 \in A$
				\item  $B_1^{c} = X \setminus B_1 \in A$
			\end{enumerate}
	\end{enumerate}
\end{definition}

\begin{definition}[Sigma Algebra]
	A Collection $S$ of subsets of a set $X$ is a \emph{$\sigma$-algebra} if 
	\begin{enumerate}
		\item $\emptyset, X \in S$
		\item If $B\in S \implies B^{c} \in S$
		\item If $S^{'} \subseteq S$ is countable then 
			$\bigcup\limits_{B \in S^{'}} B \in S$ and $\bigcap\limits_{B \in S^{'}} B \in S$
	\end{enumerate}
note: $\bigcap\limits_{B \in S^{'}} = (\bigcup\limits_{B \in S^{'}} )^{c}$

We call the pair $(X, S)$ a measurable space.
\end{definition}

\begin{definition}
	A function $f : X \to Y$ between measurable spaces is measurable if $B \in S_Y \implies f^{-1}(B) \in S_X$
\end{definition}

\begin{claim*}
	Not all $\sigma$ algebras are topologies but all important ones are.
\end{claim*}

\begin{example}
	\begin{enumerate}
		\item The power set $\mathcal{P} (X) = \{y : y \subseteq X\}$ is both a topology and a $\sigma$-algebra
		\item $\{\emptyset, X\}$ is both a topology and a $\sigma$-algebra
		\item $\{\emptyset, B, B^{c}, X\}$ is both a topology and a $\sigma$-algebra
	\end{enumerate}
\end{example}

\begin{theorem}
	For a collection $C$ of subsets of $X$ there is a smallest topology $T$ and $\sigma$-algebra $S$ with $T \supseteq C$ and $S \supseteq C$.
	Denote these by $\tau(C)$ and $\sigma(C)$ respectively.
\end{theorem}

\begin{proof}
	Let $\sigma(C) = \bigcap\limits_{S \supseteq C} S$.

	Note that $\mathcal{P}(X) \supseteq C$, the intersection is non-trivial.

	\underline{Check the Axioms}:
	\begin{enumerate}
		\item If $S \supseteq C$ is $\sigma$-algebra, then $\emptyset,X \in S$.

			Therefore  $\emptyset, X \in \bigcap \limits_{S \supseteq C \text{ is a $\sigma$-algebra}}S $
		\item $X \in \sigma(C)$ thus
			 \begin{align*}
			&\implies X \in S \text{ for } S \subseteq C \text{ a } \sigma\text{-algebra} \\
			&\implies X^{c} \in S \text{ for } S \subseteq C \text{ a } \sigma\text{-algebra} \\
			&\implies X^{c} \in \sigma(C)
			\end{align*} 

	\item $S^{'} \subseteq \sigma (C)$ countable
		 \begin{align*}
		&\implies S^{'} \subseteq S \text{ for } S \supseteq C \text{ a $\sigma$-algebra} \\
		&\implies \bigcup_{B \in S^{'}} B \in S \text{ for } S \supseteq C \text{ a $\sigma$-algebra} \\
		&\implies \bigcup_{B \in S^{'}} B \in \sigma(C)
		\end{align*} 
	\end{enumerate}

	$\sigma(C)$ is the smallest because it is contained in every $\sigma$-algebra $S \supseteq C$.

	 $\tau(C)$ is defined similarly.
\end{proof}

\begin{definition}[Measurable Function]
	We call a function $f:X \to Y$ from a measurable space to a topological space \textit{measurable} if $W \in T_Y$ implies $f^{-1}(W) \in S_X$
\end{definition}

\begin{theorem}
	This is equivalent to $f$ being a measurable function from $(X, S_X)$ to $(S, \sigma(T_Y))$
\end{theorem}

\begin{example}
	If $(X,T)$ is a topological space, then call elements of $\sigma(T)$ the \emph{Borel} subsets of $(X,T)$	
\end{example}

The Borel subsets of $\mathbb{R}^d$ and the "constructable" sets from before.

\begin{theorem}
	if $(Y, S_Y)$ is a measurable space and $f: X \to Y$ then there is a smallest $\sigma$-algebra on $X$ that makes $f$ measurable.
\end{theorem}
\begin{proof}
	Check $S_X = \{f^{-1}(B) : B \in S_Y\}$ is a $\sigma$-algebra.
\end{proof}

\begin{theorem}
	If $(X, S_X)$ is a measurable space and $f:X \to Y$ then there is a largest $\sigma$-algebra on $Y$ that makes $f$ measurable.
\end{theorem}

\begin{proof}
	Check $S_Y = \{B \subseteq Y: f^{-1}(B) \in S_X \}$ is a $\sigma$-algebra.
\end{proof}

\begin{definition}
	Basic Borel sets are open or closed.
	\begin{enumerate}
		\item $B_0$ sets are open or closed subsets of $(X,T)$
		\item $B_{K+1}$ sets of countable unions of countable intersections of sets in $B_k$
	\end{enumerate}
\end{definition}

\begin{example}
	$E \in B_4$ then

	 \begin{align*}
		 &\bigcap_i \bigcup_j \bigcap_k F_{i,j,k} \quad \text{\tiny (F open)} \qquad \text{Or} \\
		 &\bigcup_i \bigcap_j \bigcup_k G_{i,j,k} \quad \text{\tiny (G closed)}
	\end{align*}
	
\end{example}


\section{September 12th, 2022}

\epigraph{Shirt of the day: ``I have eated all the dinner''}{Charlie}

Last Time: Basic results for measurable spaces $(X, S)$.

\subsection{Measurable Space}

\begin{theorem}
  There is a smallest $\sigma$-algebra $\sigma(c)$ containing any collection $C$ of subsets of a set $X$.
\end{theorem}

\begin{lemma}
	If $(X,S_x), (Y,S_y)$ are measurable spaces, $f: X \to Y$ is $(S_x, S_y)$ measurable, $S_x \supseteq S_y$ and $S_y\supseteq S_y$ s.t. $S_x' \subseteq S_X$ and $S_Y' \subseteq S_Y$, then $f$ is $(S_x', S_y')$ measurable.
\end{lemma}

\begin{theorem}
  If $(X, S)$ is a measurable space and $f: X \to Y$, then
	\[
		f^*(s) = \left\{A \subseteq Y \mid f^{-1}(A) \in S \right\}.
	\]
	is the largest $\sigma$-algebra such that $f$ is $(S, f^*(S))$-measurable
\end{theorem}

\begin{theorem}
	If $(Y,S)$ is a measurable space and $f : X \to Y$, then
	 \begin{align*}
		 f^*(S) &= \{f^{-1}(A) : A \in S\}
	\end{align*}
	is the smallest $\sigma$-algebra such that $f$ is $(f^*(S), S)$- measurable.
\end{theorem}

\begin{theorem}
	If $(X, T_X)$ $(Y,T_Y)$ are topological spaces, and  $f : X \to Y$ is $T_X, T_Y$-continuous, then $f$ is also $(\sigma(T_X), \sigma(T_Y))$-measurable.
\end{theorem}

\begin{proof}
	$f$ is $(T_X,T_Y)-continuous$
	\begin{align*}
		&\iff \{f^{-1}(A) : A \in T_Y\} \subseteq T_X \\
		&\iff \{f^{-1}(A) : A \in \sigma(T_Y)\} \subseteq \sigma(T_X) \\
		&\iff f \text{ is } (\sigma(T_X), \sigma(T_Y)) \text{ -- measurable.}
	\end{align*}
\end{proof}

% Abuse of notation:
\begin{abuse}
If $(X, S_X)$ is a measurable space and  $(Y, T_Y)$ is a topological space, then call $f : X \to Y$
$(S_X, T_Y)$-measurable $\iff f$ is $(S_X, \sigma(T_Y))$- measurable.
\end{abuse}

\begin{theorem}
	$f : X \to Y$ is $(S_X, \sigma(T_Y))$- measurable $\iff \forall A \in T_Y$ we have $f^{-1}(A) \in S_X$.
\end{theorem}

\begin{proof}
	Since $T_Y \subseteq \sigma(T_Y)$
	then $\implies$ directions trivial.

	For the $\impliedby$ direction, observe that
	\begin{align*}
		&\{f^{-1}(A) : A \in T_Y\} \subseteq S_X \\
		&\implies \sigma(\{f^{-1}(A) : A \in T_Y\}) \subseteq \sigma(S_X) = S_X \\
		&\iff \sigma(\{f^{-1}(A) : A \in \sigma(T_Y)\}) \subseteq  S_X \text{--measurable}
	\end{align*}
\end{proof}

If $(X,T)$ is a topological space, then call $\sigma(T)$ the Borel subsets of  $X.$

If $(X, T_X)$ and $(Y,T_Y)$ are topological spaces, then $f:X \to Y$ is measurable.

If it is  $(\sigma(T_X), \sigma(T))$-measurable.

comeback too

\begin{definition}
	The extended Reals.
	$[-\infty,\infty]$ is $\mathbb{R} \cup \{-\infty, +\infty\}$.

	Add new open sets $[-\infty, a)$ and $(a, +\infty]$
	and the usual arithmetic rules
	\begin{align*}
		a + \infty = \infty, \quad a\cdot \infty = \infty, \quad a > 0
	\end{align*} etc defined by demanding $+$ and $\cdot$ extend continuously to
	\begin{align*}
		&[-\infty, \infty]^2 \setminus \{(-\infty, \infty), (\infty, -\infty)\} \text{ and } \\
		&[-\infty, \infty]^2 \setminus \{(0, \pm \infty), (\pm \infty, 0)\}
	\end{align*}
\end{definition}

\begin{corollary}
	If $(X,S)$ measurable, then $f : X \to [-\infty, \infty]$ is measurable
	$iff$ $f^{-1}((a,\infty]) \in S$ for all $a \in [-\infty,\infty]$
\end{corollary}

\begin{proof}
	The main point is
	\begin{align*}
		&\sigma(\{(a,\infty] : a \in [-\infty, a)\}) \\
		= &\sigma(\{ A \subseteq [-\infty, \infty] : A \text{ open} \})
	\end{align*}
\end{proof}

\begin{theorem}
	If $(X,S)$ is measurable space,
	$f_1, \ldots, f_d : X \to \mathbb{R}$ are measurable and $g:\mathbb{R}^d \to \mathbb{R}$ continuous, then $g \circ (f_1, \ldots, f_d) : X \to \mathbb{R}$ is measurable.
\end{theorem}

\begin{proof}
	It is enough to show that $(f_1, \ldots, f_d) : X \to \mathbb{R}^d$ is measurable. But given $A \subseteq \mathbb{R}^d$ is open, we need to show $f^{-1}(A) \in S$.

	Write $A = \bigcup_{k=1}^{\infty} R_k$ with $R_k$ open rectangles. It is enough to show $f^{-1}(R) \in S$ for any open rectangle $\mathbb{R} \subseteq \mathbb{R}^d$.

	Write $R = (a,b) \times \ldots \times (a_d, b_d)$ and compute
	$f^{-1}(R) =  f^{-1}((a,b)) \cap \ldots \cap f^{-1}((a_d, b_d)) \in S$
	by measurability of $f_1, \ldots , f_d$
\end{proof}

\begin{corollary}
	If $(X,S)$ is a measurable space and $f,g : X \to [-\infty, \infty]$ is measurable,
	then $f+g$, $f\circ g$, $\min\{f,g\}$, $\max\{f,g\}$, and $|f|$ are all measurable (when defined).
\end{corollary}

\begin{theorem}
	If $(X,S)$ measurable space and $f_1, f_2, \ldots : X \to [-\infty, \infty]$ measurable, then $f = \sup_k f_k$ is measurable.
\end{theorem}

\begin{proof}
	The main part is
	\begin{align*}
		f^{-1}((a, \infty]) = \bigcup_{k=1}^{\infty} f_k^{-1}((a,\infty]) \in S
	\end{align*}
\end{proof}

\begin{corollary}
	\begin{align*}
		\inf_k f_k, \, \lim \sup_k f_k, \, \text{and } \, \lim \inf_k f_k
	\end{align*} are also measurable.
\end{corollary}

\begin{proof}
	\begin{align*}
		&\inf_k f_k = - \sup_{k} (-f_k) \\
		&\lim_{R} \sup f_k = \inf_{k} (\sup_{j} f_{k+j}) \\
		&\lim_{k}\inf f_k = - \lim_{k} \sup (-f_k)
	\end{align*}
\end{proof}

\begin{corollary}
	If $(X,S)$ measurable space, $f_k : X \to (-\infty, \infty]$ measurable for  $k \geq 0$
	 \begin{align*}
	    f (x) = \lim_{k \to  \infty} f_k (x) \, \forall x \in X
	\end{align*} then $f$ is measurable.
\end{corollary}

\begin{proof}
	 \begin{align*}
	f = \lim_{k} f_k \implies f = \lim_{k} \sup f_k
	\end{align*}
\end{proof}

\underline{Review:} From Real analysis

check
\begin{enumerate}
	\item Locally uniform limits of continuous functions are continuous.
	\item Pointwise limits of continuous functions need not be continuous.
\end{enumerate}

\subsection{Positive Measure}

\begin{definition}[Positive Measure]
	A positive measure is a function $m : S \to [0, \infty]$ on a $\sigma$-algebra $S$ on a set $X$ such that $m(\emptyset) = 0$
	and $m( \bigcup_{k=1}^{\infty} A_k) = \sum_{k=1}^{\infty} m(A_k)$ when
	$A_1, A_2, \ldots \in S$ disjoint.

	{\tiny note $m(\emptyset) = 0$ allows us to exclude $m$ which gives everything $+\infty$ measure.}

	Call the triple $(X,S,m)$ a positive measure space.
\end{definition}

\begin{example}
	\begin{enumerate}
		\item $X = \{x_1, \ldots, x_n\}$,
			\begin{align*}
				S = \mathcal{P}(X) \text{ and } m(A) = \#A \quad \text{\tiny the counting measure}
			\end{align*}
		\item $X$, arbitrary
			\begin{align*}
				S = \mathcal{P}(X) \text{ and } m(A) =
				\begin{cases}
					\#A & A\text{-finite} \\
					+\infty & A\text{-infinite} \\
				\end{cases}
			\end{align*}

		\item The Dirac Measure:
			$X$ arbitrary, $S$ arbitrary, and $X_0 \in X$ and
			\begin{align*}
				m(A) =
				\begin{cases}
					1 & x_0 \in A \\
					0 & x_0 \notin A
				\end{cases}
			\end{align*}
	\end{enumerate}
\end{example}


\begin{definition}[Basic Properties]
	\begin{enumerate}
		\item If $A,B \in S$  and $A \subseteq B$, then $m(A) \leq m(B)$
		\item If  $A_1, \ldots, A_n \in S$ disjoint, then
			$m(A_1 \cup A_2 \cup \ldots \cup A_n) = m(A_1) + \ldots + m(A_n)$
		\item If  $A_k \in S$ and $A_1 \subseteq A_2 \subseteq \ldots$, then
			\begin{align*}
				m(\bigcup_{k=1}^{\infty} A_k ) = \lim_{k \to \infty} m(A_k)
			\end{align*}
		\item If $A_k \in S$, and $A_1 \supseteq A_2 \supseteq \ldots$ and $m(A_1) < \infty$,
			then
			 \begin{align*}
				m(\bigcap_{k=1}^{\infty} A_k) = \lim_{k \to \infty}  m(A_k)
			\end{align*}
	\end{enumerate}
\end{definition}

\begin{proof}(of properties)
	\begin{enumerate}
		\item[2] let $A_k = \emptyset$ for $k >n$. Now $A_1, A_2, \ldots$ disjoint and
			\begin{align*}
				m(A_1 \cup \ldots \cup A_n) &= m(\bigcup_{k=1}^{\infty} A_k ) \\
				= \sum_{k=1}^{\infty} m(A_k) &= m(A_1) + \cdots + m(A_n)
			\end{align*}
		\item[1] Write $B = A \cup (B \setminus A)$, and $B \setminus A \in S$ since $S$ is a $\sigma$-algebra
			\begin{align*}
				m(B) = m(A) + m(B \setminus A) \geq m(A)
			\end{align*}
		\item[3] Compute (Check)
			\begin{align*}
				m(\bigcup_{k=1}^{\infty} A_k) &= m(A_1 \cup \bigcup_{k=1}^{\infty} (A_{k+1} \setminus A_k) \\
											  &= m(A_1) + \sum_{k=1}^{\infty} m(A_{k+1} \setminus A_k) \\
											  &= m(A_1) + \lim_{n \to \infty}  \sum_{k=1}^{n} m(A_{k+1} \setminus A_k) \\
											  &= \lim_{n \to \infty} \left(m(A_1) + \sum_{k=1}^{n} m(A_{k+1} \setminus A_k) \right) \\
											  &= \lim_{n \to \infty} m \left(A_1 \cup \bigcup_{k=1}^{\infty} (A_{k+1} \setminus A_k) \right) \\
											  &= \lim_{n \to \infty} m (A_{n+1})  \\
			\end{align*}
		\item[4] Let $B_k = A_1 \setminus A_k \in S$. Observe: $B_1 \subseteq B_2 \subseteq \ldots$

			\underline{Conclude}:
			\begin{align*}
				m(\bigcup_{k=1}^{\infty} B_k) = \lim_{n \to \infty} m(B_n)
			\end{align*}
			Rewrite as:
			\[
				\quad m(A_1 \setminus \bigcap_{k=1}^{\infty} A_k) =
				\lim_{n \to \infty} m(A_1 \setminus A_n).
			\]
			Rewrite as:
			\begin{align*}
				m(A_1) - m(\bigcap_{k=1}^{\infty} A_k) = m(A_1) - \lim_{n \to \infty} m(A_n)
			\end{align*}
			Since $m(A_1) < \infty$ we can cancel.
	\end{enumerate}
\end{proof}

\begin{example}
	Let $m$ be the counting measure on  $\mathbb{N}$ and consider $A_n = \{n, n+2, n+2, \ldots \}$.
	Then $A_1 \supseteq A_2 \supseteq \ldots$ , $m(A_k) = \infty$ and
	\begin{align*}
		m(\bigcap_{k=1}^{\infty} A_k ) = m(\emptyset) = 0
	\end{align*}
\end{example}


\section{2022-09-14}

Last time: positive measure spaces $(X, S, m)$. $X$ is a set, $S$ is a $\sigma$-algebra of subsets of $X$ in (countably additive) positive measure in $S$.

Today: Integration on $(X, S, m)$.

Idea: approximate measurable $f : X \to [0, \infty]$ with simple functions.

\begin{definition}[Simple Function]
	A simple function is a measurable function $g : X \to [0, \infty)$ that takes only finitely many values.
\end{definition}

If $g$ is simple, then there are reals $\alpha_1, \ldots, \alpha_n \in [0, \infty)$, and $A_n \in S$ such that
\[
	g(x) = \sum_{n = 1}^{k} \alpha_k \1_{A_k}(x).
\]
can even demand $A_k$ disjoint and $X = A_1 \cup \ldots A_k$.

[insert sketch here]

\begin{theorem}
	If $f : X \to [0, \infty)$ measurable, then there are simple $g_n : X \to [0, \infty)$ such that for all $X$, $0 \leq g_n(x) \leq g_{n+1}(x) \leq f(x)$ and $\lim_{n \to \infty} g_n(x) = f(x)$.
\end{theorem}

\begin{proof}
  Round down to list possible values.

	[insert sketch]

	\[
		g_n(x) = \min \left\{2^n, 2^{-n} \lfloor 2^n f(x) \rfloor \right\}.
	\]
	This guarantees $0 \leq g_n(x) \leq g_{n+1}(x) \xrightarrow{n \to \infty} g(x)$.

	To see $g_n$ is measurable, observe that
	\[
		h_\varepsilon (x) = \min \left\{\varepsilon^{-1}, \varepsilon \lfloor \varepsilon^{-1} x \rfloor\right\}.
	\]
	[another sketch]

	is (Borel) measurable as a function from $[0, \infty] \to [0, \infty)$.
	\[
		g_n = h_{2^{-n}} \circ f.
	\]
\end{proof}

\begin{remark}
  We can approximate by analogous simple functions any measurable map $f : X \to M$ when metric space $M$ has a countable dense set.
\end{remark}

\subsection{Lebesgue Integration}

\begin{definition}[Lebesgue Integration]
If $g : X \to [0, \infty)$ simple and $g = \sum_{k = 1}^{K} \alpha_1 \1_{A_k}$, then
\[
	\int_{}^{} g \,dm = \sum_{n = 1}^{k} \alpha_k m(A_k).
\]
If $f : X \to [0, \infty]$ measurable, then
\[
	\int_{}^{} f \,dm = \sup_{\substack{g : X  \to [0, \infty) \textrm{ simple} \\ g \leq f}} \int_{}^{} g \,dm.
\]
\end{definition}

Basic Properties
\begin{enumerate}
	\item If $f_1, f_2 : X \to [0, \infty]$ measurable and $f_1 \leq f_2$, then $\int f_1 \, dm \leq \int f_2 \, dm$.
	\item If $f_1, f_2 : X \to [0, \infty]$ measurable and $\alpha_1, \alpha_2 \in [0, \infty)$, then
	\[
		\int \alpha_1 f_1 + \alpha_2 f_2 \, dm = \alpha_1 \int f_1 \, dm + \alpha_2 \int f_2 \, dm.
	\]
	\item If $f_1, f_2 : X \to [0, \infty]$ measurable and and $m( \left\{x : f(x) > 0 \right\}) = 0$, then $\int f \,dm = 0$. Even if $f(x) = \infty$ for some $x$!
\end{enumerate}

\subsection{Monotone Convergence Theorem}

\begin{theorem}[Monotone Convergence Theorem]
	If $f_1, f_2, f_3, \ldots : X \to [0, \infty]$ measurable and for all $x \in X$, $0 \leq f_n(x) \leq f_{n+1}(x) \xrightarrow[n \to \infty]{} f(x)$, then $\int f_n \, dm \xrightarrow[n \to \infty]{} \int f \, dm$.
\end{theorem}

\begin{proof}
	From $f_n \leq f_{n+1}$, obtain
	\[
		\int_{}^{} f_n \,dm \leq \int_{}^{} f_{n+1} \,dm \xrightarrow[n \to \infty]{} \alpha \in [0, \infty].
	\]
	From $f_n \leq f_{n+1}$, obtain $\int f \, dm \geq \alpha$. Need to prove $\int f \, dm \leq \alpha$.

	By definition of integration, it is enough to show for $g \leq f$ simple.
	\[
		\int g \, dm \leq \limsup_{n} \int f_n \, dm.
	\]
	Fix small $\delta > 0$. Let $A_k = \left\{x \in X : f_n(x) \geq (1 - \delta) g(x) \right\}$.
	Since $f_n \leq f_{n+1} \to f \geq g$, have $A_n \subseteq A_{n+1}$ and $\bigcup_{n=1}^{\infty} A_n = x$.

	Indeed for $x \in X$, break into cases:
	\begin{itemize}
		\item If $g(x) = 0$, then $x \in A_n$ for all $n \geq 1$.
		\item If $g(x) \geq 0$, then
			\[
				(1 - \delta) g(x) < g(x) \leq \lim_{n \to \infty} f_n (x).
			\]
		so $f_n(x) \geq (1 - \delta) g(x)$ for $n \geq 1$ large. Write $g = \sum_{k = 1}^{K} \beta_k \1_{B_k}$.

		Compute
		\begin{align*}
			\int f_n \, dm & \geq \int \1_{A_n} f_n \, dm \\
										 & \geq \int \1_{A_n} (1 - \delta) \, dm \\
										 & = \sum_{k = 1}^{K} (1 - \delta) \beta_k m(B_k \cap A_k) \\
										 & \xrightarrow[n \to \infty]{} \sum_{k = 1}^{K} (1 - \delta) \beta_k m(B_k \cap X) \\
										 & = (1 - \delta) \int g \, dm
		\end{align*}
		Since $\delta > 0$ arbitrary, conclude
		\[
			\limsup_{n \to \infty} \int f_n \, dm \geq \int g \, dm.
		\]
	\end{itemize}
\end{proof}

\begin{corollary}
	If $f_1, \ldots, f_n : X \to [0, \infty]$ measurable, then
	\[
		 \int \sum_{k = 1}^{\infty} f_k \, dm = \sum_{k = 1}^{\infty} \int f_k \, dm.
	\]
\end{corollary}

\begin{proof}
	$g_n = \sum_{k = 1}^{n} f_k$, then $g_n \leq g_{n+1}$ and $g_n \leq g_{n+1}$ and $g_n \to \sum_{k = 1}^{\infty} f_k$ pointwise.
\end{proof}

\subsection{Some Quick ``Applications''}

\begin{theorem}[Borel-Cantelli]
	If $A_1, A_2, \ldots \in S$ and $\sum_{k = 1}^{\infty} m(A_k) < \infty$, then the set
	\[
		B = \left\{x \in X : X \textrm{ is an element of infinitely many } A_k \right\}.
	\]
	has $m(B) = 0$.
	That is, almost every $X$ (with respect to $m$) appears in only finitely many $A_k$. 
\end{theorem}

\begin{proof}
	Let $f(x) = \sum_{k = 1}^{\infty} \1_{A_k}(x)$, observe $B = \left\{x \in X : f(x) = \infty \right\}$. 
	Compute
	\[
		\int f \, dm = \int \sum_{n = 1}^{\infty} \1_{A_k}(x) \, dm = \sum_{k = 1}^{\infty} \int \1_{A_k}(x) \, dm = \sum_{k = 1}^{\infty} m(A_k) < \infty.
	\]
	since $\int f \, dm < \infty, m(B) = 0$. 
\end{proof}

Example / Application: 
Assume we know about the Lebesgue measure, i.e. $([0,1], B, m)$ and $m((a,b)) = b - a$. 
almost every real number does \underline{not} have good rational approximations. 

\begin{theorem}[Baby Hergbtz]
	Let $m(B) = 0$ where $B = \{ x \in [0,1] \}$ : there are infinitely many pairs of integers $(p,q)$ with $\gcd (p, q) = 1$ and $\left| x - \frac{p}{q} \right| \leq \frac{1}{q^3} \left( \leq \frac{1}{5q^2} \right)$.
	Any $0 < p < q$ with $\gcd(p,q) = 1$ approximate all $x$ in $(\frac{p}{q} - \frac{1}{q^3}, \frac{p}{q} - \frac{1}{q^3}$ and $m(I_{p,q})$. 
	\[
		\sum_{\substack{1 < p < q \\ \gcd (p, q) = 1}} m(I_{p,q}) \leq \sum_{q \geq 1} \sum_{1 < p < q} m(I_{p, q}) \leq \sum_{q \geq 1}^{} \frac{2}{q^2} < \infty.
	\]
	By Borel-Cantelli, almost every $x \in [0,1]$ lies in only finitely many $I_{p,q}$. 
\end{theorem}

\begin{lemma}[Fatou's Lemma]
	If $f_n : X \to [0, \infty]$ is measurable, then 
	\[
		\liminf_{n \to \infty} f_n \, dm \leq \liminf_{n \to \infty} \int f_n \, dm.
	\]
\end{lemma}

\begin{proof}
	Let $g_n = \inf_{k \geq n} f_k$. So $g_n \leq g_{n+1} \xrightarrow[n \to \infty]{} \liminf_{k} f_k$ pointwise. 

	\begin{align*}
		\int g_n \, dm & \leq f_n \, dm \\
		\int \liminf_{k \to \infty} f_k \, dm & \leq \liminf_{n \to \infty} \int f_n \, dm
	\end{align*}
\end{proof}

Example (of why is it $\leq$ not $=$):
If $A \in S$, $0 \leq m(A) = m(A^c) < \infty$, $f_{2n} = \1_A$ and $f_{2n+1} = \1_{A^c}$, then
$\liminf_k f_k = 0$ and $\int f_k \, dm = m(A) = m(A^c) > 0$.

\section{September 19, 2022}

\epigraph{Shirt of the Day -- ``Bears will eat you''}{Charlie}

\noindent\underline{Last Time}:
$(X, S, m)$ positive measure space $f : X \to [0,\infty]$ measurable, then
\begin{align*}
	\int f \, dm = \sup_{g \leq f} \int g \, dm
\end{align*}


\begin{theorem}[Montone Convergence]
	If $f_n(x) \leq f_{n+1}(x) \to f(x)$ for ever $x$, then
	\begin{align*}
		\int f_n \xrightarrow[n \to \infty]{} \int f \, dm
	\end{align*}
\end{theorem}

\begin{corollary}[New Measure from Old]
	If we have $f: X \to [0,\infty]$ and $\tilde{m} : S \to [0,\infty]$ given by $\tilde{m} = \int \1_A f \, dm$, then $\tilde{m}$ is a measure.

	\underline{Moreover}
	\begin{align*}
		\int g \, d\tilde{m} = \int g f \, dm
	\end{align*} for all $g :X \to [0,\infty]$ measurable.

\end{corollary}

\begin{notation}
	$d \tilde{m} = f \, dm$ (equality in distributional sense)
\end{notation}

\begin{proof}
	Check countably additivity:
	Suppose $A_1, A_2, \ldots \in S$ disjoint, let  $A = \bigcup_{k=1}^{\infty} A_k$, compute
	\begin{align*}
		\tilde{m}(A) &= \int \1_{A} f \, dm
					 = \int \sum_{k=1}^{\infty} \1_{A_k} f \, dm \\
					 &= \sum_{k=1}^{\infty} \int \1_{A_k} f \, dm \quad \text{By Monotone Convergence} \\
					 &=  \sum_{k=1}^{\infty} \tilde{m}( A_k )
	\end{align*}
\end{proof}

\begin{definition}
	If $f: X \to [ - \infty, \infty]$ measurable, then  $\int f \, dm = \int f_{+} \, dm - \int f_{-} \, dm$ where
	$f_{\pm} = \max\{0, \pm f\}$
\begin{remark}
	$\int f \, dm$ defined only when $\int |f| \, dm < \infty$. Moreover,  $|\int f \, dm | \leq \int |f| \, dm$
\end{remark}
\end{definition}

Let $L^{1}(X,S, m)$ denote measurable $f: X \to [-\infty, \infty]$ with $\int | f| \, dm < \infty$.

We want $\|f\|_{L^{1}(X,S, m)} = \int |f| \, dm$ to be a norm.

Problem: If $m(\{x : f(x) \neq 0 \}) = 0$, then $\int |f| \, dm = 0$

\begin{remark}
	Notation: Identify a property $\mathbb{P}(x)$ ofa point $x \in X$ with the set $\{x \in X : \mathbb{P}(x)\}$
\end{remark}

For example, the set of $X$ where two functions  $f,g : X \to [-\infty, \infty]$ agree.

A property $\mathbb{P}$ holds $m$- almost everywhere if $m(\mathbb{P}^{c}) = 0$

\begin{example}
	If $f : X \to (-\infty, \infty)$ is measurable and $f$ vanishes almost everywere, then $\int f \, dm = 0$
\end{example}


For $f,g \in L^{1}(X,S, m)$, write $f \sim g \iff f=g$ (almost everywhere)

Observe: $\sim$ is an equivalence relation and $L^{1}(X,S, m) / \sim$ is a normed vector space.

\begin{observation}
  $\sim$ is an equivalence relation and $L^{1}(X,S, m) / \sim$ is a normed vector space.
\end{observation}

\subsection{Dominated Convergence}

\begin{definition}[Dominated Convergence]
	If
	\begin{align*}
		f_n : X &\to [-\infty, \infty] \text{ measurable }, \\
		g : X &\to [0, \infty] \text{ measurable }, \\
		\int g \, dm &< \infty
	\end{align*}
	suppose
	\begin{align*}
		f(x) = \lim_{n \to \infty} f_n(x)\, \forall x \in X
	\end{align*}
	and $|f_n (x) | \leq g(x)$ for $x \in X$ and $n \geq 0,$ \\
	then
	\begin{align}
		\lim_{n\to \infty} \int | f - f_n | \, dm = 0
	\end{align}
	and \\
	\begin{align*}
		\lim_{n\to \infty} \int f_n \, dm = \int f \, dm
	\end{align*}

\end{definition}

\begin{example}
	If $f_n : [0,1] \to [0, \infty]$ is defined by
	\begin{align*}
		f_n (x)
		\begin{cases}
			2^n & 2^{-n-1} < x < 2^{-n} \\
			0 & otherwise
		\end{cases}
	\end{align*}
	then $lim_{n \to \infty} f_n (x) = 0, \, \forall x \in [0,1]$ and  $\int f_{n} dx = \frac{1}{2} ,\, \forall n \geq 0$
	insert image
\end{example}

We will see later that the $f_{n}$ are converging (in some sense) to $\frac{1}{2}$Dirac measure at $0$. The dominating $g$ prevents this behaviour.

\begin{proof}
	\begin{align*}
		\int 2g \, dm &= \int \left( \lim_{n \to \infty} 2g - | f - f_{n}| \right) \, dm \\
				   &\leq \liminf_{n \to \infty} \int 2g - |f - f_{n}|\, dm \quad \text{ (By Fatou's Lemma)} \\
				   &= \int 2g \, dm - \limsup_{n \to \infty} \int | f - f_{n}| \, dm
	\end{align*}
	Since, $\int 2g \, dm < \infty$, conclude
	\begin{align*}
		&\limsup{n \to \infty} \int |f - f_n | \, dm = 0 \\ \text{finally \quad}
		& \left| \int f_n \, dm - \int f \, dm \right| \leq \int | f - f_n | \, dm
	\end{align*}
\end{proof}

\begin{corollary}
	If $f_{n} : X \to [-\infty, \infty]$ measurable and $\sum_{n=1}^{\infty} | f_n | \, dm < \infty$,
	then $f(x) = \sum_{n=1}^{\infty}f_n (x)$ exists for almost every $x \in X$ and $\int f \, dm = \sum_{n=1}^{\infty} \int f_{n} \, dm$
\end{corollary}

\begin{proof}
	Use dominated convergence. Let $g = \sum_{n=1}^{\infty} | f_n | : X \to [0,\infty]$.
	By hypothesis and monotone convergence, $\int g \, dm < \infty$

	Since $\int g \, dm < \infty$ and $g < \infty$ almost everywhere, if $x \in X$ and $g(x) < \infty$,
	then $\sum_{n=1}^{\infty} f_{n} (x)$ is absolutely convergent. That is $g(x) < \infty \implies f(x) = \sum_{n=1}^{\infty} f_{n}(x)$ converges.

	The partial sums $F_n (x) = \sum_{k=1}^{\infty} f_{n}(x)$ have $F_n \to f$ almost everywhere and $|F_{n}| \leq |g|$ a.e.

	By dominated convergence,
	\begin{align*}
	\sum_{k=1}^{\infty}f_{k} \, dm = \lim_{n \to \infty} \int F_n \, dm = \int f \, dm
	\end{align*}
\end{proof}

\begin{corollary}
	$L^{1}(X,S, m) / \sim$ is a \underline{complete} normed vector space.
\end{corollary}

\begin{proof}
	Suppose $\{f_n\} \subseteq L^{1}(X,S, m) / \sim$ is cauchy.

	Refining the sequence, we may assume
	\begin{align*}
	&\int_{\infty}|f_{n+1} - f_{n} | dm < 2^{-1} \\ \text{s.t. \quad}
		\sum_{n=1}^{\infty} &\int | f_{n+1} - f_{n} | dm < \infty
	\end{align*}
	Apply corollary to conclude $f(x) = \lim_{n \to \infty} f_{n}(x)$ converges for almost every $x$ and
	\begin{align*}
		\int |f - f_{n} | \, dm \to 0 \text{ as } n \to \infty
	\end{align*}
\end{proof}


\subsection{Chapter 2 - Borel Measure}

insert Diagram


generalize to:

insert diagram

\begin{definition}[Positive Linear Functional]
	Suppose $K$ is a compact metric space. A positive linear functional on $C(K)$ is a map $\Lambda : C(K) \to \mathbb{R}$ s.t.
	$\Lambda(\alpha f + \beta g) = \alpha \Lambda(f) + \beta\Lambda(g)$ for any
	$g \geq 0 \implies \lambda (g) \geq 0$
\end{definition}

\begin{example}
	\begin{enumerate}
		\item $K = [0,1]$, $\Lambda (f) = \int_{Riemann} f$
		\item $x \in X$, \, $\Lambda(f) = f(x)$
	\end{enumerate}
\end{example}

A finite positive Borel measure on $K$ is a positive measure on the Borel $\sigma$-algebra of $K$ with $m(X) < \infty$


\begin{theorem}
	There is a natural bijection between positive linear functionals on compact metric space $K$ and the finite positive Borel measures on $K$ and it is characterized by $\Lambda \leftrightarrow m$
	$\iff$
	 \begin{align*}
	\Lambda(f) = \int f \, dm \quad \forall f \in C(K)
	\end{align*}
\end{theorem}

\begin{remark}
	It is easier to prove a mor general version that uses only \underline{topological}-properties of the underlying space.
\end{remark}

\begin{definition}["Nice" topological space]
	Call a topological space $X$ nice if the following hold:
	\begin{enumerate}
		\item $X$ is Hausdorff: if $x,y \in X$ distinct, then there are open $U,V \subseteq X$ with  $x \in U$, $y \in V$ and $U \cap V  = \emptyset$
		\item $X$ is locally compact:
			if $x \in X$, then there are $K, V\subseteq X$ with \\
			$x \in \underset{open}{V} \subseteq \underset{compact}{K}$.
			% $V$ open, $K$ compact.
		\item $X$ is "really" $\sigma$-compact: if $V \subseteq X$ open, then there are compact $K_n \subseteq X$ with $V = \bigcup_{n=1}^{\infty} K_n$
	\end{enumerate}
\end{definition}


\section{September 21st, 2022}

\epigraph{Shirt of the Day -- ``You are all too tall''}{Charlie}

\begin{definition}
	$(X,T)$ \underline{nice} topological space if: Hausdorff, locally compact, and every open set is the union of countably many compact sets.
\end{definition}

\begin{example}
	$(K,d)$ a compact metric space, $C_c (x)$ space of $f \in C(x)$ such that $\{x\in X: f(x) \neq 0 \}$ is compact.

	That is, $C_c (x)$ is the space of compactly supported continuous function:
		$\Lambda : C_c(X) \to \mathbb{R}$ is posiive linear if:
	\begin{align*}
		&\Lambda(\alpha f + \beta g) = \alpha \Lambda(f) + \beta \Lambda(g) \quad \text{linearity} \\
		&\Lambda(f) \geq 0 \quad \text{ when } f \geq 0
	\end{align*} 
\end{example}


\begin{theorem}
	There is a unique measure $m$ on (X,B) s.t.
	\begin{align*}
		\Lambda(f) = \int f dm \quad \forall f \in C_c (X)
	\end{align*} 
\end{theorem}

\begin{definition}[defining m]
	for $A \subseteq X$ borel and $m$ a borel measure, $m(A) = \int \1_{A} dm$
\end{definition}

We want to approximate $\1_A$ with elements of $C_c (X)$

For $U \subseteq X $ open, let $m(U) = \sup\{\Lambda (f) : f \in C_c (X), \, 0 \leq f \leq \1_{U} \}$

insert image.

For $A \subseteq X$ arbitrary, let $m(A) = \inf\{m(U) : A \subseteq U_{open} \subseteq X\}$

We argue the restriction of $m$ to Borel $\sigma$-algebra is a measure.

\subsection{Building Test functions}

\begin{lemma}
	If $K_{1}, K_{2}$ compact and $K_{1} \cap K_2 = \emptyset$, then there are open $U_{1} , U_{2}$ with $K_{1} \subseteq U_{1}$, $K_{1} \subseteq U_{2}$, and $U_1 \cap U_2 = \emptyset$
\end{lemma}

\begin{proof}
	Since $X$ hausdorff and $K_1 \cap K_2 = \emptyset$, for  $x \in K_1$ and $y \in K_2$, there are open $U_{x,y}$, and $V_{x,y}$ with $x \in U_{x,y}$ and $y \in V_{x,y}$ and $U_{x,y} \cap V_{x,y} = \emptyset$, for any $x \in K_1$,
	\begin{align*}
		K_2 \subseteq \bigcup_{y \in K_2} V_{x,y}
	\end{align*} 

	So there are $y_1, ,\ldots , y_n$ with $K_2 \subseteq \bigcup_{k=1}^n V_{x, y_{k}} = V_{x}$

	Now have $x \in U_{x}$, $K_2 \subseteq V_x$, and $U_{x} \cap V_{x} = \emptyset$. Since $K_{1} \subseteq bigcup_{x \in K_1} V_x$, have 
	\begin{align*}
		K_1 \subseteq U_{x} \cup \ldots \cup U_{x_{n}} = U
	\end{align*} 
	Let $V = V_{x_{1}} \cap \ldots \cap V_{x_{n}}$. Have $K_{1} \subseteq U$, $K_{2} \subseteq V$ and $U \cap V = \emptyset$
\end{proof}

\begin{lemma}
	If $K$ compact, then there is $K \subseteq U_{open} \subseteq \bar{U}_{compact} \subseteq X$.
\end{lemma}

\begin{proof}
	Since $(X,T)$ locally compact, every $x \in K$ has $x \in U_{x} \subseteq \bar{U}_x \subseteq X$
	Since $K$ compact, $K \subseteq U_{x_1} \cup \ldots \cup U_{x_n} \subseteq \bar{U}_x \cup \ldots \cup \bar{U}_{x_{n}}$
\end{proof}

\begin{lemma}
	If $K_1 \subseteq U_1$, then there are  $K_1 \subseteq U_{2} \subseteq K_1 \subseteq U$
\end{lemma}

\begin{proof}
	By the previous lemma, there is $K_1 \subseteq V_{open} \subseteq \bar{V}_{compact}$.

	Replace $U_{1}$ by $U_{1} \cap V$ to obtain $\bar{V}_{1}$ compact. By 2nd previous lemma, we can separate $K$ and $\bar{U}_1 \setminus U_{1}$

	Choose open $V_1$ and $V_2$ with $K_{1} \subseteq V_{1}$,  $\bar{U}_1 \setminus U_1 \subseteq V_{2}$
and $V_{1} \cap V_{2} = \emptyset$ can assume $V_{1} \subseteq U_{1}$

Finish
\end{proof}

\begin{lemma}[Uhryson's lemma]
	If $K_{compact} \subseteq U_{open}$, then there is  $f \in C_{c} (X)$ with $\1_k \leq f \leq \1_U$
\end{lemma}

\begin{remark} dyadic rationals
		A dyadic rational is a number $q = \frac{k}{2^{n}}$ with $k,n \in \mathbb{N}$
\end{remark}

\begin{proof}
	Choose for dyadic rationals $q \in [0,1]$,
	\begin{align*}
		K \subseteq  {\underset {open} {U_{q}}} \subseteq {\underset {compact} {\bar{U}_{q}}} \subseteq U \quad \text{ s.t } 
		q \leq q' \implies \bar{U}_{q'} \subseteq U_{q}	\\
	\end{align*} 

	Now construct by induction on denominator: choose
	\begin{align*}
		U_{\frac{2^{k+1}}{2^{n+1}}} \quad {\text s.t. }\quad 
		\bar{U}_{\frac{k+1}{2^{n}}} \subseteq V_{\frac{2^{k} + 1}{2^{n+1}}} \subseteq \bar{U}_{\frac{2^{k+1}}{2^{n+1}}} \subseteq U_{\frac{k}{2^{n}}}
	\end{align*} 

	But the previous lemma does exactly this. For $x \in U_{0}$, let $f(x) = \sup\{q : q \in U_{q}\}$
	Otherwise, let $f(x) = 0$
\end{proof}

\begin{exercise}
	Prove $f$ continuous and $\1_{\bar{U}_1} \leq f \leq \1_{U_{0}}$ so $f \in C_c(X)$
\end{exercise}


\begin{lemma}
	If $K \subseteq \underset{open}{U_{1}} \cup \ldots \cup \underset{open}{U_n}$, then there are 
	$f_1, \ldots, f_n \in C_c(X)$ s.t. $f_k \subseteq \1_{U_{k}}$ for $k = 1, \ldots ..., n$ and
	$\1_k \leq f_1 + \ldots+ f_n \leq 1$
\end{lemma}

[insert image]

\begin{proof}
	Step1: for any $x \in K$, there is $K$ and $x \in \underset{open}{V_x} \subseteq \underset{compact}{\bar{V}_x} \subseteq U_x$
	Since $K$ compact, $K \subseteq V_{x_{1}} \cup \ldots \cup V_{x_N}$

	Group $V_{x_j}$ according to $x_j \in U_{k}$. Obtain compact sets
	\begin{align*}
		K_k = \bigcup_{\bar{V}_{x_j} \subseteq U_k} \bar{V}_{x_j} \subseteq U_k
	\end{align*}
	with $K \subseteq K_1 \cup \ldots \cup K_n$


Choose $g_k \in C_{c} (X)$ s.t. $\1_{K_k} \leq g_K \leq \1_{U_{K}}$. We have $g_1 + \ldots + g_n \geq \1_{K}$. We need to limit to $1$.
Let $f_1 = g_1$, $f_{k+1} = (1-g_1) \cdots (1 - g_{k}) g_{k+1} \leq \1_{U_{k+1}}$

\underline{Compute}:
\begin{align*}
\1_K \leq f_1 + \ldots + f_n = 1 - (1 - g_1) \cdots (1 - g_n) \leq 1
\end{align*} 
\end{proof}




\section{September 26th, 2022}

\epigraph{``What's one thing good about these cables [all over the place] is that you are constantly reminded
	of Thermodynamics... Entropy.''}{Charlie (said after he tripped on some cables on the classroom floor)}

\subsection{Outer Measures}
A general tool (devised by Caratheodory) to constrain measures. Outer measures is a relaxation of measure.
It is easier to construct outer measures. The main point of this section is a theorem about
constructing measures from outer measures.

\begin{definition}[Outer Measure]
	An outer measure on a measurable space $(X,S)$ is a function $m: S \to [0, \infty]$ such that
	\begin{enumerate}
		\item $m(\emptyset) = 0$
		\item $A \subseteq B$ implies $m(A) \leq m(B)$.
		\item $m \left(\bigcup_{k=1}^{\infty} A_k\right) \leq \sum_{k = 1}^{\infty} m(A_k)$.
	\end{enumerate}
	these three conditions can be summed up by: non-triviality, monotonicity, and countable subadditivity.
\end{definition}

\begin{example}
  Lebesgue outer measure.
	\[
		L^* : \R^d \to [0, \infty].
	\]
	given by
	\[
		L^*(A) = \inf \left\{\sum_k \left|Q_k\right| : A \subseteq \bigcup_{k = 1}^{\infty} Q_k \right\}.
	\]
\end{example}

\begin{question}
Is $J^*$ an outer measure?
\end{question}
No -- fails countability subadditivity condition. $J^*$ is only finitely subadditive.

\begin{definition}[m-Caratheodory]
	If $(X,S,m)$ is an outer measure space, then call $A \in S$ m-Caratheodory if
	\[
		m(B) = m(B \cap A) + m(B \cap A^c) \quad \text{ for } B \in S.
	\]
\end{definition}

Let $S_m = \{ A \in S : A \textrm{ m-Caratheodory}\}$. Caratheodory basically means ``good for splitting''.

\begin{theorem}
	$S_m$ is a $\sigma$-algebra and $(X, S_m, m|_{S_m})$ is a positive measure space.
\end{theorem}

\begin{proof}
	Step 1: $\emptyset \in S_m$.
	\begin{align*}
		m(B \cap \emptyset) + m(B \cap \emptyset^c) & = m(\emptyset) + m(B) \\
																								& = 0 + m(B)
	\end{align*}

	Step 2: $A \in S_m \implies A^c \in S_m$.
	\begin{align*}
		m(B) & = m(B \cap A) + m(B \cap A^c) \\
				 & = m(B \cap (A^c)^c) + m(B \cap A^c)
	\end{align*}

	Step 3: $A_1, A_2 \in S_m \implies A_1 \cup A_2 \in S_m$.
	\begin{align}
		m(B) & = m(B \cap A_1) + m(B \cap A_1^c) \\
				 & = m(B \cap A_1 \cap A_2) + m(B \cap A_1 \cap A_2^c) \nonumber \\
				 & \phantom{=} + m(B \cap A_1^c \cap A_2) + m(B \cap A_1^c \cap A_2^c) \label{step3.2} \\
				 & \geq m(B \cap (A_1 \cup A_2)) + m(B \cap A_1^c \cap A_2^c) \\
				 & = m(B \cap (A_1 \cup A_2)) + m(B \cap (A_1 \cup A_2)^c) \label{step3.4} \\
				 & \geq m(B)
	\end{align}

	Step 3$'$: $A_1, A_2 \in S_m$ disjoint and $m(B) \leq \infty$ implies,
	\[
		m(B \cap (A_1 \cup A_2)) = m(B \cap A_1) + m(B \cap (A_1 \cup A_2)^c).
	\]
	Using disjointness, lines \ref{step3.2} and \ref{step3.4} give
	\begin{multline*}
		m(B \cap A_1) + m(B \cap A_2) + m(B \cap (A_1 \cup A_1)^{c})  \\
		= m(B \cap (A_1 \cup A_2) + m(B \cap (A_1 \cup A_2)^{c}).
	\end{multline*}

	Step 4: $A_k \in S_m$ disjoint $\implies \bigcup_{n = 1}^{\infty} A_k \in S_m$
	and $m(\bigcup_{k=1}^{\infty} A_k = \sum_{k=1}^{\infty} m(A_k)$.
	\[
		m(B) \leq m\left(B \cap \bigcup_{k=1}^{\infty} A_k \right) + m\left(B \cap \left(\bigcup_{k=1}^{\infty} A_k\right)^{c}\right).
	\]
	We need to prove the other direction i.e. $\geq$

	We may assume $m(B) \leq \infty$.
	\begin{align*}
	  m \left(B \cap \bigcup_{k = 1}^{\infty} A_k \right) & \geq m \left(B \cap \bigcup_{k = 1}^{n} A_k \right) \\
		& = \sum_{k = 1}^{n} m(B \cap A_k) \tag{Step 3$'$} \\
		& \xrightarrow[n \to \infty]{} \sum_{k = 1}^{\infty} m(B \cap A_k) \\
		& \geq m \left( B \cap \bigcup_{k = 1}^{\infty} A_k \right)
	\end{align*}
	\begin{align*}
	  m & \left(B \cap \bigcup_{k = 1}^{\infty} A_k \right) + m \left(B \cap \left( \bigcup_{k = 1}^{\infty} A_k\right)^{c}\right) \\
		& = \lim_{n \to \infty} \left\{ m \left( B \cap \bigcup_{k = 1}^{n} A_k \right) + m \left(B \cap \left(\bigcup_{k = 1}^{\infty} A_k\right)^{c}\right)\right\} \\
		& \leq \liminf_{n \to \infty} \left\{ m \left(B \cap \bigcup_{k = 1}^{n} A_k\right) + m \left(B \cap \left(\bigcup_{k = 1}^{n} A_k\right)^{c}\right)\right\} \\
		& = m(B)
	\end{align*}
\end{proof}

\subsection{Riesz Representation Theorem (cont.)}

Suppose that $(X,T)$ is a  nice topological space.
If $K_{compact} \subseteq U_{open}$, then there is $f \in C_c(x)$ with $\1_k \leq f \leq \1_u$.
Suppose $\Lambda : C_c(x) \to \R$ is positive linear.
\begin{align*}
  & m(\underset{open}{U}) = \sup \left\{\Lambda(f) : f \in C_c(x), 0 \leq f \leq \1_k \right\} \\
  & m(\underset{arbitrary}{A}) = \inf \left\{ m(U): U \textrm{ open}, U \geq A\right\} \\
\end{align*}

\begin{lemma}
  $m$ is an outer measure on $(X, \mathcal{P}(X))$.
\end{lemma}

\begin{proof}
	$\emptyset$ open, so $m(\emptyset) = 0$.
	\[
		A \subseteq B \implies m(A) \leq m(B).
	\]
	automatically from the definition.
	We need to check
	\[
		m \left(\bigcup_{k = 1}^{\infty} A_k \right) \leq \sum_{k = 1}^{\infty} m(A_k).
	\]
	Choose $U_k \supseteq A_k$ open with $m(U_k) \leq m(A_k) + \frac{\varepsilon}{2^k}$.
	\[
		m \left(\bigcup_{k = 1}^{\infty} A_k \right) \leq m \left( \bigcup_{k = 1}^{\infty} U_k \right).
	\]
	so it is enough to show $\leq \sum_{k = 1}^{\infty} m(U_k) \leq \varepsilon + \sum_{k = 1}^{\infty} m(A_k)$.

	Suppose $0 \leq f \leq \1_{\bigcup_{n = 1}^{\infty} U_k}$. Then $f \in C_{c}(x)$.
	Since $f$ has support, there is an $n \geq 1$ so $0 \leq f \leq \1_{\bigcup_{n = 1}^{\infty} U_k}$.
	Using partitions of unity, $f = f_1 + \ldots + f_n$ with $0 \leq f_k \leq \1_{U_k}$. So
	\begin{align*}
		\Lambda(f) & = \Lambda(f_1) + \ldots + \Lambda(f_n) \\
							 & = m(U_1) + \ldots m(U_n)
	\end{align*}
	So,
	\[
		m \left(\bigcup_{k = 1}^{\infty} A_k \right) \leq \sum_{k = 1}^{\infty} m(A_k).
	\]
\end{proof}

Our next goal is to show that $S_m \supseteq \textrm{Borel}$.

\begin{lemma}
  If $U$ open, $m(U) < \infty$, and $\varepsilon > 0$, then there is compact $K \subseteq U$ with $m(U \setminus K) < \varepsilon$.
\end{lemma}

\begin{proof}
	Since $m(U) < \infty$, we can choose $f \in C_c(X)$ with $0 \leq f \leq \1_U$ 
	and $\Lambda(f) \geq m(U) - \varepsilon$.

	Let $k = \left\{x \in X : f(x) \neq 0 \right\}$. Note $K \subseteq U$ and $K$ compact. 
	If $g \in C_c(x)$ and $0 \leq g \leq \1_{U \setminus K}$, then $0 \leq f + g \leq \1_k$. 
	\[
		m(U) \geq \Lambda(f + g) = \Lambda (f) + \Lambda (g) > \Lambda(U) - \varepsilon + \Lambda(g).
	\]
	since $g$ arbitrary, $m(U \setminus K) \leq \varepsilon$.
\end{proof}

\begin{lemma}
  If $K_1, K_2$ disjoint compact, then $m(K_1 \cup K_2) = m(K_1) + m(K_2)$.
\end{lemma}

\begin{proof}
	Since $(X,T)$ nice, there are open $U_1 \supseteq K_1$ and $U_2 \supseteq K_2$ with $U_1 \cup U_2 = \emptyset$.

	[insert image]

	making $U_1, U_2$ smaller, we may assume
	\[
		m(U_1) \leq m(K) + \varepsilon \quad \textrm{ and } \quad m(U_2) \leq m(K_2) + \varepsilon.
	\]
	From the definitions, $m(U_1 \cup U_2) = m(U_1) + m(U_2)$.
	\footnote{Why? If $0 \leq f \leq \1_{U_1 \cup U_2}$, then $\1_{U_k} - f \in C_c(x)$.}
	We then calculate 
	\begin{align*}
		m(K_1 \cup K_2) & \leq m(U_1 \cup U_2) \\
										& = m(U_1) + m(U_2) \\
										& \leq m(K_1) + m(K_2)
	\end{align*}
\end{proof}

\begin{lemma}
  The Borel Sets are $m$-Caratheodory, $B \subseteq S_m$.
\end{lemma}

\begin{proof}
	Since $B$ generated by the open sets and $S_m$ is a $\sigma$-algebra,
	it is enough to show every open $U$ is Caratheodory. 
	Since $(X,T)$ locally compact, we need only consider $U$ with compact closure.
	Let $B \subseteq X$ be arbitrary. We need 
	\[
		m(B) \geq m(B \cap U) + m(B \cap U^{c}).
	\]
	Choose $V \supseteq B$ open such that $m(V) \leq m(B) + \varepsilon$. We then need, 
	\[
		m(V) \geq m(V \cap U) + m(B \cap U^{c}).
	\]
	We may assume $U \subseteq V$ (an exercise!).
	Choose compact $K \subseteq U$ and $G \subseteq V$ with $m(U \setminus K) \leq \varepsilon$ 
	and $m(U \setminus G) \leq \varepsilon$.
	\begin{align*}
		m(U) + m(V \cap U^{c}) & \leq m(U \setminus K) + m(K) + m(V \cap G^{c}) + m(G \cap U^{c}) \\
													 & \leq m(K) + m(G \cap U^{c}) + 2\varepsilon \\
													 & = m(K \cup (G \cap U^{c})) + 2\varepsilon \\
													 & \leq m(V) + 2 \varepsilon
	\end{align*}
\end{proof}


\section{ 2022-09-28 }


Reviewing: $(X,T)$ nice $\Lambda : C_c (X) \to \mathbb{R}$ positive linear

\begin{align*}
	m(U) &= \sup\{\Lambda(f) : 0 \leq f \leq \1_c\} \\
	m(A) &= \inf\{m(U) : U \contains  A\}
\end{align*} 

\begin{lemma}
	$\mu$, outer measure in  $(X,\mathcal{P}(X))$ and Borel $\subseteq$ Caratheodory.
\end{lemma}

\begin{lemma}
	restrict $m$ to Borel $\sigma$-algebra $\mathbb{B}$ so $(X, \mathcal{B}, m)$ is a positive measure space
\end{lemma}

\begin{lemma}
	For $f \in C_c (X)$, $\Lambda(f) = \int f dm$
\end{lemma}

\begin{proof}
	Use definition of $m$ and partition of unity

	Using the linearity of $\Lambda$ and  $\int dm$, we may assume $0 \leq f \leq 1.$

	Construct 
	 \begin{align*}
	    \Lambda (f) = ( \max f^+ ) \Lambda \left(\frac{f^+}{\max f^+}\right) - (\max f^- ) \Lambda \left(\frac{f^-}{max f^-} \right)
	\end{align*} 
	and same for $\int dm$


	Let $U = \{x \in X : f(x) > 0\}$. Note  $\bar{U}$ is compact and $m(U) < \infty$.

	Let  $\varepsilon = \frac{1}{N}$ for some large  $N \geq 1$.

	Let
	\begin{align*}
		E_k = f^{-1}\left( [ (k-1) \varepsilon, K \varepsilon]\right)
	\end{align*}
	replace $E_1$ with its closer $\bar{E}_1$ so that $U_k$ cover $\bar{U}_1$
	 \begin{align*}
	\sum_{K} (K -1) \varepsilon \1_{E_k} \leq f \leq \sum_{K} k \varepsilon \1_{E_k}
	\end{align*} 

	Since $E_k \in B$, we can choose $U_k \contains E_k$ open with $\delta >0 $ arbitrary.
	We then have
	\begin{align*}
		m(U_k) &\leq m(E_k) + \delta \quad \text{and} \\
		U_k &\leq f^{-1} ( [ 0, (k+1)\varepsilon))
	\end{align*} 

	Choose partition of unity: $g_k \in C_c(X)$ such that  $0 \leq g_k \leq \1_{U_k}$ and \\
	$\1_{\bar{U}} \leq \sum_{K} g_k \leq 1$ 
	Then:
	\begin{align*}
	\Lambda(f) &= \Lambda( \sum_{k} g_k f) \\
			   &=  \sum_{k} \Lambda( g_k f) \\
			   &\leq  \sum_{k} (k+1) \varepsilon \Lambda( g_k ) \\
			   &\leq  \sum_{k} (k-1) \varepsilon ( \mu( E_k ) + \delta ) \\
			   &\leq \sum_{k} (k-1) \varepsilon \mu( E_k ) + \sum_{k} (2 \varepsilon [ \mu (E_k) + \delta] + (k+1) + \delta) \\
			   &\leq \int \sum_{k} (k-1) \varepsilon \1_{E_k} dm + 2 \varepsilon  \mu( U_k ) + C N \delta  \\
			   &\leq \int f dm + 2 \varepsilon  \mu(\bar{U}) + C N \delta  \\
	\end{align*} There is a symmetric computation of other inequality.
\end{proof}




\begin{theorem}[Riesz-representation theorem]
	If $(X,T)$ is nice and  $\Lambda : C_c (X) \to \mathbb{R}$ positive linear, then there is a \underline{unique}. Positive measure $\mu$ on $(X,\mathcal{B})$ such that  $\Lambda(f) = \int f dm$
\end{theorem}

\begin{lemma}
	If $(X,T)$ nice,  $(X, \mathcal{B}, m)$ positive measure space, and $m(X) < \infty$, then m is \underline{regular}, that is
	\begin{align}
		m(A) &= \inf\{m(U) : \underset{open}{U} \contains A\} \\
			 &= \sup\{m(K) : \underset{compact}{K} \subseteq A\}
	\end{align} 
\end{lemma}

\begin{proof}
	let $S_{*} = \{A \in S: (1) \text{ holds for $A$} \}$
It is enough to show $S_{*}$ contains open sets and is closed under complement and countable union.

Since $\mathcal{B}$ is the smallest $\sigma$-algebra containing open sets, this implies $S \contains \mathcal{B}$

\begin{enumerate}
	\item[Step 1:] Suppose $U$ open, from monotonicity we have $m(U) = \inf\{m(V) : V \contains U \}$
		write: $U = \bigcup_j K_j$ with  $K_j \subseteq K_{j+1}$ compact.
		\begin{align*}
			m(U) = m(\bigcup_j K_j = \lim_{j \to \infty} m(K_j)
		\end{align*} .

		Conclude
		\begin{align*}
			m(U) = \sup\{m(K) : \underset{compact}{K} \subseteq U \}
		\end{align*}

	\item[Step 2:] Suppose $A \in S$, choose $\underset{open}{U} \contains A \underset{compact}{K}$
		so $U^{c} \subseteq A^{c} \subseteq K^{c}$ and $m(U \setminus K) = m(U \setminus A) +m (A \setminus K) < \varepsilon$

		Then
		\begin{align*}
			m(\underset{open}{K^{c}} \setminus \underset{}{U^{c}}) < \varepsilon
		\end{align*}
		Choose $\underset{compact}{G} \subseteq K^{c}$ so $m(^{c} \setminus G) < \varepsilon$. Then 
	\begin{align*}
		g \bigcap U^{c} \subseteq A^{c} &\subseteq K^{c} \quad \text{and} \\
	m(K^{c} \setminus (G \cap U^{c})) &\leq m(k^{c} \setminus V^{c}) + m(K^{c} \setminus G) \\
									 &< 2\varepsilon
	\end{align*} 	
\item[Step 3:]
	Finite unions (automatic):
	If \begin{align*}
		K_1 &\subseteq A_1 \subseteq U_1 \\
		K_2 &\subseteq A_2 \subseteq U_2, \quad \text{then} \\
		K_1 \cup K_2 &\subseteq A_1 \cup A_2 \subseteq U_1 \cup U_2
	\end{align*} 
\item[step 4:]
	Suppose $A_j \in S$, wts $\bigcup_j A_j \in S$. By steps 2 and 3, we may assume  $A_j$ disjoint.

	Next choose $K_j \subseteq A_j \subseteq U_j$ with  $m(U_j \setminus K_j) \leq \frac{\varepsilon}{2j}$ 

	Have $\cup_j K_j \subseteq \cup_j A_j \subseteq \cup_j U_j$ and  $m(\cup U_j \setminus \cup_j K_j ) \leq 2 \varepsilon$ 

	Then 
	\begin{align*}
		m( \cup)j U_j \setminus \cup_j K_j) &= \lim_{n \to \infty} m ( \cup_j U_j \setminus \cup K_j) \quad \text{choose finite $n$ so} \\
		K_1 \cup \ldots \cup K_n &\subseteq \cup_j A_j \subseteq \cup_j U_J \quad \text{and} \\
		m(\cup_j U_j &\setminus K_1 \cup \ldots \cup K_n )) \leq 3\varepsilon
	\end{align*} 
\end{enumerate}

\end{proof}

\begin{lemma}
	If $(X, T)$ nice, and $(X,\mathcal{B}, m)$ is a positive measure space, then
	\begin{align*}
		M(A) &= \inf\{m(U) : U \contains A\} \\
			 &= \sup \{m(K) : K \subseteq A \} \quad \text{whenever} \, A \in \mathcal{B},
			 \, A \subseteq U, \, m(U) < \infty
	\end{align*}
\end{lemma}

\begin{proof}
	Restrict to subspace $(U,T)$ and apply previous lemma.
\end{proof}

\subsection*{uniqueness}
It remains to show the uniqueness of the Riesz measure: suppose $m_1$ and $m_2$ are positive measures on $(X,\mathcal{B})$ and $\int f dm_1 = \int f dm_2$ for some  $f \in C_c(X)$.


We want to show that it must be  $m_1 = m_2$

\begin{proof}
	Suppose $m_1 \neq m_2$
	 \begin{enumerate}
		 \item[step 1:] there is an open $U$ and $A \subseteq U$ with $m_1 (U) < \infty$, $m_2 (U) < \infty$, and $m_1 (A) \neq m_2(A)$

		 \item[step 2:] WLOG, $m_1(A) < m_2 (A).$ Since $m_1$ and $m_2$ are regular on $U$, choose 
			 \begin{align*}
				 &\underset{compact}{K} \subseteq A \subseteq \underset{open}{U} \quad \text{so} \\
				 &m_1 (A) \leq m_1 (U) < m_2 (K) \leq m_2 (A)
			 \end{align*} 
			 and Choose $f \in C_c (X)$ so $\1_K \leq f \1_U$.
			 We then obtain
			 \begin{align*}
				 m_2 (K) &\leq \int f dm_2 \\
					 &= \int f dm_1 \leq m_1 (U)
			 \end{align*} 
			 Contradicting (star)
	\end{enumerate}
\end{proof}


Next: $Q \subseteq \mathbb{R}^d$ closed rectangles.
\begin{align*}
\Lambda : C(Q) \to \mathbb{R} \quad \text{is} \\
\Lambda(f) = \int_{Q} f \quad \text{(Riemann)}
\end{align*} Then $\Lambda(f) = \int f dL$ where $L$ is the restriction of $L^*$ to Borel sets.
\subsection*{Midterm 1 Review}
For the midterm on October 3rd, 2022, we should know basic definitions, basic examples and basic theorems. The following topics will be covered:
\begin{enumerate}
	\item Riemann integration
	\item Riemann integrability characterized on $L^*$ (Lebesgue outer measures)
	\item Measurable spaces
	\item Measurable functions
	\item Topological spaces
	\item Abstract integration
	\item Convergence Theorems (Monotone, Fatou's, Dominated Convergence)
	\item Borel measures
	\item Riesz theorem
	\item Outer measures, Caratheodory
\end{enumerate}

\section{ 2022-10-05 }

review

\begin{theorem}
	Riesz $ \Lambda : C_c (X) \to \mathbb{R}$ positive linear on nice space $X$ then $\Lambda(f) = \int f dm$ for a unique borel measure on $X$.
\end{theorem}

We have tools for constructing measure. Namely, the Caratheodory theorem for outer measure: Every outer measure is a positive measure on its Caratheodory sets.

Now we have two candidates for Lebesgue measure.

\begin{enumerate}
	\item $\Lambda (f) = \underset{Riemann}{\int f}$
		Which should be given by: $\Lambda(f) = \int f dL$
	\item The outer Lebesgue measure  $l^* (A) = \inf\{\sum_K |Q_k| : A \subseteq \bigcup_{K} Q_k \}$ Should be $L$ on its Caratheodory sets.
\end{enumerate}

These are essentially the same.

\begin{lemma}
	$L^*$ is an outer measure on $\mathcal{P}(R^d)$
\end{lemma}

\begin{proof}
	$L^* (\emptyset)= 0$ because any collection of closed rectangles covers  $\emptyset$
\end{proof}



\section{October 12, 2022}

\epigraph{*accidentally kicks over classroom phone* ``I don't understand why there is a phone is here at all.''}{Charlie}

\subsection{Outer Hausdorff Measure}
The $\alpha$-dimensional outer Hausdorff measure of a set $A \subseteq \mathbb{R}^d$ is
\[
	H^{\alpha} (A) = \sup_{r > 0} H_{r}^{\alpha} (A)
\]
where
\[
	H_{r}^{\alpha} (A) = \inf\left\{ \sum_{k \geq 0} \diam (B_k)^{\alpha} : A \subseteq \bigcup_{k \geq 0} B_k, \; \sup_{k \geq 0} \diam (B_{k} ) \leq r \right\}
\]

\begin{example}
	$H_{1}^{\alpha} \leq 1$ for each $A \subseteq B_{\frac{1}{2}}$

	$H_{1}^{1}$ does not measure length.

	$\sup_{r > 0} \{H_{r}^{1}\}$ does measure length
\end{example}

\begin{lemma}
	$H^{\alpha}$ is an outer measure.
\end{lemma}

\begin{proof} several steps.
	\begin{enumerate}
		\item[Step 1:] $H_{r}^{\alpha}$ is an outer measure by same argument for $L$.
			\[
				H_{r}^{\alpha} (\emptyset)= 0 \quad \text{and} \quad A \subseteq B \implies H_{r}^{\alpha} (A) \leq H_{r}^{\alpha} (B)
			\]
			is immediate from definition.
			The subadditivity:
			\begin{align*}
			H_{r}^{\alpha} ( \cup_{k \geq 0} A_{k}) \leq \sum_{K \geq 0} H_{r}^{\alpha} (A_k)
			\end{align*} follows from countable union of countable sets is countable.

		\item[Step 2:] It is immediate that $H^{\alpha} (\emptyset) = 0$ and
			\begin{align*}
			A \subseteq B \implies H^{\alpha} (A) \leq H^{\alpha}(B)
			\end{align*} Assume
			\begin{align*}
				\infty > H^{\alpha}( \bigcup_{k \geq 0} A_k ) &= \sup_{r > 0} H_{r}^{\alpha} ( \bigcup_{k \geq 0} A_k) \\
															  &\leq \varepsilon + H_{r}^{\alpha} ( \bigcup_{k \geq 0} A_k) \\
															  &\leq \varepsilon + \sum_{k \geq 0} \sup_{r > 0} H_{r}^{\alpha} ( A_k) \\
															  &= \varepsilon + \sum_{k \geq 0} H^{\alpha} ( A_k)
			\end{align*}
			since $\varepsilon$ arbitrary,
			\begin{align*}
				H^{\alpha} ( \bigcup_{k \geq 0} A_k) \leq \sum_{k \geq 0} H^{\alpha} ( A_k)
			\end{align*}
			In the infinite case, use a similar argument to show
			\begin{align*}
				H^{\alpha} ( \bigcup_{k \geq 0} A_k) = \infty \implies \sum_{k \geq 0} H^{\alpha} ( A_k) = \infty
			\end{align*}
	\end{enumerate}
\end{proof}

\subsection{Metric Outer Measures}

\begin{definition}[Metric outer measure]
	A metric outer measure on a metric space is an outer measure $m : \mathcal{P}(X) \to [0,\infty]$ such that
	\[
		m(A \cup B) = m(A) + m(B) \quad \text{whenever} \quad d(A,B) > 0
	\]
	where
	\[
		d(A,B) = \inf_{\substack{a \in A \\ b \in B}} d(a,b).
	\]
\end{definition}

\begin{example}
	$H^{\alpha}$ is a metric outer measure. Indeed
	\[
		H_{r}^{\alpha}(A \cup B) = H_{r}^{\alpha}(A) + H_{r}^{\alpha}(B) \quad \text{where} \quad d(A,B) > 0
	\]
	because we can effectively split any cover of $A \cup B$ into a cover of $A$ and a cover of $B$.
\end{example}

\begin{lemma}
	Borel sets are Caratheodory for metric outer measures.
\end{lemma}

\begin{proof} \hfill
	\begin{enumerate}
		\item[Step 1] Show if $A = \bigcup_{k \geq 1} A_k$, $A_{k+1} \contains A_k$,
			and $d(A_k, A \setminus A_{k+1}) >0$, then $m(A) = \lim_{k \to \infty} m(A_k)$.
			\begin{enumerate}
				\item[Case 1:]
					\begin{align*}
					&\sum_{k \geq 1} m(A_{k+1} \setminus A_k) = \infty \quad \text{for $j=1$ or $2$, then}\\
					&\sum_{k \geq 1} m(A_{j+2k+1} \setminus A_{j + 2k})  = \infty \\
					& = \lim_{n \to \infty} \sum_{k = 1}^{n} m(A_{j+2k+1} \setminus A_{j + 2k}) \\
					& = \lim_{n \to \infty} m(\bigcup{k = 1}^{n} (A_{j+2k+1} \setminus A_{j + 2k})) \\
					& \leq \lim_{n \to \infty} m(A_{j+2k+1})
					\end{align*}
					So both $m(A)$ and $\lim_{n \to \infty} m(A_n)$ are $\infty$.

				\item[Case 2:]
					\begin{align*}
						\sum_{n=1}^{\infty} m(A_{k+1} \setminus A_{k}) < \infty \\
					\text{then} \quad \lim_{k \to \infty}  m(A_{k+1} \setminus A_{k}) = 0\\
					\lim_{k \to \infty}  m(A \setminus A_{k+1}) = 0\\
					\end{align*} now
					\begin{align*}
						m(A) &\leq m(A_{k}) + m(A_{k+1} \setminus A_{k}) + m(A \setminus A_{k+1}) \\
							 &\leq \limsup_{k \to \infty} m(A_{k}) \\
							 &= \lim_{k \to \infty} m(A_{k}) \\
							 &\leq m(A)
					\end{align*}
			\end{enumerate}
		\item[Step 2] For $A$ closed, we check
			\begin{align*}
			m(B) = m(B \cap A) + m(B \cap A^{c}) \quad \forall \, B \subseteq \mathbb{R}^d
			\end{align*}
			Write $A^{c} = \bigcup_{k \geq 1} D_k$ where $D_{k} = \{x \in \mathbb{R}^d : d(x,A) > 2^{-k} \}$.

			Since $d(A^{c} \setminus D_{k+1} , D_k) > 0$, step 1 implies
			 \begin{align*}
				m(B \cap A^{c}) = \lim_{n \to \infty} m(B \cap D_k)
			\end{align*}

			And since $d(\mathbb{R}^d \setminus D_{k+1}, D_k ) > 0$, step 1 implies
			\begin{align*}
				m(B ) = \lim_{n \to \infty} m(B \cap (A \cup D_{k}))
			\end{align*}

			And finally since $d(A, D_{k}) > 0$,
			\begin{align*}
				m(B \cap (A \cup D_{k})) = m(B \cap A) + m(B \cap D_{k}).
			\end{align*}
	\end{enumerate}

	Taking $k \to \infty$ gives $m(B) = m(B \cap A) + m(B \cap A^{c})$
\end{proof}


\begin{theorem}
	The restriction of $H^{\alpha}$ to Borel sets is a measure.
\end{theorem}


\begin{definition}
	$\alpha$-dimensional Hausdorff measure is restriction of $H^{\alpha}$ to Borel sets.
\end{definition}


\begin{example}
	$H^{0}$ is counting measure.
\end{example}


\begin{example}
	$H^{\alpha} (A) = 0$ when $\alpha > 0$
	(insert image) we can cover  $[0,1]^{d}$ by $2^{nd}$ sub-cubes of size $2^{-n}$.

	This shows:
	\begin{align*}
		H^{\alpha} ([0,1]^d) \leq \lim_{n \to \infty} 2^{nd}( \sqrt{d} 2^{-n})^{\alpha} = 0
	\end{align*}
\end{example}

\begin{example}
	$H^{d} = \beta L$ for some $\beta > 0$
\end{example}

\begin{theorem}
	If $m$ Borel measure on $\mathbb{R}^{d}$, $m(x + A) = m(A)$ for $A \in \mathbb{R}^d$ and $A \subseteq \mathbb{R}^d$ Borel and $m([0,1]^{d}) =1$, then $m = L$
\end{theorem}

\begin{proof}
	Homework.
\end{proof}

\begin{definition}
	The Hausdorff dimension of a set $A \subseteq \mathbb{R}^d$ is
	\begin{align*}
		\dim_{H} (A) = \inf\{\alpha \geq 0 : H^{\alpha} (A) = 0 \}
	\end{align*}
\end{definition}

\begin{lemma}
	For any $d$, $A \subseteq \mathbb{R}^d$ there is an $\alpha \in [0,d]$ s.t.
	\begin{align*}
					H^{\beta} (A) &= 0 \quad \text{for} \quad  \beta > \alpha \\
		\text{and} \quad H^{\beta} (A) &= \infty \quad \text{for} \quad \beta < \alpha
	\end{align*}
\end{lemma}


\begin{example}
	the $\frac{1}{3}$ cantor set $\mathcal{C}$ as Hausdorff dim: $\alpha = \frac{\log 2}{\log 3}$
	\begin{proof}
		At step $K$ have cover of $\mathcal{C}$ consisting of $2^{k}$ intervals of length $3^{-k}$ this gives us
		\begin{align*}
		\lim_{k \to \infty} (2^{k}\cdot 3^{-\alpha}) \in (0, \infty) \\
		\iff \alpha = \frac{\log 2}{\log 3}
		\end{align*}
	\end{proof}

\end{example}

\begin{example}
	The Kock curve has dimension: $\alpha = \frac{\log 4}{\log 3} \in (1,2)$
	\begin{proof}
		Can cover with $4^{k}$ intervals of length $3^{-k}$
	\end{proof}

\end{example}

\begin{example}
	The Serpinski triangle has dimension: $\alpha = \frac{\log 3}{\log 2} \in (1,2)$
\end{example}


\section{10-17-2022}


Recall:

come back to

The $\alpha$-dimensional Hausdorff measure is the restriction of $H^{\alpha}$ to Borel sets.

\subsection{Hausdorff Dimension}

\begin{lemma}
	If $0 \leq \alpha \leq \beta$ then $H^{\beta}_r (A) \leq r^{\beta - \alpha} H^{\alpha}_{r} (A)$
\end{lemma}

\begin{proof}
	If $A \subseteq \bigcup_{k \geq 0} B_{k}$ and $\sup_{k \geq 0} diam (B_{k}) \leq r$, 
	then 
	\begin{align*}
		\sum_{k \geq 0} diam (B_{k})^{\beta} \leq r^{\beta - \alpha} \sum_{k \geq 0} diam (A_{k})^{\alpha}
	\end{align*} 
	using $ \beta - \alpha > 0$
\end{proof}

\begin{theorem}
	For any $A \subseteq \mathbb{R}^{d},$ 
	\begin{align*}
		diam_{H}(A) = \inf \{\alpha \geq 0 : H^{\alpha}(A) = 0\}
	\end{align*} 
	satisfies $diam_{H}(A) \in [0, d]$. moreover, if $\beta \in [0, \infty),$ then $H^{\beta}(A) = \infty$
\end{theorem}

\begin{proof}
	From last time, $H^{\beta}(A) = 0$ for any $\beta > d$. It follows that $\dim_{H}(A) \in [0, d]$.
	That $H^{\beta}(A) = \infty$ for $\beta \in [0, \infty)$ follows from the next lemma.
\end{proof}

\begin{lemma}
	If $0 \leq \alpha < \beta$m abd $H^{\alpha}(A) < \infty$, then $H^{(A) = 0}$
\end{lemma}

\begin{proof}
	If $H^{\alpha}(A) < \infty$, then
	\begin{align*}
		H^{\beta} (A) &= \sup_{r > 0} H^{\beta} (A) \sup H_{r}^{\beta} (A) \\
					  &= \lim_{r \to 0} H_{r}^{\beta} (A) \\
					  &\leq \limsup_{r \to 0} r^{\beta - \alpha} H_{r}^{\alpha} (A) \\
					  &\leq ( \limsup_{r \to 0} r^{\beta - \alpha} ) ( \limsup_{r \to 0} H_{r}^{\alpha} (A) ) \\
					  &\leq 0 \cdot H^{\alpha} (A) = 0
	\end{align*}
\end{proof}

For any $A \subseteq \mathbb{R}^d$ the graph of $\alpha \mapsto H^{\alpha} (A)$ looks like either 

insert graphs.

\begin{exercise}
	Find examples of fractals where each of the above graphs occurs.
\end{exercise}

\subsection{Fractals}

For self-similar fractals, find an equation that the fractal solves. Use the properties of the equation to study fractals.

\begin{example}
	For serpinski triangle $S \subseteq \mathbb{R}^2$:
	\begin{align*}
		\varphi_{j} (x, y) = (\frac{x}{2} + \cos(\frac{2 \pi}{3 j}), \frac{y}{2} + \cos(\frac{2 \pi}{3 j}) )
	\end{align*} for $j = 1,2,3, \ldots$, observe
	$S = \bigcup_{j = 1,2,3} \varphi_j (S) $
\end{example}

\begin{theorem}
	If $\varphi_{1}, \ldots, \varphi_{m} : \mathbb{R}^d \to \mathbb{R}^d$ are strict contracts meaning if there are $0 \leq r_{j} < 1$ s.t.
	\begin{align*}
		| Q_j (x) - Q_j (y) | \leq r_{j} |x - y |
	\end{align*} then there is a unique compact
	$K = \bigcup_{j=1}^{m} \varphi_{j} (k)$
\end{theorem}

\begin{proof}
	Use Banach contraction mapping. Let $(M,d)$ be the space of compact subsets of $\mathbb{R}^d$ equipped with Hausdorf measure.
	That is $M = \{K \subseteq \mathbb{R}^d\}$ where K is compact and non-empty
	and
	\begin{align*}
		d(K_1, K_2) = \max\{ \max_{x_1 \in K_1} \min_{x_2 \in K_2} |x_1 - x_2|, \max_{x_2 \in K_2} \min_{x_1 \in K_1} |x_1 - x_2| \}
	\end{align*}

	Define $\phi : M \to M$ by  $\phi (K) = \bigcup_{j=1}^{m} \phi_j (k)$
	\item[claim 1]: $\phi$ strict contraction
	\item[claim 2]: $(m,d)$ is complete.

	\item[Proof of claim 1]:
\end{proof}

\begin{theorem}
		If $\phi_1, \ldots, \phi_m: \mathbb{R}^d \to \mathbb{R}^d$ where $r_j \in (0,1)$ and 
		\begin{align*}
			|\phi_j (x) - \phi_j (y) | \leq r_{j} |x - y|
		\end{align*},
		then $\phi_j$ is an isometry-dilation.

		Consider $U \subseteq \mathbb{R}^d$ open, $\phi_1 (U), \ldots , \phi_m(U)$ disjoint,
		$U \contains \bigcup_{j = 1}^{m} \phi_j (U)$, then the unique compact $K$ solving $K = \bigcup_{j = 1}^{m}\phi_j (K)$ has 
		$0 < H^{\alpha}(K) < \infty$ where $\alpha$ determined by 
		$\sum_{j=1}^{m} r_{j}^{\alpha} = 1$
\end{theorem}

\begin{example}
		Let $U = (0,1)$
		\begin{align*}
			\phi_1 (x) &= \frac{x}{3} \quad \phi_{2} (x) = \frac{x}{3} + \frac{2}{3} \\
			\quad \phi_1 (U) & = (0,\frac{1}{3} ), \quad \phi_2 (U) = (\frac{2}{3}, 1) \\
		\end{align*} 
		then $2 \cdot (\frac{1}{3})^{\alpha} = 1 \iff \alpha = \frac{\log 2}{\log 3}$ 

		The open set condition guarentees $H^{\alpha}( \phi_j (K) \cap \phi_l (K)) = 0$ \,
		for $j \neq l$.
\end{example}


\section{2022-10-24}

\epigraph{``Today everything is named Hausdorff''}

\subsection{Hausdorff measure on Fractals}

\begin{remark}[hypothesis]
	if $r_1, \ldots, r_k \in [0,1)$ and $\phi_{1}, \ldots, \phi_{k} : \mathbb{R}^d \to \mathbb{R}^d$, then
	\begin{align*}
	|\phi_{j} (x) - \phi_{j} (y) | \leq r_j |x - y|
	\end{align*} 
\end{remark}

\begin{theorem}
	There is a unique compact $K \subseteq \mathbb{R}^d $ so $ K = \bigcup_{j=1}^{J} \phi_{j}(k)$
\end{theorem}

\begin{proof}
	Consider Hausdorff metric

	\begin{align*}
		d_H (K,G) = \max\{\max_{x\in K} \min_{y \in G} |x - y|, \max_{y\in G} \min_{x \in K} |y - k|\}
	\end{align*} 

	Define $\phi(K) =  \bigcup_{j=1}^{J} \phi_{j}(k)$

	\begin{enumerate}
		\item[claim 1] $d_H(\phi(K), \phi(G)) \leq  (\max_{j} r_j	) d_{H}(K,G)$

			\begin{align*}
				d_H (\phi(K) &\phi(G)) &\leq \max_{j} d_{H} (\phi_{j}(K), \phi_{j} (G)) \\
							 &\leq \max_{j} r_{j} d_{H}(K,G) \\
							 &\leq (\max_{j} r_{j}) d_{H}(K,G)
			\end{align*} 

		\item[Claim 2] the space of compact $K \subseteq \mathbb{R}^d$ with $d_H$ is complete metric space.

			Suppose $G_{K}$ compact and, $d_{H} (G_{K}, G_{K+1}) \leq 2^{-k}$

		$G = \bigcap_{n \geq 0} \overline{\bigcup_{k \geq n} G_{K}}$
		is compact since it is the limit of descending chain of compact $d_{H}(G, G_{K}) \leq 2^{-k}$

	\end{enumerate}
	Now back to the hypothesis. If we let 
	$r_1, \ldots, r_j \in (0,1)$ then
	\begin{align*}
		|\phi_j (x) - \phi_{j} (y) | = r_{j} |x - y| \qquad \text{Note equality!}
	\end{align*} 

	Note we also need an open set condition

	\begin{remark}
		There is a non-empty bounded open $U \subseteq \mathbb{R}^d$ s.t.
		\begin{align*}
			\phi_{1}(U), \ldots, \phi_{j}(U) \subseteq U \quad \text{and} \quad \phi_{1} (U), \ldots, \phi_{j} (U) \quad \text{disjoint}
		\end{align*} 

	\end{remark}
	
	\begin{example}
		$d = 2$, $j=3$.
		 \begin{align*}
		\phi_{j} (x,y) = \left(\frac{x}{2} + \cos( \frac{2\pi}{3}j), \frac{y}{2} + \sin( \frac{2\pi}{3}j)\right)
		\end{align*} 
	\end{example}
	
\end{proof}



\subsection{Moran's Theorem}

\begin{theorem}[Moran's 1946]

	For $0 < H^{\alpha} (K) < \infty$ where $\alpha \in [0,d]$, then the dimension is the unique $\alpha$ s.t.
	\begin{align*}
	r_{1}^{\alpha} + \ldots + r_{j}^{\alpha} = 1
	\end{align*} 
	
\end{theorem}


\subsubsection{scalling analysis}

$rA = \{rx : x \in A \}$ where  $r > 0$

 \begin{align*}
	 H^{\alpha} (rA) &= r^{\alpha} H^{\alpha} (A) \\
				K	&= \bigcup_{j=1}^{J} \phi_{j} (K)
\end{align*} 

If unione disjoint, then 
\begin{align*}
	H^{\alpha}(K) = \sum_{j=1}^{J} r_{j}^{\alpha} H^{\alpha} (K)
\end{align*} 
 The open set condition guarentees

 \begin{align*}
 	H^{\alpha} (\phi_{i} (k) \cap \phi_{j} (k) ) = 0 \quad \text{for $i \neq j$  (as we will see)}
 \end{align*} 

 \begin{proof}[Proof of Moran]
	 The main idea is to construct a finitely subadditive \\
	 \begin{align*}
		 m^{*} : \mathcal{P}(\R^{d}) \to [0, \infty]
	 \end{align*}
	 that lies on $K$ and then rescale so $\diam(V) = 1$

	 [insert pic]

	 \begin{align*}
		 \mathcal{U}_0 &= \{U \} \\
		 \mathcal{U}_{n+1} &= \{\phi_{j} (V) : V \in \mathcal{U}_{n}, \, j \in \{1, \ldots , J \}\} \\
		 \mathcal{U} &= \bigcup_{n \geq 0} \mathcal{U}_{n}
	 \end{align*} 
$\mathcal{U}$ is true of open sets ordered by inclusion, $\mathcal{U}_{n}$ ets at level $n$ are dosjiont (by induction).

\begin{enumerate}
	\item
		\begin{align*}
			\sum_{v \in \mathcal{U}_n} \diam(V)^{\alpha} = 1
		\end{align*} By induction 
		\begin{align*}
			\sum_{v \in \mathcal{U}_0} \diam(V)^{\alpha} &= \diam(V)^{\alpha} = 1^{\alpha} = 1 \\
			\sum_{v \in \mathcal{U}_{n+1}} \diam(V)^{\alpha} &=
			\sum_{j=1}^{J} \sum_{v \in \mathcal{U}_n} \diam(\phi(V))^{\alpha} = 
			\sum_{j=1}^{J} r_{j}^{\alpha} \sum_{v \in \mathcal{U}_n} \diam(V)^{\alpha} = \sum_{j=1}^{J} r_{j}^{\alpha} = 1
		\end{align*}

	\item $\max_{V \in \mathcal{U}_{n+1}} \diam (V) = (\max_{j} r_{j})^{n}$ goes to $0$ as $n \to 0$

	\item $K \subseteq \bigcup_{V \in \mathcal{U}_{n}} \bar{V}$ because $d_{H} ( \phi^{n}(\{x\}, K) \to 0$ and $\phi^{m}(\{x\}) \subseteq \bigcup_{V \in \mathcal{U}_{n}} V	$ for $m \geq n$

	\item $H^{\alpha} (K) \leq 1$ by (1), (2), and (3)
\end{enumerate}

For $A \subseteq \R^{d}$ define
\begin{align*}
	\mu_{n}(A) &= \sum_{\substack{V \in \mathcal{U}_n \\ \bar V \cap A \neq \emptyset}} \diam (V)^{\alpha}, \\
	\mu (A) &= \lim_{n \to \infty} \mu_{n}(A)
\end{align*}

\begin{remark}
	$\mu_n (A)$, is decreasing in $n$ because for $V \in \mathcal{U}_n$,
	\begin{align*}
		\sum_{\substack{w \in \mathcal{U}_{n+1} \\ w \subseteq V}} \diam (w)^{\alpha} = \diam (V)^{\alpha}
	\end{align*} 
\end{remark}

Check $\mu$ is a measure.

\begin{align*}
	\mu(\R^{\alpha}) &= 1 \text{   check??} \\
	A \subseteq B &\implies \mu(A) \leq \mu (B) \\
	\mu_{n} (A \cup B ) &\leq \mu_{n} (A) + \mu_{n} (B) \\
	\mu (A \cup B ) &\leq \mu (A) + \mu (B) \quad \text{by passing to limits} \\
\end{align*} 
finally,
$u(K) = 1$ : if $V \in \mathcal{U}_{n}$, then
\begin{align*}
	\bar{V} &= \phi_{j1}( \phi_{j2} ( \ldots \phi_{jn} (\bar{U}) \ldots )) \\
			&\contains \phi_{j1}( \phi_{j2} ( \ldots \phi_{jn} (K) \ldots )) \\
\end{align*} So $\mu_n (K) = 1$


Now $\mu (B_{\delta} (x) \leq C_{d,r_{1},\ldots, r_{d}, v} \delta^{\alpha}$ 

Let $V_{1}, \ldots, V_{2} \in \mathcal{U}$ be maximal elements of tree so $\bar{V} \cap B_{j} (x) \neq \emptyset$ \\
and $\diam(v) \leq \delta$ 


\begin{remark}
	Note $\diam(V_{l}) \geq (\min_{j} r_{j} ) \delta$ by maximality.

	and 
\end{remark}

\end{proof}


\begin{remark}
	this is basically using Calder\'{o}n-Zygmond. \\
	This was also the hardest proof so far.
\end{remark}

\begin{example}
	The restriction of $u$ to Borel sets $\mathcal{B}$ is given by
	\begin{align*}
		\mu ( \mathcal{B} ) = \frac{H^{\alpha} (\mathcal{B} \cap K)}{H^{\alpha} (K)}
	\end{align*} 
\end{example}

\begin{exercise}
	Suppose $C \subseteq  [0,1]$ is $\frac{1}{3}$ cantor set where $C = \bigcup_{n \geq 0} C_{n}$

	If $f \in C([0,1])$, then
	\begin{align*}
		\frac{\int f \1_{C_n}}{\int \1_{C_n}}  \xrightarrow[\enskip n \to \infty\enskip]{} 
		\frac{\int f \1_{c} d H^{\frac{\log(2)}{\log(3)}}}{\int \1_{c} d H^{\frac{\log(2)}{\log(3)}}}
	= \int f d\mu
	\end{align*} 
\end{exercise}



\section{2022-10-26}

\noindent
Last time: \\
Moran's theorem for fractals \\

\noindent
Today: \\
Return to rudin Narrative following chapter $3$: $L^{p}$-spaces.

\subsection{$L^{p}$-spaces}

Recall definition of $L^{1}$.

$m$ is a measure on $\sigma$-algebra $S$ on a set $X$.

$\mathcal{L}^{1}(m)$ is the space of measurable $f : [-\infty, \infty]$ such that

\begin{align*}
	[f]_{\mathcal{L}^{1}(m)} = \int |f| dm < \infty
\end{align*}

let $\mathcal{N}(m) = \{f : X \to [-\infty, \infty] \text{ meas }: m(\{x : f(x) \neq 0\} = 0\}$ be the set of functions that vanish away from a set of measure zero,

Observe
\begin{align*}
	[f]_{\mathcal{L}^1 (m) } = 0 \text{ for } f \in \mathcal{N}(m) \\
\end{align*} let
 \begin{align*}
	 L^{1} (m) &= \mathcal{L}^1 (m) / \mathcal{N} (m) \\
	 \|\{f\}\|_{L^1 (m) } &= \|f\|_{\mathcal{L}^1 (m) } 
\end{align*} 


\subsection{ Jensen's Inequality }
	
\begin{definition}[Convex]
	$\phi : \R \to \R$ is convex if and only if
	\begin{align*}
		\phi (( 1 - t) a + tb) \leq (1-t) \phi (a) + t \phi (b)
	\end{align*} for $a \leq b$ and $0 \leq t \leq 1$

	equivalently, 
	\begin{align*}
		\frac{\phi (c) - \phi (a)}{c - a} \leq \frac{\phi (b) - \phi (c)}{b - c} 
		\quad \text{for} \quad  a < b
	\end{align*} 
\end{definition}
	
\begin{lemma}
	Convex $\implies$ continuous
\end{lemma}

\begin{proof}
	$a < x < y_{k+1} < y_{k}$ where $y_{k} \xrightarrow[k \to \infty]{} \infty$ then
	\begin{align*}
		\frac{\phi(x) - \phi(a)}{x - a} &\leq \frac{\phi(y_{k+1}) - \phi (x)}{y_{k+1} - x} \\
										&\leq \frac{\phi(y_{k} - \phi(x)}{y_{k} - x}
	\end{align*} then
	\begin{align*}
		\frac{\phi(y_{k} - \phi(x)}{y_{k} - x} \to \alpha \quad \text{as} \quad k \to \infty \quad \text{and} \\
		\phi(y) = \phi(x) + \alpha (y - x) + \mathcal{O}( | y -x|) \quad \text{for} \quad y > x
	\end{align*} 
\end{proof}

\begin{remark}
	we could extend the notion of convexity to allow $\phi : [-\infty, \infty] \to (-\infty, \infty]$ 
	\begin{lemma}
		okay except for jumps [insert pic]
		\begin{align*}
			f(x) =
			\begin{cases}
				\infty & |x| > 1 \\
				0 &|x| \leq 1
			\end{cases}
		\end{align*} 
	\end{lemma}
	
\end{remark}

\begin{theorem}[Jensen's Inequality]
	$m(x) = 1$, $f \in C^{1}(m)$, $a < f < b$ and \\ 
	$\phi : (a,b) \to \R$ convex, then
	\begin{align*}
		\phi\left(\int f dm \right) \leq \int \phi(f) dm
	\end{align*} 
\end{theorem}

\begin{proof}
	Let $c = \int f dm$. Choose $\alpha \in \R$ so
	\begin{align*}
		\phi (x) \geq \phi (c) + \alpha (x - c) \quad x \in (a,b)
	\end{align*} 
	compute
	\begin{align*}
		\int \phi (f) dm &\geq \int \phi (c) + \alpha (f -c) dm \\
						 &= \phi (c) \int 1 dm + \alpha \int f dm - \alpha c \int 1 dm \\
						 &= \phi (c) + \alpha c - \alpha \\
						 &= \phi (c) = \phi \left( \int f dm \right)
	\end{align*}
\end{proof}

\begin{example}[Arithmatic - geometric mean inequality]
	$e^{x}$ convex
	($\frac{d^{2}}{dx^{2}} e^{x} = e^{x} \geq 0$ or $e^{x} = \sum \frac{x^{k}}{k!} =$ sum of convexterms.
\end{example}

\begin{align*}
	e^{\int f dm} \leq \int e^{f} dm \\
	e^{\log f dm} \leq \int f dm \quad f > 0 \\ 
\end{align*} let $m$ be the $\frac{1}{N} \times$ counting measure on $\{1, \ldots, N \}$ 

\begin{align*}
	e^{\frac{1}{N} \log f_{1} + \cdots + \log f_{N}} \leq \frac{1}{N} (f_{1} + \cdots + f_{N} ) \\
	(f_{1}, \ldots, f_{N})^{\frac{1}{N}} \leq \frac{f_{1} + \cdots + f_{N}}{N} \quad \text{for} \quad f_{1}, \ldots, f_{N} > 0
\end{align*} 

\begin{theorem}[Holder Inequality]
	Suppose $1 < p,q < \infty$ and $\frac{1}{p} + \frac{1}{q} = 1$.
	By convexity,
	\begin{align*}
		e^{\frac{s}{p} + \frac{t}{q}} \leq \frac{e^{s}}{p} + \frac{e^{t}}{q}
	\end{align*} 
\end{theorem}

\begin{remark}
	If $0 < a,b < \infty$, then letting $s = pbqa$ and $t = qbga$ [check]?
	gives
	\begin{align*}
		ab \leq \frac{a^{p}}{p} + b^{q}
	\end{align*} also true for $0 \leq a,b \leq \infty$
\end{remark}

\begin{theorem}[Holder]
	If $f,g: X \to [0,\infty]$ measurable, then 
	 \begin{align*}
		\int fg dm \leq \left( \int f^{p} dm\right)^{\frac{1}{p}} \left( \int g^{q} dm\right)^{\frac{1}{q}}
	\end{align*} 
\end{theorem}

\begin{proof}
	Assume $f,g \notin \mathscr{N}(m)$
	Scale so $\int f^{p} dm = \int g^{q} dm = 1$ then
	\begin{align*}
		\int fg dm &\leq \int \frac{f^{p}}{p} + \frac{g^{q}}{q} dm = \frac{1}{p} + \frac{1}{q} = 1 \\
				   &= \left( \int f^{p} dm \right)^{\frac{1}{p}} \left( \int g^{q} dm\right)^{\frac{1}{q}}
	\end{align*} 
\end{proof}

\begin{corollary}[Weak Minkowski]
\begin{align*}
	\left( \int (f + g)^{p} dm \right)^{1/p} \leq \left( \int f^{p} dm \right)^{1/q} + \left( \int g^{p} dm \right)^{1/p}
\end{align*} 	
\end{corollary}

\begin{proof}
	$\frac{1}{p} + \frac{p-1}{p} = 1$, $q = \frac{p}{p-1}$.

	\begin{align*}
		(f + g)^{p} &= (f+g) ( f+g)^{p-1} \\
		\int f ( f + g)^{p-1} dm &\leq \left( \int f^{p} dm \right)^{1/p} \left( \int (f+g)^{p} dm \right)^{\frac{p-1}{p}} \\
		\int g ( f + g)^{p-1} dm &\leq \left( \int g^{p} dm \right)^{1/p} \left( \int (f+g)^{p} dm \right)^{\frac{p-1}{p}} \\
	\int ( f + g)^{p} dm &\leq \left[ \left( \int f^{p} dm \right)^{1/p} + \left( \int g^{p} dm \right)^{1/p} \right] \left( \int (f+g)^{p} dm \right)^{\frac{p-1}{p}} \\
	\end{align*} 
\end{proof}

\subsection{$L^{p}$-Spaces}

\noindent For $1 \leq p < \infty$:

\noindent $\mathcal{L}^{p} (m)$ is space of measurable $f : X \to [-\infty, \infty]$ such that
 \begin{align*}
	\|f\|_{\mathcal{L}^{p} (m)} = \left( \int |f|^{p} dm \right)^{1/p} < \infty
\end{align*} 



\noindent $\mathcal{L}^{\infty} (m)$ is space of measurable $f : X \to [-\infty, \infty]$ such that
 \begin{align*}
 \|f\|_{\mathcal{L}^{\infty} (m)} = \inf\{\alpha \geq 0 : m ( \{ x \in X : |f(x)| > \alpha \}) =0 \}
\end{align*} 

Observe

\begin{align*}
	\|\alpha f\|_{\mathcal{L}^{p}(m)} &= \alpha \|f \|_{\mathcal{L}^{p} (m)} \quad \text{and by minkowski} \\
	\| f + g\|_{\mathcal{L}^{p}(m)} &=  \|f \|_{\mathcal{L}^{p} (m)} + \|g \|_{\mathcal{L}^{p} (m)}  \forall \alpha \geq 0 \quad \text{and} \quad f,g \in \mathcal{L}^{p} (m) \\
\end{align*} 
 let 
 \begin{align*}
 	L^{p} (m) = \mathcal{L}^{p} (m) / \mathscr{N} (m)
 \end{align*} 


 \begin{theorem}
 	$L^{p}(m)$ is complete normed vector space.
 \end{theorem}
 
 \begin{proof}
 	Need to show completeness.
Assume $f_{k} \in L^{p} (m)$ and 
$\| f_{k + 1} - f_{k} \|_{L^{p} (m)} \leq 2^{-k}$

Assume $1 \leq p < \infty$. Let  $g_{n} = \sum_{k=1}^{n} | f_{k}- f_{k+1} |$, $g = \lim_{n \to \infty} g_{n}$

By Minkowski:
\begin{align*}
	\left( \int g_{n}^{p} dm \right)^{1/p} \leq \sum_{k=1}^{n} \|f_{k} - f_{k+1} \|_{L^{p}(m)} \leq 1
\end{align*} By monotone convergence theorem. $\int g^p dm \leq 1$ 

In particular,
\begin{align*}
	f(x) = f_{1} (x) + \sum_{k=1}^{\infty} (f_{k+1}(x) - f_{k}(x))
\end{align*} is absolutely convergent away from measure zero set of $X$.

compute
\begin{align*}
	\int | f - f_{k} |^{p} dm &= \int \liminf_{j \to \infty} |f_{j} - j_{k} |^{p} dm \\
							  &\leq \liminf_{j \to \infty}\int  |f_{j} - j_{k} |^{p} dm \\
							  &\leq (2^{1-k})^{p} \xrightarrow[k \to \infty]{} 0
\end{align*} 
When $p = \infty$, let
\begin{align*}
	B = \bigcup_{k \geq 1} \{x \in X : |f_{k} (x) - f_{k +1} (x) | > 2^{-k}\}
\end{align*} by definition of $L^{\infty}$, $m(B) = 0$ for  $x \in XX \setminus B$, we have
\begin{align*}
	|f_{k +1} (x) - f_{k} (x)| &\leq 2^{-k} \\
	f(x) &=\lim_{k \to \infty} f_{k}(x) \quad \text{exists} \\
	|f (x) - f_{k} (x) | &\leq 2^{1-k}
\end{align*} So we conclude that
\begin{align*}
	\| f - f_{k} \|_{L^{\infty} (m)} \leq 2^{1 - k}
\end{align*} 
 \end{proof}
 
\subsubsection{$L^{p}$ on nice spaces}

$X$ locally compact Hausdorff. $S$ a Borel $\sigma$-algebra.

 
 \begin{theorem}[Lusin]
	 If $f: X \to [-\infty, \infty]$  measurable and $m(\{x \in X : f(x) \neq 0 \} ) < \infty$ and  $\varepsilon > 0$

	 Then there exists a $g \in C_{c} (X) $ such that $m( \{ x \in X : f(x) \neq g(x) \} ) < \varepsilon$.
	 Moreover, if $|f| \leq 1$, then $ \|g\| \leq 1$
 \end{theorem}
 
 \begin{remark}
 	This says that measurable functions are almost continuous.
 \end{remark}
 
\begin{proof}
	Reduce to case $X$ compact and $0 \leq f \leq 1$ (exercise).

	Write $f = \sum_{k=1}^{\infty} 2^{-k} \1_{A_{k}}$ where 
	\begin{align*}
		A_{1} &= \{ x \in X : f(x) \geq \frac{1}{2} \} \\
		A_{k} &= \{ x \in X : f(x) - \sum_{j=1}^{k} 2^{-k} \1_{A_{j}} \geq 2^{-k} \}
	\end{align*}
	Choose $\underset{open}{U_{k}} \contains A_{k} \contains \underset{compact}{G_{k}}$
	s.t. $m(U_{k} \setminus G_{k}) < \varepsilon 2^{-k}$ (by regularity)

	Now let $h_{k} \in C (x)$ with
	\begin{align*}
		\1_{G_{k}} \leq h_{k} \leq \1_{U_{k}}
	\end{align*} 
	let $g = \sum_{k=1}^{\infty} 2^{-k}h_{k}$
	Since $g$ is the uniform liit of a continuous function, $g$ is continuous.
	\begin{align*}
		\{ x \in X : f(x) \neq g(x) \} \subseteq \bigcup_{k \geq 0} ( U_{k} - G_{k} )
	\end{align*} and
	$\sum m( U_{k} \setminus G_{k} ) \leq \varepsilon$
\end{proof}
 
\begin{corollary}
	$C_c (X)$ is dense in $L^{p} (X)$ for $1 \leq p < \infty$
\end{corollary}

\begin{remark}
	This fails for $p = \infty$ in general.
	Any $L^{\infty}$ limit of functions in $C_c (X)$ lies in $C (X)$.
\end{remark}

Let $C_{0} (X)$ be $L^{\infty}$ closure of $C_c (X)$. These are continuous functions that "Vanish at $\infty$."

For example:
\begin{align*}
	C_{0} (\R^d ) &= \{ f\in C (\R^{d}) : \lim_{|x| \to \infty } f(x) = 0 \}
\end{align*} note by defaut $\R^{d}$ is equipped with the lebesgue measure.


\end{document}
